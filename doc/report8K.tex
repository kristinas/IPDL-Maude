\documentclass{article}

\usepackage{hyperref}
\usepackage{listings}

\title{Interface Improvements for IPDL}
\author{Kristina Sojakova \and Mihai Codescu}
\date{}
\newtheorem{remark}{Remark}

\begin{document}
\maketitle
\section*{\small Acknowledgement}
This project was funded through the NGI Assure Fund, a fund established by NLnet with financial support from the European Commission's Next Generation Internet programme, under the aegis of DG Communications Networks, Content and Technology under grant agreement No. 957073.
\section{General Design}

We have decided to integrate IPDL as a new language in the formal specification environment SpeX, developed and maintained by 
Ionu\c{t} \c{T}u\c{t}u at the Institute of Mathematics of the Romanian Academy. The main motivation in using SpeX is that it provides us
with parsing, static analysis and printing capabilities,
as well a command-line interface.  
These are things that we don't have to reimplement ourselves. 
SpeX is also implemented in Maude and has been designed with the
intention of making the cost of adding new languages and logics 
relatively small, and we greatly benefit from that. 
We have completed the integration in Spex 
of a fragment of IPDL: 
plain channels are supported but families of 
protocols are not, and the proof language is restricted to the part needed
in the helloWorld case study\footnote{See \url{https://github.com/kristinas/IPDL-Maude/blob/main/lib/csHelloWorld.ipdl}.}. For this language fragment we
have implemented the concrete syntax (CS) and its representation in the
abstract syntax (AS) takes place directly in the current prover state. 
Thus, the advantages of working with a CS are available:
the source file is more readable, we can hide from the user
some proof steps that only have to do with the way the protocols are 
represented, and we no longer have to use Maude identifiers, so names are
no longer preceeded by a prime character. The support for IPDL is still 
quite minimal (and will be fully extended to the whole language as part of the milestone 7), but the interface improvements that we planned are
already finished, as a proof of concept, for the fragment of IPDL that
is currently supported.

\section{Installation}

In order to install IPDL, Maude must be installed, using the following steps:
\begin{enumerate}
\item download the 3.3.1 version of Maude from its GitHub site\footnote{
\url{https://github.com/SRI-CSL/Maude/releases} }
\item extract the files in the downloaded ZIP archive to a convenient directory:
\begin{lstlisting}
sudo unzip Maude-linux.zip -d /usr/local/maude/
\end{lstlisting}
\item make sure that all users may run maude:
\begin{lstlisting}
sudo chmod a+x /usr/local/maude/Linux64/maude.linux64
\end{lstlisting}
\item make a discoverable link to the Maude executable, e.g.
\begin{lstlisting}
sudo ln -s /usr/local/maude/Linux64/maude.linux64
           /usr/local/bin/maude
\end{lstlisting}           
\item add to the .bashrc file an environment variable for Maude:
\begin{lstlisting}
export MAUDE_LIB=/usr/local/maude/Linux64
\end{lstlisting} 
\end{enumerate}

Now we should be able to run \textbf{maude} at the command line.

IPDL can now be installed by downloading the archive at \url{https://github.com/kristinas/IPDL-Maude/blob/main/src/IPDL.tar.xz}
and running the following commands:

\begin{lstlisting}
tar -xJf IPDL.tar.xz && cd IPDL
autoreconf -i
./configure
make
sudo make install 
\end{lstlisting}

Now we can run \textbf{ipdl} at the command line, and to test the
helloWorld example, run \textbf{load csHelloWorld.ipdl}.
 
\section{Highlighting}

We have implemented a simple coloring scheme for protocols.
For protocol compositions, which can be arbitrarily long, the color
of each operand changes by looping through a list of three colors. This
will have the benefit that in the case of branching, same protocols
will have the same color on different branches (once IPDL will be
fully supported in this implementation).

\section{Current Protocol}

The command \textbf{current-protocol} provides us the 
current protocol and the command \textbf{get-channel X} allows us to 
inspect only the channel \textbf{X} from the current protocol.
Since the result of applying a rule to a 
configuration is no longer displayed as a full
configuration, like in the original Maude implementation,
which was difficult to read,
this command becomes more important now.
 
\section{Subproofs}

The original goal here was displaying the protocol that we
need to rewrite when doing a subproof.
We have replaced this with a workflow that
provides with full support for subproofs.
The main idea can be easily explained with the help of the simple
example of the \textbf{SYM} rule (adapted to channel), that says that
if our current configuration is 
$$\textbf{C =  pConfig(Sigma, Delta2, P2, I, O2, A) }$$ and we can rewrite the configuration 
$$\textbf{C' = pConfig(Sigma, Delta1, P1, I, O1, A)}$$ \noindent for some protocol \textbf{P1} to \textbf{C},
then we can rewrite \textbf{C} to \textbf{C'}.
This means one has to provide the subproof that \textbf{C'} rewrites
to \textbf{C}. In large case studies, working on a subproof is tedious as 
it requires us to write the start configuration by hand, as a separate proof. 

The idea behind our workflow
is that instead of working with a configuration, we work with a stack of
configurations. When we want to apply a rule with subproofs, the start
configurations for them will be known and can be pushed in the stack (in our case \textbf{C'}). 
The syntax for this is 
$$\textbf{try sym from P1 : todo}$$ 
\noindent with the marker \textbf{todo} showing that
the proof still needs to be provided. 
The user can then continue working directly on
\textbf{C'} by adding proof steps (which may have subproofs themselves, and then the stack grows):
$$ \textbf{try sym from P1 : proofStep1 then proofStep2} $$
\noindent
and when the subproof is finished it can be closed
$$
\textbf{try sym from P1 : proofStep1 then proofStep2 done}
$$\noindent
and this entire construction is equivalent with
an application of the \textbf{SYM} rule:
$$
\textbf{sym from P1 ( proofStep1 then proofStep2)}
$$

This generalizes to the case of rules with two subproofs in the expected way, when the first subproof is finished we can pop from the stack and
continue working on the second subproof. IPDL rules have at most two subproofs. 

Being able to work on the subproofs directly makes proof discovery much 
easier and less error-prone.


\end{document}