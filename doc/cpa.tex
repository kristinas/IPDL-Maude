\newcommand{\key}{\mathsf{key}}
\newcommand{\msg}{\mathsf{msg}}
\newcommand{\ctxt}{\mathsf{ctxt}}
\newcommand{\gen}{\mathsf{gen}}
\newcommand{\enc}{\mathsf{enc}}
\newcommand{\zeros}{\mathsf{zeros}}
\newcommand{\Msg}{\mathsf{Msg}}
\newcommand{\MsgL}{\mathsf{Msg}_L}
\newcommand{\MsgR}{\mathsf{Msg}_R}
\newcommand{\Key}{\mathsf{Key}}
\newcommand{\Enc}{\mathsf{Enc}}

Semantic security, also known as \emph{security against chosen-plaintext attacks (CPA)}, for a symmetric-key encryption schema states that encoding $q$ fixed messages with a secret key is computationally indistinguishable from encoding any other $q$ messages. Or equivalently, encoding $q$ fixed messages with a secret key is computationally indistinguishable from encoding the same message $q$ times. We now show this equivalence in \textsf{IPDL}.

Formally, we assume types $\key, \msg, \ctxt$ of keys, messages, and ciphertexts, respectively; a chosen message $\zeros : \one \rightarrow \msg$; a probabilistic key generation algorithm $\gen_\key : \one \twoheadrightarrow \key$; and a probabilistic encryption algorithm $\enc : \msg \times \key \twoheadrightarrow \ctxt$ that takes a message and a key, and returns a ciphertext. We will write $\enc(m,k)$ in place of $\enc \ (m,k)$.

We can express the standard (\emph{left-right}) version of the CPA cryptographic assumption as the following protocol-level axiom: in the channel context $\big\{\MsgL(i) : \msg\big\}_i, \, \big\{\MsgR(i) : \msg\big\}_i, \, \Key : \key, \, \big\{\Enc(i) : \ctxt\big\}_i$ where $i \coloneqq 1,\dots,q$, the protocol
\begin{itemize}
\item $\Key \coloneqq \samp{\gen_\key}$
\item $\Enc(i) \coloneqq m_L \leftarrow \MsgL(i); \ m_R \leftarrow \MsgR(i); \ k \leftarrow \Key; \ \samp{\enc({\color{red} m_L},k)}$ for $0 \leq i < q$
\end{itemize}
with inputs $\MsgL(-)$, $\MsgR(-)$, outputs $\Enc(-)$, and an internal channel $\Key$ rewrites approximately to the protocol
\begin{itemize}
\item $\Key \coloneqq \samp{\gen_\key}$
\item $\Enc(i) \coloneqq m_L \leftarrow \MsgL(i); \ m_R \leftarrow \MsgR(i); \ k \leftarrow \Key; \ \samp{\enc({\color{red} m_R},k)}$ for $0 \leq i < q$
\end{itemize}
The \emph{chosen-message} version of CPA security is another protocol-level axiom: in the channel context $\big\{\Msg(i) : \msg\big\}_i, \Key : \key, \{\Enc(i) : \ctxt\}_i$ where $i \coloneqq 1,\dots,q$, the protocol
\begin{itemize}
\item $\Key \coloneqq \samp{\gen_\key}$
\item $\Enc(i) \coloneqq m \leftarrow \Msg(i); \ k \leftarrow \Key; \ \samp{\enc({\color{red} m},k)}$ for $0 \leq i < q$
\end{itemize}
with inputs $\Msg(-)$, outputs $\Enc(-)$, and an internal channel $\Key$ rewrites approximately to the protocol
\begin{itemize}
\item $\Key \coloneqq \samp{\gen_\key}$
\item $\Enc(i) \coloneqq m \leftarrow \Msg(i); \ k \leftarrow \Key; \ \samp{\enc{(\color{red} \zeros},k)}$ for $0 \leq i < q$
\end{itemize}

\noindent To prove that these are equivalent, assume first the \emph{left-right} version of CPA security. The protocol
\begin{itemize}
\item $\Key \coloneqq \samp{\gen_\key}$
\item $\Enc(i) \coloneqq m \leftarrow \Msg(i); \ k \leftarrow \Key; \ \samp{\enc(m,k)}$ for $0 \leq i < q$
\end{itemize}
can be factored as the composition of the protocol
\begin{itemize}
\item $\Key \coloneqq \samp{\gen_\key}$
\item $\Enc(i) \coloneqq m_L \leftarrow \MsgL(i); \ m_R \leftarrow \MsgR(i); \ k \leftarrow \Key; \ \samp{\enc(m_L,k)}$ for $0 \leq i < q$
\end{itemize}
(the left-hand side of our CPA assumption) and the protocol
\begin{itemize}
\item $\MsgL(i) \coloneqq \read{\Msg(i)}$ for $0 \leq i < q$
\item $\MsgR(i) \coloneqq \ret{\zeros}$ for $0 \leq i < q$
\end{itemize}
followed by the hiding of the channels $\MsgL(-)$, $\MsgR(-)$.

Applying the CPA assumption yields the composition of the protocol
\begin{itemize}
\item $\Key \coloneqq \samp{\gen_\key}$
\item $\Enc(i) \coloneqq m_L \leftarrow \MsgL(i); \ m_R \leftarrow \MsgR(i); \ k \leftarrow \Key; \ \samp{\enc(m_R,k)}$ for $0 \leq i < q$
\end{itemize}
and the (unchanged) protocol
\begin{itemize}
\item $\MsgL(i) \coloneqq \read{\Msg(i)}$ for $0 \leq i < q$
\item $\MsgR(i) \coloneqq \ret{\zeros}$ for $0 \leq i < q$
\end{itemize}
followed by the hiding of the channels $\MsgL(-)$, $\MsgR(-)$. 

Reversing the factorization process yields the protocol
\begin{itemize}
\item $\Key \coloneqq \samp{\gen_\key}$
\item $\Enc(i) \coloneqq m \leftarrow \Msg(i); \ k \leftarrow \Key; \ \samp{\enc(\zeros,k)}$ for $0 \leq i < q$
\end{itemize}
as desired.

For the other direction, assume the \emph{chosen-message} version of CPA security. The protocol
\begin{itemize}
\item $\Key \coloneqq \samp{\gen_\key}$
\item $\Enc(i) \coloneqq m_L \leftarrow \MsgL(i); \ m_R \leftarrow \MsgR(i); \ k \leftarrow \Key; \ \samp{\enc(m_L,k)}$ for $0 \leq i < q$
\end{itemize}
can be factored as the composition of the protocol
\begin{itemize}
\item $\Key \coloneqq \samp{\gen_\key}$
\item $\Enc(i) \coloneqq m \leftarrow \Msg(i); \ k \leftarrow \Key; \ \samp{\enc(m,k)}$ for $0 \leq i < q$
\end{itemize}
(the left-hand side of our CPA assumption) and the protocol
\begin{itemize}
\item $\Msg(i) \coloneqq m_L \leftarrow \MsgL(i); \ m_R \leftarrow \MsgR(i); \ \ret{m_L}$ for $0 \leq i < q$
\end{itemize}
followed by the hiding of the channels $\Msg(-)$.

Applying the CPA assumption yields the composition of the protocol
\begin{itemize}
\item $\Key \coloneqq \samp{\gen_\key}$
\item $\Enc(i) \coloneqq m \leftarrow \Msg(i); \ k \leftarrow \Key; \ \samp{\enc(\zeros,k)}$ for $0 \leq i < q$
\end{itemize}
and the (unchanged) protocol
\begin{itemize}
\item $\Msg(i) \coloneqq m_L \leftarrow \MsgL(i); \ m_R \leftarrow \MsgR(i); \ \ret{m_L}$ for $0 \leq i < q$
\end{itemize}
followed by the hiding of the channels $\Msg(-)$.

Substituting the channels $\Msg(-)$ away yields the protocol
\begin{itemize}
\item $\Key \coloneqq \samp{\gen_\key}$
\item $\Enc(i) \coloneqq m_L \leftarrow \MsgL(i); \ m_R \leftarrow \MsgR(i); \ k \leftarrow \Key; \ \samp{\enc(\zeros,k)}$ for $0 \leq i < q$
\end{itemize}

The above can be factored as the composition of the protocol
\begin{itemize}
\item $\Key \coloneqq \samp{\gen_\key}$
\item $\Enc(i) \coloneqq m \leftarrow \Msg(i); \ k \leftarrow \Key; \ \samp{\enc(\zeros, k)}$ for $0 \leq i < q$
\end{itemize}
(the right-hand side of our CPA assumption) and the protocol
\begin{itemize}
\item $\Msg(i) \coloneqq m_L \leftarrow \MsgL(i); \ m_R \leftarrow \MsgR(i); \ \ret{m_R}$ for $0 \leq i < q$
\end{itemize}
followed by the hiding of the channels $\Msg(-)$.

Applying the CPA assumption again (this time from right to left) yields the composition of the protocol
\begin{itemize}
\item $\Key \coloneqq \samp{\gen_\key}$
\item $\Enc(i) \coloneqq m \leftarrow \Msg(i); \ k \leftarrow \Key; \ \samp{\enc(m, k)}$ for $0 \leq i < q$
\end{itemize}
and the (unchanged) protocol
\begin{itemize}
\item $\Msg(i) \coloneqq m_L \leftarrow \MsgL(i); \ m_R \leftarrow \MsgR(i); \ \ret{m_R}$ for $0 \leq i < q$
\end{itemize}
followed by the hiding of the channels $\Msg(-)$.

Reversing the factorization process yields the protocol
\begin{itemize}
\item $\Key \coloneqq \samp{\gen_\key}$
\item $\Enc(i) \coloneqq m_L \leftarrow \MsgL(i); \ m_R \leftarrow \MsgR(i); \ k \leftarrow \Key; \ \samp{\enc(m_R, k)}$ for $0 \leq i < q$
\end{itemize}
as desired.