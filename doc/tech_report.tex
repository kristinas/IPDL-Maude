\documentclass[11pt,hidelinks]{article}
\usepackage{amsmath, amssymb, amsthm, stmaryrd}
\usepackage{url}
\usepackage{tikz}
\usetikzlibrary{positioning}
\usepackage[override]{cmtt}
\usepackage{textcomp}
\usepackage{cryptocode}
\usepackage[T1]{fontenc}
\usepackage{fancyvrb}
\usepackage{mathtools}
\usepackage{savesym}
\usepackage{physics}
\savesymbol{div}
\usepackage{listings}
\usepackage{xcolor}
\usepackage{graphics}
\usepackage{caption, subcaption}
\usepackage{xspace}
\usepackage{pl-syntax}
\usepackage{tensor}
\usepackage{adjustbox}
\usepackage{graphicx}
\usepackage{mathpartir}
\usepackage[margin=0.5in]{geometry}
\usepackage{mathtools}
\usepackage{float}
\usepackage[colorlinks]{hyperref}
\usepackage{mathabx}
\hypersetup{linkcolor=blue,filecolor=blue,citecolor=blue,urlcolor=blue}

\newcommand{\nat}{\mathbb{N}}
\newcommand{\ipdl}{\textsf{IPDL} }
\newcommand{\type}{\mathsf{t}}
\newcommand{\func}{\mathsf{f}}
\newcommand{\dist}{\mathsf{d}}
\newcommand{\one}{\mathsf{1}}
\newcommand{\Bool}{\mathsf{Bool}}
\newcommand{\true}{\mathsf{true}}
\newcommand{\false}{\mathsf{false}}
\newcommand{\fst}{\mathsf{fst}}
\newcommand{\snd}{\mathsf{snd}}
\newcommand{\ret}[1]{\mathsf{ret} \ #1}
\newcommand{\samp}[1]{\mathsf{samp} \ #1}
\renewcommand{\read}[1]{\mathsf{read} \ #1}
\newcommand{\ifte}[3]{\mathsf{if} \ #1 \ \mathsf{then} \ #2 \ \mathsf{else} \ #3}
\newcommand{\zero}{\mathsf{0}}
\newcommand{\assign}[2]{#1 \coloneqq #2}
\newcommand{\Par}[2]{#1 \; || \; #2}
\newcommand{\new}[3]{\mathsf{new} \ #1 : #2 \ \mathsf{in} \ #3}
\newcommand{\axiom}{\mathsf{axiom}}
\renewcommand{\approxeq}[6]{#1 \approx #2 : #3 \to #4 \ \mathsf{width} \ #5 \ \mathsf{length} \ #6}
\newcommand{\sem}[1]{\llbracket #1 \rrbracket}
\newcommand{\val}[1]{\mathsf{val} \ #1}
\newcommand{\outstep}[2]{\xmapsto{#1 \, \coloneqq \, #2}}
\newcommand{\stuck}{\mathsf{stuck}}
\newcommand{\final}{\mathsf{final}}
\renewcommand{\norm}[1]{\Vert #1 \Vert}
\renewcommand{\eval}[1]{#1 \! \Downarrow}
\newcommand{\lift}[1]{\mathcal{L}(#1)}
%\newcommand{\in}[2]{\langle #1 \coloneqq #2 \rangle}
\newcommand{\pull}[2]{\stackrel{\hookleftarrow}{#1 \, \coloneqq \, #2}}
\newcommand{\newNf}{\mathsf{newNF}}
\newcommand{\preNf}{\mathsf{preNF}}
\newcommand{\nf}{\mathsf{NF}}

\newtheorem{lemma}{Lemma}
\newtheorem{theorem}{Theorem}						
\newtheorem{definition}{Definition}	
\newtheorem{corollary}{Corollary}	
\newtheorem{example}{Example}

\begin{document}
\title{A Core Calculus for Equational Proofs of Distributed Cryptographic Protocols: Technical Report}

\maketitle

\section*{\small Acknowledgement}
This project was funded through the NGI Assure Fund, a fund established by NLnet with financial support from the European Commission's Next Generation Internet programme, under the aegis of DG Communications Networks, Content and Technology under grant agreement No. 957073.

\section{Syntax of \ipdl}
\newcommand{\Sim}{\mathsf{Sim}}

\ipdl is built from four layers: \emph{protocols} are networks of
mutually interacting \emph{reactions}, which are simple monadic programs. Each reaction computes an \emph{expression} probabilistically: \emph{i.e.}, the computation may include sampling from \emph{distributions}. In the context of a protocol, a reaction operates on a unique \emph{channel} and may read from other channels, thereby utilizing computations coming from other reactions. The syntax and judgements of \ipdl are outlined in Figures \ref{fig:syntax}, \ref{fig:judgements}, respectively, and are parameterized by a user-defined \emph{signature} $\Sigma$:

\begin{definition}[Signature]
An \ipdl signature $\Sigma$ is a finite collection of:
\begin{itemize}
\item type symbols $\type$;
\item typed function symbols $\func : \tau \rightarrow \sigma$; and
\item typed distribution symbols $\dist : \tau \twoheadrightarrow \sigma$.
\end{itemize}
\end{definition}

We have a minimal set of data types, including the unit type $\one$, Booleans, products, as well as arbitrary type symbols $\mathsf{t}$, drawn from the signature $\Sigma$. Expressions are used for non-probabilistic computations, and are standard. All values in \ipdl are bitstrings of a length given by data types, so we annotate the operations $\fst_{\tau \times \sigma}$ and $\snd_{\tau \times \sigma}$ with the type of the pair to determine the index to split the pair into two; for readability we omit this subscript whenever appropriate. Function symbols $\func$ must be declared in the signature $\Sigma$, and for a constant $\func : \one \rightarrow \tau$, we write $\func$ in place of $\func \ \checkmark$. Substitutions $\theta : \Gamma_1 \to \Gamma_2$ between type contexts are standard.

Analogously to function symbols, distribution symbols $\dist$ must be declared in the signature $\Sigma$, and for a constant $\dist : \one \twoheadrightarrow \tau$, we write $\samp{\dist}$ instead of $\samp{(\dist \ \checkmark)}$. As mentioned above, reactions are monadic programs which may return expressions, sample from distributions, read from channels, branch on a value of type $\Bool$, and sequentially compose. For readability, we often omit the type of the bound variable in a sequential composition, and write $x \leftarrow \read{c}; \ R$ and $x \leftarrow \samp{d}; \ R$ simply as $x \leftarrow c; \ R$ and $x \leftarrow d; \ R$ wherever appropriate. Protocols in \ipdl are given by a simple but expressive syntax: channel assignment $\assign{o}{R}$ assigns the reaction $R$ to channel $o$; parallel composition $\Par{P}{Q}$ allows $P$ and $Q$ to freely interact concurrently; and channel generation $\new{o}{\tau}{P}$ creates a new, internal channel for use in $P$. \emph{Embeddings} $\phi : \Delta_1 \to \Delta_2$ between channel contexts are injective, type-preserving mappings specifying how to rename channels in $\Delta_2$ to fit in the larger context $\Delta_1$.

\begin{figure}[ht]
\begin{syntax}
  \category[Data Types]{\tau, \sigma}
    \alternative{\type}
		\alternative{\one}
    \alternative{\Bool}
    \alternative{\tau \times \tau}

  \category[Expressions]{e}
    \alternative{x}
    \alternative{\checkmark}
	  \alternative{\true}
	  \alternative{\false}		
	  \alternative{\func \ e} 
	  \alternative{(e_1,e_2)}    
	  \alternative{\fst_{\tau \times \sigma} \ e}
		\alternative{\snd_{\tau \times \sigma} \ e}		

  \category[Distributions]{d}
		\alternative{\dist \ e} 
	
  \abstractCategory[Channels]{i, o, c}

  \category[Reactions]{R, S}
    \alternative{\ret{e}}
    \alternative{\samp{d}}
    \alternative{\read{c}}		
    \alternative{\ifte{e}{R_1}{R_2}}
    \alternative{x : \sigma \leftarrow R; \ S}         

	\category[Protocols]{P, Q}
	  \alternative{\zero}	
	  \alternative{\assign{o}{R}}
	  \alternative{\Par{P}{Q}}
	  \alternative{\new{o}{\tau}{P}}		

	\category[Channel Sets]{I, O}
	  \alternative{\{c_1, \ldots, c_n\}}
		
  \category[Type Contexts]{\Gamma}
    \alternative{\cdot}
    \alternative{\Gamma, x : \tau}

  \category[Channel Contexts]{\Delta}
    \alternative{\cdot}
    \alternative{\Delta, c : \tau}
\end{syntax}
\caption{Syntax of \textsf{\ipdl}.}
\label{fig:syntax}
\end{figure}

\begin{figure}[ht]
\begin{syntax}
  \abstractCategory[Expression Typing]{\Gamma \vdash e : \tau}
	\abstractCategory[Distribution Typing]{\Gamma \vdash d : \tau}
  \abstractCategory[Reaction Typing]{\Delta; \ \Gamma \vdash R : I \to \tau}
  \abstractCategory[Protocol Typing]{\Delta \vdash P : I \to O} \\

  \abstractCategory[Substitutions]{\theta : \Gamma_1 \to \Gamma_2}
  \abstractCategory[Embeddings]{\phi : \Delta_1 \to \Delta_2} \\

	\abstractCategory[Expression Equality]{\Gamma \vdash e_1 = e_2 : \tau}
	\abstractCategory[Distribution Equality]{\Gamma \vdash d_1 = d_2 : \tau}
  \abstractCategory[Reaction Equality]{\Delta; \ \Gamma \vdash R_1 = R_2 : I \to \tau}
  \abstractCategory[Protocol Equality (Strict)]{\Delta \vdash P_1 = P_2 : I \to O}
\end{syntax}
\caption{Judgements of the exact fragment of \textsf{ipdl}.}
\label{fig:judgements}
\end{figure}

\subsection{Typing}
We restrict our attention to well-typed \ipdl constructs. In addition to respecting data types, the typing judgments guarantee that all reads from channels in reactions are in scope, and that all channels are assigned at most one reaction in protocols. The typing $\Gamma \vdash e : \tau$ and $\Gamma \vdash d : \tau$ for expressions and distributions is standard, see Figures \ref{fig:expressions_typing} and \ref{fig:distributions_typing}. Figure \ref{fig:reactions_typing} shows the typing rules for reactions. Intuitively, $\Delta; \ \Gamma \vdash R : I \to \tau$ holds when $R$ uses variables in $\Gamma$, reads from channels in $I$ typed according to $\Delta$, and returns a value of type $\tau$. Figure \ref{fig:protocols_typing} gives the typing rules for protocols: $\Delta \vdash P : I \to O$ holds when $P$ uses inputs in $I$ to assign reactions to the channels in $O$, all typed according to $\Delta$.

Channel assignment $\assign{o}{R}$ has the type $I \to \{o\}$ 
when $R$ is well-typed with an empty variable context, making use of inputs from $I$ as well as of $o$. We allow $R$ to read from its own output $o$ to express divergence: the protocol $\assign{o}{\read{o}}$ cannot reduce, which is useful for (conditionally) deactivating certain outputs. The typing rule for parallel composition $\Par{P}{Q}$ states that $P$ may use the outputs of $Q$ as inputs while defining its own outputs, and vice versa. Importantly, the typing rules ensure that the outputs of $P$ and $Q$ are disjoint so that each channel carries a unique reaction. Finally, the rule for channel generation allows a protocol to select a fresh channel name $o$, assign it a type $\tau$, and use it for internal computation and communication. Protocol typing plays a crucial role for modeling security. 
Simulation-based security in \ipdl is modeled by the existence of a \emph{simulator}  with an appropriate typing judgment, $\Delta \vdash \Sim : I \to O$. Restricting the behavior of $\Sim$ to only use inputs along $I$ is necessary to rule out trivial results (\emph{e.g.}, $\Sim$ simply copies a secret from the specification).

\begin{figure}[ht]
\begin{mathpar}
\fbox{$\Gamma \vdash e : \tau$}\\
\inferrule*{x : \tau \in \Gamma}{\Gamma \vdash x : \tau}\and
\inferrule*{ }{\Gamma \vdash \checkmark : \one}\and
\inferrule*{ }{\Gamma \vdash \true : \Bool}\and
\inferrule*{ }{\Gamma \vdash \false : \Bool}\and
\inferrule*{\func : \sigma \rightarrow \tau \in \Sigma \\ \Gamma \vdash e : \sigma}{\Gamma \vdash \func \ e : \tau}\and
\inferrule*{\Gamma \vdash e_1 : \tau_1 \\ \Gamma \vdash e_2 : \tau_2}{\Gamma \vdash (e_1,e_2) : \tau_1 \times \tau_2}\and
\inferrule*{\Gamma \vdash e : \sigma \times \tau}{\Gamma \vdash \fst_{\sigma \times \tau} \ e : \sigma}\and
\inferrule*{\Gamma \vdash e : \sigma \times \tau}{\Gamma \vdash \snd_{\sigma \times \tau} \ e : \tau}
\end{mathpar}
\caption{Typing for \ipdl expressions.}
\label{fig:expressions_typing}
\end{figure}

\begin{figure}[ht]
\begin{mathpar}
\fbox{$\Gamma \vdash d : \tau$}\\
\inferrule*{\dist : \sigma \twoheadrightarrow \tau \in \Sigma \\ \Gamma \vdash e : \sigma}{\Gamma \vdash \dist \ e : \tau}
\end{mathpar}
\caption{Typing for \ipdl distributions.}
\label{fig:distributions_typing}
\end{figure}

\begin{figure*}[ht]
\begin{mathpar}
\fbox{$\Delta; \ \Gamma \vdash R : I \to \tau$}\\
\inferrule*{\Gamma \vdash e : \tau}{\Delta; \ \Gamma \vdash \ret{e} : I \to \tau}\and
\inferrule*{\Gamma \vdash d : \tau}{\Delta; \ \Gamma \vdash \samp{d} : I \to \tau}\and
\inferrule*{i : \tau \in \Delta \\ i \in I}{\Delta; \ \Gamma \vdash \read{i} : I \to \tau}\and
\inferrule*{\Gamma \vdash e : \Bool \\ \Delta; \ \Gamma \vdash R_1 : I \to \tau \\ \Delta; \ \Gamma \vdash R_2 : I \to \tau}{\Delta; \
\Gamma \vdash \ifte{e}{R_1}{R_2} : I \to \tau}\and
\inferrule*{\Delta; \ \Gamma \vdash R : I \to \sigma \\ \Delta; \ \Gamma, x : \sigma \vdash S : I \to \tau}{\Delta; \ \Gamma \vdash (x : \sigma \leftarrow R; \ S) : I \to \tau}
\end{mathpar}
\caption{Typing for \ipdl reactions.}
\label{fig:reactions_typing}
\end{figure*}

\begin{figure*}[ht]
\begin{mathpar}
\fbox{$\Delta \vdash P : I \to O$}\\
\inferrule*{ }{\Delta \vdash \zero : I \to \emptyset}\and
\inferrule*{o : \tau \in \Delta \\ o \notin I \\ \Delta; \ \cdot \vdash R : I \cup \{o\} \to \tau}{\Delta \vdash (\assign{o}{R}) : I \to \{o\}}\and
\inferrule*{\Delta \vdash P : I \cup O_2 \to O_1 \\ \Delta \vdash    Q : I \cup O_1 \to O_2}{\Delta \vdash \Par{P}{Q} : I \to O_1 \cup O_2}\and
\inferrule*{\Delta, o : \tau \vdash P : I \to O \cup \{o\}}{\Delta \vdash (\new{o}{\tau}{P}) : I \to O}
\end{mathpar}
\caption{Typing for \ipdl protocols.}
\label{fig:protocols_typing}
\end{figure*}

\subsection{Equational Logic}
We now present the equational logic of \ipdl. As mentioned above, the logic is divided into \emph{exact} rules that establish semantic equivalences between protocols, and \emph{approximate} rules that are used to discharge computational indistinguishability assumptions. 

\subsubsection{Exact Equality}
The bulk of the reasoning in \ipdl is done using exact equalities. At the expression level, we assume an ambient finite set of axioms of the form $\Gamma \vdash e_1 = e_2 : \tau$, where $\Gamma \vdash e_1 : \tau$ and $\Gamma \vdash e_2 : \tau$. The rules for expression and distribution equality are standard, see Figures \ref{fig:expressions_equality} and \ref{fig:distributions_equality}.

At the reaction level, we analogously assume an ambient finite set of axioms of the form $\Delta; \ \Gamma \vdash R_1 = R_2 : I \to \tau$, where $\Delta; \ \Gamma \vdash R_1 : I \to \tau$ and $\Delta; \ \Gamma \vdash R_2 : I \to \tau$. The rules for reaction equality, shown in Figures \ref{fig:reactions_equality_1} and \ref{fig:reactions_equality_2}, ensure in particular that reactions form a \emph{commutative monad}: we have \[\big(x \leftarrow R_1; \ y \leftarrow R_2; \ S(x,y)\big) = \big(y \leftarrow R_2; \ x \leftarrow R_1; \ S(x,y)\big)\] whenever $R_2$ does not depend on $x$. All expected equivalences for commutative monads hold for reactions, including the usual monad laws and congruence of equivalence under monadic bind. The \textsc{samp-pure} rule allows us to drop an unused sampling, and the \textsc{read-det} rule allows us to replace two reads from the same channel by a single one. The rules \textsc{if-left}, \textsc{if-right}, and \textsc{if-ext} allow us to manipulate conditionals.

At the protocol level, we similarly assume an ambient finite set of axioms of the form $\Delta \vdash P_1 = P_2 : I \to O$, where $\Delta \vdash P_1 : I \to O$ and $\Delta \vdash P_2 : I \to O$. We use these axioms to specify user-defined functional assumptions, \emph{e.g.}, the correctness of decryption. Exact protocol equivalences allow us to reason about communication between subprotocols and functional correctness, and to simplify intermediate computations. We will see later that exact equivalence implies the existence of a \emph{bisimulation} on protocols, which in turn implies perfect computational indistinguishability against an arbitrary distinguisher. The rules for the exact equality of protocols are in Figures~\ref{fig:protocols_equality_strict_1}, \ref{fig:protocols_equality_strict_2}; we now describe them informally.

The \textsc{comp-new} rule allows us to permute parallel composition and the creation of a new channel, and the same as \emph{scope
extrusion} in process calculi~\cite{picalc}. The \textsc{absorb-left} 
rule allows us to discard a component in a parallel composition if it has no outputs; this allows us to eliminate internal channels once they are no longer used. The \textsc{diverge} rule allows us to simplify diverging reactions: if a channel reads from itself and continues as an arbitrary reaction $R$, then we can safely discard $R$ as we will never reach it in the first place. The three (un)folding rules \textsc{fold-if-left}, \textsc{fold-if-right}, and \textsc{fold-bind} allow us to simplify composite reactions by bringing their
components into the protocol level as separate internal channels. The rule \textsc{subsume} states that channel dependency is transitive: if we depend on $o_1$, and $o_1$ in turn depends on $o_0$, then we also depend on $o_0$, and this dependency can be made explicit. The \textsc{subst} rule allows us to inline certain reactions into $\mathsf{read}$ commands. Inlining $\assign{o_1}{R_1}$ into $\assign{o_2}{x \leftarrow \read{o_1}; \ R_2}$ is sound provided $R_1$ is \emph{duplicable}: observing two independent results of evaluating $R_1$ is equivalent to observing the same result twice. This side condition is easily discharged whenever $R_1$ does not contain probabilistic
sampling. Finally, the \textsc{drop} rule allows dropping unused reads from channels in certain situations. Due to timing dependencies among channels, we only allow dropping reads from the channel $\assign{o_1}{R_1}$ in the context of $\assign{o_2}{\_ \leftarrow \read{o_1};\ R_2}$ when we have that $(\_ \leftarrow R_1; \ R_2) = R_2$. This side condition is met whenever all reads present in $R_1$ are also present in $R_2$.

\begin{figure*}[ht]
\begin{mathpar}
\fbox{$\Gamma \vdash e_1 = e_2 : \tau$}\\
\inferrule*[right=refl]{\Gamma \vdash e : \tau}{\Gamma \vdash e = e : \tau}\and
\inferrule*[right=sym]{\Gamma \vdash e_1 = e_2 : \tau}{\Gamma \vdash e_2 = e_1 : \tau}\and
\inferrule*[right=trans]{\Gamma \vdash e_1 = e_2 : \tau \\ \Gamma \vdash e_2 = e_3 : \tau}{\Gamma \vdash e_1 = e_3 : \tau}\and
\inferrule*[right=axiom]{\Gamma \vdash e_1 = e_2 : \tau \ \axiom}{\Gamma \vdash e_1 = e_2 : \tau}\and
\inferrule*[right=subst]{\theta : \Gamma_1 \to \Gamma_2 \\ \Gamma_2 \vdash e_1 = e_2 : \tau}{\Gamma_1 \vdash \theta^\star(e_1) = \theta^\star(e_2) : \tau}\and
\inferrule*[right=app-cong]{\func : \sigma \rightarrow \tau \in \Sigma \\ \Gamma \vdash e = e' : \sigma}{\Gamma \vdash \func \ e = \func \ e' : \tau}\and
\inferrule*[right=pair-cong]{\Gamma \vdash e_1 = e_1' : \tau_1 \\ \Gamma \vdash e_2 = e_2' : \tau_2}{\Gamma \vdash (e_1,e_2) = (e_1',e_2') : \tau_1 \times \tau_2}\and
\inferrule*[right=fst-cong]{\Gamma \vdash e = e' : \sigma \times \tau}{\Gamma \vdash \fst_{\sigma \times \tau} \ e = \fst_{\sigma \times \tau} \ e' : \sigma}\and
\inferrule*[right=snd-cong]{\Gamma \vdash e = e' : \sigma \times \tau}{\Gamma \vdash \snd_{\sigma \times \tau} \ e = \snd_{\sigma \times \tau} \ e' : \tau}\and
\inferrule*[right=fst-pair]{\Gamma \vdash e_1 : \tau_1 \\ \Gamma \vdash e_2 : \tau_2}{\Gamma \vdash \fst_{\tau_1 \times \tau_2} \ (e_1, e_2) = e_1 : \tau_1}\and
\inferrule*[right=snd-pair]{\Gamma \vdash e_1 : \tau_1 \\ \Gamma \vdash e_2 : \tau_2}{\Gamma \vdash \snd_{\tau_1 \times \tau_2} \ (e_1, e_2) = e_2 : \tau_2}\and
\inferrule*[right=pair-ext]{\Gamma \vdash e : \sigma \times \tau}{\Gamma \vdash e = \big(\fst_{\sigma \times \tau} \ e, \ \snd_{\sigma \times \tau} \ e\big) : \sigma \times \tau}\and
\inferrule*[right=one-ext]{\Gamma \vdash e : \one}{\Gamma \vdash e = \checkmark : \one}
\end{mathpar}
\caption{Equality for \ipdl expressions.}
\label{fig:expressions_equality}
\end{figure*}

\begin{figure*}[ht]
\begin{mathpar}
\fbox{$\Gamma \vdash d_1 = d_2 : \tau$}\\
\inferrule*[right=app-cong]{\dist : \sigma \twoheadrightarrow \tau \in \Sigma \\ \Gamma \vdash d = d' : \sigma}{\Gamma \vdash \dist \ e = \dist \ e' : \tau}
\end{mathpar}
\caption{Equality for \ipdl distributions.}
\label{fig:distributions_equality}
\end{figure*}

\begin{figure*}[ht]
\begin{mathpar}
\fbox{$\Delta; \ \Gamma \vdash R_1 = R_2 : I \to \tau$}\\
\inferrule*[right=refl]{\Delta; \ \Gamma \vdash R : I \to \tau}{\Delta; \ \Gamma \vdash R = R : I \to \tau}\and
\inferrule*[right=sym]{\Delta; \ \Gamma \vdash R_1 = R_2 : I \to \tau}{\Delta; \ \Gamma \vdash R_2 = R_1 : I \to \tau}\and
\inferrule*[right=trans]{\Delta; \ \Gamma \vdash R_1 = R_2 : I \to \tau \\ \Delta; \ \Gamma \vdash R_2 = R_3 : I \to \tau}{\Delta; \ \Gamma \vdash R_1 = R_3 : I \to \tau}\and
\inferrule*[right=axiom]{\Delta; \ \Gamma \vdash R_1 = R_2 : I \to \tau \ \axiom}{\Delta; \ \Gamma \vdash R_1 = R_2 : I \to \tau}\and
\inferrule*[right=input-unused]{i \notin I \\ \Delta; \ \Gamma \vdash R_1 = R_2 : I \to \tau}{\Delta; \ \Gamma \vdash R_1 = R_2 : I \cup \{i\} \to \tau}\and
\inferrule*[right=subst]{\theta : \Gamma_1 \to \Gamma_2 \\ \Delta; \ \Gamma_2 \vdash R_1 = R_2 : I \to \tau}{\Delta; \ \Gamma_1 \vdash \theta^\star(R_1) = \theta^\star(R_2) : I \to \tau}\and
\inferrule*[right=embed]{\phi : \Delta_1 \to \Delta_2 \\ \Delta_2; \ \Gamma \vdash R_1 = R_2 : I \to \tau}{\Delta_1; \ \Gamma \vdash \phi^\star(R_1) = \phi^\star(R_2) : \phi^\star(I) \to \tau}\and
\inferrule*[right=cong-ret]{\Gamma \vdash e = e' : \tau}{\Delta; \ \Gamma \vdash \ret{e} = \ret{e'} : I \to \tau}\and
\inferrule*[right=cong-samp]{\Gamma \vdash d = d' : \sigma}{\Delta; \ \Gamma \vdash \samp{d} = \samp{d'} : I \to \tau}\and
\inferrule*[right=cong-if]{\Gamma \vdash e = e' : \Bool \\ \Delta; \ \Gamma \vdash R_1 = R_1' : I \to \tau \\ \Delta; \ \Gamma \vdash R_2 = R_2' : I \to \tau}{\Delta; \ \Gamma \vdash \big(\ifte{e}{R_1}{R_2}\big) = \big(\ifte{e'}{R_1'}{R_2'}\big) : I \to \tau}\and
\inferrule*[right=cong-bind]{\Delta; \ \Gamma \vdash R = R' : I \to \sigma \\ \Delta; \ \Gamma, x : \sigma \vdash S = S' : I \to \tau}{\Delta; \ \Gamma \vdash (x : \sigma \leftarrow R; \ S) = (x : \sigma \leftarrow R'; \ S') : I \to \tau}
\end{mathpar}
\caption{Equality for \ipdl reactions. Additional rules are given in
Figure~\ref{fig:reactions_equality_2}.}
\label{fig:reactions_equality_1}
\end{figure*}

\begin{figure*}[ht]
\begin{mathpar}
\fbox{$\Delta; \ \Gamma \vdash R_1 = R_2 : I \to \tau$}\\
\inferrule*[right=ret-bind]{\Gamma \vdash e : \sigma \\ \Delta; \ \Gamma, x : \sigma \vdash R : I \to \tau}{\Delta; \ \Gamma \vdash (x : \sigma \leftarrow \ret{e}; \ R) = R[\assign{x}{e}] : I \to \tau}\and
\inferrule*[right=bind-ret]{\Delta; \ \Gamma \vdash R : I \to \tau}{\Delta; \ \Gamma \vdash (x : \tau \leftarrow R; \ \ret{x}) = R : I \to \tau}\and
\inferrule*[right=bind-bind]{\Delta; \ \Gamma \vdash R_1 : I \to \sigma_1 \\ \Delta; \ \Gamma, x_1 : \sigma_1 \vdash R_2 : I \to \sigma_2 \\ \Delta; \ \Gamma, x_2 : \sigma_2 \vdash S : I \to \tau}{\Delta; \ \Gamma \vdash \big(x_2 : \sigma_2 \leftarrow (x_1 : \sigma_1 \leftarrow R_1; \ R_2); \ S\big) = \big(x_1 : \sigma_1 \leftarrow R_1; \ x_2 : \sigma_2 \leftarrow  R_2; \ S\big) : I \to \tau}\and
\inferrule*[right=exch]{\Delta; \ \Gamma \vdash R_1 : I \to \sigma_1 \\ \Delta; \ \Gamma \vdash R_2 : I \to \sigma_2 \\ \Delta; \ \Gamma, x_1 : \sigma_1, x_2 : \sigma_2 \vdash S : I \to \tau}{\Delta; \ \Gamma \vdash \big(x_1 : \sigma_1 \leftarrow R_1; \ x_2 : \sigma_2 \leftarrow R_2; \ S\big) = \big(x_2 : \sigma_2 \leftarrow R_2; \ x_1 : \sigma_1 \leftarrow R_1; \ S\big) : I \to \tau}\and
\inferrule*[right=samp-pure]{\Gamma \vdash d : \sigma \\ \Delta; \ \Gamma \vdash R : I \to \tau}{\Delta; \ \Gamma \vdash (x : \sigma \leftarrow \samp{d}; \ R) = R : I \to \tau}\and
\inferrule*[right=read-det]{i : \sigma \in \Delta \\ i \in I \\ \Delta; \ \Gamma, x : \sigma, y : \sigma \vdash R : I \to \tau}{\Delta; \ \Gamma \vdash \big(x : \sigma \leftarrow \read{i}; \ y : \sigma \leftarrow \read{i}; \ R\big) = \big(x : \sigma \leftarrow \read{i}; \ R[\assign{y}{x}]\big) : I \to \tau}\and
\inferrule*[right=if-left]{\Delta; \ \Gamma \vdash R_1 : I \to \tau \\ \Delta; \ \Gamma \vdash R_2 : I \to \tau}{\Delta; \ \Gamma \vdash \big(\ifte{\true}{R_1}{R_2}\big) = R_1 : I \to \tau}\and
\inferrule*[right=if-right]{\Delta; \ \Gamma \vdash R_1 : I \to \tau \\ \Delta; \ \Gamma \vdash R_2 : I \to \tau}{\Delta; \ \Gamma \vdash \big(\ifte{\false}{R_1}{R_2}\big) = R_2 : I \to \tau}\and
\inferrule*[right=if-ext]{\Delta; \ \Gamma, x : \Bool \vdash R : I \to \tau \\ \Gamma \vdash e : \Bool}{\Delta; \ \Gamma \vdash \big(\ifte{e}{R[\assign{x}{\true}]}{R[\assign{x}{\false}]}\big) = R[\assign{x}{e}] : I \to \tau}
\end{mathpar}
\caption{Equality for \ipdl reactions.}
\label{fig:reactions_equality_2}
\end{figure*}

\begin{figure*}[ht]
\begin{mathpar}
\fbox{$\Delta \vdash P = Q : I \to O$}\\
\inferrule*[right=refl]{\Delta \vdash P : I \to O}{\Delta \vdash P = P : I \to O}\and
\inferrule*[right=sym]{\Delta \vdash P_1 = P_2 : I \to O}{\Delta \vdash P_2 = P_1 : I \to O}\and
\inferrule*[right=trans]{\Delta \vdash P_1 = P_2 : I \to O \\ \Delta \vdash P_2 = P_3 : I \to O}{\Delta \vdash P_1 = P_3 : I \to O}\and
\inferrule*[right=axiom]{\Delta \vdash P = Q : I \to O \ \axiom}{\Delta \vdash P = Q : I \to O}\and
\inferrule*[right=input-unused]{i \notin I \cup O \\ \Delta \vdash P_1 = P_2 : I \to O}{\Delta \vdash P_1 = P_2 : I \cup \{i\} \to O}\and
\inferrule*[right=embed]{\phi : \Delta_1 \to \Delta_2 \\ \Delta_2 \vdash P = Q : I \to O}{\Delta_1 \vdash \phi^\star(P) = \phi^\star(Q) : \phi^\star(I) \to \phi^\star(O)}\and
\inferrule*[right=cong-react]{o : \tau \in \Delta \\ o \notin I \\ \Delta; \ \cdot \vdash R = R' : I \cup \{o\} \to \tau}{\Delta \vdash (\assign{o}{R}) = (\assign{o}{R'}) : I \to \{o\}}\and
\inferrule*[right=cong-comp-left]{\Delta \vdash P = P' : I \cup O_2 \to O_1 \\ \Delta \vdash Q : I \cup O_1 \to O_2}{\Delta \vdash \Par{P}{Q} = \Par{P'}{Q} : I \to O_1 \cup O_2}\and
\inferrule*[right=cong-new]{\Delta, o : \tau \vdash P = P' : I \to O \cup \{o\}}{\Delta \vdash (\new{o}{\tau}{P}) = (\new{o}{\tau}{P'}) : I \to O}\and
\inferrule*[right=comp-comm]{\Delta \vdash P_1 : I \cup O_2 \to O_1 \\ \Delta \vdash P_2 : I \cup O_1 \to O_2}{\Delta \vdash \Par{P_1}{P_2} = \Par{P_2}{P_1} : I \to O_1 \cup O_2}\and
\inferrule*[right=comp-assoc]{\Delta \vdash P_1 : I \cup O_2 \cup O_3 \to O_1 \\ \Delta \vdash P_2 : I \cup O_1 \cup O_3 \to O_2 \\ \Delta \vdash P_3 : I \cup O_1 \cup O_2 \to O_3}{\Delta \vdash \Par{(\Par{P_1}{P_2})}{P_3} = \Par{P_1}{(\Par{P_2}{P_3})} : I \to O_1 \cup O_2 \cup O_3}\and
\inferrule*[right=new-exch]{\Delta, o_1 : \tau_1, o_2 : \tau_2 \vdash P : I \to O \cup \{o_1,o_2\}}{\Delta \vdash \big(\new{o_1}{\tau_1}{\new{o_2}{\tau_2}{P}}\big) = \big(\new{o_2}{\tau_2}{\new{o_1}{\tau_1}{P}}\big) : I \to O}\and
\inferrule*[right=comp-new]{\Delta \vdash P : I \cup O_2 \to O_1 \\ \Delta, o : \tau \vdash Q : I \cup O_1 \to O_2 \cup \{o\}}{\Delta \vdash \Par{P}{(\new{o}{\tau}{Q})} = \new{o}{\tau}{(\Par{P}{Q})} : I \to O_1 \cup O_2}\and
\inferrule*[right=absorb-left]{\Delta \vdash P : I \to O \\ \Delta \vdash Q : I \cup O \to \emptyset}{\Delta \vdash \Par{P}{Q} = P : I \to O}
\end{mathpar}
\caption{Exact equality for \ipdl protocols. Additional rules are given in Figure~\ref{fig:protocols_equality_strict_2}.}
\label{fig:protocols_equality_strict_1}
\end{figure*}

\begin{figure*}[ht]
\begin{mathpar}
\fbox{$\Delta \vdash P = Q : I \to O$}\\
\inferrule*[right=diverge]{o : \tau \in \Delta \\ o \notin I \\ \Delta; \ \cdot \vdash R : I \cup \{o\} \to \tau}{\Delta \vdash (\assign{o}{x : \tau \leftarrow \read{o}; \ R}) = (\assign{o}{\read{o}}) : I \to \{o\}}\and
\inferrule*[right=fold-if-left]{o \notin I \\ b \in I \\ b : \Bool, o : \tau \in \Delta \\ \Delta; \ \cdot \vdash S_1 : I \cup \{o\} \to \tau \\ \Delta; \ \cdot \vdash S_2 : I \cup \{o\} \to \tau}{\Delta \vdash \big(\new{l}{\tau}{\Par{\assign{o}{x : \Bool \leftarrow \read{b}; \ \ifte{x}{{\color{red} \read{l}}}{S_2}}}{{\color{red} \assign{l}{S_1}}}}\big) = \\ \big(\assign{o}{x : \Bool \leftarrow \read{b}; \ \ifte{x}{{\color{red} S_1}}{S_2}}\big) : I \to \{o\}\hspace{-55pt}}\and
\inferrule*[right=fold-if-right]{o \notin I \\ b \in I \\ b : \Bool, o : \tau \in \Delta \\ \Delta; \ \cdot \vdash S_1 : I \cup \{o\} \to \tau \\ \Delta; \ \cdot \vdash S_2 : I \cup \{o\} \to \tau}{\Delta \vdash \big(\new{r}{\tau}{\Par{\assign{o}{x : \Bool \leftarrow \read{b}; \ \ifte{x}{S_1}{{\color{red} \read{r}}}}}{{\color{red} \assign{r}{S_2}}}}\big) = \\ \big(\assign{o}{x : \Bool \leftarrow \read{b}; \ \ifte{x}{S_1}{{\color{red} S_2}}}\big) : I \to \{o\}\hspace{-53pt}}\and
\inferrule*[right=fold-bind]{o \notin I \\ o : \tau_2 \in \Delta \\ \Delta; \ \cdot \vdash R_1 : I \cup \{o\} \to \tau_1 \\ \Delta; \ x : \tau_1 \vdash R_2 : I \cup \{o\} \to \tau_2}{\Delta \vdash \big(\new{c}{\sigma}{\Par{\assign{o}{{\color{red} x : \tau_1 \leftarrow \read{c};} \ R_2}}{{\color{red} \assign{c}{R_1}}}}\big) = (\assign{o}{{\color{red} x : \tau_1 \leftarrow R_1;} \ R_2}) : I \to \{o\}}\and
\inferrule*[right=subst]{o_1 \neq o_2 \\ o_1, o_2 \notin I \\ o_1 : \tau_1, o_2 : \tau_2 \in \Delta \\ \Delta; \ \cdot \vdash R_1 : I \cup \{o_1,o_2\} \to \tau_1 \\ \Delta; \ x_1 : \tau_1 \vdash R_2 : I \cup \{o_1,o_2\} \to \tau_2 \\ \Delta; \ \cdot \vdash \big(x_1 : \tau_1 \leftarrow R_1; \ {\color{red} x'_1 : \tau_1 \leftarrow R_1; \ } \ret{(x_1,{\color{red} x'_1})}\big) = \big(x_1 : \tau_1 \leftarrow R_1; \ \ret{(x_1,{\color{red} x_1})}\big) : I \cup \{o_1,o_2\} \to \tau_1 \times \tau_1}{\Delta \vdash \big(\Par{\assign{o_1}{R_1}}{\assign{o_2}{{\color{red} x_1 : \tau_1 \leftarrow \read{o_1}; \ } R_2}}\big) = \big(\Par{\assign{o_1}{R_1}}{\assign{o_2}{{\color{red} x_1 : \tau_1 \leftarrow R_1; \ } R_2}}\big) : I \to \{o_1, o_2\}}\and
\inferrule*[right=drop]{o_1 \neq o_2 \\ o_1, o_2 \notin I \\ o_1 : \tau_1, o_2 : \tau_2 \in \Delta \\ \Delta; \ \cdot \vdash R_1 : I \cup \{o_1,o_2\} \to \tau_1 \\ \Delta; \ \cdot \vdash R_2 : I \cup \{o_1,o_2\} \to \tau_2 \\ \Delta; \ \cdot \vdash ({\color{red} x_1 : \tau_1 \leftarrow R_1; \ } R_2) = R_2 : I \cup \{o_1,o_2\} \to \tau_2}{\Delta \vdash \big(\Par{\assign{o_1}{R_1}}{\assign{o_2}{{\color{red} x_1 \leftarrow \read{o_1}; \ } R_2}}\big) = (\Par{\assign{o_1}{R_1}}{\assign{o_2}{R_2}}) : I \to \{o_1, o_2\}}
\end{mathpar}
\caption{Additional rules for exact equality of \ipdl protocols. Distinguishing changes of equalities are highlighed in {\color{red} red}.}
\label{fig:protocols_equality_strict_2}
\end{figure*}

\subsubsection{Approximate Equality}
The equational theory for the approximate fragment of \ipdl consists of two layers: one for the \emph{approximate equality} of protocols, and one for the \emph{asymptotic equality} of protocol families as functions of the security parameter $\lambda \in \nat$. The approximate equality judgement $\Delta \vdash \approxeq{P}{Q}{I}{O}{k}{l}$ equates two protocols $\Delta \vdash P : I \to O$ and $\Delta \vdash Q : I \to O$ with identical typing judgements. We think of these as corresponding to a specific security parameter $\lambda$. Analogously to exact protocol equality, we assume an ambient finite set of \emph{approximate axioms} of the form $\Delta \vdash P \approx Q : I \to O$, where $\Delta \vdash P : I \to O$ and $\Delta \vdash Q : I \to O$. These axioms capture cryptographic assumptions on computational indistinguishability.

The parameters $k, l \in \nat$ track the size of the derivation. The \emph{width} parameter $k$ simply counts the number of invocations of axioms applied during the proof: applying a single approximate axiom incurs $k = 1$, and we sum up the two values of $k$ whenever we use transitivity. In the asymptotic equality judgement, $k$ becomes a function of the security parameter $\lambda$ and we require that it be bounded by a polynomial in $\lambda$: even though each individual axiom invocation introduces a negligible error, the sum of exponentially many negligible errors may not be negligible anymore.

Since most nontrivial reasoning in \ipdl is done in the exact half, the approximate equality rules are used mostly to apply indistinguishability assumptions nested deeply inside protocols. The \emph{length} parameter $l$ tracks the largest size of such a nesting -- also known as a \emph{program context}. In the asymptotic equality judgement, we again require that $l$ be bounded as a function of $\lambda$ by a polynomial: exponentially large \ipdl contexts could in principle be used to encode exponential-time probabilistic computations. An \ipdl program context surrounding an indistinguishability assumption is formally a part of the adversary, and as such it must be resource-bounded for the indistinguishability assumption to apply.

In \textsf{IPDL}, the bound on resources is given by a \emph{symbolic size} function $|\cdot|$ defined for expressions, reactions, and protocols. Since we assume that all function symbols will be interpreted by  functions computable in poly-time, our symbolic size for expressions simply counts the number of variables and function applications present.
\begin{align*}
|x| & \coloneqq 1 \\
|\checkmark| & \coloneqq 0 \\
|\true| & \coloneqq 1 \\
|\false| & \coloneqq 1 \\
|\func \ e| & \coloneqq |e| + 1 \\
|(e_1, e_2)| & \coloneqq |e_1| + |e_2| \\
|\fst_{\sigma \times \tau} \ e| & \coloneqq |e| \\
|\snd_{\sigma \times \tau} \ e| & \coloneqq |e|
\end{align*}
For distributions, we follow a similar principle: we assume that all distribution symbols will be interpreted by probabilistic functions approximately computable in poly-time.
\begin{align*}
|\dist \ e| & \coloneqq |e| + 1
\end{align*}
For reactions, we sum up the symbolic sizes of all expressions and distributions occurring inside the reaction, with the exception of the conditional: here we pick the size of the larger branch and add it to the size of the condition.
\begin{align*}
|\ret{e}| & \coloneqq |e| \\
|\samp{d}| & \coloneqq |d| \\
|\read{c}| & \coloneqq 1 \\
|\ifte{e}{R_1}{R_2}| & \coloneqq |e| + \mathsf{max} \, (|R_1|, |R_2|) \\
|x : \sigma \leftarrow R; \ S| & \coloneqq |R| + |S|
\end{align*}
Since protocols in \ipdl are finite networks of channels that do not contain recursion, the size of a protocol is simply the sum of the symbolic sizes of all reactions occurring in the protocol.
\begin{align*}
|\zero| & \coloneqq 0 \\
|\assign{o}{R}| & \coloneqq |R| \\
|\Par{P}{Q}| & \coloneqq |P| + |Q| \\
|\new{o}{\tau}{P}| & \coloneqq |P|
\end{align*}

Figure \ref{fig:protocols_equality_approx} shows the rules for the approximate equality of \ipdl protocols; crucially, rule \textsc{strict} allows us to descend to the exact half of the proof system. Whenever we need to make the ambient theory with approximate axioms $\Delta^1 \vdash P^1 \approx Q^1 : I^1 \to O^1, \ldots, \Delta^n \vdash P^n \approx Q^n : I^n \to O^n$ explicit, we write the approximate equality judgement as $\Delta^1 \vdash P^1 \approx Q^1 : I^1 \to O^1, \ldots, \Delta^n \vdash P^n \approx Q^n : I^n \to O^n \Rightarrow \Delta \vdash \approxeq{P}{Q}{I}{O}{k}{l}$.

\begin{figure*}
\begin{mathpar}
\fbox{$\Delta \vdash \approxeq{P}{Q}{I}{O}{k}{l}$}\\
\inferrule*[right=strict]{\Delta \vdash P = Q : I \to O}{\Delta \vdash \approxeq{P}{Q}{I}{O}{0}{0}} \and
\inferrule*[right=sym]{\Delta \vdash \approxeq{P_1}{P_2}{I}{O}{k}{l}}{\Delta \vdash \approxeq{P_2}{P_1}{I}{O}{k}{l}}\and
\inferrule*[right=trans]{\Delta \vdash \approxeq{P_1}{P_2}{I}{O}{k_1}{l_1} \\ \Delta \vdash \approxeq{P_2}{P_3}{I}{O}{k_2}{l_2}}{\Delta \vdash \approxeq{P_1}{P_3}{I}{O}{k_1 + k_2}{\mathsf{max} \, (l_1,l_2)}}\and
\inferrule*[right=axiom]{\Delta \vdash P \approx Q : I \to O \ \axiom}{\Delta \vdash \approxeq{P}{Q}{I}{O}{1}{0}}\and
\inferrule*[right=input-unused]{i \notin I \cup O \\ \Delta \vdash \approxeq{P}{Q}{I}{O}{k}{l}}{\Delta \vdash \approxeq{P}{Q}{I \cup \{i\}}{O}{k}{l+1}}\and
\inferrule*[right=embed]{\theta : \Delta_1 \to \Delta_2 \\ \Delta_1 \vdash \approxeq{P}{Q}{I}{O}{k}{l}}{\Delta_2 \vdash \approxeq{\theta^\star(P)}{\theta^\star(Q)}{\theta^\star(I)}{\theta^\star(O)}{k}{l}}\and
\inferrule*[right=cong-comp-left]{\Delta \vdash \approxeq{P}{P'}{I \cup O_2}{O_1}{k}{l} \\ \Delta \vdash_\Sigma Q : I \cup O_1 \to O_2}{\Delta \vdash \approxeq{\Par{P}{Q}}{\Par{P'}{Q}}{I}{O_1 \cup O_2}{k}{l + |Q|}} \and
\inferrule*[right=cong-new]{\Delta, o : A \vdash \approxeq{P}{P'}{I}{O \cup \{o\}}{k}{l}}{\Delta \vdash \approxeq{(\new{o}{A}{P})}{(\new{o}{A}{P'})}{I}{O}{k}{l}}
\end{mathpar}
\caption{Approximate equality for \ipdl protocols.}
\label{fig:protocols_equality_approx}
\end{figure*}

For the asymptotic equality of \ipdl protocols, we assume a finite set $\mathbb{T}_\approx$ of \emph{axiom families} of the form $\{\Delta_\lambda \vdash P_\lambda \approx Q_\lambda : I_\lambda \to O_\lambda\}_{\lambda \in \nat}$. In this setting, the asymptotic equivalence of two protocol families $\{\Delta_\lambda \vdash P_\lambda : I_\lambda \to O_\lambda\}_{\lambda \in \nat}$ and $\{\Delta_\lambda \vdash Q_\lambda : I_\lambda \to O_\lambda\}_{\lambda \in \nat}$ with pointwise-identical typing judgements takes the form of the judgement $\mathbb{T}_\approx \Rightarrow \{\Delta_\lambda \vdash P_\lambda \approx Q_\lambda : I_\lambda \to O_\lambda\}_{\lambda \in \nat}$, see Figure \ref{fig:protocols_equivalence_asympto}.
 
Specifically, for any fixed $\lambda$ we obtain an approximate theory by selecting from each axiom family in $\mathbb{T}_\approx$ the axiom corresponding to $\lambda$. Similarly, from each of the two protocol families we select the protocol corresponding to $\lambda$, which gives us two concrete protocols to equate approximately. We recall that an approximate equality judgement is tagged by a pair of parameters $k$ and $l$. Letting $\lambda \in \nat$ vary thus gives us two functions $k_\lambda$ and $l_\lambda$, and we require that these be bounded by a polynomial. We can summarize the asymptotic judgement as saying that the protocol families are pointwise approximately equal, and both the width and length of the derivation, as well as the number of input and output channels are bounded by a polynomial in $\lambda$.

\begin{figure*}
\begin{mathpar}
\fbox{$\{\Delta^1_\lambda \vdash P^1_\lambda \approx Q^1_\lambda : I^1_\lambda \to O^1_\lambda\}_{\lambda \in \nat}, \ldots, \{\Delta^n_\lambda \vdash P^n_\lambda \approx Q^n_\lambda : I^n_\lambda \to O^n_\lambda\}_{\lambda \in \nat} \Rightarrow \{\Delta_\lambda \vdash P_\lambda \approx Q_\lambda : I_\lambda \to O_\lambda\}_{\lambda \in \nat}$}
\\
\inferrule{\forall \lambda, \Delta^1_\lambda \vdash P^1_\lambda \approx
    Q^1_\lambda : I^1_\lambda \to O^1_\lambda, \ldots, \Delta^n_\lambda \vdash
    P^n_\lambda \approx Q^n_\lambda : I^n_\lambda \to O^n_\lambda \Rightarrow
    \Delta_\lambda \vdash
    \approxeq{P_\lambda}{Q_\lambda}{I_\lambda}{O_\lambda}{k_\lambda}{l_\lambda}
    \\ k_\lambda = \mathsf{O}(\mathsf{poly}(\lambda)) \\ l_\lambda = \mathsf{O}(\mathsf{poly}(\lambda)) \\ |I_\lambda| = \mathsf{O}(\mathsf{poly}(\lambda)) \\ |O_\lambda| = \mathsf{O}(\mathsf{poly}(\lambda))}{\{\Delta^1_\lambda \vdash P^1_\lambda \approx Q^1_\lambda : I^1_\lambda \to O^1_\lambda\}_{\lambda \in \nat}, \ldots, \{\Delta^n_\lambda \vdash P^n_\lambda \approx Q^n_\lambda : I^n_\lambda \to O^n_\lambda\}_{\lambda \in \nat} \Rightarrow \{\Delta_\lambda \vdash P_\lambda \approx Q_\lambda : I_\lambda \to O_\lambda\}_{\lambda \in \nat}}
\end{mathpar}
\caption{Asymptotic equivalence for \ipdl protocol families.}
\label{fig:protocols_equivalence_asympto}
\end{figure*}

Whenever we need to make the underlying exact theory $\mathbb{T}$ explicit, we write the asymptotic equality judgement as $\mathbb{T}; \mathbb{T}_\approx \Rightarrow \{\Delta_\lambda \vdash P_\lambda \approx Q_\lambda : I_\lambda \to O_\lambda\}_{\lambda \in \nat}$.

\section{Operational Semantics of \ipdl}
In this section we define an operational semantics for expressions, reactions, and protocols. This semantics will validate the \emph{exact} fragment of our equational logic and prove perfect observational equivalence. To give semantics to user-defined symbols, we define interpretations:

\begin{definition}[Interpretation]
An interpretation $\sem{-}$ for a signature $\Sigma$ associates:
\begin{itemize}
\item to each type symbol $\type$ a bitstring length $\sem{\type} \in \nat$;
\item to each function symbol $\func : \sigma \to \tau$ a function $\sem{\func}$ from bitstrings $\{0,1\}^{\sem{\sigma}}$ to bitstrings $\{0,1\}^{\sem{\tau}}$;
\item to each distribution symbol $\dist : \sigma \to \tau$ a function $\sem{\dist}$ from bitstrings $\{0,1\}^{\sem{\sigma}}$ to \emph{distributions} on bitstrings $\{0,1\}^{\sem{\tau}}$. 
\end{itemize}
\end{definition}

\noindent In the above, we naturally lift the interpretation $\sem{-}$ to data types by setting
\begin{align*}
\sem{\one} & \coloneqq 0 \\
\sem{\Bool} & \coloneqq 1 \\
\sem{\tau \times \sigma} & \coloneqq \sem{\tau} + \sem{\sigma}
\end{align*}

\noindent To handle partial computations, we augment the syntax of expressions, reactions, and protocols to contain intermediate bitstring values $v \in \{0,1\}^\star$:\smallskip

\begin{syntax}
\category[Valued Expressions]{e}
\alternative{{\color{red} v}} \alternative{\dots}
\category[Valued Reactions]{R, S}
\alternative{{\color{red} \val{v}}} \alternative{\dots}
\category[Valued Protocols]{P}
\alternative{{\color{red} \assign{o}{v}}} \alternative{\dots}
\end{syntax}\smallskip

\noindent Given an ambient interpretation $\sem{-}$ for the signature $\Sigma$, we can type the valued counterpart of \ipdl constructs as expected: in addition to the regular typing rules, we have

\begin{mathpar}
\inferrule*{v \in \{0,1\}^{\sem{\tau}}}{\Gamma \vdash v : \tau}\and
\inferrule*{v \in \{0,1\}^{\sem{\tau}}}{\Delta; \ \Gamma \vdash \val{v} : I \to \tau}\and
\inferrule*{o : \tau \in \Delta \\ o \notin I \\ v \in \{0,1\}^{\sem{\tau}}}{\Delta \vdash (\assign{o}{v}) : I \to \{o\}}
\end{mathpar}

\noindent The big-step semantics $e \Downarrow v$ for expressions is straightforward -- see Figure \ref{fig:expressions_semantics}, where we denote the empty bitstring by $()$ and use $v_1 v_2$ for bitstring concatenation. Pairing is given by the aforementioned bitstring concatenation (rule \textsc{pair}), and the projections $\fst_{\sigma \times \tau}$ and $\snd_{\sigma \times \tau}$ unambiguously split the pair according to $\sem{\sigma}$ and $\sem{\tau}$, respectively (rules \textsc{fst} and \textsc{snd}).

\begin{lemma}[Determinism of $\Downarrow$ for expressions]
For any well-typed expression $\Gamma \vdash e : \tau$ there exists a unique value $v$ such that $e \Downarrow v$, and $v \in \{0,1\}^{\sem{\tau}}$.
\end{lemma}

Reactions have a straightforward small-step semantics of the form $R \to \eta$, where $\eta$ is a probability distribution over reactions. Figure \ref{fig:reactions_semantics} shows the rules, where we write $1[R]$ for the distribution with unit mass at the reaction $R$, and freely use a distribution in place of a value (rule \textsc{samp}) or a reaction (rule \textsc{bind-react}) to indicate the obvious lifting of the corresponding construct to distributions on reactions. All distributions are implicitly finitely supported. Crucially, there is no semantic rule for stepping the reaction $\read{c}$ -- we model communication via semantics for protocols, which substitute all instances of $\mathsf{read}$ for values. 

We give semantics to protocols via two main small-step rules, see Figure \ref{fig:protocols_semantics}, where we analogously write $1[P]$ for the distribution with unit mass at the protocol $P$, and freely use a distribution in place of a reaction (in rule \textsc{step-react}) or a protocol (rules \textsc{step-comp-left}, \textsc{step-comp-right}, and \textsc{step-new}) to indicate the obvious lifting of the corresponding construct to distributions on protocols.

First we have the \emph{output} relation $P \outstep{o}{v} Q$, which is enabled when the reaction for channel $o$ in $P$ terminates, resulting in value $v$ (rule \textsc{out-val}). When this happens, the value of $o$ is broadcast through the protocol context enveloping $P$ (rules \textsc{out-comp-left}, \textsc{out-comp-left}, and \textsc{out-new}), resulting in each $\read{o}$ command in other reactions to be substituted with $\val{v}$. Note that the value of $o$ is not broadcast above the $\mathsf{new}$ quantifier when the local channel introduced is equal to $o$.

Next we have the \emph{internal stepping} relation $P \to \eta$, specified similarly to the small-step relation for reactions. The rule \textsc{step-react} lifts the stepping relation for $R$ to the stepping relation for $\assign{o}{R}$, while the three rules \textsc{step-comp-left}, \textsc{step-comp-right}, \textsc{step-new} simply propagate the stepping relation through parallel composition and the $\mathsf{new}$ quantifier. The last rule \textsc{out-new-hide} links the output relation with the stepping relation: whenever $P$ steps to $P'$, resulting in the output $\assign{o}{v}$, we have that $\new{o}{\tau}{P}$ steps with unit mass to $\new{o}{\tau}{P'}$.

\begin{figure}
\begin{mathpar}
\fbox{$e \Downarrow v$}\\
\inferrule*[right=val]{ }{v \Downarrow v}\and
\inferrule*[right=check]{ }{\checkmark \Downarrow ()}\and
\inferrule*[right=true]{ }{\true \Downarrow 1}\and
\inferrule*[right=false]{ }{\false \Downarrow 0}\and
\inferrule*[right=app]{e \Downarrow v}{(\func \ e) \Downarrow \sem{\func}(v)}\and
\inferrule*[right=pair]{e_1 \Downarrow v_1 \\ e_2 \Downarrow v_2}{(e_1,e_2) \Downarrow v_1 v_2}\and
\inferrule*[right=fst]{e \Downarrow v_1 v_2 \\ v_1 \in \{0,1\}^{\sem{\sigma}}}{(\fst_{\sigma \times \tau} \ e) \Downarrow v_1}\and
\inferrule*[right=snd]{e \Downarrow v_1 v_2 \\ v_2 \in \{0,1\}^{\sem{\tau}}}{(\snd_{\sigma \times \tau}) \ e \Downarrow v_2}
\end{mathpar}
\caption{Big-step operational semantics for $\ipdl$ expressions.}
\label{fig:expressions_semantics}
\end{figure}

\begin{figure}
\begin{mathpar}
\fbox{$R \to \eta$}\\
\inferrule*[right=ret]{e \Downarrow v}{\ret{e} \to 1[\val{v}]}\and
\inferrule*[right=samp]{e \Downarrow v}{\samp{(\dist \ e)} \to \val{\sem{\dist}(v)}}\and
\inferrule*[right=if-true]{e \Downarrow 1}{(\ifte{e}{R_1}{R_2}) \to 1[R_1]}\and
\inferrule*[right=if-false]{e \Downarrow 0}{(\ifte{e}{R_1}{R_2}) \to 1[R_2]}\and
\inferrule*[right=bind-val]{ }{(x : \sigma \leftarrow \val{v}; \ S) \to 1\big[S[\assign{x}{v}]\big]}\and
\inferrule*[right=bind-react]{R \to \eta}{(x : \sigma \leftarrow R; \ S) \to {(x : \sigma \leftarrow \eta; \ S)}}
\end{mathpar}
\caption{Small-step operational semantics for $\ipdl$ reactions.} 
\label{fig:reactions_semantics}
\end{figure}

\begin{figure}
\begin{mathpar}
\fbox{$P \outstep{o}{v} Q$}\\
\inferrule*[right=out-react]{ }{(\assign{o}{\val{v}}) \outstep{o}{v} (\assign{o}{v})}\and
\inferrule*[right=out-comp-left]{P \outstep{o}{v} P'}{(\Par{P}{Q}) \outstep{o}{v} \big(\Par{P'}{Q[\assign{\read{o}}{\val{v}}]}\big)}\and
\inferrule*[right=out-comp-right]{Q \outstep{o}{v} Q'}{(\Par{P}{Q}) \outstep{o}{v} \big(\Par{P[\assign{\read{o}}{\val{v}}]}{Q'}\big)}\and
\inferrule*[right=out-new]{P \outstep{o}{v} P' \\ o \neq c}{(\new{c}{\tau}{P}) \outstep{o}{v} (\new{c}{\tau}{P'})}\\\\
\fbox{$P \to \eta$}\\
\inferrule*[right=step-react]{R \to \eta}{(\assign{o}{R}) \to (\assign{o}{\eta})}\and
\inferrule*[right=step-comp-left]{P \to \eta}{(\Par{P}{Q}) \to (\Par{\eta}{Q})}\and
\inferrule*[right=step-comp-right]{Q \to \eta}{(\Par{P}{Q}) \to (\Par{P}{\eta})}\and
\inferrule*[right=step-new]{P \to \eta}{(\new{c}{\tau}{P}) \to (\new{c}{\tau}{\eta})}\and
\inferrule*[right=out-new-hide]{P \outstep{c}{v} P'}{(\new{c}{\tau}{P}) \to 1[\new{c}{\tau}{P'}]}
\end{mathpar}
\caption{Small-step operational semantics for \ipdl protocols.}
\label{fig:protocols_semantics}
\end{figure}

The big-step operational semantics for reactions $R \Downarrow \eta$, see Figure \ref{fig:reactions_big_step}, performs as many steps as possible in an attempt to compute $R$, resulting in a distribution $\eta$ on reactions. A reaction that cannot step any further is \emph{final}. We can syntactically describe final reactions as those that have either yielded a final value or have an unresolved read in the leading position (\emph{i.e.}, are \emph{stuck}).

Similarly, the big-step operational semantics for protocols $P \Downarrow \eta$, see Figure \ref{fig:protocols_big_step}, performs as many output and internal steps as possible in an attempt to compute $P$, resulting in a distribution $\eta$ on protocols. Analogously to reactions, a protocol that cannot step any further is \emph{final}. We can syntactically describe final protocols as those where every channel, including the internal ones, carries either a final value or a reaction that is stuck.

\begin{figure}
\begin{mathpar}
\fbox{$R \ \stuck$}\\
\inferrule*{ }{(\read{c}) \ \stuck}\and
\inferrule*{R \ \stuck}{(x : \tau \leftarrow R; \ S) \ \stuck}\\
\fbox{$R \ \final$}\\
\inferrule*{ }{(\val{v}) \ \final}\and
\inferrule*{R \ \stuck}{R \ \final}\\
\fbox{$R \Downarrow \eta$}\\
\inferrule*{}{ }\and
\inferrule*{R \to \sum_i c_i \, 1[R_i] \\ R_i \Downarrow \eta_i}{R \Downarrow \sum_i c_i \, \eta_i}\and \\
\inferrule*{R \ \final}{R \Downarrow 1[R]}
\end{mathpar}
\caption{Big-step operational semantics for \ipdl reactions.}
\label{fig:reactions_big_step}
\end{figure}

\begin{figure}
\begin{mathpar}
\fbox{$P \ \final$}\\
\inferrule*{ }{\zero \ \final}\and
\inferrule*{ }{(\assign{o}{v}) \ \final}\and
\inferrule*{R \ \stuck}{(\assign{o}{R}) \ \final}\and
\inferrule*{P \ \final \\ Q \ \final}{(\Par{P}{Q}) \ \final}\and
\inferrule*{P \ \final}{(\new{o}{\tau}{P}) \ \final}\\\\
\fbox{$P \Downarrow \eta$}\\
\inferrule*{}{ }\and
\inferrule*{P \to \sum_i c_i \ 1[P_i] \\ P_i \Downarrow \eta_i}{P \Downarrow \sum_i c_i \ \eta_i}\and \\
\inferrule*{P \outstep{o}{v} Q \\ Q \Downarrow \eta}{P \Downarrow \eta}\and
\inferrule*{P \ \final}{P \Downarrow 1[P]}
\end{mathpar}
\caption{Big-step operational semantics for \ipdl protocols.}
\label{fig:protocols_big_step}
\end{figure}

Note that while the semantics for reactions is sequential, both output and internal step relations for protocols are non-deterministic. Indeed, any two channels in a protocol may output in any order. Ordinarily, this presents a problem for reasoning about cryptography, since non-deterministic choice may present a security leak. However, our language introduces \emph{no} way to exploit this extra non-determinism, essentially due to the $\mathsf{read}$ commands in reactions being blocking. This is formalized by a \emph{confluence} result for \textsf{IPDL}:

\begin{lemma}[Confluence]
If $\Delta \vdash P : I \to O$, then:
\begin{itemize}
\item If $P \outstep{o}{v_1} Q_1$ and $P \outstep{o}{v_2} Q_2$, then $v_1 = v_2$ and $Q_1 = Q_2$.
\item If $P \outstep{o_1}{v_1} Q_1$ and $P \outstep{o_2}{v_2} Q_2$ with
$o_1 \neq o_2$, then there exists $Q$ such that $Q_1 \outstep{o_2}{v_2} Q$ and $Q_2 \outstep{o_1}{v_1} Q$.
\item If $P \outstep{o}{v} Q$ and $P \to \eta$, then there exists $\eta'$ such that $\eta \outstep{o}{v} \eta'$ and $Q \to \eta'$.
\item If $P \to \eta_1$ and $P \to \eta_2$, then either $\eta_1 = \eta_2$ or there exists $\eta$ such that $\eta_1 \to \eta$ and $\eta_2 \to \eta$.
\end{itemize}
\end{lemma}

\noindent In the above lemma, we lift the two protocol stepping relations $\outstep{o}{v}$ and $\to$ to distributions in the natural way. \smallskip

To guarantee termination of the semantics for reactions, we count the maximum number of steps the reaction would take \emph{provided all reads were resolved}:
\begin{align*}
\norm{\val{v}} & \coloneqq 0 \\
\norm{\ret{e}} & \coloneqq 1 \\
\norm{\samp{(\dist \ e)}} & \coloneqq 1 \\
\norm{\read{c}} & \coloneqq 0 \\
\norm{\ifte{e}{R_1}{R_2}} & \coloneqq \mathsf{max} \, (\norm{R_1}, \norm{R_2}) + 1 \\
\norm{x : \tau \leftarrow R; \ S} & \coloneqq (\norm{R} + \norm{S}) + 1
\end{align*}
We note that $\norm{-}$ for reactions is invariant under substitutions, embeddings, and input assignment. As expected, stepping reduces the number of steps left, guaranteeing termination:

\begin{lemma}
If $R \to \sum_i c_i \ 1[R_i]$, $c_i \neq 0$, then $\norm{R_i} < \norm{R}$.
\end{lemma}

\begin{corollary}[Determinism of $\Downarrow$ for reactions]
For any well-typed reaction $\Delta; \cdot \vdash R : I \to \tau$ there exists a unique distribution $\eta$ such that $R \Downarrow \eta$. We will denote $\eta$  by $\eval{R}$.
\end{corollary}

\noindent To guarantee termination of the semantics for protocols, we analogously count the maximum number of steps the protocol would take \emph{provided all reads in reactions were resolved}:
\begin{align*}
\norm{\zero} & \coloneqq 0 \\
\norm{\assign{o}{v}} & \coloneqq 0 \\
\norm{\assign{o}{R}} & \coloneqq \norm{R} + 1 \\
\norm{\Par{P}{Q}} & \coloneqq \norm{P} + \norm{Q} \\
\norm{\new{c}{\tau}{P}} & \coloneqq \norm{P}
\end{align*}
As for reactions, $\norm{-}$ for protocols is invariant under embeddings and input assignment, and stepping reduces the number of steps left:

\begin{lemma}
If $P \outstep{o}{v} Q$, then $\norm{Q} < \norm{P}$, and if $P \to \sum_i c_i \ 1[P_i]$, $c_i \neq 0$, then $\norm{P_i} < \norm{P}$.
\end{lemma}

\noindent Together with confluence, termination gives us the desired result:

\begin{corollary}[Determinism of $\Downarrow$ for protocols]
For any well-typed protocol $\Delta \vdash P : I \to O$ there exists a unique distribution $\eta$ such that $P \Downarrow \eta$. We will denote $\eta$ by $\eval{P}$.
\end{corollary}

\section{Soundness of Exact Equality in \ipdl}
Soundness of equality at the expression level means that if we substitute the same valued expression for each free variable, the resulting closed expressions will compute to the same value:

\begin{definition}
An axiom $\Gamma \vdash e_1 = e_2 : \tau$ is \emph{sound} if for any valued substitution $\theta : \cdot \to \Gamma$, we have $\eval{\theta^\star(e_1)} = \eval{\theta^\star(e_2)}$.
\end{definition}

\noindent The ambient \ipdl theory for expressions is said to be sound if each of its axioms is sound. It is straightforward to show that this implies overall soundness:
 
\begin{lemma}[Soundness of equality of expressions]
If the ambient \ipdl theory for expressions is sound, then for any equal expressions $\Gamma \vdash e_1 = e_2 : \tau$ and any valued substitution $\theta : \cdot \to \Gamma$, we have that $\eval{\theta^\star(e_1)} = \eval{\theta^\star(e_1)}$.
\end{lemma}

At the reaction level, two equal reactions should behave in a way that is indistinguishable by an external observer. We formally capture this notion of indistinguishability by a logical relation known as a \emph{bisimulation} -- a binary relation on distributions on reactions that satisfies certain closure properties, together with the crucial \emph{valuation property} that allows us to jointly partition two related distributions so that any two corresponding components are again related and have the same \emph{value}: a reaction $R$ is said to have value $v$ if $R$ is of the form $\val{v}$ (otherwise the value is undefined), and we lift this notion to distributions on reactions in the obvious way. At the reaction level, we only require the valuation property for those distributions that are \emph{final}, \emph{i.e.}, no reaction in the support steps.

\begin{definition}[Reaction bisimulation]
A \emph{reaction bisimulation} $\sim$ is a binary relation on distributions on reactions $\Delta; \ \cdot \vdash R : I \to \tau$ satisfying the following conditions:
\begin{itemize}
\item \emph{Closure under convex combinations}: For any distributions $\eta_1 \sim \varepsilon_1$ and $\eta_2 \sim \varepsilon_2$, and any coefficients $c_1, c_2 > 0$ with $c_1 + c_2 = 1$, we have $c_1 \eta_1 + c_2 \eta_2 \sim c_1 \varepsilon_1 + c_2 \varepsilon_2$.

\item \emph{Closure under input assignment}: For any distributions $\eta \sim \varepsilon$, input channel $i \in I$ of type $\tau$, and value $v \in \{0,1\}^{\sem{\tau}}$, we have $\eta[\read{i} \coloneqq \val{v}] \sim \varepsilon[\read{i} \coloneqq \val{v}]$.

\item \emph{Closure under computation}: For any distributions $\eta \sim \varepsilon$, we have $\eval{\eta} \sim \eval{\varepsilon}$.

\item \emph{Valuation property}: For any distributions $\eta \sim \varepsilon$ that are final, there exists a joint convex combination \[\eta = \sum_i c_i \, \eta_i \; \sim \, \sum_i c_i \, \varepsilon_i = \varepsilon\]
with $c_i > 0$ and $\sum_i c_i = 1$, such that
\begin{itemize}
\item the respective components $\eta_i \sim \varepsilon_i$ are again related, and
\item the distributions $\eta_i$ and $\varepsilon_j$ have the same value $v$ or lack thereof if and only if $i = j$.
\end{itemize}
\end{itemize}
\end{definition}

\noindent Crucially, we note that the joint convex combination in the valuation property is unique up to the order of the summands. We now describe one canonical way to construct reaction bisimulations:

\begin{definition}
Let $\sim$ be an arbitrary binary relation on distributions on reactions $\Delta; \ \cdot \vdash R : I \to \tau$. The \emph{lifting} $\lift(\sim)$ is the closure of $\sim$ under joint convex combinations. Explicitly, $\lift(\sim)$ is defined by
\[\sum_i c_i \, \eta_i \; \lift(\sim) \; \sum_i c_i \, \varepsilon_i\]
for coefficients $c_i > 0$ with $\sum_i c_i = 1$ and distributions $\eta_i \sim \varepsilon_i$.
\end{definition}

\begin{lemma}\label{lem:reaction_seed}
Let $\sim$ be a binary relation on distributions on reactions $\Delta; \ \cdot \vdash R : I \to \tau$ with the following properties:
\begin{itemize}
\item \emph{Closure under input assignment}: For any distributions $\eta \sim \varepsilon$, input channel $i \in I$ of type $\tau$, and value $v \in \{0,1\}^{\sem{\tau}}$, we have $\eta[\read{i} \coloneqq \val{v}] \sim \varepsilon[\read{i} \coloneqq \val{v}]$.

\item \emph{Lifting closure under computation}: For any distributions $\eta \sim \varepsilon$, we have $\eval{\eta} \lift(\sim) \, \eval{ \varepsilon}$.

\item \emph{Valuation property}: For any distributions $\eta \sim \varepsilon$ that are final, there exists a joint convex combination \[\eta = \sum_i c_i \, \eta_i \; \sim \, \sum_i c_i \, \varepsilon_i = \varepsilon\]
with $c_i > 0$ and $\sum_i c_i = 1$, such that
\begin{itemize}
\item the respective components $\eta_i \sim \varepsilon_i$ are again related, and
\item the distributions $\eta_i$ and $\varepsilon_j$ have the same value $v$ or lack thereof if and only if $i = j$.
\end{itemize}
\end{itemize}
Then the lifting $\lift(\sim)$ is a reaction bisimulation.
\end{lemma}

\begin{lemma}
We have the following: 
\begin{itemize}
\item The identity relation is a reaction bisimulation.
\item The inverse of a reaction bisimulation is a reaction bisimulation.
\item The composition of two reaction bisimulations is a reaction bisimulation.
\end{itemize}
\end{lemma}

\begin{example}
Fix two expressions $\cdot \vdash e_1 : \sigma$ and $\cdot \vdash e_2 : \sigma$ such that $\eval{e_1} = \eval{e_2}$. Then the relation $\sim$ defined by
\begin{itemize}
\item $1[R(x \coloneqq e_1)] \sim 1[R(x \coloneqq e_2)]$ for reaction $\Delta; \ x : \sigma \vdash R : I \to \tau$
\end{itemize}
is a reaction bisimulation.
\end{example}

\noindent Having defined reaction bisimulations, we can now formally state what it means for reaction equality to be sound:

\begin{definition}
An axiom $\Delta; \ \Gamma \vdash R_1 = R_2 : I \to \tau$ is \emph{sound} if there is a reaction bisimulation $\sim$ such that for any valued substitution $\theta : \cdot \to \Gamma$, we have $1[\theta^\star(R_1)] \sim 1[\theta^\star(R_2)]$.
\end{definition}

\noindent The ambient \ipdl theory for reactions is said to be sound if each of its axioms is sound. We now show that this implies overall soundness:

\begin{lemma}[Soundness of equality of reactions]
If the ambient \ipdl theory for reactions is sound, then for any equal reactions $\Delta; \ \Gamma \vdash R_1 = R_2 : I \to \tau$,     there exists a reaction bisimulation $\sim$ such that for any valued substitution $\theta : \cdot \to \Gamma$, we have $1[\theta^\star(R_1)] \sim 1[\theta^\star(R_2)]$.
\end{lemma}

\begin{proof}
We first replace the exchange rule \textsc{exch} by the three rules \textsc{exch-samp-samp}, \textsc{exch-samp-read}, and \textsc{exch-read-read} in Figure \ref{fig:exch_alt}; it is easy to see that this new set of rules is equivalent to the original one. We now proceed by induction on the alternative set of rules for reaction equality. We will freely use a distribution in place of a value (rule \textsc{exch-samp-read}) or a reaction (rules \textsc{embed}, \textsc{cong-bind}) to indicate the obvious lifting of the corresponding construct to distributions on reactions.

\begin{itemize}
\item \textsc{refl}: Our desired bisimulation is the identity relation.
\item \textsc{sym}: Our desired bisimulation is the inverse of the bisimulation obtained from the premise.
\item \textsc{trans}: Our desired bisimulation is the composition of the two bisimulations obtained from the two premises.
\item \textsc{axiom}: The desired bisimulation exists by assumption.
\item \textsc{input-unused}: Our desired bisimulation is precisely the bisimulation obtained from the premise, seen as a bisimulation on distributions on reactions with the additional input $i$.
\item \textsc{subst}: Our desired bisimulation is precisely the bisimulation obtained from the premise.
\item \textsc{embed}: Let $\sim$ be the bisimulation obtained from the premise. Our desired bisimulation $\sim_\phi$ is defined by
\begin{itemize}
\item $\phi^\star(\eta) \sim_\phi \phi^\star(\varepsilon)$ if $\eta \sim \varepsilon$
\end{itemize}
\item \textsc{cong-ret}: Our desired bisimulation is the lifting of the relation $\sim$ defined by
\begin{itemize}
\item $1[\ret{e}] \sim 1[\ret{e'}]$ for
\begin{itemize}
\item expressions $\cdot \vdash e : \tau$ and $\cdot \vdash e' : \tau$such that $\eval{e} = \eval{e'}$
\end{itemize}
\item $1[\val{v}] \sim 1[\val{v}]$ for value $v \in \{0,1\}^{\sem{\tau}}$
\end{itemize}
\item \textsc{cong-samp}: Our desired bisimulation is the lifting of the relation $\sim$ defined by
\begin{itemize}
\item $1[\samp{(\dist \ e)}] \sim 1[\samp{(\dist \ e')}]$ for
\begin{itemize}
\item expressions $\cdot \vdash e : \tau$ and $\cdot \vdash e' : \tau$such that $\eval{e} = \eval{e'}$
\end{itemize}
\item $1[\val{v}] \sim 1[\val{v}]$ for value $v \in \{0,1\}^{\sem{\tau}}$
\end{itemize}
\item \textsc{cong-if}: Let $\sim_1$ and $\sim_2$ be the two bisimulations obtained from the two premises. Our desired bisimulation is the lifting of the relation $\sim_\mathsf{if}$ defined by
\begin{itemize}
\item $1[\ifte{e}{R_1}{R_2}] \sim_\mathsf{if} \, 1[\ifte{e'}{R'_1}{R'_2}]$ for 
\begin{itemize}
\item expressions $\cdot \vdash e : \Bool$ and $\cdot \vdash e' : \Bool $such that $\eval{e} = \eval{e'}$
\item reactions $\Delta; \ \cdot \vdash R_1 : I \to \tau$ and $\Delta; \ \cdot \vdash R'_1 : I \to \tau$ such that $1[R_1] \sim_1 1[R'_1]$
\item reactions $\Delta; \ \cdot \vdash R_2 : I \to \tau$ and $\Delta; \ \cdot \vdash R'_2 : I \to \tau$ such that $1[R_2] \sim_2 1[R'_2]$
\end{itemize}
\item $\eta_1 \sim_\mathsf{if} \eta'_1$ if $\eta_1 \sim_1 \eta_1'$
\item $\eta_2 \sim_\mathsf{if} \eta'_2$ if $\eta_2 \sim_2 \eta_2'$
\end{itemize}
\item \textsc{cong-bind}: Let $\sim_1$ and $\sim_2$ be the two bisimulations obtained from the two premises. Our desired bisimulation is the lifting of the relation $\sim_\mathsf{bind}$ defined by
\begin{itemize}
\item $(x \leftarrow \eta; \ S) \sim_\mathsf{bind} (x \leftarrow \eta'; \ S')$ for
\begin{itemize}
\item distributions $\eta \sim_1 \eta'$
\item reactions $\Delta; \ x : \sigma \vdash S : I \to \tau$ and $\Delta; \ x : \sigma \vdash S' : I \to \tau$ such that for any value $v \in \{0,1\}^{\sem{\sigma}}$, we have $1[S(x \coloneqq v)] \sim_2 1[S'(x \coloneqq v)]$
\end{itemize}
\item $\varepsilon \sim_\mathsf{bind} \varepsilon'$ if $\varepsilon \sim_2 \varepsilon'$
\end{itemize}
\item \textsc{ret-bind}: Our desired bisimulation is the lifting of the relation $\sim$ defined by
\begin{itemize}
\item $1[x \leftarrow \ret{e}; \ R] \sim 1[R(x \coloneqq e)]$ for expression $\cdot \vdash e : \sigma$ and reaction $\Delta; \ x : \sigma \vdash R : I \to \tau$
\item $1[R(x \coloneqq v)] \sim 1[R(x \coloneqq e)]$ for
\begin{itemize}
\item reaction $\Delta; \ x : \sigma \vdash R : I \to \tau$
\item expression $\cdot \vdash e : \sigma$ and value $v \in \{0,1\}^{\sem{\sigma}}$ such that $e \Downarrow v$
\end{itemize}
\end{itemize}
\item \textsc{bind-ret}: Our desired bisimulation is the lifting of the relation $\sim$ defined by
\begin{itemize}
\item $1[x \leftarrow R; \ \ret{x}] \sim 1[R]$ for reaction $\Delta; \ \cdot \vdash R : I \to \tau$
\item $1[\val{v}] \sim 1[\val{v}]$ for value $v \in \{0,1\}^{\sem{\tau}}$
\end{itemize}
\item \textsc{bind-bind}: Our desired bisimulation is the lifting of the relation $\sim$ defined by
\begin{itemize}
\item $1[x_2 \leftarrow (x_1 \leftarrow R_1; \ R_2); \ S] \sim 1[x_1 \leftarrow R_1; \ x_2 \leftarrow R_2; \ S]$ for
\begin{itemize}
\item reaction $\Delta; \ \cdot \vdash R_1 : I \to \sigma_1$
\item reaction $\Delta; \ x_1 : \sigma_1 \vdash R_2 : I \to \sigma_2$
\item reaction $\Delta; \ x_2 : \sigma_2 \vdash S : I \to \tau$
\end{itemize}
\item $1[x_2 \leftarrow R_2; \ S] \sim 1[x_2 \leftarrow R_2; \ S]$ for
\begin{itemize}
\item reaction $\Delta; \ \cdot \vdash R_2 : I \to \sigma_2$
\item reaction $\Delta; \ x_2 : \sigma_2 \vdash S : I \to \tau$
\end{itemize}
\item $1[S] \sim 1[S]$ for reaction $\Delta; \ \cdot \vdash S : I \to \tau$
\end{itemize}
\item \textsc{samp-pure}: Our desired bisimulation is the lifting of the relation $\sim$ defined by
\begin{itemize}
\item $1[x \leftarrow \samp{(\dist \ e)}; \ R] \sim 1[R]$ for reaction $\Delta; \ \cdot \vdash R : I \to \tau$
\item $1[R] \sim 1[R]$ for reaction $\Delta; \ \cdot \vdash R : I \to \tau$
\end{itemize}
\item \textsc{read-det}: Our desired bisimulation is the lifting of the relation $\sim$ defined by
\begin{itemize}
\item $1[x \leftarrow \read{i}; \ y \leftarrow \read{i}; \ R] \sim 1[x \leftarrow \read{i}; \ R(y \coloneqq x)]$ for reaction $\Delta; \ x : \sigma, y : \sigma \vdash R : I \to \tau$
\item $1[x \leftarrow \val{v}; \ y \leftarrow \val{v}; \ R] \sim 1[x \leftarrow \val{v}; \ R(y \coloneqq x)]$ for
\begin{itemize}
\item reaction $\Delta; \ x : \sigma, y : \sigma \vdash R : I \to \tau$ 
\item value $v \in \{0,1\}^{\sem{\sigma}}$
\end{itemize}
\item $1[R] \sim 1[R]$ for reaction $\Delta; \ \cdot \vdash R : I \to \tau$
\end{itemize}
\item \textsc{if-left}: Our desired bisimulation is the lifting of the relation $\sim$ defined by
\begin{itemize}
\item $1[\ifte{\true}{R_1}{R_2}] \sim 1[R_1]$ for reactions $\Delta; \ \cdot \vdash R_1 : I \to \tau$ and $\Delta; \ \cdot \vdash R_2 : I \to \tau$
\item $1[R_1] \sim 1[R_1]$ for reaction $\Delta; \ \cdot \vdash R_1 : I \to \tau$
\end{itemize}
\item \textsc{if-right}: Our desired bisimulation is the lifting of the relation $\sim$ defined by
\begin{itemize}
\item $1[\ifte{\false}{R_1}{R_2}] \sim 1[R_2]$ for reactions $\Delta; \ \cdot \vdash R_1 : I \to \tau$ and $\Delta; \ \cdot \vdash R_2 : I \to \tau$
\item $1[R_2] \sim 1[R_2]$ for reaction $\Delta; \ \cdot \vdash R_2 : I \to \tau$
\end{itemize}
\item \textsc{if-ext}: Our desired bisimulation is the lifting of the relation $\sim$ defined by
\begin{itemize}
\item $1[R(x \coloneqq e)] \sim 1[\ifte{e}{R(x \coloneqq \true)}{R(x \coloneqq \false)]}$ for
\begin{itemize}
\item reaction $\Delta; \ x : \Bool \vdash R : I \to \tau$
\item expression $\cdot \vdash e : \Bool$
\end{itemize}
\item $1[R(x \coloneqq e)] \sim 1[R(x \coloneqq \true)]$ for
\begin{itemize}
\item reaction $\Delta; \ x : \Bool \vdash R : I \to \tau$
\item expression $\cdot \vdash e : \Bool$ such that $\eval{e} = 1$
\end{itemize}
\item $1[R(x \coloneqq e)] \sim 1[R(x \coloneqq \false)]$ for
\begin{itemize}
\item reaction $\Delta; \ x : \Bool \vdash R : I \to \tau$
\item expression $\cdot \vdash e : \Bool$ such that $\eval{e} = 0$
\end{itemize}
\end{itemize}
\item \textsc{exch-samp-samp}: Our desired bisimulation is the lifting of the relation $\sim$ defined by
\begin{itemize}
\item $1[x_1 \leftarrow \samp{(\dist_1 \ e_1)}; \ x_2 \leftarrow \samp{(\dist_2 \ e_2)}; \ \ret{(x_1,x_2)}] \sim \\ 1[x_2 \leftarrow \samp{(\dist_2 \ e_2)}; \ x_1 \leftarrow \samp{(\dist_1 \ e_1)}; \ \ret{(x_1,x_2)}]$ for
\begin{itemize}
\item expressions $\cdot \vdash e_1 : \sigma_1$ and $\cdot \vdash e_2 : \sigma_2$
\end{itemize}
\item $1[\val{v_1 v_2}] \sim 1[\val{v_1 v_2}]$ for values $v_1 \in \{0,1\}^{\sem{\tau_1}}$ and $v_2 \in \{0,1\}^{\sem{\tau_2}}$
\end{itemize}
\item \textsc{exch-samp-read}: Our desired bisimulation is the lifting of the relation $\sim$ defined by
\begin{itemize}
\item $1[x_1 \leftarrow \samp{(\dist \ e)}; \ x_2 \leftarrow \read{i}; \ \ret{(x_1,x_2)}] \sim 1[x_2 \leftarrow \read{i}; \ x_1 \leftarrow \samp{(\dist \ e)}; \ \ret{(x_1,x_2)}]$ for
\begin{itemize}
\item expression $\cdot \vdash e : \sigma$
\end{itemize}
\item $1[x_1 \leftarrow \samp{(\dist \ e)}; \ x_2 \leftarrow \val{v_2}; \ \ret{(x_1,x_2)}] \sim 1[x_2 \leftarrow \val{v_2}; \ x_1 \leftarrow \samp{(\dist \ e)}; \ \ret{(x_1,x_2)}]$ for
\begin{itemize}
\item expression $\cdot \vdash e : \sigma$
\item value $v_2 \in \{0,1\}^{\sem{\tau_2}}$
\end{itemize}
\item $\big(x_2 \leftarrow \read{i}; \ \ret{(\sem{\dist}(v),x_2)}\big) \sim 1[x_2 \leftarrow \read{i}; \ x_1 \leftarrow \samp{(\dist \ e)}; \ \ret{(x_1,x_2)}]$ for
\begin{itemize}
\item expression $\cdot \vdash e : \sigma$ and value $v \in \{0,1\}^{\sem{\sigma}}$ such that $e \Downarrow v$
\end{itemize}
\item $\big(x_2 \leftarrow \val{v_2}; \ \ret{(\sem{\dist}(\eval{e}),x_2)}\big) \sim 1[x_2 \leftarrow \val{v_2}; \ x_1 \leftarrow \samp{(\dist \ e)}; \ \ret{(x_1,x_2)}]$ for
\begin{itemize}
\item expression $\cdot \vdash e : \sigma$
\item value $v_2 \in \{0,1\}^{\sem{\tau_2}}$
\end{itemize}
\item $1[\val{v_1 v_2}] \sim 1[\val{v_1 v_2}]$ for values $v_1 \in \{0,1\}^{\sem{\tau_1}}$ and $v_2 \in \{0,1\}^{\sem{\tau_2}}$
\end{itemize}
\item \textsc{exch-read-read}: Our desired bisimulation is the lifting of the relation $\sim$ defined by
\begin{itemize}
\item $1[x_1 \leftarrow \read{i_1}; \ x_2 \leftarrow \read{i_2}; \ \ret{(x_1,x_2)}] \sim 1[x_2 \leftarrow \read{i_2}; \ x_1 \leftarrow \read{i_1}; \ \ret{(x_1,x_2)}]$
\item $1[x_1 \leftarrow \val{v_1}; \ x_2 \leftarrow \read{i_2}; \ \ret{(x_1,x_2)}] \sim 1[x_2 \leftarrow \read{i_2}; \ x_1 \leftarrow \val{v_1}; \ \ret{(x_1,x_2)}]$ for
\begin{itemize}
\item value $v_1 \in \{0,1\}^{\sem{\tau_1}}$
\end{itemize}
\item $1[x_1 \leftarrow \read{i_1}; \ x_2 \leftarrow \val{v_2}; \ \ret{(x_1,x_2)}] \sim 1[x_2 \leftarrow \val{v_2}; \ x_1 \leftarrow \read{i_1}; \ \ret{(x_1,x_2)}]$ for
\begin{itemize}
\item value $v_2 \in \{0,1\}^{\sem{\tau_2}}$
\end{itemize}
\item $1[x_1 \leftarrow \val{v_1}; \ x_2 \leftarrow \val{v_2}; \ \ret{(x_1,x_2)}] \sim 1[x_2 \leftarrow \val{v_2}; \ x_1 \leftarrow \val{v_1}; \ \ret{(x_1,x_2)}]$ for
\begin{itemize}
\item values $v_1 \in \{0,1\}^{\sem{\tau_1}}$ and $v_2 \in \{0,1\}^{\sem{\tau_2}}$
\end{itemize}
\item $1[x_2 \leftarrow \read{i_2}; \ \ret{(v_1,x_2)}] \sim 1[x_2 \leftarrow \read{i_2}; \ x_1 \leftarrow \val{v_1}; \ \ret{(x_1,x_2)}]$ for value $v_1 \in \{0,1\}^{\sem{\tau_1}}$
\item $1[x_1 \leftarrow \read{i_1}; \ x_2 \leftarrow \val{v_2}; \ \ret{(x_1,x_2)}] \sim 1[x_1 \leftarrow \read{i_1}; \ \ret{(x_1,v_2)}]$ for value $v_2 \in \{0,1\}^{\sem{\tau_2}}$
\item $1[x_2 \leftarrow \val{v_2}; \ \ret{(v_1,x_2)}] \sim 1[x_2 \leftarrow \val{v_2}; \ x_1 \leftarrow \val{v_1}; \ \ret{(x_1,x_2)}]$ for
\begin{itemize}
\item values $v_1 \in \{0,1\}^{\sem{\tau_1}}$ and $v_2 \in \{0,1\}^{\sem{\tau_2}}$
\end{itemize}
\item $1[x_1 \leftarrow \val{v_1}; \ x_2 \leftarrow \val{v_2}; \ \ret{(x_1,x_2)}] \sim 1[x_1 \leftarrow \val{v_1}; \ \ret{(x_1,v_2)}]$ for
\begin{itemize}
\item values $v_1 \in \{0,1\}^{\sem{\tau_1}}$ and $v_2 \in \{0,1\}^{\sem{\tau_2}}$
\end{itemize}
\item $1[\val{v_1 v_2}] \sim 1[\val{v_1 v_2}]$ for values $v_1 \in \{0,1\}^{\sem{\tau_1}}$ and $v_2 \in \{0,1\}^{\sem{\tau_2}}$
\end{itemize}
\end{itemize}
\end{proof}

\begin{figure}[ht!]
\begin{mathpar}
\inferrule*[right=exch-samp-samp]{\dist_1 : \sigma_1 \twoheadleftarrow \tau_1, \dist_2 : \sigma_2 \twoheadleftarrow \tau_2 \in \Sigma \\ \Gamma \vdash e_1 : \sigma_1 \\ \Gamma \vdash e_2 : \sigma_2}{\Delta; \ \Gamma \vdash \big(x_1 : \tau_1 \leftarrow \samp{(\dist_1 \ e_1)}; \ x_2 : \tau_2 \leftarrow \samp{(\dist_2 \ e_2)}; \ \ret{(x_1,x_2)}\big) = \\ \big(x_2 : \tau_2 \leftarrow \samp{(\dist_2 \ e_2)}; \ x_1 : \tau_1 \leftarrow \samp{(\dist_1 \ e_1)}; \ \ret{(x_1,x_2)}\big) : I \to \tau_1 \times \tau_2}\and
\inferrule*[right=exch-samp-read]{\dist : \sigma \to \tau_1 \in \Sigma \\ \Gamma \vdash e : \sigma \\ i : \tau_2 \in \Delta \\ i \in I}{\Delta; \ \Gamma \vdash \big(x_1 : \tau_1 \twoheadleftarrow \samp{(\dist \ e)}; \ x_2 : \tau_2 \leftarrow \read{i}; \ \ret{(x_1,x_2)}\big) =  \\ \big(x_2 : \tau_2 \leftarrow \read{i}; \ x_1 : \tau_1 \leftarrow \samp{(\dist \ e)}; \ \ret{(x_1,x_2)}\big) : I \to \tau_1 \times \tau_2}\and
\inferrule*[right=exch-read-read]{i_1 : \tau_1, i_2 : \tau_2 \in \Delta \\ i_1, i_2 \in I}{\Delta; \ \Gamma \vdash \big(x_1 : \tau_1 \leftarrow \read{i_1}; \ x_2 : \tau_2 \leftarrow \read{i_2}; \ \ret{(x_1,x_2)}\big) = \hspace{52.3pt} \\ \big(x_2 : \tau_2 \leftarrow \read{i_2}; \ x_1 : \tau_1 \leftarrow \read{i_1}; \ \ret{(x_1,x_2)}\big) : I \to \tau_1 \times \tau_2 \hspace{-39.8pt}}
\end{mathpar}
\caption{Alternative formulation of the \textsc{exch} rule for reaction equality.}
\label{fig:exch_alt}
\end{figure}

At last we get to the protocol level. A protocol bisimulation is entirely analogous to a reaction bisimulation, except we require the valuation property to hold: \emph{i)} per output channel $o$, and \emph{ii)} for all distributions (not necessarily final).

\begin{definition}[Protocol bisimulation]
A \emph{protocol bisimulation} $\sim$ is a binary relation on distributions on protocols $\Delta \vdash P : I \to O$ satisfying the following conditions:
\begin{itemize}
\item \emph{Closure under convex combinations}: For any distributions $\eta_1 \sim \varepsilon_1$ and $\eta_2 \sim \varepsilon_2$, and any coefficients $c_1, c_2 > 0$ with $c_1 + c_2 = 1$, we have $c_1 \eta_1 + c_2 \eta_2 \sim c_1 \varepsilon_1 + c_2 \varepsilon_2$.

\item \emph{Closure under input assignment}: For any distributions $\eta \sim \varepsilon$, input channel $i \in I$ of type $\tau$, and value $v \in \{0,1\}^{\sem{\tau}}$, we have $\eta[\read{i} \coloneqq \val{v}] \sim \varepsilon[\read{i} \coloneqq \val{v}]$.

\item \emph{Closure under computation}: For any distributions $\eta \sim \varepsilon$, we have $\eval{\eta} \sim \eval{\varepsilon}$.

\item \emph{Valuation property}: For any output channel $o \in O$, and any distributions $\eta \sim \varepsilon$, there exists a joint convex combination \[\eta = \sum_i c_i \, \eta_i \; \sim \, \sum_i c_i \, \varepsilon_i = \varepsilon\]
with $c_i > 0$ and $\sum_i c_i = 1$, such that
\begin{itemize}
\item the respective components $\eta_i \sim \varepsilon_i$ are again related, and
\item the distributions $\eta_i$ and $\varepsilon_j$ have the same value $v$ or lack thereof on $o$ if and only if $i = j$.
\end{itemize}
\end{itemize}
\end{definition}

\noindent Just like for reaction bisimulations, the joint sum in the valuation property is unique up to the order of the summands. We have an analogous canonical way of constructing protocol bisimulations:

\begin{definition}
Let $\sim$ be an arbitrary binary relation on distributions on protocols $\Delta \vdash P : I \to O$. The \emph{lifting} $\lift(\sim)$ is the closure of $\sim$ under joint convex combinations. Explicitly, $\lift(\sim)$ is defined by
\[\sum_i c_i \, \eta_i \; \lift(\sim) \; \sum_i c_i \, \varepsilon_i\]
for coefficients $c_i > 0$ with $\sum_i c_i = 1$ and distributions $\eta_i \sim \varepsilon_i$.
\end{definition}

\begin{lemma}\label{lem:protocol_seed}
Let $\sim$ be a binary relation on distributions on protocols $\Delta \vdash P : I \to O$ with the following properties:
\begin{itemize}
\item \emph{Closure under input assignment}: For any distributions $\eta \sim \varepsilon$, input channel $i \in I$ of type $\tau$, and value $v \in \{0,1\}^{\sem{\tau}}$, we have $\eta[\read{i} \coloneqq \val{v}] \sim \varepsilon[\read{i} \coloneqq \val{v}]$.

\item \emph{Lifting closure under computation}: For any distributions $\eta \sim \varepsilon$, we have $\eval{\eta} \lift(\sim) \, \eval{\varepsilon}$.

\item \emph{Valuation property}: For any output channel $o \in O$, and any distributions $\eta \sim \varepsilon$, there exists a joint convex combination \[\eta = \sum_i c_i \, \eta_i \; \sim \, \sum_i c_i \, \varepsilon_i = \varepsilon\]
with $c_i > 0$ and $\sum_i c_i = 1$, such that
\begin{itemize}
\item the respective components $\eta_i \sim \varepsilon_i$ are again related, and
\item the distributions $\eta_i$ and $\varepsilon_j$ have the same value $v$ or lack thereof on $o$ if and only if $i = j$.
\end{itemize}
\end{itemize}
Then the lifting $\lift(\sim)$ is a protocol bisimulation.
\end{lemma}

\begin{lemma}
We have the following: 
\begin{itemize}
\item The identity relation is a protocol bisimulation.
\item The inverse of a protocol bisimulation is a protocol bisimulation.
\item The composition of two protocol bisimulations is a protocol bisimulation.
\end{itemize}
\end{lemma}

\noindent We can now formally state what it means for exact protocol equality to be sound:

\begin{definition}
An axiom $\Delta \vdash P_1 = P_2 : I \to O$ is \emph{sound} if there is a protocol bisimulation $\sim$ such that $1[P_1] \sim 1[P_2]$.
\end{definition}

\noindent The ambient \ipdl theory for protocols is said to be sound if each of its axioms is sound. We now show that this implies overall soundness for exact equality:

\begin{lemma}[Soundness of exact equality of protocols]
If the ambient \ipdl theory for protocols is sound, then for any equal protocols $\Delta \vdash P_1 = P_2 : I \to O$, there exists a protocol bisimulation $\sim$ such that $1[P_1] \sim 1[P_2]$.
\end{lemma}

\begin{proof}
We first replace the rules \textsc{fold-if-left} and \textsc{fold-if-right} by the equivalent formulation in Figure \ref{fig:fold_if_alt}. We now proceed by induction on this alternative set of rules for exact protocol equality. We will freely use a measure in place of a reaction (rule \textsc{cong-react}) or a protocol (rules \textsc{embed}, \textsc{absorb-left}) to indicate the obvious lifting of the corresponding construct to measures on protocols.

\begin{itemize}
\item \textsc{refl}: Our desired bisimulation is the identity relation.
\item \textsc{sym}: Our desired bisimulation is the inverse of the bisimulation obtained from the premise.
\item \textsc{trans}: Our desired bisimulation is the composition of the two bisimulations obtained from the two premises.
\item \textsc{axiom}: The desired bisimulation exists by assumption.
\item \textsc{input-unused}: Our desired bisimulation is precisely the bisimulation obtained from the premise, seen as a bisimulation on distributions on protocols with the additional input $i$.
\item \textsc{embed}: Let $\sim$ be the bisimulation obtained from the premise. Our desired bisimulation $\sim_\phi$ is defined by
\begin{itemize}
\item $\phi^\star(\eta) \sim_\phi \phi^\star(\varepsilon)$ if $\eta \sim \varepsilon$
\end{itemize}
\item \textsc{cong-react}: Let $\sim$ be the reaction bisimulation obtained from the premise. Our desired bisimulation is the lifting of the relation $\sim_{\mathsf{react}}$ defined by
\begin{itemize}
\item $(o \coloneqq \eta) \sim_{\mathsf{react}} (o \coloneqq \eta')$ for distributions $\eta \sim \eta'$
\item $1[o \coloneqq v] \sim_{\mathsf{react}} 1[o \coloneqq v]$ for  value $v \in \{0,1\}^{\sem{\tau}}$
\end{itemize}
\item \textsc{cong-comp-left}: Let $\sim$ be the bisimulation obtained from the premise. Our desired bisimulation is the lifting of the relation $\sim_{\mathsf{par}}$ defined by
\begin{itemize}
\item $(\Par{\eta}{Q}) \sim_{\mathsf{par}} (\Par{\eta'}{Q})$ for $\eta \sim \eta'$ and protocol $\Delta \vdash Q : I \cup O_1 \to O_2$
\end{itemize}
The fact that this is indeed a bisimulation requires a fair amount of work; see Lemma \ref{lem:compositionality_exact}.
\item \textsc{cong-new}: Let $\sim$ be the bisimulation obtained from the premise. Our desired bisimulation $\sim_{\mathsf{new}}$ is defined by
\begin{itemize}
\item $(\new{o}{\tau}{\eta}) \sim_{\mathsf{new}} (\new{o}{\tau}{\eta'})$ if $\eta \sim \eta'$
\end{itemize}
\item \textsc{comp-comm}: Our desired bisimulation is the lifting of the relation $\sim$ defined by
\begin{itemize}
\item $1[\Par{P_1}{P_2}] \sim 1[\Par{P_2}{P_1}]$ for protocols $\Delta \vdash P_1 : I \cup O_2 \to O_1$ and $\Delta \vdash P_2 : I \cup O_1 \to O_2$
\end{itemize}
\item \textsc{comp-assoc}: Our desired bisimulation is the lifting of the relation $\sim$ defined by
\begin{itemize}
\item $1\big[\Par{(\Par{P_1}{P_2})}{P_3}\big] \sim 1\big[\Par{P_1}{(\Par{P_2}{P_3})}\big]$ for
\begin{itemize}
\item protocol $\Delta \vdash P_1 : I \cup O_2 \cup O_3 \to O_1$
\item protocol $\Delta \vdash P_2 : I \cup O_1 \cup O_3 \to O_2$
\item protocol $\Delta \vdash P_3 : I \cup O_1 \cup O_2 \to O_3$
\end{itemize}
\end{itemize}
\item \textsc{new-exch}: The desired bisimulation is the lifting of the relation $\sim$ defined by
\begin{itemize}
\item $1[\new{o_1}{\tau_1}{\new{o_2}{\tau_2}{P}}] \sim 1[\new{o_2}{\tau_2}{\new{o_1}{\tau_1}{P}}]$ for
\begin{itemize}
\item protocol $\Delta, o_1 : \tau_1, o_2 : \tau_2 \vdash P : I \to O \cup \{o_1,o_2\}$
\end{itemize}
\end{itemize}
\item \textsc{comp-new}: Our desired bisimulation is the lifting of the relation $\sim$ defined by
\begin{itemize}
\item $1[\Par{P}{(\new{o}{\tau}{Q})}] \sim 1[\new{o}{\tau}{(\Par{P}{Q})}]$ for
\begin{itemize}
\item protocol $\Delta \vdash P : I \cup O_2 \to O_1$
\item protocol $\Delta, o : \tau \vdash Q : I \cup O_1 \to O_2 \cup \{o\}$
\end{itemize}
\end{itemize}
\item \textsc{absorb-left}: Our desired bisimulation is the lifting of the relation $\sim$ defined by
\begin{itemize}
\item $1[\Par{P}{Q}] \sim 1[P]$ for protocols $\Delta \vdash P : I \to O$ and $\Delta \vdash Q : I \cup O \to \emptyset$
\end{itemize}
\item \textsc{diverge}: Our desired bisimulation is the lifting of the relation $\sim$ defined by
\begin{itemize}
\item $1[\assign{o}{x \leftarrow \read{o}; \ R}] \sim 1[\assign{o}{\read{o}}]$ for reaction $\Delta; \ \cdot \vdash R : I \cup \{o\} \to \tau$
\end{itemize}
\item \textsc{fold-if-left}: Our desired bisimulation is the lifting of the relation $\sim$ defined by
\begin{itemize}
\item $1[\new{l}{\tau}{\Par{\assign{o}{x \leftarrow \read{b}; \ \ifte{x}{\read{l}}}{S_2}}{\assign{l}{x \leftarrow \read{b}; \ S_1}}}] \sim \\ 1[\assign{o}{x \leftarrow \read{b}; \ \ifte{x}{S_1}{S_2}}]$ for
\begin{itemize}
\item reaction $\Delta; \ \cdot \vdash S_1 : I \cup \{o\} \to \tau$
\item reaction $\Delta; \ \cdot \vdash S_2 : I \cup \{o\} \to \tau$
\end{itemize}
\item $1[\new{l}{\tau}{\Par{\assign{o}{x \leftarrow \val{v}; \ \ifte{x}{\read{l}}}{S_2}}{\assign{l}{x \leftarrow \val{v}; \ S_1}}}] \sim \\ 1[\assign{o}{x \leftarrow \val{v}; \ \ifte{x}{S_1}{S_2}}]$ for
\begin{itemize}
\item value $v \in \{0,1\}$
\item reaction $\Delta; \ \cdot \vdash S_1 : I \cup \{o\} \to \tau$
\item reaction $\Delta; \ \cdot \vdash S_2 : I \cup \{o\} \to \tau$
\end{itemize}
\item $1[\new{l}{\tau}{\Par{\assign{o}{\read{l}}}{\assign{l}{S_1}}}] \sim 1[\assign{o}{S_1}]$ for reaction $\Delta; \ \cdot \vdash S_1 : I \cup \{o\} \to \tau$
\item $1[\new{l}{\tau}{\Par{\assign{o}{S_2}}{\assign{l}{S_1}}}] \sim 1[\assign{o}{S_2}]$ for
\begin{itemize}
\item reaction $\Delta; \ \cdot \vdash S_1 : I \cup \{o\} \to \tau$
\item reaction $\Delta; \ \cdot \vdash S_2 : I \cup \{o\} \to \tau$
\end{itemize}
\item $1[\new{l}{\tau}{\Par{\assign{o}{v_2}}{\assign{l}{S_1}}}] \sim 1[\assign{o}{v_2}]$ for
\begin{itemize}
\item reaction $\Delta; \ \cdot \vdash S_1 : I \cup \{o\} \to \tau$
\item value $v_2 \in \{0,1\}^{\sem{\tau}}$
\end{itemize}
\item $1[\new{l}{\tau}{\Par{\assign{o}{S_2}}{\assign{l}{v_1}}}] \sim 1[\assign{o}{S_2}]$ for
\begin{itemize}
\item value $v_1 \in \{0,1\}^{\sem{\tau}}$
\item reaction $\Delta; \ \cdot \vdash S_2 : I \cup \{o\} \to \tau$
\end{itemize}
\item $1[\new{l}{\tau}{\Par{\assign{o}{v_2}}{\assign{l}{v_1}}}] \sim 1[\assign{o}{v_2}]$ for values $v_1,v_2 \in \{0,1\}^{\sem{\tau}}$
\end{itemize}
\item \textsc{fold-if-right}: Our desired bisimulation is the lifting of the relation $\sim$ defined by
\begin{itemize}
\item $1[\new{r}{\tau}{\Par{\assign{o}{x \leftarrow \read{b}; \ \ifte{x}{S_1}{\read{r}}}}{\assign{r}{x \leftarrow \read{b}; \ S_2}}}] \sim \\ 1[\assign{o}{x \leftarrow \read{b}; \ \ifte{x}{S_1}{S_2}}]$ for
\begin{itemize}
\item reaction $\Delta; \ \cdot \vdash S_1 : I \cup \{o\} \to \tau$
\item reaction $\Delta; \ \cdot \vdash S_2 : I \cup \{o\} \to \tau$
\end{itemize}
\item $1[\new{r}{\tau}{\Par{\assign{o}{x \leftarrow \val{v}; \ \ifte{x}{S_1}{\read{r}}}}{\assign{r}{x \leftarrow \val{v}; \ S_2}}}] \sim \\ 1[\assign{o}{x \leftarrow \val{v}; \ \ifte{x}{S_1}{S_2}}]$ for
\begin{itemize}
\item value $v \in \{0,1\}$
\item reaction $\Delta; \ \cdot \vdash S_1 : I \cup \{o\} \to \tau$
\item reaction $\Delta; \ \cdot \vdash S_2 : I \cup \{o\} \to \tau$
\end{itemize}
\item $1[\new{r}{\tau}{\Par{\assign{o}{\read{r}}}{\assign{r}{S_2}}}] \sim 1[\assign{o}{S_2}]$ for reaction $\Delta; \ \cdot \vdash S_2 : I \cup \{o\} \to \tau$
\item $1[\new{r}{\tau}{\Par{\assign{o}{S_1}}{\assign{r}{S_2}}}] \sim 1[\assign{o}{S_1}]$ for
\begin{itemize}
\item reaction $\Delta; \ \cdot \vdash S_1 : I \cup \{o\} \to \tau$
\item reaction $\Delta; \ \cdot \vdash S_2 : I \cup \{o\} \to \tau$
\end{itemize}
\item $1[\new{r}{\tau}{\Par{\assign{o}{v_1}}{\assign{r}{S_2}}}] \sim 1[\assign{o}{v_1}]$ for
\begin{itemize}
\item value $v_1 \in \{0,1\}^{\sem{\tau}}$
\item reaction $\Delta; \ \cdot \vdash S_2 : I \cup \{o\} \to \tau$
\end{itemize}
\item $1[\new{r}{\tau}{\Par{\assign{o}{S_1}}{\assign{r}{v_2}}}] \sim 1[\assign{o}{S_1}]$ for
\begin{itemize}
\item reaction $\Delta; \ \cdot \vdash S_1 : I \cup \{o\} \to \tau$
\item value $v_2 \in \{0,1\}^{\sem{\tau}}$
\end{itemize}
\item $1[\new{r}{\tau}{\Par{\assign{o}{v_1}}{\assign{r}{v_2}}}] \sim 1[\assign{o}{v_1}]$ for values $v_1, v_2 \in \{0,1\}^{\sem{\tau}}$
\end{itemize}
\item \textsc{fold-bind}: Our desired bisimulation is the lifting of the relation $\sim$ defined by
\begin{itemize}
\item $1[\new{c}{\tau_1}{\Par{\assign{o}{x \leftarrow \read{c};} \ R_2}{\assign{c}{R_1}}}] \sim 1[\assign{o}{x \leftarrow R_1; \ R_2}]$ for
\begin{itemize}
\item reaction $\Delta; \ \cdot \vdash R_1 : I \cup \{o\} \to \tau_1$
\item reaction $\Delta; \ x : \tau_1 \vdash R_2 : I \cup \{o\} \to \tau_2$
\end{itemize}
\item $1[\new{c}{\tau_1}{\Par{\assign{o}{R_2}}{\assign{c}{v_1}}}] \sim 1[\assign{o}{R_2}]$ for
\begin{itemize}
\item value $v_1 \in \{0,1\}^{\sem{\tau_1}}$
\item reaction $\Delta; \ \cdot \vdash R_2 : I \cup \{o\} \to \tau_2$
\end{itemize}
\item $1[\new{c}{\tau_1}{\Par{\assign{o}{v_2}}{\assign{c}{v_1}}}] \sim 1[\assign{o}{v_2}]$ for
\begin{itemize}
\item values $v_1 \in \{0,1\}^{\sem{\tau_1}}$ and $v_2 \in \{0,1\}^{\sem{\tau_2}}$
\end{itemize}
\end{itemize}
\item \textsc{subsume}: Our desired bisimulation is the lifting of the relation $\sim$ defined by
\begin{itemize}
\item $1[\Par{\assign{o_1}{x_0 \leftarrow \read{o_0}; \ R_1}}{\assign{o_2}{x_0 \leftarrow \read{o_0}; \ x_1 \leftarrow \read{o_1}; \ R_2}}] \sim \\ 1[\Par{\assign{o_1}{x_0 \leftarrow \read{o_0}; \ R_1}}{\assign{o_2}{x_1 \leftarrow \read{o_1}; \ R_2}}]$ for
\begin{itemize}
\item reaction $\Delta; \ x_0 : \tau_0 \vdash R_1 : I \cup \{o_1,o_2\} \to \tau_1$
\item reaction $\Delta; \ x_1 : \tau_1 \vdash R_2 : I \cup \{o_1,o_2\} \to \tau_2$
\end{itemize}
\item $1[\Par{\assign{o_1}{x_0 \leftarrow \val{v_0}; \ R_1}}{\assign{o_2}{x_0 \leftarrow \val{v_0}; \ x_1 \leftarrow \read{o_1}; \ R_2}}] \sim \\ 1[\Par{\assign{o_1}{x_0 \leftarrow \val{v_0}; \ R_1}}{\assign{o_2}{x_1 \leftarrow \read{o_1}; \ R_2}}]$ for
\begin{itemize}
\item value $v_0 \in \{0,1\}^{\sem{\tau_0}}$
\item reaction $\Delta; \ x_0 : \tau_0 \vdash R_1 : I \cup \{o_1,o_2\} \to \tau_1$
\item reaction $\Delta; \ x_1 : \tau_1 \vdash R_2 : I \cup \{o_1,o_2\} \to \tau_2$
\end{itemize}
\item $1[\Par{\assign{o_1}{R_1}}{\assign{o_2}{x_1 \leftarrow \read{o_1}; \ R_2}}] \sim 1[\Par{\assign{o_1}{R_1}}{\assign{o_2}{x_1 \leftarrow \read{o_1}; \ R_2}}]$ for
\begin{itemize}
\item reaction $\Delta; \ \cdot \vdash R_1 : I \cup \{o_1,o_2\} \to \tau_1$
\item reaction $\Delta; \ x_1 : \tau_1 \vdash R_2 : I \cup \{o_1,o_2\} \to \tau_2$
\end{itemize}
\item $1[\Par{\assign{o_1}{v_1}}{\assign{o_2}{R_2}}] \sim 1[\Par{\assign{o_1}{v_1}}{\assign{o_2}{R_2}}]$ for
\begin{itemize}
\item value $v_1 \in \{0,1\}^{\sem{\tau_1}}$
\item reaction $\Delta; \ \cdot \vdash R_2 : I \cup \{o_1,o_2\} \to \tau_2$
\end{itemize}
\item $1[\Par{\assign{o_1}{v_1}}{\assign{o_2}{v_2}}] \sim 1[\Par{\assign{o_1}{v_1}}{\assign{o_2}{v_2}}]$ for values $v_1 \in \{0,1\}^{\sem{\tau_1}}$ and $v_2 \in \{0,1\}^{\sem{\tau_2}}$
\end{itemize}


\item \textsc{subst}: Let $\sim$ be the reaction bisimulation obtained from the premise. Our desired bisimulation is the lifting of the relation $\sim_\mathsf{subst}$ defined by
\begin{itemize}
\item $\big(\Par{\assign{o_1}{\eta}}{\assign{o_2}{x_1 \leftarrow \read{o_1}; \ R_2}}\big) \sim_\mathsf{subst} \big(\Par{\assign{o_1}{\eta}}{\assign{o_2}{x_1 \leftarrow \eta; \ R_2}}\big)$ for
\begin{itemize}
\item distribution $\eta$ on reactions $\Delta; \ \cdot \vdash R_1 : I \cup \{o_1,o_2\} \to \tau_1$
\item reaction $\Delta; \ \cdot \vdash R_1 : I \cup \{o_1,o_2\} \to \tau_1$ evaluating to the same distribution as $\eta$
\item reaction $\Delta; \ x_1 : \tau_1 \vdash R_2 : I \cup \{o_1,o_2\} \to \tau_2$
\end{itemize}
such that $1[x_1 \leftarrow R_1; \ x_1' \leftarrow R_1; \ \ret{(x_1,x_1')}] \sim 1[x_1 \leftarrow R_1; \ \ret{(x_1,x_1)}]$
\item $1[\Par{\assign{o_1}{v_1}}{\assign{o_2}{R_2}}] \sim_\mathsf{subst} 1[\Par{\assign{o_1}{v_1}}{\assign{o_2}{R_2}}]$ for
\begin{itemize}
\item value $v_1 \in \{0,1\}^{\sem{\tau_1}}$
\item reaction $\Delta; \ \cdot \vdash R_2 : I \cup \{o_1,o_2\} \to \tau_2$
\end{itemize}
\item $1[\Par{\assign{o_1}{v_1}}{\assign{o_2}{v_2}}] \sim_\mathsf{subst} 1[\Par{\assign{o_1}{v_1}}{\assign{o_2}{v_2}}]$ for values $v_1 \in \{0,1\}^{\sem{\tau_1}}$ and $v_2 \in \{0,1\}^{\sem{\tau_2}}$
\end{itemize}
\item \textsc{drop}: Let $\sim$ be the reaction bisimulation obtained from the premise. Our desired bisimulation is the lifting of the relation $\sim_{\mathsf{drop}}$ defined by
\begin{itemize}
\item $\big(\Par{\assign{o_1}{\eta_1}}{\assign{o_2}{x_1 \leftarrow \read{o_1}; \ R_2}}\big) \sim_{\mathsf{drop}} \big(\Par{\assign{o_1}{\eta_1}}{\assign{o_2}{\eta_2}}\big)$ for
\begin{itemize}
\item measure $\eta_1$ on reactions $\Delta; \ \cdot \vdash R_1 : I \cup \{o_1,o_2\} \to \tau_1$
\item reaction $\Delta; \ \cdot \vdash R_1 : I \cup \{o_1,o_2\} \to \tau_1$ such that
\begin{itemize}
\item[\emph{i)}] $R_1$ either evaluates to the same distribution as $\eta_1$, or
\item[\emph{ii)}] there exists a measure $\overline{\eta_1}$ on reactions $\Delta; \ \cdot \vdash R_1 : I \cup \{o_1,o_2\} \to \tau_1$ such that $R_1$ evaluates to the same distribution as $\eta_1 + \overline{\eta_1}$
\end{itemize}
\item distribution $\eta_2$ on reactions $\Delta; \ \cdot \vdash R_2 : I \cup \{o_1,o_2\} \to \tau_2$
\item reaction $\Delta; \ \cdot \vdash R_2 : I \cup \{o_1,o_2\} \to \tau_2$ evaluating to the same distribution as $\eta_2$
\end{itemize}
such that $1[x_1 \leftarrow R_1; \ R_2] \sim 1[R_2]$
\item $(\Par{\assign{o_1}{v_1}}{\assign{o_2}{R_2}}) \sim_{\mathsf{drop}} (\Par{\assign{o_1}{v_1}}{\assign{o_2}{R_2}})$ for
\begin{itemize}
\item value $v_1 \in \{0,1\}^{\sem{\tau_1}}$
\item reaction $\Delta; \ \cdot \vdash R_2 : I \cup \{o_1,o_2\} \to \tau_2$
\end{itemize}
\item $(\Par{\assign{o_1}{v_1}}{\assign{o_2}{v_2}}) \sim_{\mathsf{drop}} (\Par{\assign{o_1}{v_1}}{\assign{o_2}{v_2}})$ for values $v_1 \in \{0,1\}^{\sem{\tau_1}}$ and $v_2 \in \{0,1\}^{\sem{\tau_2}}$
\end{itemize}
\end{itemize}
\end{proof}

\begin{figure*}[h]
\begin{mathpar}
\inferrule*[right=fold-if-left]{o \notin I \\ b \in I \\ b : \Bool, o : \tau \in \Delta \\ \Delta; \ \cdot \vdash S_1 : I \cup \{o\} \to \tau \\ \Delta; \ \cdot \vdash S_2 : I \cup \{o\} \to \tau}{\Delta \vdash \big(\new{l}{\tau}{\Par{\assign{o}{x : \Bool \leftarrow \read{b}; \ \ifte{x}{{\color{red} \read{l}}}{S_2}}}{{\color{red} \assign{l}{x : \Bool \leftarrow \read{b}; \ S_1}}}}\big) = \\ \big(\assign{o}{x : \Bool \leftarrow \read{b}; \ \ifte{x}{{\color{red} S_1}}{S_2}}\big) : I \to \{o\}\hspace{15pt}}\and
\inferrule*[right=fold-if-right]{o \notin I \\ b \in I \\ b : \Bool, o : \tau \in \Delta \\ \Delta; \ \cdot \vdash S_1 : I \cup \{o\} \to \tau \\ \Delta; \ \cdot \vdash S_2 : I \cup \{o\} \to \tau}{\Delta \vdash \big(\new{r}{\tau}{\Par{\assign{o}{x : \Bool \leftarrow \read{b}; \ \ifte{x}{S_1}{{\color{red} \read{r}}}}}{{\color{red} \assign{r}{x : \Bool \leftarrow \read{b}; \ S_2}}}}\big) = \\ \big(\assign{o}{x : \Bool \leftarrow \read{b}; \ \ifte{x}{S_1}{{\color{red} S_2}}}\big) : I \to \{o\}\hspace{25pt}}
\end{mathpar}
\caption{Alternative formulation of the \textsc{fold-if-left} and \textsc{fold-if-right} rules.}
\label{fig:fold_if_alt}
\end{figure*}

\noindent The remainder of this section is devoted to proving the following lemma:

\begin{lemma}[Compositionality for the exact equality of protocols]\ref{lem:compositionality_exact}
Let $\sim$ be a bisimulation on protocols $\Delta \vdash P : I \cup O_2 \to O_1$. Then the lifting of the relation $\sim_{\mathsf{par}}$ defined by
\begin{itemize}
\item $(\Par{\eta}{Q}) \sim_{\mathsf{par}} (\Par{\eta'}{Q})$ for $\eta \sim \eta'$ and protocol $\Delta \vdash Q : I \cup O_1 \to O_2$
\end{itemize}
is a protocol bisimulation.
\end{lemma}

\begin{proof}
The one property difficult to verify is lifting closure under computation: for any protocol $\Delta \vdash Q : I \cup O_1 \to O_2$, and any distributions $\eta \sim \eta'$, we have $\eval{(\Par{\eta}{Q})} \lift(\sim_\mathsf{par}) \eval{(\Par{\eta'}{Q})}$. The difficulty arises from the \emph{global} nature of the protocol semantics: in the composition $\Par{P}{Q}$, a step of the form $P \outstep{o}{v} P'$ \emph{changes} the protocol $Q$ (specifically to $Q[\assign{\read{o}}{\val{v}}]$). This makes it hard to express the computation of $\Par{P}{Q}$ in terms of the computation of $P$, because in the course of the latter we are simultaneously probabilistically updating $Q$.
%
%We solve this problem by defining an alternate \emph{local} form of interaction for \ipdl protocols, and showing that the resulting operational semantics agrees with the original one. Informally speaking, the global semantics of \ipdl protocols has a \emph{push} character -- the moment a value $v$ on a channel $o$ is computed, every $\read{o}$ command in all other reactions is replaced by $\val{v}$. In contrast, the local form of the semantics that we are about to define has a \emph{pull} character -- a reaction containing a $\read{o}$ command extracts the value $v$ from channel $o$, if possible, and replaces this \emph{particular} occurrence of $\read{o}$ by $\val{v}$. We formally define this mechanism in Figure {fig:protocols_semantics_local}. The relation 
%
%
%For protocols, the relation $P \in{o}{v} Q$ indicates that we have replaced one occurrence of the command $\read{o}$ in a \emph{single} reaction in $P$ by $\val{v}$, yielding the protocol $Q$. The protocol $Q$ may be seen as a single-step approximation towards the protocol obtained by performing the output assignment $o \coloneqq v$ in $P$.
%
%\begin{figure}
%\begin{mathpar}
%\fbox{$R \in{o}{v} S$}\\
%\inferrule*{ }{\read{o} \in{o}{v} \val{v}}\and
%\inferrule*{R_1 \in{o}{v} R_1'}{(\ifte{e}{R_1}{R_2}) \in{o}{v} (\ifte{e}{R_1'}{R_2})}\and
%\inferrule*{R_2 \in{o}{v} R_2'}{(\ifte{e}{R_1}{R_2}) \in{o}{v} (\ifte{e}{R_2'}{R_2})}\and
%\inferrule*{R \in{o}{v} R'}{(x : \sigma \leftarrow R; \ S) \in{o}{v} {(x : \sigma \leftarrow R'; \ S)}}\and
%\inferrule*{S \in{o}{v} S'}{(x : \sigma \leftarrow R; \ S) \in{o}{v} {(x : \sigma \leftarrow R; \ S')}}\\\\
%\fbox{$P \in{o}{v} Q$}\\
%\inferrule*{R \in{o}{v} S}{(\assign{o}{R}) \in{o}{v} (\assign{o}{S})}\and
%\inferrule*{P \in{o}{v} P'}{(\Par{P}{Q}) \in{o}{v} (\Par{P'}{Q})}\and
%\inferrule*{Q \in{o}{v} Q'}{(\Par{P}{Q}) \in{o}{v} (\Par{P}{Q'})}\and
%\inferrule*{P \in{o}{v} P' \\ o \neq c}{(\new{c}{\tau}{P}) \in{o}{v} (\new{c}{\tau}{P'})}\\\\
%\fbox{$P \hookrightarrow Q$}\\
%\inferrule*[right=pull-comp-left]{P \outstep{o}{v} P' \\ Q \in{o}{v} Q'}{(\Par{P}{Q}) \pull{o}{v} (\Par{P}{Q'})}\and
%\inferrule*[right=pull-comp-right]{Q \outstep{o}{v} Q' \\ P \in{o}{v} P'}{(\Par{P}{Q}) \pull{o}{v} (\Par{P'}{Q})}\and
%\inferrule*[right=pull-new]{P \pull{o}{v} P' \\ o \neq c}{(\new{c}{\tau}{P}) \pull{o}{v} (\new{c}{\tau}{P'})}
%\end{mathpar}
%\caption{The local form of small-step operational semantics for \ipdl protocols.}
%\label{fig:protocols_semantics_local}
%\end{figure}
%
%
We now have all the preliminaries necessary to prove that $\sim_\mathsf{par}$ enjoys lifting closure under computation.


 Since the set $O_1$ of outputs is finite, we can apply the valuation property of $\sim$ in succession for each output channel $o \in O_1$, until we end up with the special case when $\eta$ and $\eta'$ have the same value $v$ or lack thereof on each output channel. In other words, it suffices to prove the following:
%
%\emph{\begin{center}
%Claim 1: For any protocol $\Delta \vdash Q : I \cup O_1 \to O_2$, and any measures $\eta \sim \eta'$ that have the same value $v$ or lack thereof on any output channel $o \in O_1$, if $(\Par{\eta}{Q}) \Downarrow \varepsilon$ and $(\Par{\eta'}{Q}) \Downarrow \varepsilon'$, then $\varepsilon \lift(\sim_\mathsf{par}) \, \varepsilon'$.
%\end{center}}
%
%The remainder of this section is devoted to proving this claim.
%
%
%
%
%\emph{\begin{center}
%Claim 2: For any protocol $\Delta \vdash Q : I \cup O_1 \to O_2$, and any measures $\eta$ and $\eta'$ that have the same value $v$ or lack thereof on any output channel $o \in O_1$, if $(\Par{\eta}{Q}) \Downarrow \varepsilon$ and $(\Par{\eta'}{Q}) \Downarrow \varepsilon'$, then $\varepsilon \lift(\sim_\mathsf{par}) \, \varepsilon'$.
%\end{center}}
%
%
%The local form of the big-step operational semantics for protocols $P \Downarrow \eta$, see Figure \ref{fig:protocols_big_step_local}, performs as many output and internal steps as possible in an attempt to compute
%
%\begin{figure}
%\begin{mathpar}
%\fbox{$P \outstep{O} Q$}\\
%
%\inferrule*[right=out-val]{ }{(\assign{o}{\val{v}}) \outstep{o}{v} (\assign{o}{v})}\and
%\inferrule*[right=out-comp-left]{P \outstep{o}{v} P'}{(\Par{P}{Q}) \outstep{o}{v} \big(\Par{P'}{Q[\assign{\read{o}}{\val{v}}]}\big)}\and
%\inferrule*[right=out-comp-right]{Q \outstep{o}{v} Q'}{(\Par{P}{Q}) \outstep{o}{v} \big(\Par{P[\assign{\read{o}}{\val{v}}]}{Q'}\big)}\and
%\inferrule*[right=out-new]{P \outstep{o}{v} P' \\ o \neq c}{(\new{c}{\tau}{P}) \outstep{o}{v} (\new{c}{\tau}{P'})}\\\\
%
%
%\inferrule*{P \Downarrow \eta \\ P \outset{\out{P}} \eta}{P \ \lfinal}\\\\
%
%
%\fbox{$P \outset{O} Q$}\\
%\inferrule*{P \Downarrow \eta \\ P \outset{\out{P}} \eta}{P \ \lfinal}\\\\
%
%
%\fbox{$P \ \lfinal$}\\
%\inferrule*{P \Downarrow 1[Q] \\ P \outset{\out{P}} Q}{P \ \lfinal}\\\\
%\fbox{$P \Rightarrow \eta$}\\
%\inferrule*{}{ }\and
%\inferrule*{P \to \sum_i c_i \ 1[P_i] \\ P_i \Rightarrow \eta_i}{P \Rightarrow \sum_i c_i \ \eta_i}\and
%\inferrule*{P \hookrightarrow Q \\ Q \Rightarrow \eta}{P \Rightarrow \eta}\and
%\inferrule*{P \ \lfinal}{P \Rightarrow 1[P]}
%\end{mathpar}
%\caption{The local form of big-step operational semantics for \ipdl protocols.}
%\label{fig:protocols_big_step_local}
%\end{figure}
%
%
%
%
\end{proof}
%
%\begin{definition}[Sound exact theory]
%Fix a signature $\Sigma$ and an interpretation $\Int$. An \emph{exact} \ipdl theory $\mathbb{T}$ is a triple $(\mathbb{T}_e,\mathbb{T}_r,\mathbb{T}_p)$ of expression-level, reaction-level, and protocol-level \ipdl theories, respectively. The exact theory $\mathbb{T}$ is \emph{sound} with respect to $\Int$, written $\Int \vDash \mathbb{T}$, if each of $\mathbb{T}_e$, $\mathbb{T}_r$, and $\mathbb{T}_p$ is sound with respect to $\Int$.
%\end{definition}

\end{document}

%\section{Computational Semantics of \ipdl}
%The second step is to define an \emph{interaction}, seen as a \emph{security game}, between an \ipdl program and a resource-bounded, probabilistic \emph{distinguisher}. The interaction semantics is used to validate \emph{approximate} observational equivalences: these are used for cryptographic hardness assumptions, such as the security of encryption schemes or Diffie-Hellman, as well as top-level statements of security for protocols.

%\section{Case Studies in \ipdl}
%In this section, we briefly describe the case studies we have completed in \ipdl so far. The full proofs are available in a separate document \cite{case_studies}. We demonstrate through lines of code that the proof effort of \ipdl scales well with increasing
%protocol complexity, see Figure~\ref{fig:cases}.
%
%\begin{figure}
%\begin{center}
%\begin{tabular}{|l|l|}
%\hline
%Case study & Lines of Code \\
%\hline
%A2S: CPA & 239 LoC \\
%\hline
%OT: Pre-Processing & 480 LoC \\
%\hline
%Multi-Party Coin Flip & 2019 LoC \\
%\hline
%\end{tabular}
%\end{center}
%\caption{Case Studies in \textsf{ipdl}.}
%\label{fig:cases}
%\end{figure}
%
%
%
%In this section, we briefly describe the case studies we have completed in \ipdl, and outline several key proof steps that conveniently employ equational reasoning. Our case studies range from simple communication protocols to a two-party GMW protocol and a multi-party coin flip protocol. We demonstrate through lines of code that the proof effort of \ipdl scales well with increasing protocol complexity, see Figure~\ref{fig:cases}.

%\subsection{Communication Protocols}
%%We prove secure two different communication protocols that construct a secure communication channel from an authenticated one. The authenticated channels allow the adversary to observe in-flight messages and schedule delivery of them; in contrast, the secure communication channels only allow the adversary to observe the \emph{presence} of the channel, but none of the message contents.
%
%\subsubsection{Secure Communication from CPA Security}
%In our first case study (Section 1 of \cite{case_studies}), we prove secure a communication protocol that constructs a secure communication channel from an authenticated one. The authenticated channels allow the adversary to observe in-flight messages and schedule delivery of them; in contrast, the secure communication channels allow the adversary to schedule messages but grant no access to the message contents. We note that in this setup, the ability of the adversary to delay the messages sent over the secure channel is inherited from the corresponding control of the adversary over the authenticated channel.
%
%%is a generalization of our example from Section~\ref{sec:overview} to allow the adversary to schedule the delivery of each message. In line with Section~\ref{sec:overview}, we prove that a CPA-secure encryption scheme may be used alongside an authenticated channel to achieve a secure one. 
%
%\subsection{Oblivious Transfer Protocols}
%We next consider a particular Oblivious Transfer (OT) construction. This example (Section 2 of \cite{case_studies}) is proven secure in the \emph{semi-honest} (or \emph{honest-but-curious}) setting, where all parties operate correctly, but corrupt parties leak all private data to the adversary. We prove that the leaked values reveal no private information about the honest parties. To encode semi-honest corruption, we augment the protocols with appropriate \emph{leakage} functions that forward to the adversary all values visible to the corrupted parties. In turn, the simulator must take as input the leakages in the ideal protocol
%(usually minimal), and output suitable leakages in the real protocol.
%
%
%
%
%%We next consider several Oblivious Transfer (OT) constructions secure. These examples are proven in the \emph{semi-honest} (or \emph{honest-but-curious}) setting, where we assume the parties operate correctly, but corrupt parties leak all private data to the adversary. We prove that leaked values reveal no private information about the honest parties. To encode semi-honest corruption, we augment the protocols with \emph{leakage} functions that send all values visible to the corrupted party to the adversary. 
%
%In (1-out-of-2) OT, Bob wishes to obtain exactly one of Alice's two messages, without revealing his choice~\cite{gmw}. Alice doesn't learn which message Bob asked for, while Bob doesn't learn the other of the two messages. The ideal functionality simply receives the two messages $m_0,m_1$ from the sender, the choice bit $c$ from the receiver, and outputs $m_i$. We analyze the most interesting case when Bob is semi-honest and Alice is honest. Hence, the real-world leakages are derived solely from the input $i$ coming from the receiver, and the output $m_i$ coming from the ideal functionality, with no access to any information about message $m_{1-i}$.
%
%%We prove the security of three main OT constructions from the literature:
%%first, we show that 1-out-of-4 OT, which is used by our GMW example, can
%%be realized from three instances of an ideal 1-out-of-2 OT~\cite{naor1999oblivious};
%%then, we show a \emph{preprocessing} result for OT, which allows Alice
    %%and Bob to establish an OT in an offline phase, then use this OT for a
        %%fast online phase~\cite{beaver1995precomputing}; finally, we show that 1-out-of-2 OT can be realized using a trapdoor permutation and a hard-core bit predicate~\cite{gmw}.
%%
%%To illustrate how \ipdl allows us to carry out probabilistic reasoning, we outline here a few key steps from the second construction. 
%%In the pre-processing phase, Alice randomly generates a new pair of keys $(k_0,
%%k_1)$, while Bob randomly decides on one of these keys, obtaining a
%%choice bit $j$. They then
%%use the underlying (idealized) OT to securely transfer the randomly chosen key
%%$k_j$ to Bob.
%%
%%In the online phase, Bob encrypts his actual choice bit $i$ by $\mathsf{xor}$-ing it
%%with $j$, chosen randomly in the prior phase. He sends his encrypted choice $i
%%\oplus j$ to Alice, who responds by first swapping her two keys if $i \oplus j$ is
%%true, then sending Bob her two keys, $\mathsf{xor}$-ed with their respective
%%messages. Bob has enough information to recover his chosen message, but the
%%other one appears uniformly random. 
%%
%%To prove that Bob does not learn any information about the message he did not ask for, we carry out two probabilistic arguments. The first, which we call \emph{decoupling}, observes that selecting two keys $k_0,k_1$ from the same distribution $\mu$, and then randomly deciding to return either $(k_0,k_1)$ or $(k_1,k_0)$ is perfectly indistinguishable from just returning $(k_0,k_1)$. To see this, consider the protocol
%%where $\Key(0)$ and $\Key(1)$ are assigned the reaction $\samp{\mu}$, and 
%%\begin{align*}
    %%\KeyPair &\coloneqq f \leftarrow \samp{\flip}; \ k_0 \leftarrow \Key(0); \
    %%k_1 \leftarrow \Key(1); \ \ifte{f}{\ret{(k_1,k_0)}}{\ret{(k_0,k_1)}}.
%%\end{align*}
%%Here the channels $\Key(0),\Key(1)$ are internal and the channel $\KeyPair$ is an output. We fold the two key samplings into the channel $\KeyPair$:
%%\begin{align*}
    %%\KeyPair &\coloneqq f \leftarrow \samp{\flip}; \ \mathsf{if} \ f \ \mathsf{then} \ {\color{red} k_0 \leftarrow \samp{\mu}; \ k_1 \leftarrow \samp{\mu};} \ \ret{(k_1,k_0)} \\ 
    %%& {\color{white} \KeyPair \coloneqq f \leftarrow \Flip; \ \mathsf{if} \ f \ }{\mathsf{else} \ {\color{red} k_0 \leftarrow \samp{\mu}; \ k_1 \leftarrow \samp{\mu};} \ \ret{(k_0,k_1)}}
%%\end{align*}
%%
%%Since the samplings are interchangeable, we end up doing the same thing either way:
%%\begin{align*}
    %%\KeyPair & \coloneqq f \leftarrow \samp{\flip}; \ \mathsf{if} \ f \ \mathsf{then} \ {\color{red} k_1 \leftarrow \samp{\mu}; \ k_0 \leftarrow \samp{\mu};} \ \ret{(k_1,k_0)} \\ 
    %%&{\color{white} \KeyPair \coloneqq f \leftarrow \Flip; \ \mathsf{if} \ f \ }{\mathsf{else} \ k_0 \leftarrow \samp{\mu}; \ k_1 \leftarrow \samp{\mu};} \ \ret{(k_0,k_1)}
%%\end{align*}
%%
%%So we may just as well not flip:
%%\[ \KeyPair \coloneqq k_0 \leftarrow \samp{\mu}; \ k_1 \leftarrow \samp{\mu}; \
%%\ret{(k_0,k_1)}. \]
%%We emphasize that no complex probabilistic reasoning is necessary in the
%%argument, but only a few simple application of equational proof rules. 
%%
%%The second probabilistic argument concerns the distribution $\mu$, which
%%represents uniform randomness. 
%%Rather than modeling uniform randomness intrinsically in Coq, we only need to
%%introduce the (sound) axiom that $\mu = (x \leftarrow \mu;\ \unit{x \oplus y})$ for any
%%bitstring $y$. 
%
%\subsection{Multi-Party Coin Flip Protocol}
%Our first large case study (Section 3 of \cite{case_studies}) is for a protocol that allows an arbitrary number of mutually distrusting parties to collaboratively generate fair randomness, due to Blum~\cite{blum1983coin}. To do so, each party locally generates randomness, and commits it to all other parties. We assume an idealized commitment functionality that also bakes in a notion of broadcast, to prevent equivocation. Each party decommits their randomness once all other commitments have been collected; the output of the protocol is the Boolean sum of all decommitments.
%
%Unlike previous examples, this example is secure in the \emph{malicious} model. We model malicious parties by assigning them a \emph{shell}, which simply forwards all information between the protocol and the adversary.
%
%\section{Maude Formalization}
%\subsection{Normal Forms}
We work with protocols that start with a list of declarations of 
internal channels, using $\mathsf{new}$, followed by a parallel compositions of channel assignments. The reactions in these
assignments can be transformed into a list of binds of the form
$x : \tau \leftarrow read(c)$, called
bind-read reactions, followed by a reaction without binds.
The list of binds can be regarded as commutative, 
as two reactions with the same list of binds in different order
are equivalent due to the reaction equivalence rule \textsc{exch}.
Similarly, different order of declarations of internal channels gives
equivalent protocols, by using the protocol equivalence rule
\textsc{new-exch}. When writing equivalence proofs, we do not want to 
make the use of these rules explicit. Therefore, we introduce
normal forms of reactions and protocols. The normal form of
a reaction 
$\nf(L, R, O)$
consists of a commutative list $L$ of bind-read reactions,
a bind-free reaction $R$ 
and a chosen order $O$ of the names of the variables occuring in
the binds in $L$.
The latter will be used to determine how to turn the normal form 
of a reaction into a regular reaction. 
During equivalence proofs, we may obtain either arbitrary binds in
$L$ or reactions $R$ that are not bind-free.
This will be represented as a pre-normal-form, 
written $\preNf(L, R, O)$, which is a normal
form without restrictions on the occuring reactions. The general
strategy will be to transform pre-normal-forms to normal 
forms by rule applications.
The normal form of a 
protocol 
$\newNf(L, P, O)$
consists of a commutative list $L$ of declarations of internal
channels, a protocol $P$ that does not start with internal channel declarations
and again a designated order $O$ 
for the names of internal channels occuring in the declarations in $L$.
Since in both cases the lists are commutative, we can write
equivalence rules for normal forms where we assume that an
internal channel declaration or a bind-read reaction are first in the 
lists.

\bibliography{bib} 
\bibliographystyle{ieeetr}

\end{document}