\subsection{Normal Forms}
We work with protocols that start with a list of declarations of 
internal channels, using $\mathsf{new}$, followed by a parallel compositions of channel assignments. The reactions in these
assignments can be transformed into a list of binds of the form
$x : \tau \leftarrow read(c)$, called
bind-read reactions, followed by a reaction without binds.
The list of binds can be regarded as commutative, 
as two reactions with the same list of binds in different order
are equivalent due to the reaction equivalence rule \textsc{exch}.
Similarly, different order of declarations of internal channels gives
equivalent protocols, by using the protocol equivalence rule
\textsc{new-exch}. When writing equivalence proofs, we do not want to 
make the use of these rules explicit. Therefore, we introduce
normal forms of reactions and protocols. The normal form of
a reaction 
$\nf(L, R, O)$
consists of a commutative list $L$ of bind-read reactions,
a bind-free reaction $R$ 
and a chosen order $O$ of the names of the variables occuring in
the binds in $L$.
The latter will be used to determine how to turn the normal form 
of a reaction into a regular reaction. 
During equivalence proofs, we may obtain either arbitrary binds in
$L$ or reactions $R$ that are not bind-free.
This will be represented as a pre-normal-form, 
written $\preNf(L, R, O)$, which is a normal
form without restrictions on the occuring reactions. The general
strategy will be to transform pre-normal-forms to normal 
forms by rule applications.
The normal form of a 
protocol 
$\newNf(L, P, O)$
consists of a commutative list $L$ of declarations of internal
channels, a protocol $P$ that does not start with internal channel declarations
and again a designated order $O$ 
for the names of internal channels occuring in the declarations in $L$.
Since in both cases the lists are commutative, we can write
equivalence rules for normal forms where we assume that an
internal channel declaration or a bind-read reaction are first in the 
lists.