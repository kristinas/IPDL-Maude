\documentclass[11pt,hidelinks]{article}

\usepackage{amsmath, amssymb, amsthm, stmaryrd}
\usepackage{url}
\usepackage{tikz}
\usepackage[override]{cmtt}
\usepackage{textcomp}
\usepackage{cryptocode}
\usepackage{mathtools}
\usepackage{listings}
\usepackage{xcolor}
\usepackage{graphics}
\usepackage{float}
\usepackage{xspace}
\usepackage{graphicx}
\usepackage{mathpartir}
\usepackage[margin=0.45in]{geometry}
\usepackage[colorlinks]{hyperref}
\usepackage{pl-syntax}
\usepackage{mathabx}
\usepackage{relsize}
\hypersetup{linkcolor=blue,filecolor=blue,citecolor=blue,urlcolor=blue}

\makeatletter
\newcommand*{\downmapsto}{\mathrel{\mathpalette\down@mapsto\relax}}
\newcommand*{\down@mapsto}[2]{%
   \vcenter{%
      \m@th
      \setbox\z@=\hbox{$#1\dabar@$}%
      \ialign{\hfil$#1##$\hfil\cr\noalign{\kern-.45\ht\z@}
         \copy\z@\cr\noalign{\nointerlineskip\kern-.55\ht\z@}
         \downarrow\cr
      }%
   }%
}
\makeatother

\newcommand{\nat}{\mathbb{N}}
\newcommand{\rat}{\mathbb{Q}}
\newcommand{\ipdl}{\textsf{IPDL} }
\newcommand{\type}{\mathsf{t}}
\newcommand{\func}{\mathsf{f}}
\newcommand{\dist}{\mathsf{d}}
\newcommand{\one}{\mathsf{1}}
\newcommand{\Bool}{\mathsf{Bool}}
\newcommand{\true}{\mathsf{true}}
\newcommand{\false}{\mathsf{false}}
\newcommand{\fst}{\mathsf{fst}}
\newcommand{\snd}{\mathsf{snd}}
\newcommand{\Var}[2]{\mathsf{var}(#1 : #2)}
\newcommand{\App}[4]{\mathsf{app}_{#2 \to #3} \ #1 \ #4}
\newcommand{\ret}[1]{\mathsf{ret} \ #1}
\newcommand{\Samp}[4]{\mathsf{samp}_{#2 \twoheadrightarrow #3} \ #1 \ #4}
\newcommand{\samp}[2]{\mathsf{samp} \ #1 \ #2}
\newcommand{\Read}[2]{\mathsf{read}(#1 : #2)}
\renewcommand{\read}[1]{\mathsf{read} \ #1}
\newcommand{\ifte}[3]{\mathsf{if} \ #1 \ \mathsf{then} \ #2 \ \mathsf{else} \ #3}
\newcommand{\zero}{\mathsf{0}}
\newcommand{\assign}[2]{#1 \coloneqq #2}
\newcommand{\Par}[2]{#1 \; \| \; #2}
\newcommand{\new}[3]{\mathsf{new} \ #1 : #2 \ \mathsf{in} \ #3}
\newcommand{\axiom}{\mathsf{axiom}}
\newcommand{\approxcong}[5]{#1 \cong #2 : #3 \to #4 \ \mathsf{len} \ #5}
\renewcommand{\approxeq}[6]{#1 \approx #2 : #3 \to #4 \ \mathsf{wid} \ #5 \ \mathsf{len} \ #6}
\renewcommand{\int}[1]{\llbracket #1 \rrbracket}
\newcommand{\val}[1]{\mathsf{val} \ #1}
\newcommand{\outstep}[2]{\xmapsto{#1 \, \coloneqq \, #2}}
\newcommand{\stuck}{\mathsf{stuck}}
\newcommand{\final}{\mathsf{final}}
\newcommand{\snorm}[1]{\Vert #1 \Vert_{\mathsf{str}}}
\newcommand{\eval}[1]{{#1 \hspace{-2.5pt} \Downarrow}}
\newcommand{\valueof}[1]{#1 \, |_{\mathsf{val}}}
\newcommand{\lift}[1]{\mathcal{L}(#1)}
\newcommand{\valueat}[2]{#1|_{\mathsf{val}(#2)}}
\newcommand{\localvalueat}[2]{#1|^{\mathsf{react}}_{\mathsf{val}(#2)}}
\newcommand{\pull}[2]{\xhookrightarrow{#1 \, \coloneqq \, #2}}
\newcommand{\localassign}[2]{\downarrow_{#1 \, \coloneqq \, #2}}
\renewcommand{\max}{\mathsf{max}}
\newcommand{\size}[1]{\mathsf{size}(#1)}
\newcommand{\tmnorm}[1]{\Vert #1 \Vert_{\scriptscriptstyle \mathsf{TM}}}
\newcommand{\TM}{\mathsf{TM}}
\newcommand{\Adv}{\mathsf{Adv}}
\newcommand{\round}{\mathsf{round}}
\newcommand{\tape}{\mathsf{tape}}
\newcommand{\var}{\mathsf{var}}
\newcommand{\chan}{\mathsf{chan}}
\newcommand{\St}{\mathsf{St}}
\newcommand{\Symb}{\mathsf{Symb}}
\newcommand{\Step}{\mathsf{T}}
\newcommand{\In}{\mathsf{I}}
\newcommand{\Out}{\mathsf{O}}
\newcommand{\Dec}{\mathsf{D}}
\newcommand{\err}{\mathsf{err}}
\newcommand{\sem}{\mathsf{sem}}
\newcommand{\adv}{\mathsf{adv}}
\newcommand{\width}{\mathsf{wid}}
\newcommand{\length}{\mathsf{len}}
\newcommand{\poly}{\mathsf{poly}}
\newcommand{\lapproxcong}[6]{#1 \cong_{#6} #2 : #3 \to #4 \ \mathsf{len} \ #5}
\newcommand{\lapproxeq}[7]{#1 \approx_{#7} #2 : #3 \to #4 \ \mathsf{wid} \ #5 \ \mathsf{len} \ #6}
\newcommand{\interaction}[3]{#1 \xrightleftharpoons{#3} #2}
\newcommand{\bnorm}[1]{\Vert #1 \Vert_{\mathsf{var}}}
\newcommand{\nnorm}[1]{\Vert #1 \Vert_{\mathsf{chan}}}
\newcommand{\qnorm}[1]{\Vert #1 \Vert_{\mathsf{query}}}
\newcommand{\tmenc}[1]{\mathsf{Enc}_{\scriptscriptstyle \mathsf{TM}}[#1]}
\newcommand{\QAnnRead}[3]{\mathsf{read}[a](#1 : #2)}
\newcommand{\query}{\;\mathsf{query}\;}
\newcommand{\QAnn}[2]{\mathsf{QueryAnn}_{#2}[#1]}
\newcommand{\MinQIn}[2]{\mathsf{MinQueryIn}_{#2}[#1]}

\newtheorem{lemma}{Lemma}
\newtheorem{theorem}{Theorem}						
\newtheorem{definition}{Definition}	
\newtheorem{corollary}{Corollary}	
\newtheorem{example}{Example}
\newtheorem{claim}{Claim}

% maude macros

\newcommand{\code}[1]{\texttt{#1}}
\lstset{
  basicstyle=\ttfamily,
  columns=fullflexible,
  literate={~} {$\sim$}{1}
}

\begin{document}
\title{A Core Calculus for Equational Proofs of Distributed Cryptographic Protocols: Technical Report}
\author{Kristina Sojakova \and Mihai Codescu \and Joshua Gancher}

\maketitle

\section*{\small Acknowledgement}
This project was funded through the NGI Assure Fund, a fund established by NLnet with financial support from the European Commission's Next Generation Internet programme, under the aegis of DG Communications Networks, Content and Technology under grant agreement No. 957073.

\section{Syntax of IPDL}
\newcommand{\Sim}{\mathsf{Sim}}

\ipdl is built from four layers: \emph{protocols} are networks of
mutually interacting \emph{reactions}, which are simple monadic programs. Each reaction computes an \emph{expression} probabilistically: \emph{i.e.}, the computation may include sampling from \emph{distributions}. In the context of a protocol, a reaction operates on a unique \emph{channel} and may read from other channels, thereby utilizing computations coming from other reactions. The syntax and judgements of \ipdl are outlined in Figures \ref{fig:syntax}, \ref{fig:judgements}, respectively, and are parameterized by a user-defined \emph{signature} $\Sigma$:

\begin{definition}[Signature]
An \ipdl signature $\Sigma$ is a finite collection of:
\begin{itemize}
\item type symbols $\type$;
\item typed function symbols $\func : \tau \rightarrow \sigma$; and
\item typed distribution symbols $\dist : \tau \twoheadrightarrow \sigma$.
\end{itemize}
\end{definition}

We have a minimal set of data types, including the unit type $\one$, Booleans, products, as well as arbitrary type symbols $\mathsf{t}$, drawn from the signature $\Sigma$. Expressions are used for non-probabilistic computations, and are standard. All values in \ipdl are bitstrings of a length given by data types, so we annotate the operations $\fst_{\tau \times \sigma}$ and $\snd_{\tau \times \sigma}$ with the type of the pair to determine the index to split the pair into two; for readability we omit this subscript whenever appropriate. Function symbols $\func$ must be declared in the signature $\Sigma$, and for a constant $\func : \one \rightarrow \tau$, we write $\func$ in place of $\func \ \checkmark$. Substitutions $\theta : \Gamma_1 \to \Gamma_2$ between type contexts are standard.

Analogously to function symbols, distribution symbols $\dist$ must be declared in the signature $\Sigma$, and for a constant $\dist : \one \twoheadrightarrow \tau$, we write $\samp{\dist}$ instead of $\samp{(\dist \ \checkmark)}$. As mentioned above, reactions are monadic programs which may return expressions, sample from distributions, read from channels, branch on a value of type $\Bool$, and sequentially compose. For readability, we often omit the type of the bound variable in a sequential composition, and write $x \leftarrow \read{c}; \ R$ and $x \leftarrow \samp{d}; \ R$ simply as $x \leftarrow c; \ R$ and $x \leftarrow d; \ R$ wherever appropriate. Protocols in \ipdl are given by a simple but expressive syntax: channel assignment $\assign{o}{R}$ assigns the reaction $R$ to channel $o$; parallel composition $\Par{P}{Q}$ allows $P$ and $Q$ to freely interact concurrently; and channel generation $\new{o}{\tau}{P}$ creates a new, internal channel for use in $P$. \emph{Embeddings} $\phi : \Delta_1 \to \Delta_2$ between channel contexts are injective, type-preserving mappings specifying how to rename channels in $\Delta_2$ to fit in the larger context $\Delta_1$.

\begin{figure}[ht]
\begin{syntax}
  \category[Data Types]{\tau, \sigma}
    \alternative{\type}
		\alternative{\one}
    \alternative{\Bool}
    \alternative{\tau \times \tau}

  \category[Expressions]{e}
    \alternative{x}
    \alternative{\checkmark}
	  \alternative{\true}
	  \alternative{\false}		
	  \alternative{\func \ e} 
	  \alternative{(e_1,e_2)}    
	  \alternative{\fst_{\tau \times \sigma} \ e}
		\alternative{\snd_{\tau \times \sigma} \ e}		

  \category[Distributions]{d}
		\alternative{\dist \ e} 
	
  \abstractCategory[Channels]{i, o, c}

  \category[Reactions]{R, S}
    \alternative{\ret{e}}
    \alternative{\samp{d}}
    \alternative{\read{c}}		
    \alternative{\ifte{e}{R_1}{R_2}}
    \alternative{x : \sigma \leftarrow R; \ S}         

	\category[Protocols]{P, Q}
	  \alternative{\zero}	
	  \alternative{\assign{o}{R}}
	  \alternative{\Par{P}{Q}}
	  \alternative{\new{o}{\tau}{P}}		

	\category[Channel Sets]{I, O}
	  \alternative{\{c_1, \ldots, c_n\}}
		
  \category[Type Contexts]{\Gamma}
    \alternative{\cdot}
    \alternative{\Gamma, x : \tau}

  \category[Channel Contexts]{\Delta}
    \alternative{\cdot}
    \alternative{\Delta, c : \tau}
\end{syntax}
\caption{Syntax of \textsf{\ipdl}.}
\label{fig:syntax}
\end{figure}

\begin{figure}[ht]
\begin{syntax}
  \abstractCategory[Expression Typing]{\Gamma \vdash e : \tau}
	\abstractCategory[Distribution Typing]{\Gamma \vdash d : \tau}
  \abstractCategory[Reaction Typing]{\Delta; \ \Gamma \vdash R : I \to \tau}
  \abstractCategory[Protocol Typing]{\Delta \vdash P : I \to O} \\

  \abstractCategory[Substitutions]{\theta : \Gamma_1 \to \Gamma_2}
  \abstractCategory[Embeddings]{\phi : \Delta_1 \to \Delta_2} \\

	\abstractCategory[Expression Equality]{\Gamma \vdash e_1 = e_2 : \tau}
	\abstractCategory[Distribution Equality]{\Gamma \vdash d_1 = d_2 : \tau}
  \abstractCategory[Reaction Equality]{\Delta; \ \Gamma \vdash R_1 = R_2 : I \to \tau}
  \abstractCategory[Protocol Equality (Strict)]{\Delta \vdash P_1 = P_2 : I \to O}
\end{syntax}
\caption{Judgements of the exact fragment of \textsf{ipdl}.}
\label{fig:judgements}
\end{figure}

\subsection{Typing}
We restrict our attention to well-typed \ipdl constructs. In addition to respecting data types, the typing judgments guarantee that all reads from channels in reactions are in scope, and that all channels are assigned at most one reaction in protocols. The typing $\Gamma \vdash e : \tau$ and $\Gamma \vdash d : \tau$ for expressions and distributions is standard, see Figures \ref{fig:expressions_typing} and \ref{fig:distributions_typing}. Figure \ref{fig:reactions_typing} shows the typing rules for reactions. Intuitively, $\Delta; \ \Gamma \vdash R : I \to \tau$ holds when $R$ uses variables in $\Gamma$, reads from channels in $I$ typed according to $\Delta$, and returns a value of type $\tau$. Figure \ref{fig:protocols_typing} gives the typing rules for protocols: $\Delta \vdash P : I \to O$ holds when $P$ uses inputs in $I$ to assign reactions to the channels in $O$, all typed according to $\Delta$.

Channel assignment $\assign{o}{R}$ has the type $I \to \{o\}$ 
when $R$ is well-typed with an empty variable context, making use of inputs from $I$ as well as of $o$. We allow $R$ to read from its own output $o$ to express divergence: the protocol $\assign{o}{\read{o}}$ cannot reduce, which is useful for (conditionally) deactivating certain outputs. The typing rule for parallel composition $\Par{P}{Q}$ states that $P$ may use the outputs of $Q$ as inputs while defining its own outputs, and vice versa. Importantly, the typing rules ensure that the outputs of $P$ and $Q$ are disjoint so that each channel carries a unique reaction. Finally, the rule for channel generation allows a protocol to select a fresh channel name $o$, assign it a type $\tau$, and use it for internal computation and communication. Protocol typing plays a crucial role for modeling security. 
Simulation-based security in \ipdl is modeled by the existence of a \emph{simulator}  with an appropriate typing judgment, $\Delta \vdash \Sim : I \to O$. Restricting the behavior of $\Sim$ to only use inputs along $I$ is necessary to rule out trivial results (\emph{e.g.}, $\Sim$ simply copies a secret from the specification).

\begin{figure}[ht]
\begin{mathpar}
\fbox{$\Gamma \vdash e : \tau$}\\
\inferrule*{x : \tau \in \Gamma}{\Gamma \vdash x : \tau}\and
\inferrule*{ }{\Gamma \vdash \checkmark : \one}\and
\inferrule*{ }{\Gamma \vdash \true : \Bool}\and
\inferrule*{ }{\Gamma \vdash \false : \Bool}\and
\inferrule*{\func : \sigma \rightarrow \tau \in \Sigma \\ \Gamma \vdash e : \sigma}{\Gamma \vdash \func \ e : \tau}\and
\inferrule*{\Gamma \vdash e_1 : \tau_1 \\ \Gamma \vdash e_2 : \tau_2}{\Gamma \vdash (e_1,e_2) : \tau_1 \times \tau_2}\and
\inferrule*{\Gamma \vdash e : \sigma \times \tau}{\Gamma \vdash \fst_{\sigma \times \tau} \ e : \sigma}\and
\inferrule*{\Gamma \vdash e : \sigma \times \tau}{\Gamma \vdash \snd_{\sigma \times \tau} \ e : \tau}
\end{mathpar}
\caption{Typing for \ipdl expressions.}
\label{fig:expressions_typing}
\end{figure}

\begin{figure}[ht]
\begin{mathpar}
\fbox{$\Gamma \vdash d : \tau$}\\
\inferrule*{\dist : \sigma \twoheadrightarrow \tau \in \Sigma \\ \Gamma \vdash e : \sigma}{\Gamma \vdash \dist \ e : \tau}
\end{mathpar}
\caption{Typing for \ipdl distributions.}
\label{fig:distributions_typing}
\end{figure}

\begin{figure*}[ht]
\begin{mathpar}
\fbox{$\Delta; \ \Gamma \vdash R : I \to \tau$}\\
\inferrule*{\Gamma \vdash e : \tau}{\Delta; \ \Gamma \vdash \ret{e} : I \to \tau}\and
\inferrule*{\Gamma \vdash d : \tau}{\Delta; \ \Gamma \vdash \samp{d} : I \to \tau}\and
\inferrule*{i : \tau \in \Delta \\ i \in I}{\Delta; \ \Gamma \vdash \read{i} : I \to \tau}\and
\inferrule*{\Gamma \vdash e : \Bool \\ \Delta; \ \Gamma \vdash R_1 : I \to \tau \\ \Delta; \ \Gamma \vdash R_2 : I \to \tau}{\Delta; \
\Gamma \vdash \ifte{e}{R_1}{R_2} : I \to \tau}\and
\inferrule*{\Delta; \ \Gamma \vdash R : I \to \sigma \\ \Delta; \ \Gamma, x : \sigma \vdash S : I \to \tau}{\Delta; \ \Gamma \vdash (x : \sigma \leftarrow R; \ S) : I \to \tau}
\end{mathpar}
\caption{Typing for \ipdl reactions.}
\label{fig:reactions_typing}
\end{figure*}

\begin{figure*}[ht]
\begin{mathpar}
\fbox{$\Delta \vdash P : I \to O$}\\
\inferrule*{ }{\Delta \vdash \zero : I \to \emptyset}\and
\inferrule*{o : \tau \in \Delta \\ o \notin I \\ \Delta; \ \cdot \vdash R : I \cup \{o\} \to \tau}{\Delta \vdash (\assign{o}{R}) : I \to \{o\}}\and
\inferrule*{\Delta \vdash P : I \cup O_2 \to O_1 \\ \Delta \vdash    Q : I \cup O_1 \to O_2}{\Delta \vdash \Par{P}{Q} : I \to O_1 \cup O_2}\and
\inferrule*{\Delta, o : \tau \vdash P : I \to O \cup \{o\}}{\Delta \vdash (\new{o}{\tau}{P}) : I \to O}
\end{mathpar}
\caption{Typing for \ipdl protocols.}
\label{fig:protocols_typing}
\end{figure*}

\subsection{Equational Logic}
We now present the equational logic of \ipdl. As mentioned above, the logic is divided into \emph{exact} rules that establish semantic equivalences between protocols, and \emph{approximate} rules that are used to discharge computational indistinguishability assumptions. 

\subsubsection{Exact Equality}
The bulk of the reasoning in \ipdl is done using exact equalities. At the expression level, we assume an ambient finite set of axioms of the form $\Gamma \vdash e_1 = e_2 : \tau$, where $\Gamma \vdash e_1 : \tau$ and $\Gamma \vdash e_2 : \tau$. The rules for expression and distribution equality are standard, see Figures \ref{fig:expressions_equality} and \ref{fig:distributions_equality}.

At the reaction level, we analogously assume an ambient finite set of axioms of the form $\Delta; \ \Gamma \vdash R_1 = R_2 : I \to \tau$, where $\Delta; \ \Gamma \vdash R_1 : I \to \tau$ and $\Delta; \ \Gamma \vdash R_2 : I \to \tau$. The rules for reaction equality, shown in Figures \ref{fig:reactions_equality_1} and \ref{fig:reactions_equality_2}, ensure in particular that reactions form a \emph{commutative monad}: we have \[\big(x \leftarrow R_1; \ y \leftarrow R_2; \ S(x,y)\big) = \big(y \leftarrow R_2; \ x \leftarrow R_1; \ S(x,y)\big)\] whenever $R_2$ does not depend on $x$. All expected equivalences for commutative monads hold for reactions, including the usual monad laws and congruence of equivalence under monadic bind. The \textsc{samp-pure} rule allows us to drop an unused sampling, and the \textsc{read-det} rule allows us to replace two reads from the same channel by a single one. The rules \textsc{if-left}, \textsc{if-right}, and \textsc{if-ext} allow us to manipulate conditionals.

At the protocol level, we similarly assume an ambient finite set of axioms of the form $\Delta \vdash P_1 = P_2 : I \to O$, where $\Delta \vdash P_1 : I \to O$ and $\Delta \vdash P_2 : I \to O$. We use these axioms to specify user-defined functional assumptions, \emph{e.g.}, the correctness of decryption. Exact protocol equivalences allow us to reason about communication between subprotocols and functional correctness, and to simplify intermediate computations. We will see later that exact equivalence implies the existence of a \emph{bisimulation} on protocols, which in turn implies perfect computational indistinguishability against an arbitrary distinguisher. The rules for the exact equality of protocols are in Figures~\ref{fig:protocols_equality_strict_1}, \ref{fig:protocols_equality_strict_2}; we now describe them informally.

The \textsc{comp-new} rule allows us to permute parallel composition and the creation of a new channel, and the same as \emph{scope
extrusion} in process calculi~\cite{picalc}. The \textsc{absorb-left} 
rule allows us to discard a component in a parallel composition if it has no outputs; this allows us to eliminate internal channels once they are no longer used. The \textsc{diverge} rule allows us to simplify diverging reactions: if a channel reads from itself and continues as an arbitrary reaction $R$, then we can safely discard $R$ as we will never reach it in the first place. The three (un)folding rules \textsc{fold-if-left}, \textsc{fold-if-right}, and \textsc{fold-bind} allow us to simplify composite reactions by bringing their
components into the protocol level as separate internal channels. The rule \textsc{subsume} states that channel dependency is transitive: if we depend on $o_1$, and $o_1$ in turn depends on $o_0$, then we also depend on $o_0$, and this dependency can be made explicit. The \textsc{subst} rule allows us to inline certain reactions into $\mathsf{read}$ commands. Inlining $\assign{o_1}{R_1}$ into $\assign{o_2}{x \leftarrow \read{o_1}; \ R_2}$ is sound provided $R_1$ is \emph{duplicable}: observing two independent results of evaluating $R_1$ is equivalent to observing the same result twice. This side condition is easily discharged whenever $R_1$ does not contain probabilistic
sampling. Finally, the \textsc{drop} rule allows dropping unused reads from channels in certain situations. Due to timing dependencies among channels, we only allow dropping reads from the channel $\assign{o_1}{R_1}$ in the context of $\assign{o_2}{\_ \leftarrow \read{o_1};\ R_2}$ when we have that $(\_ \leftarrow R_1; \ R_2) = R_2$. This side condition is met whenever all reads present in $R_1$ are also present in $R_2$.

\begin{figure*}[ht]
\begin{mathpar}
\fbox{$\Gamma \vdash e_1 = e_2 : \tau$}\\
\inferrule*[right=refl]{\Gamma \vdash e : \tau}{\Gamma \vdash e = e : \tau}\and
\inferrule*[right=sym]{\Gamma \vdash e_1 = e_2 : \tau}{\Gamma \vdash e_2 = e_1 : \tau}\and
\inferrule*[right=trans]{\Gamma \vdash e_1 = e_2 : \tau \\ \Gamma \vdash e_2 = e_3 : \tau}{\Gamma \vdash e_1 = e_3 : \tau}\and
\inferrule*[right=axiom]{\Gamma \vdash e_1 = e_2 : \tau \ \axiom}{\Gamma \vdash e_1 = e_2 : \tau}\and
\inferrule*[right=subst]{\theta : \Gamma_1 \to \Gamma_2 \\ \Gamma_2 \vdash e_1 = e_2 : \tau}{\Gamma_1 \vdash \theta^\star(e_1) = \theta^\star(e_2) : \tau}\and
\inferrule*[right=app-cong]{\func : \sigma \rightarrow \tau \in \Sigma \\ \Gamma \vdash e = e' : \sigma}{\Gamma \vdash \func \ e = \func \ e' : \tau}\and
\inferrule*[right=pair-cong]{\Gamma \vdash e_1 = e_1' : \tau_1 \\ \Gamma \vdash e_2 = e_2' : \tau_2}{\Gamma \vdash (e_1,e_2) = (e_1',e_2') : \tau_1 \times \tau_2}\and
\inferrule*[right=fst-cong]{\Gamma \vdash e = e' : \sigma \times \tau}{\Gamma \vdash \fst_{\sigma \times \tau} \ e = \fst_{\sigma \times \tau} \ e' : \sigma}\and
\inferrule*[right=snd-cong]{\Gamma \vdash e = e' : \sigma \times \tau}{\Gamma \vdash \snd_{\sigma \times \tau} \ e = \snd_{\sigma \times \tau} \ e' : \tau}\and
\inferrule*[right=fst-pair]{\Gamma \vdash e_1 : \tau_1 \\ \Gamma \vdash e_2 : \tau_2}{\Gamma \vdash \fst_{\tau_1 \times \tau_2} \ (e_1, e_2) = e_1 : \tau_1}\and
\inferrule*[right=snd-pair]{\Gamma \vdash e_1 : \tau_1 \\ \Gamma \vdash e_2 : \tau_2}{\Gamma \vdash \snd_{\tau_1 \times \tau_2} \ (e_1, e_2) = e_2 : \tau_2}\and
\inferrule*[right=pair-ext]{\Gamma \vdash e : \sigma \times \tau}{\Gamma \vdash e = \big(\fst_{\sigma \times \tau} \ e, \ \snd_{\sigma \times \tau} \ e\big) : \sigma \times \tau}\and
\inferrule*[right=one-ext]{\Gamma \vdash e : \one}{\Gamma \vdash e = \checkmark : \one}
\end{mathpar}
\caption{Equality for \ipdl expressions.}
\label{fig:expressions_equality}
\end{figure*}

\begin{figure*}[ht]
\begin{mathpar}
\fbox{$\Gamma \vdash d_1 = d_2 : \tau$}\\
\inferrule*[right=app-cong]{\dist : \sigma \twoheadrightarrow \tau \in \Sigma \\ \Gamma \vdash d = d' : \sigma}{\Gamma \vdash \dist \ e = \dist \ e' : \tau}
\end{mathpar}
\caption{Equality for \ipdl distributions.}
\label{fig:distributions_equality}
\end{figure*}

\begin{figure*}[ht]
\begin{mathpar}
\fbox{$\Delta; \ \Gamma \vdash R_1 = R_2 : I \to \tau$}\\
\inferrule*[right=refl]{\Delta; \ \Gamma \vdash R : I \to \tau}{\Delta; \ \Gamma \vdash R = R : I \to \tau}\and
\inferrule*[right=sym]{\Delta; \ \Gamma \vdash R_1 = R_2 : I \to \tau}{\Delta; \ \Gamma \vdash R_2 = R_1 : I \to \tau}\and
\inferrule*[right=trans]{\Delta; \ \Gamma \vdash R_1 = R_2 : I \to \tau \\ \Delta; \ \Gamma \vdash R_2 = R_3 : I \to \tau}{\Delta; \ \Gamma \vdash R_1 = R_3 : I \to \tau}\and
\inferrule*[right=axiom]{\Delta; \ \Gamma \vdash R_1 = R_2 : I \to \tau \ \axiom}{\Delta; \ \Gamma \vdash R_1 = R_2 : I \to \tau}\and
\inferrule*[right=input-unused]{i \notin I \\ \Delta; \ \Gamma \vdash R_1 = R_2 : I \to \tau}{\Delta; \ \Gamma \vdash R_1 = R_2 : I \cup \{i\} \to \tau}\and
\inferrule*[right=subst]{\theta : \Gamma_1 \to \Gamma_2 \\ \Delta; \ \Gamma_2 \vdash R_1 = R_2 : I \to \tau}{\Delta; \ \Gamma_1 \vdash \theta^\star(R_1) = \theta^\star(R_2) : I \to \tau}\and
\inferrule*[right=embed]{\phi : \Delta_1 \to \Delta_2 \\ \Delta_2; \ \Gamma \vdash R_1 = R_2 : I \to \tau}{\Delta_1; \ \Gamma \vdash \phi^\star(R_1) = \phi^\star(R_2) : \phi^\star(I) \to \tau}\and
\inferrule*[right=cong-ret]{\Gamma \vdash e = e' : \tau}{\Delta; \ \Gamma \vdash \ret{e} = \ret{e'} : I \to \tau}\and
\inferrule*[right=cong-samp]{\Gamma \vdash d = d' : \sigma}{\Delta; \ \Gamma \vdash \samp{d} = \samp{d'} : I \to \tau}\and
\inferrule*[right=cong-if]{\Gamma \vdash e = e' : \Bool \\ \Delta; \ \Gamma \vdash R_1 = R_1' : I \to \tau \\ \Delta; \ \Gamma \vdash R_2 = R_2' : I \to \tau}{\Delta; \ \Gamma \vdash \big(\ifte{e}{R_1}{R_2}\big) = \big(\ifte{e'}{R_1'}{R_2'}\big) : I \to \tau}\and
\inferrule*[right=cong-bind]{\Delta; \ \Gamma \vdash R = R' : I \to \sigma \\ \Delta; \ \Gamma, x : \sigma \vdash S = S' : I \to \tau}{\Delta; \ \Gamma \vdash (x : \sigma \leftarrow R; \ S) = (x : \sigma \leftarrow R'; \ S') : I \to \tau}
\end{mathpar}
\caption{Equality for \ipdl reactions. Additional rules are given in
Figure~\ref{fig:reactions_equality_2}.}
\label{fig:reactions_equality_1}
\end{figure*}

\begin{figure*}[ht]
\begin{mathpar}
\fbox{$\Delta; \ \Gamma \vdash R_1 = R_2 : I \to \tau$}\\
\inferrule*[right=ret-bind]{\Gamma \vdash e : \sigma \\ \Delta; \ \Gamma, x : \sigma \vdash R : I \to \tau}{\Delta; \ \Gamma \vdash (x : \sigma \leftarrow \ret{e}; \ R) = R[\assign{x}{e}] : I \to \tau}\and
\inferrule*[right=bind-ret]{\Delta; \ \Gamma \vdash R : I \to \tau}{\Delta; \ \Gamma \vdash (x : \tau \leftarrow R; \ \ret{x}) = R : I \to \tau}\and
\inferrule*[right=bind-bind]{\Delta; \ \Gamma \vdash R_1 : I \to \sigma_1 \\ \Delta; \ \Gamma, x_1 : \sigma_1 \vdash R_2 : I \to \sigma_2 \\ \Delta; \ \Gamma, x_2 : \sigma_2 \vdash S : I \to \tau}{\Delta; \ \Gamma \vdash \big(x_2 : \sigma_2 \leftarrow (x_1 : \sigma_1 \leftarrow R_1; \ R_2); \ S\big) = \big(x_1 : \sigma_1 \leftarrow R_1; \ x_2 : \sigma_2 \leftarrow  R_2; \ S\big) : I \to \tau}\and
\inferrule*[right=exch]{\Delta; \ \Gamma \vdash R_1 : I \to \sigma_1 \\ \Delta; \ \Gamma \vdash R_2 : I \to \sigma_2 \\ \Delta; \ \Gamma, x_1 : \sigma_1, x_2 : \sigma_2 \vdash S : I \to \tau}{\Delta; \ \Gamma \vdash \big(x_1 : \sigma_1 \leftarrow R_1; \ x_2 : \sigma_2 \leftarrow R_2; \ S\big) = \big(x_2 : \sigma_2 \leftarrow R_2; \ x_1 : \sigma_1 \leftarrow R_1; \ S\big) : I \to \tau}\and
\inferrule*[right=samp-pure]{\Gamma \vdash d : \sigma \\ \Delta; \ \Gamma \vdash R : I \to \tau}{\Delta; \ \Gamma \vdash (x : \sigma \leftarrow \samp{d}; \ R) = R : I \to \tau}\and
\inferrule*[right=read-det]{i : \sigma \in \Delta \\ i \in I \\ \Delta; \ \Gamma, x : \sigma, y : \sigma \vdash R : I \to \tau}{\Delta; \ \Gamma \vdash \big(x : \sigma \leftarrow \read{i}; \ y : \sigma \leftarrow \read{i}; \ R\big) = \big(x : \sigma \leftarrow \read{i}; \ R[\assign{y}{x}]\big) : I \to \tau}\and
\inferrule*[right=if-left]{\Delta; \ \Gamma \vdash R_1 : I \to \tau \\ \Delta; \ \Gamma \vdash R_2 : I \to \tau}{\Delta; \ \Gamma \vdash \big(\ifte{\true}{R_1}{R_2}\big) = R_1 : I \to \tau}\and
\inferrule*[right=if-right]{\Delta; \ \Gamma \vdash R_1 : I \to \tau \\ \Delta; \ \Gamma \vdash R_2 : I \to \tau}{\Delta; \ \Gamma \vdash \big(\ifte{\false}{R_1}{R_2}\big) = R_2 : I \to \tau}\and
\inferrule*[right=if-ext]{\Delta; \ \Gamma, x : \Bool \vdash R : I \to \tau \\ \Gamma \vdash e : \Bool}{\Delta; \ \Gamma \vdash \big(\ifte{e}{R[\assign{x}{\true}]}{R[\assign{x}{\false}]}\big) = R[\assign{x}{e}] : I \to \tau}
\end{mathpar}
\caption{Equality for \ipdl reactions.}
\label{fig:reactions_equality_2}
\end{figure*}

\begin{figure*}[ht]
\begin{mathpar}
\fbox{$\Delta \vdash P = Q : I \to O$}\\
\inferrule*[right=refl]{\Delta \vdash P : I \to O}{\Delta \vdash P = P : I \to O}\and
\inferrule*[right=sym]{\Delta \vdash P_1 = P_2 : I \to O}{\Delta \vdash P_2 = P_1 : I \to O}\and
\inferrule*[right=trans]{\Delta \vdash P_1 = P_2 : I \to O \\ \Delta \vdash P_2 = P_3 : I \to O}{\Delta \vdash P_1 = P_3 : I \to O}\and
\inferrule*[right=axiom]{\Delta \vdash P = Q : I \to O \ \axiom}{\Delta \vdash P = Q : I \to O}\and
\inferrule*[right=input-unused]{i \notin I \cup O \\ \Delta \vdash P_1 = P_2 : I \to O}{\Delta \vdash P_1 = P_2 : I \cup \{i\} \to O}\and
\inferrule*[right=embed]{\phi : \Delta_1 \to \Delta_2 \\ \Delta_2 \vdash P = Q : I \to O}{\Delta_1 \vdash \phi^\star(P) = \phi^\star(Q) : \phi^\star(I) \to \phi^\star(O)}\and
\inferrule*[right=cong-react]{o : \tau \in \Delta \\ o \notin I \\ \Delta; \ \cdot \vdash R = R' : I \cup \{o\} \to \tau}{\Delta \vdash (\assign{o}{R}) = (\assign{o}{R'}) : I \to \{o\}}\and
\inferrule*[right=cong-comp-left]{\Delta \vdash P = P' : I \cup O_2 \to O_1 \\ \Delta \vdash Q : I \cup O_1 \to O_2}{\Delta \vdash \Par{P}{Q} = \Par{P'}{Q} : I \to O_1 \cup O_2}\and
\inferrule*[right=cong-new]{\Delta, o : \tau \vdash P = P' : I \to O \cup \{o\}}{\Delta \vdash (\new{o}{\tau}{P}) = (\new{o}{\tau}{P'}) : I \to O}\and
\inferrule*[right=comp-comm]{\Delta \vdash P_1 : I \cup O_2 \to O_1 \\ \Delta \vdash P_2 : I \cup O_1 \to O_2}{\Delta \vdash \Par{P_1}{P_2} = \Par{P_2}{P_1} : I \to O_1 \cup O_2}\and
\inferrule*[right=comp-assoc]{\Delta \vdash P_1 : I \cup O_2 \cup O_3 \to O_1 \\ \Delta \vdash P_2 : I \cup O_1 \cup O_3 \to O_2 \\ \Delta \vdash P_3 : I \cup O_1 \cup O_2 \to O_3}{\Delta \vdash \Par{(\Par{P_1}{P_2})}{P_3} = \Par{P_1}{(\Par{P_2}{P_3})} : I \to O_1 \cup O_2 \cup O_3}\and
\inferrule*[right=new-exch]{\Delta, o_1 : \tau_1, o_2 : \tau_2 \vdash P : I \to O \cup \{o_1,o_2\}}{\Delta \vdash \big(\new{o_1}{\tau_1}{\new{o_2}{\tau_2}{P}}\big) = \big(\new{o_2}{\tau_2}{\new{o_1}{\tau_1}{P}}\big) : I \to O}\and
\inferrule*[right=comp-new]{\Delta \vdash P : I \cup O_2 \to O_1 \\ \Delta, o : \tau \vdash Q : I \cup O_1 \to O_2 \cup \{o\}}{\Delta \vdash \Par{P}{(\new{o}{\tau}{Q})} = \new{o}{\tau}{(\Par{P}{Q})} : I \to O_1 \cup O_2}\and
\inferrule*[right=absorb-left]{\Delta \vdash P : I \to O \\ \Delta \vdash Q : I \cup O \to \emptyset}{\Delta \vdash \Par{P}{Q} = P : I \to O}
\end{mathpar}
\caption{Exact equality for \ipdl protocols. Additional rules are given in Figure~\ref{fig:protocols_equality_strict_2}.}
\label{fig:protocols_equality_strict_1}
\end{figure*}

\begin{figure*}[ht]
\begin{mathpar}
\fbox{$\Delta \vdash P = Q : I \to O$}\\
\inferrule*[right=diverge]{o : \tau \in \Delta \\ o \notin I \\ \Delta; \ \cdot \vdash R : I \cup \{o\} \to \tau}{\Delta \vdash (\assign{o}{x : \tau \leftarrow \read{o}; \ R}) = (\assign{o}{\read{o}}) : I \to \{o\}}\and
\inferrule*[right=fold-if-left]{o \notin I \\ b \in I \\ b : \Bool, o : \tau \in \Delta \\ \Delta; \ \cdot \vdash S_1 : I \cup \{o\} \to \tau \\ \Delta; \ \cdot \vdash S_2 : I \cup \{o\} \to \tau}{\Delta \vdash \big(\new{l}{\tau}{\Par{\assign{o}{x : \Bool \leftarrow \read{b}; \ \ifte{x}{{\color{red} \read{l}}}{S_2}}}{{\color{red} \assign{l}{S_1}}}}\big) = \\ \big(\assign{o}{x : \Bool \leftarrow \read{b}; \ \ifte{x}{{\color{red} S_1}}{S_2}}\big) : I \to \{o\}\hspace{-55pt}}\and
\inferrule*[right=fold-if-right]{o \notin I \\ b \in I \\ b : \Bool, o : \tau \in \Delta \\ \Delta; \ \cdot \vdash S_1 : I \cup \{o\} \to \tau \\ \Delta; \ \cdot \vdash S_2 : I \cup \{o\} \to \tau}{\Delta \vdash \big(\new{r}{\tau}{\Par{\assign{o}{x : \Bool \leftarrow \read{b}; \ \ifte{x}{S_1}{{\color{red} \read{r}}}}}{{\color{red} \assign{r}{S_2}}}}\big) = \\ \big(\assign{o}{x : \Bool \leftarrow \read{b}; \ \ifte{x}{S_1}{{\color{red} S_2}}}\big) : I \to \{o\}\hspace{-53pt}}\and
\inferrule*[right=fold-bind]{o \notin I \\ o : \tau_2 \in \Delta \\ \Delta; \ \cdot \vdash R_1 : I \cup \{o\} \to \tau_1 \\ \Delta; \ x : \tau_1 \vdash R_2 : I \cup \{o\} \to \tau_2}{\Delta \vdash \big(\new{c}{\sigma}{\Par{\assign{o}{{\color{red} x : \tau_1 \leftarrow \read{c};} \ R_2}}{{\color{red} \assign{c}{R_1}}}}\big) = (\assign{o}{{\color{red} x : \tau_1 \leftarrow R_1;} \ R_2}) : I \to \{o\}}\and
\inferrule*[right=subst]{o_1 \neq o_2 \\ o_1, o_2 \notin I \\ o_1 : \tau_1, o_2 : \tau_2 \in \Delta \\ \Delta; \ \cdot \vdash R_1 : I \cup \{o_1,o_2\} \to \tau_1 \\ \Delta; \ x_1 : \tau_1 \vdash R_2 : I \cup \{o_1,o_2\} \to \tau_2 \\ \Delta; \ \cdot \vdash \big(x_1 : \tau_1 \leftarrow R_1; \ {\color{red} x'_1 : \tau_1 \leftarrow R_1; \ } \ret{(x_1,{\color{red} x'_1})}\big) = \big(x_1 : \tau_1 \leftarrow R_1; \ \ret{(x_1,{\color{red} x_1})}\big) : I \cup \{o_1,o_2\} \to \tau_1 \times \tau_1}{\Delta \vdash \big(\Par{\assign{o_1}{R_1}}{\assign{o_2}{{\color{red} x_1 : \tau_1 \leftarrow \read{o_1}; \ } R_2}}\big) = \big(\Par{\assign{o_1}{R_1}}{\assign{o_2}{{\color{red} x_1 : \tau_1 \leftarrow R_1; \ } R_2}}\big) : I \to \{o_1, o_2\}}\and
\inferrule*[right=drop]{o_1 \neq o_2 \\ o_1, o_2 \notin I \\ o_1 : \tau_1, o_2 : \tau_2 \in \Delta \\ \Delta; \ \cdot \vdash R_1 : I \cup \{o_1,o_2\} \to \tau_1 \\ \Delta; \ \cdot \vdash R_2 : I \cup \{o_1,o_2\} \to \tau_2 \\ \Delta; \ \cdot \vdash ({\color{red} x_1 : \tau_1 \leftarrow R_1; \ } R_2) = R_2 : I \cup \{o_1,o_2\} \to \tau_2}{\Delta \vdash \big(\Par{\assign{o_1}{R_1}}{\assign{o_2}{{\color{red} x_1 \leftarrow \read{o_1}; \ } R_2}}\big) = (\Par{\assign{o_1}{R_1}}{\assign{o_2}{R_2}}) : I \to \{o_1, o_2\}}
\end{mathpar}
\caption{Additional rules for exact equality of \ipdl protocols. Distinguishing changes of equalities are highlighed in {\color{red} red}.}
\label{fig:protocols_equality_strict_2}
\end{figure*}

\subsubsection{Approximate Equality}
The equational theory for the approximate fragment of \ipdl consists of two layers: one for the \emph{approximate equality} of protocols, and one for the \emph{asymptotic equality} of protocol families as functions of the security parameter $\lambda \in \nat$. The approximate equality judgement $\Delta \vdash \approxeq{P}{Q}{I}{O}{k}{l}$ equates two protocols $\Delta \vdash P : I \to O$ and $\Delta \vdash Q : I \to O$ with identical typing judgements. We think of these as corresponding to a specific security parameter $\lambda$. Analogously to exact protocol equality, we assume an ambient finite set of \emph{approximate axioms} of the form $\Delta \vdash P \approx Q : I \to O$, where $\Delta \vdash P : I \to O$ and $\Delta \vdash Q : I \to O$. These axioms capture cryptographic assumptions on computational indistinguishability.

The parameters $k, l \in \nat$ track the size of the derivation. The \emph{width} parameter $k$ simply counts the number of invocations of axioms applied during the proof: applying a single approximate axiom incurs $k = 1$, and we sum up the two values of $k$ whenever we use transitivity. In the asymptotic equality judgement, $k$ becomes a function of the security parameter $\lambda$ and we require that it be bounded by a polynomial in $\lambda$: even though each individual axiom invocation introduces a negligible error, the sum of exponentially many negligible errors may not be negligible anymore.

Since most nontrivial reasoning in \ipdl is done in the exact half, the approximate equality rules are used mostly to apply indistinguishability assumptions nested deeply inside protocols. The \emph{length} parameter $l$ tracks the largest size of such a nesting -- also known as a \emph{program context}. In the asymptotic equality judgement, we again require that $l$ be bounded as a function of $\lambda$ by a polynomial: exponentially large \ipdl contexts could in principle be used to encode exponential-time probabilistic computations. An \ipdl program context surrounding an indistinguishability assumption is formally a part of the adversary, and as such it must be resource-bounded for the indistinguishability assumption to apply.

In \textsf{IPDL}, the bound on resources is given by a \emph{symbolic size} function $|\cdot|$ defined for expressions, reactions, and protocols. Since we assume that all function symbols will be interpreted by  functions computable in poly-time, our symbolic size for expressions simply counts the number of variables and function applications present.
\begin{align*}
|x| & \coloneqq 1 \\
|\checkmark| & \coloneqq 0 \\
|\true| & \coloneqq 1 \\
|\false| & \coloneqq 1 \\
|\func \ e| & \coloneqq |e| + 1 \\
|(e_1, e_2)| & \coloneqq |e_1| + |e_2| \\
|\fst_{\sigma \times \tau} \ e| & \coloneqq |e| \\
|\snd_{\sigma \times \tau} \ e| & \coloneqq |e|
\end{align*}
For distributions, we follow a similar principle: we assume that all distribution symbols will be interpreted by probabilistic functions approximately computable in poly-time.
\begin{align*}
|\dist \ e| & \coloneqq |e| + 1
\end{align*}
For reactions, we sum up the symbolic sizes of all expressions and distributions occurring inside the reaction, with the exception of the conditional: here we pick the size of the larger branch and add it to the size of the condition.
\begin{align*}
|\ret{e}| & \coloneqq |e| \\
|\samp{d}| & \coloneqq |d| \\
|\read{c}| & \coloneqq 1 \\
|\ifte{e}{R_1}{R_2}| & \coloneqq |e| + \mathsf{max} \, (|R_1|, |R_2|) \\
|x : \sigma \leftarrow R; \ S| & \coloneqq |R| + |S|
\end{align*}
Since protocols in \ipdl are finite networks of channels that do not contain recursion, the size of a protocol is simply the sum of the symbolic sizes of all reactions occurring in the protocol.
\begin{align*}
|\zero| & \coloneqq 0 \\
|\assign{o}{R}| & \coloneqq |R| \\
|\Par{P}{Q}| & \coloneqq |P| + |Q| \\
|\new{o}{\tau}{P}| & \coloneqq |P|
\end{align*}

Figure \ref{fig:protocols_equality_approx} shows the rules for the approximate equality of \ipdl protocols; crucially, rule \textsc{strict} allows us to descend to the exact half of the proof system. Whenever we need to make the ambient theory with approximate axioms $\Delta^1 \vdash P^1 \approx Q^1 : I^1 \to O^1, \ldots, \Delta^n \vdash P^n \approx Q^n : I^n \to O^n$ explicit, we write the approximate equality judgement as $\Delta^1 \vdash P^1 \approx Q^1 : I^1 \to O^1, \ldots, \Delta^n \vdash P^n \approx Q^n : I^n \to O^n \Rightarrow \Delta \vdash \approxeq{P}{Q}{I}{O}{k}{l}$.

\begin{figure*}
\begin{mathpar}
\fbox{$\Delta \vdash \approxeq{P}{Q}{I}{O}{k}{l}$}\\
\inferrule*[right=strict]{\Delta \vdash P = Q : I \to O}{\Delta \vdash \approxeq{P}{Q}{I}{O}{0}{0}} \and
\inferrule*[right=sym]{\Delta \vdash \approxeq{P_1}{P_2}{I}{O}{k}{l}}{\Delta \vdash \approxeq{P_2}{P_1}{I}{O}{k}{l}}\and
\inferrule*[right=trans]{\Delta \vdash \approxeq{P_1}{P_2}{I}{O}{k_1}{l_1} \\ \Delta \vdash \approxeq{P_2}{P_3}{I}{O}{k_2}{l_2}}{\Delta \vdash \approxeq{P_1}{P_3}{I}{O}{k_1 + k_2}{\mathsf{max} \, (l_1,l_2)}}\and
\inferrule*[right=axiom]{\Delta \vdash P \approx Q : I \to O \ \axiom}{\Delta \vdash \approxeq{P}{Q}{I}{O}{1}{0}}\and
\inferrule*[right=input-unused]{i \notin I \cup O \\ \Delta \vdash \approxeq{P}{Q}{I}{O}{k}{l}}{\Delta \vdash \approxeq{P}{Q}{I \cup \{i\}}{O}{k}{l+1}}\and
\inferrule*[right=embed]{\theta : \Delta_1 \to \Delta_2 \\ \Delta_1 \vdash \approxeq{P}{Q}{I}{O}{k}{l}}{\Delta_2 \vdash \approxeq{\theta^\star(P)}{\theta^\star(Q)}{\theta^\star(I)}{\theta^\star(O)}{k}{l}}\and
\inferrule*[right=cong-comp-left]{\Delta \vdash \approxeq{P}{P'}{I \cup O_2}{O_1}{k}{l} \\ \Delta \vdash_\Sigma Q : I \cup O_1 \to O_2}{\Delta \vdash \approxeq{\Par{P}{Q}}{\Par{P'}{Q}}{I}{O_1 \cup O_2}{k}{l + |Q|}} \and
\inferrule*[right=cong-new]{\Delta, o : A \vdash \approxeq{P}{P'}{I}{O \cup \{o\}}{k}{l}}{\Delta \vdash \approxeq{(\new{o}{A}{P})}{(\new{o}{A}{P'})}{I}{O}{k}{l}}
\end{mathpar}
\caption{Approximate equality for \ipdl protocols.}
\label{fig:protocols_equality_approx}
\end{figure*}

For the asymptotic equality of \ipdl protocols, we assume a finite set $\mathbb{T}_\approx$ of \emph{axiom families} of the form $\{\Delta_\lambda \vdash P_\lambda \approx Q_\lambda : I_\lambda \to O_\lambda\}_{\lambda \in \nat}$. In this setting, the asymptotic equivalence of two protocol families $\{\Delta_\lambda \vdash P_\lambda : I_\lambda \to O_\lambda\}_{\lambda \in \nat}$ and $\{\Delta_\lambda \vdash Q_\lambda : I_\lambda \to O_\lambda\}_{\lambda \in \nat}$ with pointwise-identical typing judgements takes the form of the judgement $\mathbb{T}_\approx \Rightarrow \{\Delta_\lambda \vdash P_\lambda \approx Q_\lambda : I_\lambda \to O_\lambda\}_{\lambda \in \nat}$, see Figure \ref{fig:protocols_equivalence_asympto}.
 
Specifically, for any fixed $\lambda$ we obtain an approximate theory by selecting from each axiom family in $\mathbb{T}_\approx$ the axiom corresponding to $\lambda$. Similarly, from each of the two protocol families we select the protocol corresponding to $\lambda$, which gives us two concrete protocols to equate approximately. We recall that an approximate equality judgement is tagged by a pair of parameters $k$ and $l$. Letting $\lambda \in \nat$ vary thus gives us two functions $k_\lambda$ and $l_\lambda$, and we require that these be bounded by a polynomial. We can summarize the asymptotic judgement as saying that the protocol families are pointwise approximately equal, and both the width and length of the derivation, as well as the number of input and output channels are bounded by a polynomial in $\lambda$.

\begin{figure*}
\begin{mathpar}
\fbox{$\{\Delta^1_\lambda \vdash P^1_\lambda \approx Q^1_\lambda : I^1_\lambda \to O^1_\lambda\}_{\lambda \in \nat}, \ldots, \{\Delta^n_\lambda \vdash P^n_\lambda \approx Q^n_\lambda : I^n_\lambda \to O^n_\lambda\}_{\lambda \in \nat} \Rightarrow \{\Delta_\lambda \vdash P_\lambda \approx Q_\lambda : I_\lambda \to O_\lambda\}_{\lambda \in \nat}$}
\\
\inferrule{\forall \lambda, \Delta^1_\lambda \vdash P^1_\lambda \approx
    Q^1_\lambda : I^1_\lambda \to O^1_\lambda, \ldots, \Delta^n_\lambda \vdash
    P^n_\lambda \approx Q^n_\lambda : I^n_\lambda \to O^n_\lambda \Rightarrow
    \Delta_\lambda \vdash
    \approxeq{P_\lambda}{Q_\lambda}{I_\lambda}{O_\lambda}{k_\lambda}{l_\lambda}
    \\ k_\lambda = \mathsf{O}(\mathsf{poly}(\lambda)) \\ l_\lambda = \mathsf{O}(\mathsf{poly}(\lambda)) \\ |I_\lambda| = \mathsf{O}(\mathsf{poly}(\lambda)) \\ |O_\lambda| = \mathsf{O}(\mathsf{poly}(\lambda))}{\{\Delta^1_\lambda \vdash P^1_\lambda \approx Q^1_\lambda : I^1_\lambda \to O^1_\lambda\}_{\lambda \in \nat}, \ldots, \{\Delta^n_\lambda \vdash P^n_\lambda \approx Q^n_\lambda : I^n_\lambda \to O^n_\lambda\}_{\lambda \in \nat} \Rightarrow \{\Delta_\lambda \vdash P_\lambda \approx Q_\lambda : I_\lambda \to O_\lambda\}_{\lambda \in \nat}}
\end{mathpar}
\caption{Asymptotic equivalence for \ipdl protocol families.}
\label{fig:protocols_equivalence_asympto}
\end{figure*}

Whenever we need to make the underlying exact theory $\mathbb{T}$ explicit, we write the asymptotic equality judgement as $\mathbb{T}; \mathbb{T}_\approx \Rightarrow \{\Delta_\lambda \vdash P_\lambda \approx Q_\lambda : I_\lambda \to O_\lambda\}_{\lambda \in \nat}$.

\section{Operational Semantics of IPDL}
In this section we define an operational semantics for \ipdl expressions, reactions, and protocols. This semantics will validate the \emph{exact} fragment of our equational logic and prove perfect observational equivalence. To give semantics to user-defined symbols, we define interpretations:

\begin{definition}[Interpretation]
An interpretation $\int{-}$ for a signature $\Sigma$ associates to:
\begin{itemize}
\item each type symbol $\type$ a subset $\subseteq \{0,1\}^{|\type|}$ of bitstrings of a fixed length $|\type| \geq 0$;
\item each function symbol $\func : \sigma \to \tau$ a function $\int{\func}$ from $\int{\sigma}$ to $\int{\tau}$;
\item each distribution symbol $\dist : \sigma \twoheadrightarrow \tau$ a function $\int{\dist}$ from $\int{\sigma}$ to \emph{distributions} on $\int{\tau}$.
\end{itemize}
\end{definition}
\noindent In the above, we naturally lift the interpretation $\int{-}$ to all types by setting
\begin{align*}
\int{\one} & \coloneqq \{()\} \\
\int{\Bool} & \coloneqq \{0,1\} \\
\int{\tau \times \sigma} & \coloneqq \big\{v_1 v_2 \, | \, v_1 \in \int{\tau}, v_2 \in \int{\sigma}\big\}
\end{align*}
where $()$ denotes the empty bitstring and $v_1 v_2$ stands for the concatenation of bitstrings $v_1$ and $v_2$. We similarly define the length $|\sigma|$ of a type $\sigma$ in the obvious way, letting $|\one| \coloneqq 0$, $|\Bool| \coloneqq 1$, and $|\tau \times \sigma| \coloneqq |\tau| + |\sigma|$.

To handle partial computations, we augment the syntax of \ipdl expressions, reactions, and protocols to contain intermediate bitstring values $v \in \{0,1\}^\star$:\smallskip

\begin{syntax}
\category[Valued Expressions]{e}
\alternative{{\color{red} v}} \alternative{\dots}
\category[Valued Reactions]{R, S}
\alternative{{\color{red} \val{v}}} \alternative{\dots}
\category[Valued Protocols]{P, Q}
\alternative{{\color{red} \assign{o}{v}}} \alternative{\dots}
\end{syntax}\smallskip

\noindent We extend the notion of a Turing Machine bound to valued \ipdl constructs as follows, where $|v|$ denotes the length of the bitstring $v$:
\begin{align*}
\tmnorm{v} & \coloneqq |v| \\
\tmnorm{\val{v}} & \coloneqq |v| + 3 \\
\tmnorm{\assign{o}{v}} & \coloneqq |v| + 4
\end{align*}

\noindent Given an ambient interpretation $\int{-}$ for the signature $\Sigma$, we can type the valued counterpart of \ipdl constructs as expected: in addition to the regular typing rules, we have

\begin{mathpar}
\inferrule*{v \in \int{\tau}}{\Gamma \vdash v : \tau}\and
\inferrule*{v \in \int{\tau}}{\Delta; \ \Gamma \vdash \val{v} : I \to \tau}\and
\inferrule*{o : \tau \in \Delta \\ o \notin I \\ v \in \int{\tau}}{\Delta \vdash \big(\assign{o}{v}\big) : I \to \{o\}}
\end{mathpar}

\subsection{Small-Step Operational Semantics of IPDL}

The small-step semantics $e \to e'$ for expressions is standard, see Figure \ref{fig:expressions_semantics}. Pairing is given by bitstring concatenation (rule \textsc{pair}), and the projections $\fst_{\sigma \times \tau}$ and $\snd_{\sigma \times \tau}$ unambiguously split the pair according to $|\sigma|$ and $|\tau|$, respectively (rules \textsc{fst-eval} and \textsc{snd-eval}).

Reactions have a straightforward small-step semantics of the form $R \to \eta$, where $\eta$ is a probability distribution over reactions. Figure \ref{fig:reactions_semantics} shows the rules, where we write $1[R]$ for the distribution with unit mass at the reaction $R$, and freely use a distribution in place of a value (rule \textsc{samp-eval}) or a reaction (rule \textsc{bind-react}) to indicate the obvious lifting of the corresponding construct to distributions on reactions. All distributions are implicitly finitely supported. Crucially, there is no semantic rule for stepping the reaction $\read{c}$; we model communication via semantics for protocols, which substitute all instances of $\mathsf{read}$ for values.

We give semantics to protocols via two main small-step rules, see Figure \ref{fig:protocols_semantics}, where we analogously write $1[P]$ for the distribution with unit mass at the protocol $P$, and freely use a distribution in place of a reaction (in rule \textsc{step-react}) or a protocol (rules \textsc{step-comp-left}, \textsc{step-comp-right}, and \textsc{step-new}) to indicate the obvious lifting of the corresponding construct to distributions on protocols.

First we have the \emph{output} relation $P \outstep{o}{v} Q$, which is enabled when the reaction for channel $o$ in $P$ terminates, resulting in value $v$ (rule \textsc{out-val}). When this happens, the value of $o$ is broadcast through the protocol context enveloping $P$ (rules \textsc{out-comp-left}, \textsc{out-comp-left}, and \textsc{out-new}), resulting in each $\read{o}$ command in other reactions to be substituted with $\val{v}$. We note that the value of $o$ is not broadcast above the $\mathsf{new}$ quantifier when the local channel introduced is equal to $o$.

Next we have the \emph{internal stepping} relation $P \to \eta$, specified similarly to the small-step relation for reactions. The rule \textsc{step-react} lifts the stepping relation for $R$ to the stepping relation for $\assign{o}{R}$, while the three rules \textsc{step-comp-left}, \textsc{step-comp-right}, \textsc{step-new} simply propagate the stepping relation through parallel composition and the $\mathsf{new}$ quantifier. The last rule \textsc{out-new-hide} links the output relation with the stepping relation: whenever $P$ steps to $P'$, resulting in the output $\assign{o}{v}$, we have that $\new{o}{\tau}{P}$ steps with unit mass to $\new{o}{\tau}{P'}$.

\begin{figure}
\begin{mathpar}
\fbox{$e \to e'$}\\
\inferrule*[right=check]{ }{\checkmark \to ()}\and
\inferrule*[right=true]{ }{\true \to 1}\and
\inferrule*[right=false]{ }{\false \to 0}\and
\inferrule*[right=app-cong]{e \to e'}{\func \ e \to \func \ e'}\and
\inferrule*[right=app-eval]{ }{\func \ v \to \int{\func}(v)}\and
\inferrule*[right=pair-cong-fst]{e_1 \to e'_1}{(e_1,e_2) \to (e'_1,e_2)}\and
\inferrule*[right=pair-cong-snd]{e_2 \to e'_2}{(e_1,e_2) \to (e_1,e'_2)}\and
\inferrule*[right=pair-eval]{ }{(v_1,v_2) \to v_1 v_2}\and
\inferrule*[right=fst-cong]{e \to e'}{\fst_{\tau_1 \times \tau_2} \ e \to \fst_{\tau_1 \times \tau_2} \ e'}\and
\inferrule*[right=fst-eval]{v_1 \in \{0,1\}^{|\tau_1|} \\ v_2 \in \{0,1\}^{|\tau_2|}}{\fst_{\tau_1 \times \tau_2} \ v_1 v_2 \to v_1}\and
\inferrule*[right=snd-cong]{e \to e'}{\snd_{\tau_1 \times \tau_2} \ e \to \snd_{\tau_1 \times \tau_2} \ e'}\and
\inferrule*[right=snd-eval]{v_1 \in \{0,1\}^{|\tau_1|} \\ v_2 \in \{0,1\}^{|\tau_2|}}{\snd_{\tau_1 \times \tau_2} \ v_1 v_2 \to v_2}
\end{mathpar}
\caption{Small-step operational semantics for $\ipdl$ expressions.} 
\label{fig:expressions_semantics}
\end{figure}

\begin{figure}
\begin{mathpar}
\fbox{$R \to \eta$}\\
\inferrule*[right=ret-step]{e \to e'}{\ret{e} \to 1[\ret{e'}]}\and
\inferrule*[right=ret-eval]{ }{\ret{v} \to 1[\val{v}]}\and
\inferrule*[right=samp-step]{e \to e'}{\samp{\dist}{e} \to 1[\samp{\dist}{e'}]}\and
\inferrule*[right=samp-eval]{ }{\samp{\dist}{v} \to \val{\int{\dist}(v)}}\and
\inferrule*[right=if-step]{e \to e'}{\big(\ifte{e}{R_1}{R_2}\big) \to 1[\ifte{e'}{R_1}{R_2}]}\and
\inferrule*[right=if-true]{ }{\big(\ifte{1}{R_1}{R_2}\big) \to 1[R_1]}\and
\inferrule*[right=if-false]{ }{\big(\ifte{0}{R_1}{R_2}\big) \to 1[R_2]}\and
\inferrule*[right=bind-val]{ }{\big(x : \sigma \leftarrow \val{v}; \ S\big) \to 1\big[S[\assign{x}{v}]\big]}\and
\inferrule*[right=bind-react]{R \to \eta}{\big(x : \sigma \leftarrow R; \ S\big) \to {\big(x : \sigma \leftarrow \eta; \ S\big)}}
\end{mathpar}
\caption{Small-step operational semantics for $\ipdl$ reactions.} 
\label{fig:reactions_semantics}
\end{figure}

\begin{figure}
\begin{mathpar}
\fbox{$P \outstep{o}{v} Q$}\\
\inferrule*[right=out-react]{ }{\big(\assign{o}{\val{v}}\big) \outstep{o}{v} \big(\assign{o}{v}\big)}\and
\inferrule*[right=out-comp-left]{P \outstep{o}{v} P'}{\Par{P}{Q} \outstep{o}{v} \Par{P'}{Q[\assign{\read{o}}{\val{v}}]}}\and
\inferrule*[right=out-comp-right]{Q \outstep{o}{v} Q'}{\Par{P}{Q} \outstep{o}{v} \Par{P[\assign{\read{o}}{\val{v}}]}{Q'}}\and
\inferrule*[right=out-new]{P \outstep{o}{v} P' \\ o \neq c}{\big(\new{c}{\tau}{P}\big) \outstep{o}{v} \big(\new{c}{\tau}{P'}\big)}\\\\
\fbox{$P \to \eta$}\\
\inferrule*[right=step-react]{R \to \eta}{\big(\assign{o}{R}\big) \to \big(\assign{o}{\eta}\big)}\and
\inferrule*[right=step-comp-left]{P \to \eta}{\Par{P}{Q} \to \Par{\eta}{Q}}\and
\inferrule*[right=step-comp-right]{Q \to \eta}{\Par{P}{Q} \to \Par{P}{\eta}}\and
\inferrule*[right=step-new]{P \to \eta}{\big(\new{c}{\tau}{P}\big) \to \big(\new{c}{\tau}{\eta}\big)}\and
\inferrule*[right=out-new-hide]{P \outstep{c}{v} P'}{\big(\new{c}{\tau}{P}\big) \to 1[\new{c}{\tau}{P'}]}
\end{mathpar}
\caption{Small-step operational semantics for \ipdl protocols.}
\label{fig:protocols_semantics}
\end{figure}

\subsection{Big-Step Operational Semantics of IPDL}

The big-step semantics for expressions $e \Downarrow v$, see Figure \ref{fig:expressions_big_step}, performs a sequence of small steps to compute $e$ to a value. The big-step semantics for reactions $R \Downarrow \eta$, see Figure \ref{fig:reactions_big_step}, performs as many steps as possible in an attempt to compute $R$, resulting in a distribution $\eta$ on reactions. A reaction that cannot step any further is \emph{final}. We can syntactically describe final reactions as those that have either yielded a value or have an unresolved read in the leading position (\emph{i.e.}, are \emph{stuck}). The big-step operational semantics for protocols $P \Downarrow \eta$, see Figure \ref{fig:protocols_big_step}, performs as many output and internal steps as possible in an attempt to compute $P$, resulting in a distribution $\eta$ on protocols. A protocol that cannot step any further using either of the two stepping relations is \emph{final}. We can syntactically describe final protocols as those where every channel, including the internal ones, carries either a value or a reaction that is stuck.

\begin{figure}
\begin{mathpar}
\fbox{$e \Downarrow v$}\\
\inferrule*{ }{v \Downarrow v}\and
\inferrule*{e \to e' \\ e' \Downarrow v}{e \Downarrow v}
\end{mathpar}
\caption{Big-step operational semantics for \ipdl expressions.}
\label{fig:expressions_big_step}
\end{figure}

\begin{figure}
\begin{mathpar}
\fbox{$R \ \stuck$}\\
\inferrule*{ }{\big(\read{c}\big) \ \stuck}\and
\inferrule*{R \ \stuck}{\big(x : \sigma \leftarrow R; \ S\big) \ \stuck}\\
\fbox{$R \ \final$}\\
\inferrule*{ }{\big(\val{v}\big) \ \final}\and
\inferrule*{R \ \stuck}{R \ \final}\\
\fbox{$R \Downarrow \eta$}\\
\inferrule*{}{ }\and
\inferrule*{R \ \final}{R \Downarrow 1[R]}\and \\
\inferrule*{R \to \sum_i c_i * 1[R_i] \\ R_i \Downarrow \eta_i}{R \Downarrow \sum_i c_i * \eta_i}\and
\end{mathpar}
\caption{Big-step operational semantics for \ipdl reactions.}
\label{fig:reactions_big_step}
\end{figure}

\begin{figure}
\begin{mathpar}
\fbox{$P \ \final$}\\
\inferrule*{ }{\zero \ \final}\and
\inferrule*{ }{\big(\assign{o}{v}\big) \ \final}\and
\inferrule*{R \ \stuck}{\big(\assign{o}{R}\big) \ \final}\and
\inferrule*{P \ \final \\ Q \ \final}{\Par{P}{Q} \ \final}\and
\inferrule*{P \ \final}{\big(\new{o}{\tau}{P}\big) \ \final}\\\\
\fbox{$P \Downarrow \eta$}\\
\inferrule*{}{ }\and
\inferrule*{P \ \final}{P \Downarrow 1[P]}\and
\inferrule*{P \outstep{o}{v} Q \\ Q \Downarrow \eta}{P \Downarrow \eta}\and \\
\inferrule*{P \to \sum_i c_i * 1[P_i] \\ P_i \Downarrow \eta_i}{P \Downarrow \sum_i c_i * \eta_i}
\end{mathpar}
\caption{Big-step operational semantics for \ipdl protocols.}
\label{fig:protocols_big_step}
\end{figure}

Note that while the small-step semantics for reactions is sequential, both output and internal step relations for protocols are non-deterministic. Indeed, any two channels in a protocol may output in any order. Ordinarily, this presents a problem for reasoning about cryptography, since non-deterministic choice may present a security leak. However, our language introduces \emph{no} way to exploit this extra non-determinism, essentially due to the $\mathsf{read}$ commands in reactions being blocking. This is formalized by a number of \emph{confluence} results for \textsf{IPDL}:

\begin{lemma}[Confluence for expressions]
For any expression $\Gamma \vdash e : \tau$ we have the following:
\begin{itemize}
\item If $e \to e_1$ and $e \to e_2$ then either $e_1 = e_2$ or there is $e'$ such that $e_1 \to e'$ and $e_2 \to e'$.
\item If $e \to e'$ and $e \Downarrow v$ then $e' \Downarrow v$.
\end{itemize}
\end{lemma}

\begin{lemma}[Confluence for reactions]
For any reaction $\Delta; \ \Gamma \vdash R : I \to \tau$ we have the following:
\begin{itemize}
\item We have $R[\read{i} \coloneqq \val{v}][\read{i} \coloneqq \val{v}] = R[\read{i} \coloneqq \val{v}]$.
\item Let $i_1 \neq i_2$. Then $R[\read{i_1} \coloneqq \val{v_1}][\read{i_2} \coloneqq \val{v_2}] = R[\read{i_2} \coloneqq \val{v_2}][\read{i_1} \coloneqq \val{v_1}]$.
\item If $R \to \eta$ then $R[\read{i} \coloneqq \val{v}] \to \eta[\read{i} \coloneqq \val{v}]$.
\end{itemize}
\end{lemma}

\begin{lemma}[Confluence for protocols, part I]
For any protocol $\Delta \vdash P : I \to O$ we have the following:
\begin{itemize}
\item We have $P[\read{i} \coloneqq \val{v}][\read{i} \coloneqq \val{v}] = P[\read{i} \coloneqq \val{v}]$.
\item Let $i_1 \neq i_2$. Then $P[\read{i_1} \coloneqq \val{v_1}][\read{i_2} \coloneqq \val{v_2}] = P[\read{i_2} \coloneqq \val{v_2}][\read{i_1} \coloneqq \val{v_1}]$.
\item Let $o_1 \neq o_2$. If $P \outstep{o_1}{v_1} Q$ then $P[\read{o_2} \coloneqq \val{v_2}] \outstep{o_1}{v_1} Q[\read{o_2} \coloneqq \val{v_2}]$.
\item If $P \to \eta$ then $P[\read{o} \coloneqq \val{v}] \to \eta[\read{o} \coloneqq \val{v}]$.
\end{itemize}
\end{lemma}

\begin{lemma}[Confluence for protocols, part II]
For any protocol $\Delta \vdash P : I \to O$ we have the following:
\begin{itemize}
\item Let $o_1 \neq o_2$. If $P \outstep{o_1}{v_1} Q$ and $P \outstep{o_2}{v_2} P'$ then there is $Q'$ such that $Q \outstep{o_2}{v_2} Q'$ and $P' \outstep{o_1}{v_1} Q'$.
\item If $P \outstep{o}{v} P'$ and $P \to \eta$ then there is $\eta'$ such that $\eta \outstep{o}{v} \eta'$ and $P' \to \eta'$.
\item If $P \to \eta_1$ and $P \to \eta_2$ then either $\eta_1 = \eta_2$ or there is $\eta$ such that $\eta_1 \to \eta$ and $\eta_2 \to \eta$.
\end{itemize}
\end{lemma}

\noindent In the above lemma, we lift the two protocol stepping relations $\outstep{o}{v}$ and $\to$ to distributions in the natural way.

\begin{lemma}[Confluence for protocols, part III]
For any protocol $\Delta \vdash P : I \to O$ we have the following:
\begin{itemize}
\item If $P \outstep{o}{v} Q$ and $P \Downarrow \varepsilon$ then $Q \Downarrow \varepsilon$.
\item If $P \to \eta$ and $P \Downarrow \varepsilon$ then $\eta \Downarrow \varepsilon$.
\end{itemize}
\end{lemma}

\noindent To guarantee termination (for protocols especially), we introduce the notion of a \emph{structure bound}. The structure bound for expressions, defined below, is invariant under substitutions.
\begin{align*}
\snorm{v} & \coloneqq 0 \\
\snorm{\checkmark} & \coloneqq 1 \\
\snorm{\true} & \coloneqq 1 \\
\snorm{\false} & \coloneqq 1 \\
\snorm{\func \ e} & \coloneqq \snorm{e} + 1 \\
\snorm{(e_1,e_2)} & \coloneqq \snorm{e_1} + \snorm{e_2} + 1 \\
\snorm{\fst \ e} & \coloneqq \snorm{e} + 1 \\
\snorm{\snd \ e} & \coloneqq \snorm{e} + 1
\end{align*}
\noindent The structure bound for reactions, defined below, is invariant under substitutions, embeddings, and input assignment.
\begin{align*}
\snorm{\val{v}} & \coloneqq 0 \\
\snorm{\ret{e}} & \coloneqq \snorm{e} + 1 \\
\snorm{\samp{\dist}{e}} & \coloneqq \snorm{e} + 1 \\
\snorm{\read{c}} & \coloneqq 0 \\
\snorm{\ifte{e}{R_1}{R_2}} & \coloneqq \snorm{e} + \mathsf{max} \, \big(\snorm{R_1}, \snorm{R_2}\big) + 1 \\
\snorm{x \leftarrow R; \ S} & \coloneqq \snorm{R} + \snorm{S} + 1
\end{align*}
\noindent The structure bound for protocols, defined below, is invariant under embeddings and input assignment.
\begin{align*}
\snorm{\zero} & \coloneqq 0 \\
\snorm{\assign{o}{v}} & \coloneqq 0 \\
\snorm{\assign{o}{R}} & \coloneqq \snorm{R} + 1 \\
\snorm{\Par{P}{Q}} & \coloneqq \snorm{P} + \snorm{Q} \\
\snorm{\new{c}{\tau}{P}} & \coloneqq \snorm{P}
\end{align*}

\begin{lemma}[Progress for expressions]
For any expression $\cdot \vdash e : \tau$ we have the following:
\begin{itemize}
\item If $e \to e'$ then $\snorm{e'} < \snorm{e}$.
\end{itemize}
\end{lemma}

\begin{lemma}[Progress for reactions]
For any reaction $\Delta; \ \cdot \vdash R : I \to \tau$ we have the following:
\begin{itemize}
\item If $R \to \sum_i c_i * 1[R_i]$ with $c_i \neq 0$, then $\snorm{R_i} < \snorm{R}$.
\end{itemize}
\end{lemma}

\begin{lemma}[Progress for protocols]
For any protocol $\Delta \vdash P : I \to O$ we have the following:
\begin{itemize}
\item If $P \outstep{o}{v} Q$ then $\snorm{Q} < \snorm{P}$.
\item If $P \to \sum_i c_i * 1[P_i]$ with $c_i \neq 0$, then $\snorm{P_i} < \snorm{P}$.
\end{itemize}
\end{lemma}

\noindent The confluence and progress lemmas together imply that our semantics is indeed well-defined:

\begin{corollary}[Determinism of $\Downarrow$ for expressions]
For any expression $\Gamma \vdash e : \tau$, there exists a unique bitstring $v$ such that $e \Downarrow v$. We will denote $v$ by $\eval{e}$.
\end{corollary}

\begin{corollary}[Determinism of $\Downarrow$ for reactions]
For any reaction $\Delta; \ \cdot \vdash R : I \to \tau$, there exists a unique distribution $\eta$ on reactions such that $R \Downarrow \eta$. We will denote $\eta$ by $\eval{R}$.
\end{corollary}

\begin{corollary}[Determinism of $\Downarrow$ for protocols]
For any protocol $\Delta \vdash P : I \to O$, there exists a unique distribution $\eta$ on protocols such that $P \Downarrow \eta$. We will denote $\eta$ by $\eval{P}$.
\end{corollary}

\section{Soundness of Exact Equality in IPDL}
Soundness of equality at the expression level means that if we substitute the same valued expression for each free variable, the resulting closed expressions will compute to the same value:

\begin{definition}
An axiom $\Gamma \vdash e_1 = e_2 : \tau$ is \emph{sound} if for any valued substitution $\theta : \cdot \to \Gamma$ we have $\eval{\theta^\star(e_1)} = \eval{\theta^\star(e_2)}$.
\end{definition}

\noindent The ambient \ipdl theory $\mathbb{T}_\mathsf{exp}$ for expressions is said to be sound if each of its axioms is sound. It is straightforward to show that this implies overall soundness:
 
\begin{lemma}[Soundness of equality of expressions]
If the ambient \ipdl theory for expressions is sound, then for any expressions $\Gamma \vdash e_1 = e_2 : \tau$ and any valued substitution $\theta : \cdot \to \Gamma$ we have that $\eval{\theta^\star(e_1)} = \eval{\theta^\star(e_2)}$.
\end{lemma}

At the reaction level, two equal reactions should behave in a way that is perfectly indistinguishable by an external observer. We formally capture this notion of indistinguishability by a logical relation known as a \emph{bisimulation} -- a binary relation on distributions on reactions that satisfies certain closure properties, together with the crucial \emph{valuation property} that allows us to jointly partition two related final distributions so that any two corresponding components are again related and have the same \emph{value}: a final reaction $\Delta; \ \cdot \vdash R : I \to \tau$ is said to have value $v \in \int{\tau}$ if $R$ is of the form $\val{v}$; if $R$ is stuck we define the value to be $\bot$. Given a $v_\bot \in \{\bot\} \cup \int{\tau}$, we write $\valueof{R} = v_\bot$ to indicate that the value of $R$ is $v_\bot$, and lift this notation to distributions in the obvious way.

We emphasize that at the reaction level, we only require the valuation property to hold for those distributions that are \emph{final}, \emph{i.e.}, no reaction in the support steps.

\begin{definition}[Reaction bisimulation]
A \emph{reaction bisimulation} $\sim$ is a binary relation on distributions on reactions of type $\Delta; \ \cdot \vdash I \to \tau$ satisfying the following conditions:
\begin{itemize}
\item \emph{Closure under convex combinations}: For any distributions $\eta_1 \sim \varepsilon_1$ and $\eta_2 \sim \varepsilon_2$, and any coefficients $c_1, c_2 > 0$ with $c_1 + c_2 = 1$, we have $(c_1 * \eta_1 + c_2 * \eta_2) \sim (c_1 * \varepsilon_1 + c_2 * \varepsilon_2)$.

\item \emph{Closure under input assignment}: For any distributions $\eta \sim \varepsilon$, input channel $i \in I$ with $i : \tau \in \Delta$, and value $v \in \int{\tau}$, we have $\eta[\read{i} \coloneqq \val{v}] \sim \varepsilon[\read{i} \coloneqq \val{v}]$.

\item \emph{Closure under computation}: For any distributions $\eta \sim \varepsilon$, we have $(\eval{\eta}) \sim (\eval{\varepsilon})$.

\item \emph{Valuation property}: For any final distributions $\eta \sim \varepsilon$, there exists a joint convex combination \[\eta = \sum_i c_i * \eta_i \; \sim \, \sum_i c_i * \varepsilon_i = \varepsilon\]
with $c_i > 0$ and $\sum_i c_i = 1$ such that
\begin{itemize}
\item the respective components $\eta_i \sim \varepsilon_i$ are again related, and
\item $\valueof{\eta_i} = v_\bot = \valueof{\varepsilon_j}$ for the same $v_\bot \in \{\bot\} \cup \int{\tau}$ if and only if $i = j$.
\end{itemize}
\end{itemize}
\end{definition}

\noindent Crucially, we note that the joint convex combination in the valuation property is unique up to the order of the summands.

\begin{lemma}
We have the following: 
\begin{itemize}
\item The identity relation is a reaction bisimulation.
\item The inverse of a reaction bisimulation is a reaction bisimulation.
\item The composition of two reaction bisimulations is a reaction bisimulation.
\end{itemize}
\end{lemma}

\noindent We now describe one canonical way to construct reaction bisimulations:

\begin{definition}
Let $\sim$ be an arbitrary binary relation on distributions on reactions of type $\Delta; \ \cdot \vdash I \to \tau$. The \emph{expansion} $\lift{\sim}$ is the closure of $\sim$ under joint convex combinations. Explicitly, $\lift{\sim}$ is defined by
\[\Big(\sum_i c_i * \eta_i\Big) \; \lift{\sim} \; \Big(\sum_i c_i * \varepsilon_i\Big)\]
for coefficients $c_i > 0$ with $\sum_i c_i = 1$ and distributions $\eta_i \sim \varepsilon_i$.
\end{definition}

\begin{lemma}\label{lem:reaction_seed}
Let $\sim$ be a binary relation on distributions on reactions of type $\Delta; \ \cdot \vdash I \to \tau$ with the following properties:
\begin{itemize}
\item \emph{Closure under input assignment}: For any distributions $\eta \sim \varepsilon$, input channel $i \in I$ with $i : \tau \in \Delta$, and value $v \in \int{\tau}$, we have $\eta[\read{i} \coloneqq \val{v}] \sim \varepsilon[\read{i} \coloneqq \val{v}]$.

\item \emph{Expansion closure under computation}: For any distributions $\eta \sim \varepsilon$, we have $(\eval{\eta}) \; \lift{\sim} \; (\eval{ \varepsilon})$.

\item \emph{Valuation property}: For any final distributions $\eta \sim \varepsilon$, there exists a joint convex combination \[\eta = \sum_i c_i * \eta_i \; \sim \, \sum_i c_i * \varepsilon_i = \varepsilon\]
with $c_i > 0$ and $\sum_i c_i = 1$ such that
\begin{itemize}
\item the respective components $\eta_i \sim \varepsilon_i$ are again related, and
\item $\valueof{\eta_i} = v_\bot = \valueof{\varepsilon_j}$ for the same $v_\bot \in \{\bot\} \cup \int{\tau}$ if and only if $i = j$.
\end{itemize}
\end{itemize}
Then the expansion $\lift{\sim}$ is a reaction bisimulation.
\end{lemma}

\begin{example}
Fix two expressions $\cdot \vdash e_1 : \sigma$ and $\cdot \vdash e_2 : \sigma$ such that $\eval{e_1} = \eval{e_2}$. Then the relation $\sim$ defined by
\begin{itemize}
\item $1[R(x \coloneqq e_1)] \sim 1[R(x \coloneqq e_2)]$ for reaction $\Delta; \ x : \sigma \vdash R : I \to \tau$
\end{itemize}
satisfies the hypotheses of Lemma \ref{lem:reaction_seed} (and its lifting is hence a reaction bisimulation).
\end{example}

\noindent Having defined reaction bisimulations, we can now formally state what it means for reaction equality to be sound:

\begin{definition}
An axiom $\Gamma \vdash R_1 = R_2 : \tau$ is \emph{sound} if there is a reaction bisimulation $\sim$ such that for any valued substitution $\theta : \cdot \to \Gamma$ we have $1[\theta^\star(R_1)] \sim 1[\theta^\star(R_2)]$.
\end{definition}

\noindent The ambient \ipdl theory $\mathbb{T}_\mathsf{dist}$ for reactions is said to be sound if each of its axioms is sound. We now show that this implies overall soundness:

\begin{lemma}[Soundness of equality of reactions]
If the ambient \ipdl theories for expressions and reactions are sound, then for any reactions $\Delta; \ \Gamma \vdash R_1 = R_2 : I \to \tau$ there exists a reaction bisimulation $\sim$ such that for any valued substitution $\theta : \cdot \to \Gamma$ we have $1[\theta^\star(R_1)] \sim 1[\theta^\star(R_2)]$.
\end{lemma}

\begin{proof}
We first replace the exchange rule \textsc{exch} by the three rules \textsc{exch-samp-samp}, \textsc{exch-samp-read}, and \textsc{exch-read-read} in Figure \ref{fig:exch_alt}; it is easy to see that this new set of rules is equivalent to the original one. We now proceed by induction on the alternative set of rules for reaction equality. We will freely use a distribution in place of a value (rule \textsc{exch-samp-read}) or a reaction (rules \textsc{embed}, \textsc{cong-bind}) to indicate the obvious lifting of the corresponding construct to distributions on reactions.

\begin{itemize}
\item \textsc{refl}: Our desired bisimulation is the identity relation.
\item \textsc{sym}: Our desired bisimulation is the inverse of the bisimulation obtained from the premise.
\item \textsc{trans}: Our desired bisimulation is the composition of the two bisimulations obtained from the two premises.
\item \textsc{axiom}: Our desired bisimulation is precisely the bisimulation obtained from the soundness of the axiom.
\item \textsc{input-unused}: Our desired bisimulation is precisely the bisimulation obtained from the premise, seen as a bisimulation on distributions on reactions with the additional input $i$.
\item \textsc{subst}: Our desired bisimulation is precisely the bisimulation obtained from the premise.
\item \textsc{embed}: Let $\sim$ be the bisimulation obtained from the premise. Our desired bisimulation $\sim_\phi$ is defined by
\begin{itemize}
\item $\phi^\star(\eta) \sim_\phi \phi^\star(\varepsilon)$ if $\eta \sim \varepsilon$
\end{itemize}
\item \textsc{cong-ret}: Our desired bisimulation is the expansion of the relation $\sim$ defined by
\begin{itemize}
\item $1[\ret{e}] \sim 1[\ret{e'}]$ for expressions $\cdot \vdash e : \tau$ and $\cdot \vdash e' : \tau$ such that $\eval{e} = \eval{e'}$
\item $1[\val{v}] \sim 1[\val{v}]$ for value $v \in \int{\tau}$
\end{itemize}
\item \textsc{cong-samp}: Our desired bisimulation is the expansion of the relation $\sim$ defined by
\begin{itemize}
\item $1[\samp{\dist}{e}] \sim 1[\samp{\dist}{e'}]$ for expressions $\cdot \vdash e : \sigma$ and $\cdot \vdash e' : \sigma$ such that $\eval{e} = \eval{e'}$
\item $1[\val{v}] \sim 1[\val{v}]$ for value $v \in \int{\tau}$
\end{itemize}
\item \textsc{cong-if}: Let $\sim_1$ and $\sim_2$ be the two bisimulations obtained from the two premises. Our desired bisimulation is the expansion of the relation $\sim_\mathsf{if}$ defined by
\begin{itemize}
\item $1[\ifte{e}{R_1}{R_2}] \sim_\mathsf{if} \, 1[\ifte{e'}{R'_1}{R'_2}]$ for 
\begin{itemize}
\item expressions $\cdot \vdash e : \Bool$ and $\cdot \vdash e' : \Bool $ such that $\eval{e} = \eval{e'}$
\item reactions $\Delta; \ \cdot \vdash R_1 : I \to \tau$ and $\Delta; \ \cdot \vdash R'_1 : I \to \tau$ such that $1[R_1] \sim_1 1[R'_1]$
\item reactions $\Delta; \ \cdot \vdash R_2 : I \to \tau$ and $\Delta; \ \cdot \vdash R'_2 : I \to \tau$ such that $1[R_2] \sim_2 1[R'_2]$
\end{itemize}
\item $\eta_1 \sim_\mathsf{if} \eta'_1$ if $\eta_1 \sim_1 \eta_1'$
\item $\eta_2 \sim_\mathsf{if} \eta'_2$ if $\eta_2 \sim_2 \eta_2'$
\end{itemize}
\item \textsc{cong-bind}: Let $\sim_1$ and $\sim_2$ be the two bisimulations obtained from the two premises. Our desired bisimulation is the expansion of the relation $\sim_\mathsf{bind}$ defined by
\begin{itemize}
\item $\big(x \leftarrow \eta; \ S\big) \sim_\mathsf{bind} \big(x \leftarrow \eta'; \ S'\big)$ for
\begin{itemize}
\item distributions $\eta \sim_1 \eta'$
\item reactions $\Delta; \ x : \sigma \vdash S : I \to \tau$ and $\Delta; \ x : \sigma \vdash S' : I \to \tau$ such that for any value $v \in \int{\sigma}$ we have $1[S(x \coloneqq v)] \sim_2 1[S'(x \coloneqq v)]$
\end{itemize}
\item $\varepsilon \sim_\mathsf{bind} \varepsilon'$ if $\varepsilon \sim_2 \varepsilon'$
\end{itemize}
\item \textsc{ret-bind}: Our desired bisimulation is the expansion of the relation $\sim$ defined by
\begin{itemize}
\item $1[x \leftarrow \ret{e}; \ R] \sim 1[R(x \coloneqq e)]$ for expression $\cdot \vdash e : \sigma$ and reaction $\Delta; \ x : \sigma \vdash R : I \to \tau$
\item $1[R(x \coloneqq \eval{e})] \sim 1[R(x \coloneqq e)]$ for reaction $\Delta; \ x : \sigma \vdash R : I \to \tau$ and expression $\cdot \vdash e : \sigma$
\end{itemize}
\item \textsc{bind-ret}: Our desired bisimulation is the expansion of the relation $\sim$ defined by
\begin{itemize}
\item $1[x \leftarrow R; \ \ret{x}] \sim 1[R]$ for reaction $\Delta; \ \cdot \vdash R : I \to \tau$
\item $1[\val{v}] \sim 1[\val{v}]$ for value $v \in \int{\tau}$
\end{itemize}
\item \textsc{bind-bind}: Our desired bisimulation is the expansion of the relation $\sim$ defined by
\begin{itemize}
\item $1[x_2 \leftarrow (x_1 \leftarrow R_1; \ R_2); \ S] \sim 1[x_1 \leftarrow R_1; \ x_2 \leftarrow R_2; \ S]$ for
\begin{itemize}
\item reaction $\Delta; \ \cdot \vdash R_1 : I \to \sigma_1$
\item reaction $\Delta; \ x_1 : \sigma_1 \vdash R_2 : I \to \sigma_2$
\item reaction $\Delta; \ x_2 : \sigma_2 \vdash S : I \to \tau$
\end{itemize}
\item $1[x_2 \leftarrow R_2; \ S] \sim 1[x_2 \leftarrow R_2; \ S]$ for reactions $\Delta; \ \cdot \vdash R_2 : I \to \sigma_2$ and $\Delta; \ x_2 : \sigma_2 \vdash S : I \to \tau$
\item $1[S] \sim 1[S]$ for reaction $\Delta; \ \cdot \vdash S : I \to \tau$
\end{itemize}
\item \textsc{samp-pure}: Our desired bisimulation is the expansion of the relation $\sim$ defined by
\begin{itemize}
\item $1[x \leftarrow \samp{\dist}{e}; \ R] \sim 1[R]$ for expression $\cdot \vdash e : \rho$ and reaction $\Delta; \ \cdot \vdash R : I \to \tau$
\item $1[R] \sim 1[R]$ for reaction $\Delta; \ \cdot \vdash R : I \to \tau$
\end{itemize}
\item \textsc{read-det}: Our desired bisimulation is the expansion of the relation $\sim$ defined by
\begin{itemize}
\item $1[x \leftarrow \read{i}; \ y \leftarrow \read{i}; \ R] \sim 1[x \leftarrow \read{i}; \ R(y \coloneqq x)]$ for reaction $\Delta; \ x : \sigma, y : \sigma \vdash R : I \to \tau$
\item $1[x \leftarrow \val{v}; \ y \leftarrow \val{v}; \ R] \sim 1[x \leftarrow \val{v}; \ R(y \coloneqq x)]$ for
\begin{itemize}
\item reaction $\Delta; \ x : \sigma, y : \sigma \vdash R : I \to \tau$ 
\item value $v \in \int{\sigma}$
\end{itemize}
\item $1[R] \sim 1[R]$ for reaction $\Delta; \ \cdot \vdash R : I \to \tau$
\end{itemize}
\item \textsc{if-left}: Our desired bisimulation is the expansion of the relation $\sim$ defined by
\begin{itemize}
\item $1[\ifte{\true}{R_1}{R_2}] \sim 1[R_1]$ for reactions $\Delta; \ \cdot \vdash R_1 : I \to \tau$ and $\Delta; \ \cdot \vdash R_2 : I \to \tau$
\item $1[R_1] \sim 1[R_1]$ for reaction $\Delta; \ \cdot \vdash R_1 : I \to \tau$
\end{itemize}
\item \textsc{if-right}: Our desired bisimulation is the expansion of the relation $\sim$ defined by
\begin{itemize}
\item $1[\ifte{\false}{R_1}{R_2}] \sim 1[R_2]$ for reactions $\Delta; \ \cdot \vdash R_1 : I \to \tau$ and $\Delta; \ \cdot \vdash R_2 : I \to \tau$
\item $1[R_2] \sim 1[R_2]$ for reaction $\Delta; \ \cdot \vdash R_2 : I \to \tau$
\end{itemize}
\item \textsc{if-ext}: Our desired bisimulation is the expansion of the relation $\sim$ defined by
\begin{itemize}
\item $1[\ifte{e}{R(x \coloneqq \true)}{R(x \coloneqq \false)]} \sim 1[R(x \coloneqq e)]$ for
\begin{itemize}
\item reaction $\Delta; \ x : \Bool \vdash R : I \to \tau$
\item expression $\cdot \vdash e : \Bool$
\end{itemize}
\item $1[R(x \coloneqq \true)] \sim 1[R(x \coloneqq e)]$ for
\begin{itemize}
\item reaction $\Delta; \ x : \Bool \vdash R : I \to \tau$
\item expression $\cdot \vdash e : \Bool$ such that $e \Downarrow 1$
\end{itemize}
\item $1[R(x \coloneqq \false)] \sim 1[R(x \coloneqq e)]$ for
\begin{itemize}
\item reaction $\Delta; \ x : \Bool \vdash R : I \to \tau$
\item expression $\cdot \vdash e : \Bool$ such that $e \Downarrow 0$
\end{itemize}
\end{itemize}
\item \textsc{exch-samp-samp}: Our desired bisimulation is the expansion of the relation $\sim$ defined by
\begin{itemize}
\item $1[x_1 \leftarrow \samp{\dist_1}{e_1}; \ x_2 \leftarrow \samp{\dist_2}{e_2}; \ \ret{(x_1,x_2)}] \sim \\ 1[x_2 \leftarrow \samp{\dist_2}{e_2}; \ x_1 \leftarrow \samp{\dist_1}{e_1}; \ \ret{(x_1,x_2)}]$ for
\begin{itemize}
\item expressions $\cdot \vdash e_1 : \sigma_1$ and $\cdot \vdash e_2 : \sigma_2$
\end{itemize}
\item $1[\val{v_1 v_2}] \sim 1[\val{v_1 v_2}]$ for values $v_1 \in \int{\tau_1}$ and $v_2 \in \int{\tau_2}$
\end{itemize}
\item \textsc{exch-samp-read}: Our desired bisimulation is the expansion of the relation $\sim$ defined by
\begin{itemize}
\item $1[x_1 \leftarrow \samp{\dist}{e}; \ x_2 \leftarrow \read{i}; \ \ret{(x_1,x_2)}] \sim 1[x_2 \leftarrow \read{i}; \ x_1 \leftarrow \samp{\dist}{e}; \ \ret{(x_1,x_2)}]$ for
\begin{itemize}
\item expression $\cdot \vdash e : \sigma$
\end{itemize}
\item $1[x_1 \leftarrow \samp{\dist}{e}; \ x_2 \leftarrow \val{v_2}; \ \ret{(x_1,x_2)}] \sim 1[x_2 \leftarrow \val{v_2}; \ x_1 \leftarrow \samp{\dist}{e}; \ \ret{(x_1,x_2)}]$ for
\begin{itemize}
\item expression $\cdot \vdash e : \sigma$
\item value $v_2 \in \int{\tau_2}$
\end{itemize}
\item $\big(x_2 \leftarrow \read{i}; \ \ret{(\int{\dist}(\eval{e}),x_2)}\big) \sim 1[x_2 \leftarrow \read{i}; \ x_1 \leftarrow \samp{\dist}{e}; \ \ret{(x_1,x_2)}]$ for
\begin{itemize}
\item expression $\cdot \vdash e : \sigma$
\end{itemize}
\item $\big(x_2 \leftarrow \val{v_2}; \ \ret{(\int{\dist}(\eval{e}),x_2)}\big) \sim 1[x_2 \leftarrow \val{v_2}; \ x_1 \leftarrow \samp{\dist}{e}; \ \ret{(x_1,x_2)}]$ for
\begin{itemize}
\item expression $\cdot \vdash e : \sigma$
\item value $v_2 \in \int{\tau_2}$
\end{itemize}
\item $1[\val{v_1 v_2}] \sim 1[\val{v_1 v_2}]$ for values $v_1 \in \int{\tau_1}$ and $v_2 \in \int{\tau_2}$
\end{itemize}
\item \textsc{exch-read-read}: Our desired bisimulation is the expansion of the relation $\sim$ defined by
\begin{itemize}
\item $1[x_1 \leftarrow \read{i_1}; \ x_2 \leftarrow \read{i_2}; \ \ret{(x_1,x_2)}] \sim 1[x_2 \leftarrow \read{i_2}; \ x_1 \leftarrow \read{i_1}; \ \ret{(x_1,x_2)}]$
\item $1[x_1 \leftarrow \val{v_1}; \ x_2 \leftarrow \read{i_2}; \ \ret{(x_1,x_2)}] \sim 1[x_2 \leftarrow \read{i_2}; \ x_1 \leftarrow \val{v_1}; \ \ret{(x_1,x_2)}]$ for value $v_1 \in \int{\tau_1}$
\item $1[x_1 \leftarrow \read{i_1}; \ x_2 \leftarrow \val{v_2}; \ \ret{(x_1,x_2)}] \sim 1[x_2 \leftarrow \val{v_2}; \ x_1 \leftarrow \read{i_1}; \ \ret{(x_1,x_2)}]$ for value $v_2 \in \int{\tau_2}$
\item $1[x_1 \leftarrow \val{v_1}; \ x_2 \leftarrow \val{v_2}; \ \ret{(x_1,x_2)}] \sim 1[x_2 \leftarrow \val{v_2}; \ x_1 \leftarrow \val{v_1}; \ \ret{(x_1,x_2)}]$ for
\begin{itemize}
\item values $v_1 \in \int{\tau_1}$ and $v_2 \in \int{\tau_2}$
\end{itemize}
\item $1[x_2 \leftarrow \read{i_2}; \ \ret{(v_1,x_2)}] \sim 1[x_2 \leftarrow \read{i_2}; \ x_1 \leftarrow \val{v_1}; \ \ret{(x_1,x_2)}]$ for value $v_1 \in \int{\tau_1}$
\item $1[x_1 \leftarrow \read{i_1}; \ x_2 \leftarrow \val{v_2}; \ \ret{(x_1,x_2)}] \sim 1[x_1 \leftarrow \read{i_1}; \ \ret{(x_1,v_2)}]$ for value $v_2 \in \int{\tau_2}$
\item $1[x_2 \leftarrow \val{v_2}; \ \ret{(v_1,x_2)}] \sim 1[x_2 \leftarrow \val{v_2}; \ x_1 \leftarrow \val{v_1}; \ \ret{(x_1,x_2)}]$ for values $v_1 \in \int{\tau_1}$ and $v_2 \in \int{\tau_2}$
\item $1[x_1 \leftarrow \val{v_1}; \ x_2 \leftarrow \val{v_2}; \ \ret{(x_1,x_2)}] \sim 1[x_1 \leftarrow \val{v_1}; \ \ret{(x_1,v_2)}]$ for
values $v_1 \in \int{\tau_1}$ and $v_2 \in \int{\tau_2}$
\item $1[\val{v_1 v_2}] \sim 1[\val{v_1 v_2}]$ for values $v_1 \in \int{\tau_1}$ and $v_2 \in \int{\tau_2}$
\end{itemize}
\end{itemize}
\end{proof}

\begin{figure}
\begin{mathpar}
\inferrule*[right=exch-samp-samp]{\dist_1 : \sigma_1 \twoheadrightarrow \tau_1, \dist_2 : \sigma_2 \twoheadrightarrow \tau_2 \in \Sigma \\ \Gamma \vdash e_1 : \sigma_1 \\ \Gamma \vdash e_2 : \sigma_2}{\Delta; \ \Gamma \vdash \big(x_1 : \tau_1 \leftarrow \Samp{\dist_1}{\sigma_1}{\tau_1}{e_1}; \ x_2 : \tau_2 \leftarrow \Samp{\dist_2}{\sigma_2}{\tau_2}{e_2}; \ \ret{(x_1,x_2)}\big) = \hspace{52.3pt} \\ \big(x_2 : \tau_2 \leftarrow \Samp{\dist_2}{\sigma_2}{\tau_2}{e_2}; \ x_1 : \tau_1 \leftarrow \Samp{\dist_1}{\sigma_1}{\tau_1}{e_1}; \ \ret{(x_1,x_2)}\big) : I \to \tau_1 \times \tau_2\hspace{-39.8pt}}\and
\inferrule*[right=exch-samp-read]{\dist : \sigma \twoheadrightarrow \tau_1 \in \Sigma \\ \Gamma \vdash e : \sigma \\ i : \tau_2 \in \Delta \\ i \in I}{\Delta; \ \Gamma \vdash \big(x_1 : \tau_1 \leftarrow \Samp{\dist}{\sigma}{\tau_1}{e}; \ x_2 : \tau_2 \leftarrow \read{i}; \ \ret{(x_1,x_2)}\big) = \hspace{52.3pt} \\ \big(x_2 : \tau_2 \leftarrow \read{i}; \ x_1 : \tau_1 \leftarrow \Samp{\dist}{\sigma}{\tau_1}{e}; \ \ret{(x_1,x_2)}\big) : I \to \tau_1 \times \tau_2\hspace{-39.8pt}}\and
\inferrule*[right=exch-read-read]{i_1 : \tau_1, i_2 : \tau_2 \in \Delta \\ i_1, i_2 \in I}{\Delta; \ \Gamma \vdash \big(x_1 : \tau_1 \leftarrow \read{i_1}; \ x_2 : \tau_2 \leftarrow \read{i_2}; \ \ret{(x_1,x_2)}\big) = \hspace{52.3pt} \\ \big(x_2 : \tau_2 \leftarrow \read{i_2}; \ x_1 : \tau_1 \leftarrow \read{i_1}; \ \ret{(x_1,x_2)}\big) : I \to \tau_1 \times \tau_2 \hspace{-39.8pt}}
\end{mathpar}
\caption{Alternative formulation of the \textsc{exch} rule for reaction equality.}
\label{fig:exch_alt}
\end{figure}

At last we get to the protocol level. A protocol $\Delta \vdash P : I \to O$ is said to have value $v \in \int{\tau}$ on channel $o \in O$ with $o : \tau \in \Delta$ if $P$ contains the assignment $\assign{o}{v}$; otherwise we define the value to be $\bot$. We write $\valueat{P}{o} = v_\bot$ to indicate that the value of $P$ on $o$ is $v_\bot \in \{\bot\} \cup \int{\tau}$, and lift this notation to distributions in the obvious way. In the case when $\valueat{P}{o} = \bot$, $P$ contains the assignment $\assign{o}{R}$ for some reaction $\Delta; \ \cdot \vdash R : I \to \tau$. The value of $R$ will be called the \emph{local value} of $P$ at $o$ (this terminology indicates that the computation of the channel $o$ to $v$ has not yet been communicated to the rest of the protocol). We write $\localvalueat{P}{o} = v_\bot$ to indicate that the local value of $P$ on $o$ is $v_\bot \in \{\bot\} \cup \int{\tau}$ (therefore $\valueof{R} = v_\bot$), and lift this notation to distributions $\eta$ in the obvious way. Thus, $\localvalueat{\eta}{o} = v$ indicates that each protocol in the support of $\eta$ contains the assignment $o \coloneqq \val{v}$, whereas $\localvalueat{\eta}{o} = \bot$ indicates that each protocol in the support of $\eta$ carries a stuck reaction on $o$.

A protocol bisimulation is entirely analogous to a reaction bisimulation, except we require the valuation property to hold: \emph{i)} per output channel $o$, and \emph{ii)} for all distributions (not necessarily final).

\begin{definition}[Protocol bisimulation]
A \emph{protocol bisimulation} $\sim$ is a binary relation on distributions on protocols of type $\Delta \vdash I \to O$ satisfying the following conditions:
\begin{itemize}
\item \emph{Closure under convex combinations}: For any distributions $\eta_1 \sim \varepsilon_1$ and $\eta_2 \sim \varepsilon_2$, and any coefficients $c_1, c_2 > 0$ with $c_1 + c_2 = 1$, we have $(c_1 * \eta_1 + c_2 * \eta_2) \sim (c_1 * \varepsilon_1 + c_2 * \varepsilon_2)$.

\item \emph{Closure under input assignment}: For any distributions $\eta \sim \varepsilon$, input channel $i \in I$ with $i : \tau \in \Delta$, and value $v \in \int{\tau}$, we have $\eta[\read{i} \coloneqq \val{v}] \sim \varepsilon[\read{i} \coloneqq \val{v}]$.

\item \emph{Closure under computation}: For any distributions $\eta \sim \varepsilon$, we have $(\eval{\eta}) \sim (\eval{\varepsilon})$.

\item \emph{Valuation property}: For any output channel $o \in O$ with $o : \tau \in \Delta$ and any distributions $\eta \sim \varepsilon$, there exists a joint convex combination \[\eta = \sum_i c_i * \eta_i \; \sim \, \sum_i c_i * \varepsilon_i = \varepsilon\]
with $c_i > 0$ and $\sum_i c_i = 1$ such that
\begin{itemize}
\item the respective components $\eta_i \sim \varepsilon_i$ are again related, and
\item $\valueat{\eta_i}{o} = v_\bot = \valueat{\varepsilon_j}{o}$ for the same $v_\bot \in \{\bot\} \cup \int{\tau}$ if and only if $i = j$.
\end{itemize}
\end{itemize}
\end{definition}

\noindent We have the analogous results for bisimulations on the protocol level:

\begin{lemma}
We have the following: 
\begin{itemize}
\item The identity relation is a protocol bisimulation.
\item The inverse of a protocol bisimulation is a protocol bisimulation.
\item The composition of two protocol bisimulations is a protocol bisimulation.
\end{itemize}
\end{lemma}

\begin{definition}
Let $\sim$ be an arbitrary binary relation on distributions on protocols of type $\Delta \vdash I \to O$. The \emph{expansion} $\lift{\sim}$ is the closure of $\sim$ under joint convex combinations. Explicitly, $\lift{\sim}$ is defined by
\[\Big(\sum_i c_i * \eta_i\Big) \; \lift{\sim} \; \Big(\sum_i c_i * \varepsilon_i\Big)\]
for coefficients $c_i > 0$ with $\sum_i c_i = 1$ and distributions $\eta_i \sim \varepsilon_i$.
\end{definition}

\begin{lemma}\label{lem:protocol_seed}
Let $\sim$ be a binary relation on distributions on protocols of type $\Delta \vdash I \to O$ with the following properties:
\begin{itemize}
\item \emph{Closure under input assignment}: For any distributions $\eta \sim \varepsilon$, input channel $i \in I$ with $i : \tau \in \Delta$, and value $v \in \int{\tau}$, we have $\eta[\read{i} \coloneqq \val{v}] \sim \varepsilon[\read{i} \coloneqq \val{v}]$.

\item \emph{Expansion closure under computation}: For any distributions $\eta \sim \varepsilon$, we have $(\eval{\eta}) \; \lift{\sim} \; (\eval{ \varepsilon})$.

\item \emph{Valuation property}: For any output channel $o \in O$ with $o : \tau \in \Delta$ and any distributions $\eta \sim \varepsilon$, there exists a joint convex combination \[\eta = \sum_i c_i * \eta_i \; \sim \, \sum_i c_i * \varepsilon_i = \varepsilon\]
with $c_i > 0$ and $\sum_i c_i = 1$ such that
\begin{itemize}
\item the respective components $\eta_i \sim \varepsilon_i$ are again related, and
\item $\valueat{\eta_i}{o} = v_\bot = \valueat{\varepsilon_j}{o}$ for the same $v_\bot \in \{\bot\} \cup \int{\tau}$ if and only if $i = j$.
\end{itemize}
\end{itemize}
Then the expansion $\lift{\sim}$ is a protocol bisimulation.
\end{lemma}

\noindent We now formally state what it means for exact protocol equality to be sound:

\begin{definition}
An axiom $\Delta \vdash P_1 = P_2 : I \to O$ is \emph{sound} if there is a protocol bisimulation $\sim$ such that $1[P_1] \sim 1[P_2]$.
\end{definition}

\noindent The ambient exact \ipdl theory $\mathbb{T}_\mathsf{prot}$ for protocols is said to be sound if each of its axioms is sound. We now show that this implies overall soundness for exact equality:

\begin{lemma}[Soundness of exact equality of protocols]
If the ambient (exact) \ipdl theories for expressions, reactions, and protocols are sound, then for any protocols $\Delta \vdash P_1 = P_2 : I \to O$ there exists a protocol bisimulation $\sim$ such that $1[P_1] \sim 1[P_2]$.
\end{lemma}

\begin{proof}
We first replace the rules \textsc{fold-if-left} and \textsc{fold-if-right} by the equivalent formulation in Figure \ref{fig:fold_if_alt}. We now proceed by induction on this alternative set of rules for exact protocol equality. We will freely use a distribution in place of a reaction (rule \textsc{cong-react}) or a protocol (rules \textsc{embed}, \textsc{absorb-left}) to indicate the obvious lifting of the corresponding construct to distributions on protocols.

\begin{itemize}
\item \textsc{refl}: Our desired bisimulation is the identity relation.
\item \textsc{sym}: Our desired bisimulation is the inverse of the bisimulation obtained from the premise.
\item \textsc{trans}: Our desired bisimulation is the composition of the two bisimulations obtained from the two premises.
\item \textsc{axiom}: Our desired bisimulation is precisely the bisimulation obtained from the soundness of the axiom.
\item \textsc{input-unused}: Our desired bisimulation is precisely the bisimulation obtained from the premise, seen as a bisimulation on distributions on protocols with the additional input $i$.
\item \textsc{embed}: Let $\sim$ be the bisimulation obtained from the premise. Our desired bisimulation $\sim_\phi$ is defined by
\begin{itemize}
\item $\phi^\star(\eta) \sim_\phi \phi^\star(\varepsilon)$ if $\eta \sim \varepsilon$
\end{itemize}
\item \textsc{cong-react}: Let $\sim$ be the reaction bisimulation obtained from the premise. Our desired bisimulation is the expansion of the relation $\sim_{\mathsf{react}}$ defined by
\begin{itemize}
\item $(o \coloneqq \eta) \sim_{\mathsf{react}} (o \coloneqq \eta')$ for distributions $\eta \sim \eta'$
\item $1[o \coloneqq v] \sim_{\mathsf{react}} 1[o \coloneqq v]$ for  value $v \in \int{\tau}$
\end{itemize}
\item \textsc{cong-comp-left}: Let $\sim$ be the bisimulation obtained from the premise. The expansion of the relation $\sim_{\mathsf{par}}$ defined by
\begin{itemize}
\item $(\Par{\eta}{Q}) \sim_{\mathsf{par}} (\Par{\eta'}{Q})$ for $\eta \sim \eta'$ and protocol $\Delta \vdash Q : I \cup O_1 \to O_2$
\end{itemize}
is the natural candidate for our desired bisimulation. Proving that this is indeed a bisimulation requires a fair amount of work (see Lemma \ref{lem:composability_exact}).
\item \textsc{cong-new}: Let $\sim$ be the bisimulation obtained from the premise. Our desired bisimulation $\sim_{\mathsf{new}}$ is defined by
\begin{itemize}
\item $\big(\new{o}{\tau}{\eta}\big) \sim_{\mathsf{new}} \big(\new{o}{\tau}{\eta'}\big)$ if $\eta \sim \eta'$
\end{itemize}
\item \textsc{comp-comm}: Our desired bisimulation is the expansion of the relation $\sim$ defined by
\begin{itemize}
\item $1[\Par{P_1}{P_2}] \sim 1[\Par{P_2}{P_1}]$ for protocols $\Delta \vdash P_1 : I \cup O_2 \to O_1$ and $\Delta \vdash P_2 : I \cup O_1 \to O_2$
\end{itemize}
\item \textsc{comp-assoc}: Our desired bisimulation is the expansion of the relation $\sim$ defined by
\begin{itemize}
\item $1\big[\Par{(\Par{P_1}{P_2})}{P_3}\big] \sim 1\big[\Par{P_1}{(\Par{P_2}{P_3})}\big]$ for
\begin{itemize}
\item protocol $\Delta \vdash P_1 : I \cup O_2 \cup O_3 \to O_1$
\item protocol $\Delta \vdash P_2 : I \cup O_1 \cup O_3 \to O_2$
\item protocol $\Delta \vdash P_3 : I \cup O_1 \cup O_2 \to O_3$
\end{itemize}
\end{itemize}
\item \textsc{new-exch}: The desired bisimulation is the expansion of the relation $\sim$ defined by
\begin{itemize}
\item $1[\new{o_1}{\tau_1}{\new{o_2}{\tau_2}{P}}] \sim 1[\new{o_2}{\tau_2}{\new{o_1}{\tau_1}{P}}]$ for
\begin{itemize}
\item protocol $\Delta, o_1 : \tau_1, o_2 : \tau_2 \vdash P : I \to O \cup \{o_1,o_2\}$
\end{itemize}
\end{itemize}
\item \textsc{comp-new}: Our desired bisimulation is the expansion of the relation $\sim$ defined by
\begin{itemize}
\item $1[\Par{P}{(\new{o}{\tau}{Q}\big})] \sim 1[\new{o}{\tau}{(\Par{P}{Q})}]$ for
\begin{itemize}
\item protocol $\Delta \vdash P : I \cup O_2 \to O_1$
\item protocol $\Delta, o : \tau \vdash Q : I \cup O_1 \to O_2 \cup \{o\}$
\end{itemize}
\end{itemize}
\item \textsc{absorb-left}: Our desired bisimulation is the expansion of the relation $\sim$ defined by
\begin{itemize}
\item $1[\Par{P}{Q}] \sim 1[P]$ for protocols $\Delta \vdash P : I \to O$ and $\Delta \vdash Q : I \cup O \to \emptyset$
\end{itemize}
\item \textsc{diverge}: Our desired bisimulation is the expansion of the relation $\sim$ defined by
\begin{itemize}
\item $1[\assign{o}{x \leftarrow \read{o}; \ R}] \sim 1[\assign{o}{\read{o}}]$ for reaction $\Delta; \ \cdot \vdash R : I \to \tau$
\end{itemize}
\item \textsc{fold-if-left}: Our desired bisimulation is the expansion of the relation $\sim$ defined by
\begin{itemize}
\item $1[\new{l}{\tau}{\Par{\assign{o}{x \leftarrow \read{b}; \ \ifte{x}{\read{l}}}{S_2}}{\assign{l}{x \leftarrow \read{b}; \ S_1}}}] \sim \\ 1[\assign{o}{x \leftarrow \read{b}; \ \ifte{x}{S_1}{S_2}}]$ for
\begin{itemize}
\item reaction $\Delta; \ \cdot \vdash S_1 : I \to \tau$
\item reaction $\Delta; \ \cdot \vdash S_2 : I \to \tau$
\end{itemize}
\item $1[\new{l}{\tau}{\Par{\assign{o}{x \leftarrow \val{v}; \ \ifte{x}{\read{l}}}{S_2}}{\assign{l}{x \leftarrow \val{v}; \ S_1}}}] \sim \\ 1[\assign{o}{x \leftarrow \val{v}; \ \ifte{x}{S_1}{S_2}}]$ for
\begin{itemize}
\item value $v \in \{0,1\}$
\item reaction $\Delta; \ \cdot \vdash S_1 : I \to \tau$
\item reaction $\Delta; \ \cdot \vdash S_2 : I \to \tau$
\end{itemize}
\item $1[\new{l}{\tau}{\Par{\assign{o}{\read{l}}}{\assign{l}{S_1}}}] \sim 1[\assign{o}{S_1}]$ for reaction $\Delta; \ \cdot \vdash S_1 : I \to \tau$
\item $1[\new{l}{\tau}{\Par{\assign{o}{S_2}}{\assign{l}{S_1}}}] \sim 1[\assign{o}{S_2}]$ for reactions $\Delta; \ \cdot \vdash S_1 : I \to \tau$ and $\Delta; \ \cdot \vdash S_2 : I \to \tau$
\item $1[\new{l}{\tau}{\Par{\assign{o}{v_2}}{\assign{l}{S_1}}}] \sim 1[\assign{o}{v_2}]$ for reaction $\Delta; \ \cdot \vdash S_1 : I \to \tau$ and value $v_2 \in \int{\tau}$
\item $1[\new{l}{\tau}{\Par{\assign{o}{S_2}}{\assign{l}{v_1}}}] \sim 1[\assign{o}{S_2}]$ for value $v_1 \in \int{\tau}$ and reaction $\Delta; \ \cdot \vdash S_2 : I \to \tau$
\item $1[\new{l}{\tau}{\Par{\assign{o}{v_2}}{\assign{l}{v_1}}}] \sim 1[\assign{o}{v_2}]$ for values $v_1,v_2 \in\int{\tau}$
\end{itemize}
\item \textsc{fold-if-right}: Our desired bisimulation is the expansion of the relation $\sim$ defined by
\begin{itemize}
\item $1[\new{r}{\tau}{\Par{\assign{o}{x \leftarrow \read{b}; \ \ifte{x}{S_1}{\read{r}}}}{\assign{r}{x \leftarrow \read{b}; \ S_2}}}] \sim \\ 1[\assign{o}{x \leftarrow \read{b}; \ \ifte{x}{S_1}{S_2}}]$ for
\begin{itemize}
\item reaction $\Delta; \ \cdot \vdash S_1 : I \to \tau$
\item reaction $\Delta; \ \cdot \vdash S_2 : I \to \tau$
\end{itemize}
\item $1[\new{r}{\tau}{\Par{\assign{o}{x \leftarrow \val{v}; \ \ifte{x}{S_1}{\read{r}}}}{\assign{r}{x \leftarrow \val{v}; \ S_2}}}] \sim \\ 1[\assign{o}{x \leftarrow \val{v}; \ \ifte{x}{S_1}{S_2}}]$ for
\begin{itemize}
\item value $v \in \{0,1\}$
\item reaction $\Delta; \ \cdot \vdash S_1 : I \to \tau$
\item reaction $\Delta; \ \cdot \vdash S_2 : I \to \tau$
\end{itemize}
\item $1[\new{r}{\tau}{\Par{\assign{o}{\read{r}}}{\assign{r}{S_2}}}] \sim 1[\assign{o}{S_2}]$ for reaction $\Delta; \ \cdot \vdash S_2 : I \to \tau$
\item $1[\new{r}{\tau}{\Par{\assign{o}{S_1}}{\assign{r}{S_2}}}] \sim 1[\assign{o}{S_1}]$ for reactions $\Delta; \ \cdot \vdash S_1 : I \to \tau$ and $\Delta; \ \cdot \vdash S_2 : I \to \tau$
\item $1[\new{r}{\tau}{\Par{\assign{o}{S_1}}{\assign{r}{v_2}}}] \sim 1[\assign{o}{S_1}]$ for reaction $\Delta; \ \cdot \vdash S_1 : I \to \tau$ and value $v_2 \in \int{\tau}$
\item $1[\new{r}{\tau}{\Par{\assign{o}{v_1}}{\assign{r}{S_2}}}] \sim 1[\assign{o}{v_1}]$ for value $v_1 \in \int{\tau}$ and reaction $\Delta; \ \cdot \vdash S_2 : I \to \tau$
\item $1[\new{r}{\tau}{\Par{\assign{o}{v_1}}{\assign{r}{v_2}}}] \sim 1[\assign{o}{v_1}]$ for values $v_1, v_2 \in \int{\tau}$
\end{itemize}
\item \textsc{fold-bind}: Our desired bisimulation is the expansion of the relation $\sim$ defined by
\begin{itemize}
\item $1[\new{c}{\sigma}{\Par{\assign{o}{x \leftarrow \read{c};} \ S}{\assign{c}{R}}}] \sim 1[\assign{o}{x \leftarrow R; \ S}]$ for
\begin{itemize}
\item reaction $\Delta; \ \cdot \vdash R : I \to \sigma$
\item reaction $\Delta; \ x : \sigma \vdash S : I \to \tau$
\end{itemize}
\item $1[\new{c}{\sigma}{\Par{\assign{o}{S}}{\assign{c}{u}}}] \sim 1[\assign{o}{S}]$ for value $u \in \int{\sigma}$ and reaction $\Delta; \ \cdot \vdash S : I \to \tau$
\item $1[\new{c}{\sigma}{\Par{\assign{o}{v}}{\assign{c}{u}}}] \sim 1[\assign{o}{u}]$ for values $u \in \int{\sigma}$ and $v \in \int{\tau}$
\end{itemize}
\item \textsc{subst}: Let $\sim$ be the reaction bisimulation obtained from the premise
\[\Delta; \ \cdot \vdash \big(x_1 \leftarrow R_1; \ x'_1 \leftarrow R_1; \ \ret{(x_1,x'_1)}\big) = \big(x_1 \leftarrow R_1; \ \ret{(x_1,x_1)}\big) : I \to \tau_1 \times \tau_1\]
Our desired bisimulation is the expansion of the relation $\sim_\mathsf{subst}$ defined by
\begin{itemize}
\item $\big(\Par{\assign{o_1}{\eta}}{\assign{o_2}{x_1 \leftarrow \read{o_1}; \ R_2}}\big) \sim_\mathsf{subst} \big(\Par{\assign{o_1}{\eta}}{\assign{o_2}{x_1 \leftarrow \eta; \ R_2}}\big)$ for
\begin{itemize}
\item distribution $\eta$ on reactions $\Delta; \ \cdot \vdash R_1 : I \to \tau_1$
\item reaction $\Delta; \ \cdot \vdash R_1 : I \to \tau_1$ such that $\eval{R_1} = \eval{\eta}$
\item reaction $\Delta; \ x_1 : \tau_1 \vdash R_2 : I \to \tau_2$
\end{itemize}
such that $1[x_1 \leftarrow R_1; \ x_1' \leftarrow R_1; \ \ret{(x_1,x_1')}] \sim 1[x_1 \leftarrow R_1; \ \ret{(x_1,x_1)}]$
\item $1[\Par{\assign{o_1}{v_1}}{\assign{o_2}{R_2}}] \sim_\mathsf{subst} 1[\Par{\assign{o_1}{v_1}}{\assign{o_2}{R_2}}]$ for value $v_1 \in \int{\tau_1}$ and reaction $\Delta; \ \cdot \vdash R_2 : I \to \tau_2$
\item $1[\Par{\assign{o_1}{v_1}}{\assign{o_2}{v_2}}] \sim_\mathsf{subst} 1[\Par{\assign{o_1}{v_1}}{\assign{o_2}{v_2}}]$ for values $v_1 \in \int{\tau_1}$ and $v_2 \in \int{\tau_2}$
\end{itemize}
\item \textsc{drop}: Let $\sim$ be the reaction bisimulation obtained from the premise
\[\Delta; \ \cdot \vdash \big(x_1 \leftarrow R_1; \ R_2\big) = R_2 : I \to \tau_2\]
Our desired bisimulation is the expansion of the relation $\sim_{\mathsf{drop}}$ defined by
\begin{itemize}
\item $\big(\Par{\assign{o_1}{\eta_1}}{\assign{o_2}{x_1 \leftarrow \read{o_1}; \ R_2}}\big) \sim_{\mathsf{drop}} \big(\Par{\assign{o_1}{\eta_1}}{\assign{o_2}{\eta_2}}\big)$ for
\begin{itemize}
\item distribution $\eta_1$ on reactions $\Delta; \ \cdot \vdash R_1 : I \to \tau_1$
\item reaction $\Delta; \ \cdot \vdash R_1 : I \to \tau_1$ such that either
\begin{itemize}
\item[\emph{i)}] $\eval{R_1} = \eval{\eta_1}$, or
\item[\emph{ii)}] $\eval{R_1} = c_1 (\eval{\eta_1}) + c_2 \mu$ for some distribution $\mu$ and some $c_1,c_2 > 0$ with $c_1 + c_2 = 1$
\end{itemize}
\item distribution $\eta_2$ on reactions $\Delta; \ \cdot \vdash R_2 : I \to \tau_2$
\item reaction $\Delta; \ \cdot \vdash R_2 : I \to \tau_2$ such that $\eval{R_2} = \eval{\eta_2}$
\end{itemize}
such that $1[x_1 \leftarrow R_1; \ R_2] \sim 1[R_2]$
\item $1[\Par{\assign{o_1}{v_1}}{\assign{o_2}{R_2}}] \sim_{\mathsf{drop}} 1[\Par{\assign{o_1}{v_1}}{\assign{o_2}{R_2}}]$ for value $v_1 \in \int{\tau_1}$ and reaction $\Delta; \ \cdot \vdash R_2 : I \to \tau_2$
\item $1[\Par{\assign{o_1}{v_1}}{\assign{o_2}{v_2}}] \sim_{\mathsf{drop}} 1[\Par{\assign{o_1}{v_1}}{\assign{o_2}{v_2}}]$ for values $v_1 \in \int{\tau_1}$ and $v_2 \in \int{\tau_2}$
\end{itemize}
\end{itemize}
\end{proof}

\begin{figure*}
\begin{mathpar}
\inferrule*[right=fold-if-left]{b \neq o \\ b \in I \\ b : \Bool, o : \tau \in \Delta \\ \Delta; \ \cdot \vdash S_1 : I \to \tau \\ \Delta; \ \cdot \vdash S_2 : I \to \tau}{\Delta \vdash \big(\new{l}{\tau}{\Par{\assign{o}{x \leftarrow \read{b}; \ \ifte{x}{{\color{red} \read{l}}}{S_2}}}{{\color{red} \assign{l}{x \leftarrow \read{b}; \ S_1}}}}\big) = \\ \big(\assign{o}{x \leftarrow \read{b}; \ \ifte{x}{{\color{red} S_1}}{S_2}}\big) : I \ \setminus \ \{o\} \to \{o\}\hspace{-23pt}}\and
\inferrule*[right=fold-if-right]{b \neq o \\ b \in I \\ b : \Bool, o : \tau \in \Delta \\ \Delta; \ \cdot \vdash S_1 : I \to \tau \\ \Delta; \ \cdot \vdash S_2 : I \to \tau}{\Delta \vdash \big(\new{r}{\tau}{\Par{\assign{o}{x \leftarrow \read{b}; \ \ifte{x}{S_1}{{\color{red} \read{r}}}}}{{\color{red} \assign{r}{x \leftarrow \read{b}; \ S_2}}}}\big) = \\ \big(\assign{o}{x \leftarrow \read{b}; \ \ifte{x}{S_1}{{\color{red} S_2}}}\big) : I \ \setminus \ \{o\} \to \{o\}\hspace{-21pt}}
\end{mathpar}
\caption{Alternative formulation of the \textsc{fold-if-left} and \textsc{fold-if-right} rules.}
\label{fig:fold_if_alt}
\end{figure*}

\noindent The remainder of this section is devoted to proving the following lemma:

\begin{lemma}[Composability for bisimulations]\label{lem:composability_exact}
Let $\sim$ be a bisimulation on protocols of type $\Delta \vdash I \cup O_2 \to O_1$. Then the expansion of the relation $\sim_{\mathsf{par}}$ defined by
\begin{itemize}
\item $(\Par{\eta}{Q}) \sim_{\mathsf{par}} (\Par{\eta'}{Q})$ for $\eta \sim \eta'$ and protocol $\Delta \vdash Q : I \cup O_1 \to O_2$
\end{itemize}
is again a protocol bisimulation.
\end{lemma}

The one property hard to verify is expansion closure under computation: \emph{for any protocol $\Delta \vdash Q : I \cup O_1 \to O_2$ and any distributions $\eta \sim \eta'$, we have $\eval{\big(\Par{\eta}{Q}\big)} \;\; \lift{\sim_\mathsf{par}} \;\; \eval{\big(\Par{\eta'}{Q}\big)}$}. The difficulty arises from the global nature of the protocol semantics: in the composition $\Par{P}{Q}$, a step of the form $P \outstep{o}{v} P'$ changes the protocol $Q$ (specifically to $Q[\assign{\read{o}}{\val{v}}]$). This makes it hard to express the computation of $\Par{P}{Q}$ in terms of the computation of $P$, because in the course of the latter we are simultaneously probabilistically updating $Q$.

We solve this problem by defining an alternate \emph{local} form of operational semantics for \ipdl protocols, where we can use local values (that is, an assignment of the form $\assign{o}{\val{v}}$ rather than $\assign{o}{v}$) to compute in $P$ without having to update $Q$ at the same time. The relation $P \pull{o}{v} Q$ (``\emph{pull}'') for protocols formalizes this notion.
\begin{mathpar}
\inferrule*[right=pull-react]{ }{\big(\assign{o}{\val{v}}\big) \pull{o}{v} \big(\assign{o}{\val{v}}\big)}\and
\inferrule*[right=pull-comp-left]{P \pull{o}{v} P'}{\Par{P}{Q} \pull{o}{v} \Par{P'}{Q[\assign{\read{o}}{\val{v}}]}}\and
\inferrule*[right=pull-comp-right]{Q \pull{o}{v} Q'}{\Par{P}{Q} \pull{o}{v} \Par{P[\assign{\read{o}}{\val{v}}]}{Q'}}\and
\inferrule*[right=pull-new]{P \pull{o}{v} P' \\ o \neq c}{\big(\new{c}{\tau}{P}\big) \pull{o}{v} \big(\new{c}{\tau}{P'}\big)}\\\\
\end{mathpar}
The only difference between $P \pull{o}{v} Q$ and $P \outstep{o}{v} Q$ is that the latter turns the assignment $\assign{o}{\val{v}}$ into $\assign{o}{v}$. We extract this simple extra step into the dual relation $P \localassign{o}{v} Q$ (``\emph{local assign}''), which, crucially, does not involve any manipulation of $\mathsf{read}$s.
\begin{mathpar}
\inferrule*[right=loc-assign-react]{ }{\big(\assign{o}{\val{v}}\big) \localassign{o}{v} \big(\assign{o}{v}\big)}\and
\inferrule*[right=loc-assign-comp-left]{P \localassign{o}{v} P'}{\big(\Par{P}{Q}\big) \localassign{o}{v} \big(\Par{P'}{Q}\big)}\and
\inferrule*[right=loc-assign-comp-right]{Q \localassign{o}{v} Q'}{\big(\Par{P}{Q}\big) \localassign{o}{v} \big(\Par{P}{Q'}\big)}\and
\inferrule*[right=loc-assign-new]{P \localassign{o}{v} P' \\ o \neq c}{\big(\new{c}{\tau}{P}\big) \localassign{o}{v} \big(\new{c}{\tau}{P'}\big)}
\end{mathpar}
The big-step form $P \Rightarrow \eta$ of our local operational semantics strings together a sequence of internal and pull steps.
\begin{mathpar}
\inferrule*{}{ }\and
\inferrule*{ }{P \Rightarrow 1[P]}\and
\inferrule*{P \pull{o}{v} Q \\ Q \Rightarrow \eta}{P \Rightarrow \eta}\and \\
\inferrule*{P \to \sum_i c_i * 1[P_i] \\ P_i \Rightarrow \eta_i}{P \Rightarrow \sum_i c_i * \eta_i}
\end{mathpar}
To bridge the gap between the local and the global versions of our operational semantics, we use the big-step form $P \downmapsto Q$, which strings together a sequence of local assign steps \emph{that coincide with output steps}.
\begin{mathpar}
\inferrule*{ }{P \downmapsto P}\and
\inferrule*{P \localassign{o}{v} P' \\ P \outstep{o}{v} P' \\ P' \downmapsto Q}{P \downmapsto Q}
\end{mathpar}
So if $P \downmapsto Q$ then $Q$ is a computation of $P$ that has been obtained chiefly by performing local assign steps.

\begin{lemma}[Lifting]
For any protocol $\Delta \vdash P : I \to O$ we have the following:
\begin{itemize}
\item If $P \localassign{o_1}{v_1} Q$ and $Q \outstep{o_2}{v_2} Q'$ then there is $P'$ such that $P \outstep{o_2}{v_2} P'$ and $P' \localassign{o_1}{v_1} Q'$.
\item If $P \localassign{o}{v} P'$ and $P' \to \eta'$ then there is $\eta$ such that $P \to \eta$ and $\eta \localassign{o}{v} \eta'$.
\item If $P \localassign{o_1}{v_1} Q$ and $P \outstep{o_1}{v_1} Q$, and
$Q \localassign{o_2}{v_2} Q'$ and $Q \outstep{o_2}{v_2} Q'$, then there is $P'$ such that \\ $P \localassign{o_2}{v_2} P'$ and $P \outstep{o_2}{v_2} P'$, and $P' \localassign{o_1}{v_1} Q'$ and $P' \outstep{o_1}{v_1} Q'$.
\item If $P \downmapsto Q$ and $Q \outstep{o}{v} Q'$ then there is $P'$ such that $P \outstep{o}{v} P'$ and $P' \downmapsto Q'$.
\item If $P \downmapsto P'$ and $P' \to \eta'$ then there is $\eta$ such that $P \to \eta$ and $\eta \downmapsto \eta'$.
\end{itemize}
\end{lemma}

\noindent In the above lemma, we lift the relations $\localassign{o}{v}$ and $\downmapsto$ to distributions in the natural way.

\begin{lemma}[Approximation]
For any protocol $\Delta \vdash P : I \to O$ we have the following:
\begin{itemize}
\item If $P \pull{o}{v} P'$ then there is $Q$ such that $P \outstep{o}{v} Q$ and $P' \outstep{o}{v} Q$.
\end{itemize}
\end{lemma}

\begin{lemma}[Factoring]
For any protocol $\Delta \vdash P : I \to O$ we have the following:
\begin{itemize}
\item If $P \outstep{o}{v} Q$ then there is $P'$ such that $P \pull{o}{v} P'$ and $P' \localassign{o}{v} Q$ and $P' \outstep{o}{v} Q$.
\end{itemize}
\end{lemma}

\begin{lemma}[Correctness]
For any protocol $\Delta \vdash P : I \to O$ we have the following:
\begin{itemize}
\item If $P \Rightarrow \eta$ then $\eval{P} = \eval{\eta}$.
\item If $P \downmapsto Q$ then $\eval{P} = \eval{Q}$.
\end{itemize}
\end{lemma}



\noindent The following lemma expresses the computation of $\Par{P}{Q}$ in terms of the local computation of $P$.

\begin{lemma}[Composability for local semantics]
For protocols $\Delta \vdash P : I \cup O_2 \to O_1$ and $\Delta \vdash Q : I \cup O_1 \to O_2$, if $P \Rightarrow \eta$ then $\eval{(\Par{P}{Q})} = \eval{\big(\Par{\eta}{Q}\big)}$.
\end{lemma}

\begin{lemma}[Termination]
For any protocol $\Delta \vdash P : I \to O$ there are distributions $\eta$ and $\varepsilon$ such that $P \Rightarrow \eta$ and $\eta \downmapsto \varepsilon$ and $\varepsilon$ is final.
\end{lemma}

\begin{proof}[Sketch]
We generalize the statement. \emph{Given any $n \in \nat$, any protocol $\Delta \vdash P : I \to O$, and any protocol $Q$ such that $P \downmapsto Q$ and $\snorm{Q} = n$, there are distributions $\eta$ and $\varepsilon$ such that $P \Rightarrow \eta$ and $\eta \downmapsto \varepsilon$ and $\varepsilon$ is final.} We prove this statement by induction on the structure bound $n$ using the lifting and factoring lemmas.
\end{proof}

\noindent We now have all the preliminaries necessary to prove that $\sim_\mathsf{par}$ enjoys expansion closure under computation.

\begin{proof}
We proceed by induction on the structure bound of the protocol $Q$:

\begin{claim}[Expansion closure under computation]
Given any $n \in \nat$, any protocol $\Delta \vdash Q : I \cup O_1 \to O_2$ such that $\snorm{Q} = n$, and any distributions $\eta \sim \eta'$ on protocols of type $\Delta \vdash I \cup O_2 \to O_1$, we have $\eval{\big(\Par{\eta}{Q}\big)} \;\; \lift{\sim_\mathsf{par}} \;\; \eval{\big(\Par{\eta'}{Q}\big)}$.
\end{claim}

In the remainder of this section we will work with a fixed $n \in \nat$. Since the set $O_1$ is finite, we can start off by successively applying the valuation property of the bisimulation $\sim$ to $\eta \sim \eta'$ for each output channel $o \in O_1$, until we end up with the special case when $\eta$ and $\eta'$ have the same value on each $o$. In other words, it suffices to prove the following:

\begin{claim}
Given any protocol $\Delta \vdash Q : I \cup O_1 \to O_2$ such that $\snorm{Q} = n$, and any distributions $\eta \sim \eta'$ on protocols of type $\Delta \vdash I \cup O_2 \to O_1$ such that
\begin{itemize}
\item for each channel $o \in O_1$ with $o : \tau \in \Delta$ we have $\valueat{\eta}{o} = v_\bot = \valueat{\eta'}{o}$ for some $v_\bot \in \{\bot\} \cup \int{\tau}$,
\end{itemize}
we have $\eval{\big(\Par{\eta}{Q}\big)} \;\; \lift{\sim_\mathsf{par}} \;\; \eval{\big(\Par{\eta'}{Q}\big)}$.
\end{claim}

By performing internal and pull steps on the distributions $\eta$ and $\eta'$, we can approximate their computations without changing any channel valuations. The resulting distributions will not be necessarily related by $\sim$ but that's okay: their computations will again be related, as these coincide with the computations of $\eta$ and $\eta'$, respectively. Specifically, by the correctness, composability, and termination lemmas for our local semantics it suffices to prove the following:

\begin{claim}
Given any protocol $\Delta \vdash Q : I \cup O_1 \to O_2$ such that $\snorm{Q} = n$, any distributions $\eta$ and $\eta'$ on protocols of type $\Delta \vdash I \cup O_2 \to O_1$ such that
\begin{itemize}
\item for each channel $o \in O_1$ with $o : \tau \in \Delta$ we have $\valueat{\eta}{o} = v_\bot = \valueat{\eta'}{o}$ for some $v_\bot \in \{\bot\} \cup \int{\tau}$,
\end{itemize}
and any final distributions $\varepsilon \sim \varepsilon'$ such that $\eta \downmapsto \varepsilon$ and $\eta' \downmapsto \varepsilon'$, we have $\eval{\big(\Par{\eta}{Q}\big)} \;\; \lift{\sim_\mathsf{par}} \;\; \eval{\big(\Par{\eta'}{Q}\big)}$.
\end{claim}

\noindent We now establish an analogue of the valuation property for the local semantics of protocols:

\begin{claim}[Local valuation property]
For any channel $o \in O_1$ with $o : \tau \in \Delta$, any distributions $\eta$ and $\eta'$ on protocols of type $\Delta \vdash I \cup O_2 \to O_1$ such that $\valueat{\eta}{o} = \bot = \valueat{\eta'}{o}$, and any final distributions $\varepsilon \sim \varepsilon'$ such that $\eta \downmapsto \varepsilon$ and $\eta' \downmapsto \varepsilon'$, there exist convex combinations \[\eta = \sum_i c_i * \eta_i, \;\; \eta' = \sum_i c_i * \eta'_i, \;\; \varepsilon = \sum_i c_i * \varepsilon_i, \;\; \varepsilon' = \sum_i c_i * \varepsilon'_i\]
with $c_i > 0$ and $\sum_i c_i = 1$ such that
\begin{itemize}
\item $\eta_i \downmapsto \varepsilon_i$ and $\eta'_i \downmapsto \varepsilon'_i$,
\item the respective components $\varepsilon_i \sim \varepsilon_i'$ are again related, and
\item $\localvalueat{\eta_i}{o} = v_\bot = \localvalueat{\eta'_j}{o}$ for the same $v_\bot \in \{\bot\} \cup \int{\tau}$ if and only if $i = j$.
\end{itemize}
\end{claim}

\noindent The local valuation property follows easily from the valuation property of the bisimulation $\sim$ and the fact that the relation $P \downmapsto Q$ turns local values on $o \in O$ into global ones: if $\localvalueat{P}{o} = v_\bot$ and $Q$ is final then $\valueat{Q}{o} = v_\bot$.

We can now carry out the analogous argument from earlier but for local valuation: since the set $O_1$ of output channels is finite, we can successively apply the local valuation property to $\eta$, $\eta'$ for each output channel $o$ where $\valueat{\eta}{o} = \bot = \valueat{\eta'}{o}$, until we end up with the special case when $\eta$ and $\eta'$ have the same local value on each such $o$. In other words, it suffices to prove the following:

\begin{claim}
Given any protocol $\Delta \vdash Q : I \cup O_1 \to O_2$ such that $\snorm{Q} = n$, any distributions $\eta$ and $\eta'$ on protocols of type $\Delta \vdash I \cup O_2 \to O_1$ such that
\begin{itemize}
\item for each channel $o \in O_1$ with $o : \tau \in \Delta$ we have $\valueat{\eta}{o} = v_\bot = \valueat{\eta'}{o}$ for some $v_\bot \in \{\bot\} \cup \int{\tau}$, and
\begin{itemize}
\item in the case that $\valueat{\eta}{o} = \bot = \valueat{\eta'}{o}$, we have $\localvalueat{\eta}{o} = v_\bot = \localvalueat{\eta'}{o}$ for some $v_\bot \in \{\bot\} \cup \int{\tau}$,
\end{itemize}
\end{itemize}
and any final distributions $\varepsilon \sim \varepsilon'$ such that $\eta \downmapsto \varepsilon$ and $\eta' \downmapsto \varepsilon'$, we have $\eval{\big(\Par{\eta}{Q}\big)} \;\; \lift{\sim_\mathsf{par}} \;\; \eval{\big(\Par{\eta'}{Q}\big)}$.
\end{claim}

To prove the latest claim, we let $\eta$ and $\eta'$ step simultaneously on the distinct output channels $o_1,\ldots,o_n \in O_1$ as follows, where the lifting lemma guarantees that the order in which we execute these steps is immaterial:
\begin{itemize}
\item $\eta = \mu_0 \localassign{o_1}{v_1} \mu_1 \localassign{o_2}{v_2} \ldots \localassign{o_{n-1}}{v_{n-1}} \mu_{n-1} \localassign{o_n}{v_n} \mu_n = \varepsilon$,
\item $\eta = \mu_0 \outstep{o_1}{v_1} \mu_1 \outstep{o_2}{v_2} \ldots \outstep{o_{n-1}}{v_{n-1}} \mu_{n-1} \outstep{o_n}{v_n} \mu_n = \varepsilon$, and
\item $\eta' = \mu'_0 \localassign{o_1}{v_1} \mu'_1 \localassign{o_2}{v_2} \ldots \localassign{o_{n-1}}{v_{n-1}} \mu'_{n-1} \localassign{o_n}{v_n} \mu'_n = \varepsilon'$,
\item $\eta' = \mu'_0 \outstep{o_1}{v_1} \mu'_1 \outstep{o_2}{v_2} \ldots \outstep{o_{n-1}}{v_{n-1}} \mu'_{n-1} \outstep{o_n}{v_n} \mu'_n = \varepsilon'$.
\end{itemize}

The valuation assumptions on $\eta$ and $\eta'$, and consequently on $\mu_i$ and $\mu'_i$, guarantee that the corresponding steps $\mu_i \outstep{o_{i+1}}{v_{i+1}} \mu_{i+1}$ and $\mu'_i \outstep{o_{i+1}}{v_{i+1}} \mu'_{i+1}$ exert the same effect on the common context, thereby yielding the same sequence $Q = Q_0, \ldots, Q_n$ of updates: we have
\begin{align*}
\Par{\mu_i}{Q_i} & \outstep{o_{i+1}}{v_{i+1}} \Par{\mu_{i+1}}{Q_i[\read{o_{i+1}} \coloneqq \val{v_{i+1}}]} \\
\Par{\mu'_i}{Q_i} & \outstep{o_{i+1}}{v_{i+1}} \Par{\mu'_{i+1}}{Q_i[\read{o_{i+1}} \coloneqq \val{v_{i+1}}]}
\end{align*}
and hence $Q_{i+1} \coloneqq Q_i[\read{o_{i+1}} \coloneqq \val{v_{i+1}}]$. It thus suffices to prove the following: 

\begin{claim}
Given any protocol $\Delta \vdash Q : I \cup O_1 \to O_2$ with $\snorm{Q} = n$, and any final distributions $\eta \sim \eta'$, we have $\eval{\big(\Par{\eta}{Q}\big)} \;\; \lift{\sim_\mathsf{par}} \;\; \eval{\big(\Par{\eta'}{Q}\big)}$.
\end{claim}

At last we have the opportunity to use the induction hypothesis: either $Q$ is itself final, in which case the claim follows at once from the definition of $\sim_\mathsf{par}$, or $Q$ takes a step using one of the stepping relations $\to$ and $\outstep{o}{v}$, in which case its structure bound becomes strictly smaller and the induction hypothesis applies.
\end{proof}

\section{Computational Semantics of IPDL}
When using Turing Machines to compute (probabilistic) functions, we only consider (probabilistic) Turing Machines that have a finite runtime $N \in \nat$; \emph{i.e.}, for every input in the domain, after $N$ (probabilistic) transitions the TM ends up in a configuration where no further transitions are possible. That is, the TM has either reached an accepting state or it has halted after reading a symbol for which no transition is possible in the current state.

Intuitively, a family of interpretations is PPT (probabilistic polynomial-time) if it assigns polynomial lengths to type symbols $\type$, and PPT-computable functions to function symbols $\func$ and distribution symbols $\dist$. A small caveat is that a random distribution on a subset $S \subseteq \{0,1\}^n$ of bitstrings is in general computable by a probabilistic Turing Machine only up to a small error $\varepsilon$, which is the probability that the TM does not end up in an accepting state. In effect, the TM computes a distribution $\mu$ on $S \cup \{\bot\}$ with $\mu(\bot) = \varepsilon$. To relate $\mu$ to our original distribution on $S$, we introduce the following:

\begin{definition}[Approximating distributions]
Let $S \subseteq \{0,1\}^n$ be a subset of bitstrings of a fixed length. We say that a distribution $\mu_1$ on $S \cup \{\bot\}$ \emph{approximates} a distribution $\mu_2$ on $S$ with error $0 \leq \varepsilon \leq 1$ if there are distributions $\eta_1, \eta_2$ on $S$ such that $\mu_1 = (1 - \varepsilon) \eta_1 + \varepsilon 1[\bot]$ and $\mu_2 = (1 - \varepsilon) \eta_1 + \varepsilon \eta_2$.
\end{definition}

\noindent We recall that a function $\varepsilon : \nat \to \rat_{\geq 0}$ is negligible if it is eventually smaller than the inverse of any polynomial: \emph{for any $n \in \nat$, there exists $N \in \nat$ such that for all $\lambda \geq N$ we have $\varepsilon(\lambda) \leq \frac{1}{\lambda^n}$}.

\begin{definition}[PPT family of interpretations]\label{def:ppt-interpretation}
Let $\Sigma$ be an \ipdl signature. A family $\big\{\int{-}_\lambda\big\}_{\lambda \in \nat}$ of interpretations for $\Sigma$ is \emph{probabilistic polynomial-time (PPT)} if there is a polynomial $p(\lambda)$, a negligible function $\eta(\lambda)$, and a natural number $N \in \nat$ such that the following holds:
\begin{itemize}
\item For all type symbols $\type$, $|\type|_\lambda \leq p(\lambda)$ if $\lambda \geq N$.

\item For all function symbols $\func : \sigma \to \tau$, the function $\int{\func}_\lambda$ from bitstrings $\int{\sigma}_\lambda$ to bitstrings $\int{\tau}_\lambda$ is computable by a deterministic Turing Machine $\TM_\lambda$ with symbols $\mathsf{0}, \mathsf{1}$. Both the number of states and the runtime of $\TM_\lambda$ are $\leq p(\lambda)$ if $\lambda \geq N$.

\item For all distribution symbols $\dist : \sigma \twoheadrightarrow \tau$, the function $\int{\dist}_\lambda$ from bitstrings $\int{\sigma}_\lambda$ to distributions on bitstrings $\int{\tau}_\lambda$ is computable up to an error $\eta(\lambda)$ by a probabilistic Turing Machine $\TM_\lambda$ with symbols $\mathsf{0}, \mathsf{1}$. Specifically, for every $v \in \int{\sigma}_\lambda$, the distribution $\TM_\lambda(v)$ on $\int{\tau}_\lambda \cup \{\bot\}$ approximates $\int{\dist}_\lambda(v)$ with error $\leq \eta(\lambda)$. Both the number of states and the runtime of $\TM_\lambda$ are $\leq p(\lambda)$ if $\lambda \geq N$.
\end{itemize}
\end{definition}

To seamlessly account for the possible renaming of channel names, a distinguisher for protocols of type $\Delta \vdash I \to O$ is allowed to operate in a larger context $\Delta'$ that subsumes the original context $\Delta$ via an embedding $\phi : \Delta'\to \Delta$. In this larger context, we specify the adversarial input channels $I'$ that the distinguisher will query for a value, and the adversarial output channels $O'$ that the distinguisher will assign values to. The adversarial inputs $I'$ will be a subset of the protocol outputs $O$, appropriately translated along $\phi$. Dually, the protocol inputs $I$, appropriately translated along $\phi$, will be a subset of the adversarial outputs $O'$. In the interaction between the adversary and the protocol, every query for a value of a channel $o \in I'$ will extract the value of the channel $o \in \phi^\star(O)$ as computed by the the protocol, and pass it on to the distinguisher. Conversely, an input on channel $i \in \phi^\star(I)$ to the protocol occurs after the distinguisher computes the value of the channel $i \in O'$.

Since our distinguishers will be resource-bounded, we need to bind the number of interactions or \emph{rounds} between the distinguisher and the protocol. In each round, the adversary examines its internal state to determine the type of interaction to perform next, and steps to a new state. This transition function is a partial probabilistic function of type $\St \rightharpoonup \big(\{\bot\} \cup I' \cup O'\big) \times \St$. That is, for any internal state $s$ the adversary probabilistically decides among: \emph{1)} no interaction, coupled with stepping to a new state $s'$; \emph{2)} querying a channel $o \in I'$, coupled with stepping to a new state $s'$; \emph{3)} an assignment to a channel $i \in O'$, coupled with stepping to a new state $s'$; or \emph{4)} halting, in which case the game between the distinguisher and the protocol ends without a decision Boolean. We use this last option to capture probabilistic computations that only succeed up to a negligible error.

If a distinguisher chooses to query the channel $o \in I'$ and receives a value $v$ as a response to the query, it will update its internal state according to an input assignment function of type $\int{\tau} \times \St \to \St$, where $\tau$ is the type of the channel $o$ in $\Delta'$. That is, for any value $v \in \int{\tau}$ and any state $s$, the distinguisher steps to a new state $s'$ that records the value $v$ as a result of the query. If a distinguisher chooses a value assignment to a channel $i \in O'$, the value $v$ -- if any -- is determined by an output valuation function of type $\St \to \int{\tau} \cup \{\bot\}$, where $\tau$ is the type of the channel $i$ in $\Delta'$. After completing the designated number of rounds, the distinguisher converts its internal state to a final decision Boolean according to a function of type $\St \to \{0,1\}$.

To bind the complexity of the aforementioned operations, we implement them as Turing Machines. For convenience, we allow TMs with multiple tapes. As is standard, in the initial configuration all tapes except the first are fully blank. The internal state of the distinguisher is typically encoded as a bitstring, containing \emph{e.g.}, register values together with the sequence of instructions to be executed, if we view the distinguisher as an essentially arbitrary probabilistic program. For our purposes, it will be convenient to allow additional symbols besides $0$, $1$ on our TM tapes: when justifying the \textsc{comp-cong-left} rule of the approximate fragment of \textsc{ipdl}, we will suitably compose the distinguisher with the common context $Q$, which is an \ipdl protocol. The protocol $Q$ thus becomes integrated into the new distinguisher's code. Instead of encoding \ipdl protocols as bitstrings, we will suitably enrich our baseline set of symbols so that we can faithfully capture \ipdl code.

\begin{definition}[Distinguishers]\label{def:disting}
Fix an \ipdl signature $\Sigma$ and an interpretation $\int{-}$ for $\Sigma$. A \emph{distinguisher} $\Adv$ for protocols of type $\Delta \vdash I \to O$ is a tuple $\big(\Delta', I', O', \phi, \#_\round, \#_\tape, \Symb, \St, s_\star, \Step, \big\{\In_o\big\}_{o \, \in \, I'}, \big\{\Out_i\big\}_{i \, \in \, O'}, \Dec\big)$, where

\begin{itemize}
\item $\Delta'$ is a channel context;

\item $I' \subseteq \Delta'$ is a set of channels that the adversary can query for a value;

\item $O' \subseteq \Delta'$ is a set of channels to which the adversary can assign a value;

\item $\phi : \Delta' \to \Delta$ is an embedding of $\Delta$ into $\Delta'$;

\item $\#_\round \geq 1$ is the number of rounds the adversary will perform;

\item $\#_\tape \geq 1$ is the number of TM tapes at our disposal;

\item $\Symb$ is a finite set of additional symbols that will be used to encode the distinguisher's internal state;

\item $\St \subseteq \big(\{0,1\} \, \bigsqcup \, \Symb\big)^k$ is a set of strings of a fixed length $k \geq 1$ consisting of symbols drawn from the disjoint union of the sets $\{0,1\}$ and $\Symb$;

\item $s_\star \in \St$ is the initial state;

\item $\Step$ is a probabilistic TM that computes a partial function $\St \rightharpoonup \big(\{\bot\} \cup I' \cup O'\big) \times \St$, with $\#_\tape$-many tapes and using symbols from the set $\{0,1\} \, \bigsqcup \ \{\bot\} \, \bigsqcup \ (I' \cup O') \, \bigsqcup \, \Symb$,

\item each $\In_o$ with $o : \tau \in \Delta$ is a deterministic TM that computes a function $\St \times \int{\tau} \to \St$, with $\#_\tape$-many tapes and using symbols from the set $\{0,1\} \, \bigsqcup \, \Symb$,

\item each $\Out_{i}$ with $i : \tau \in \Delta$ is a deterministic TM that computes a function $\St \to \int{\tau} \cup \{\bot\}$, with $\#_\tape$-many tapes and using symbols from the set $\{0,1\} \, \bigsqcup \ \{\bot\} \, \bigsqcup \, \Symb$,

\item $\Dec$ is a deterministic TM that computes a function $\St \to \{0,1\}$, with $\#_\tape$-many tapes and using symbols from the set $\{0,1\} \, \bigsqcup \, \Symb$.
\end{itemize}
We furthermore require that
\begin{itemize}
\item $I' \subseteq \phi^\star(O)$, and
\item $\phi^\star(I) \subseteq O'$,
\item $\phi^\star(O) \cap O' = \emptyset$.
\end{itemize}
\end{definition}

\noindent The probabilistic Turing Machine $\Step$ that computes the transition function can terminate in a non-accepting state with probability $> 0$. We will be interested in families of distinguishers where this \emph{error} as a function of the security parameter is negligible.

\begin{definition}[Distinguisher error]
Fix a distinguisher $\Adv$ as in Definition \ref{def:disting}. We say that $\Adv$ \emph{has error up to} $\varepsilon \in \rat_{\geq 0}$, written $\err(\Adv) \leq \varepsilon$, if for any state $s \in \St$ the transition function $\Step(s)$ is undefined with probability $\leq \varepsilon$. In other words, when $\Step$ is run with the initial tape contents $s$, it halts in a non-accepting state with probability $\leq \varepsilon$.
\end{definition}

\noindent To ensure that a distinguisher does not have access to computationally expensive functions such as the discrete logarithm, we need to impose a bound on its computational resources. We will be interested in families of distinguishers where the bound as the function of the security parameter is polynomial.

\begin{definition}[Resource-bounded distinguishers]
Fix a distinguisher $\Adv$ as in Definition \ref{def:disting}. We say that $\Adv$ \emph{is bounded by} $K \in \nat$, written $|\Adv| \leq K$, if:
\begin{itemize}
\item $\#_\round$, $\#_\tape \leq K$;
\item $|I'| \leq K$ and for each $o \in I'$ with $o : \tau \in \Delta'$, we have $|\tau| \leq K$,
\item $|O'| \leq K$ and for each $i \in O'$ with $i : \tau \in \Delta'$, we have $|\tau| \leq K$,
\item $|\Symb| \leq K$;
\item the length $k$ of a state $s \in \St$ is $\leq K$;
\item the number of states of each TM $\Step, \In_o, \Out_i, \Dec$ is $\leq K$;
\item the runtime of each TM $\Step, \In_o, \Out_i, \Dec$ is $\leq K$.
\end{itemize}
\end{definition}

\noindent We note that instead of having a separate Turing Machine $\In_o$ for each channel $o \in I'$ we could have required a single Turing Machine that performs the computation across all channels in $I'$, and analogously for $\Out_i$. However, this is unnecessary as the number of channels in $I'$ and $O'$ is $\mathsf{O(poly(\lambda))}$, and the current formulation is more convenient for our purposes. We can now formally define the interaction between a distinguisher $\Adv$ and a protocol $P$.

\begin{definition}[Interaction]
Fix an \ipdl signature $\Sigma$ and an interpretation $\int{-}$ for $\Sigma$. Let $\Adv$ be a distinguisher for protocols of type $\Delta \vdash I \to O$ and let $\Delta \vdash P : I \to O$. We define $\interaction{\Adv}{P}{\int{-}}$ to be the probability sub-distribution on Booleans induced by the algorithm in Figure~\ref{fig:interaction}.
\end{definition}

\begin{figure}
\fbox{% 
\begin{minipage}{0.9 \textwidth}
\begin{center}
Algorithm $\interaction{\Adv}{P}{\int{-}}$:
\end{center}
\hrule
\begin{align*}
& s \coloneqq s_\star \\
& P \coloneqq \phi^\star(P) \\
& \mathsf{for \ } \#_\round \ \mathsf{rounds} \\
& \ \ \ \ P' \leftarrow \eval{P} \\
& \ \ \ \ (q, s') \leftarrow \Step(s) \\
& \ \ \ \ \mathsf{if} \ q = \bot \ \mathsf{then} \\
& \ \ \ \ \ \ \ \ s := s' \\
& \ \ \ \ \ \ \ \ P := P' \\
& \ \ \ \ \mathsf{else \ if} \ q = i \in O' \ \mathsf{then} \\
& \ \ \ \ \ \ \ \ \mathsf{if} \ \Out_i(s') = v \ \textit{for some $v$} \ \mathsf{then} \\
& \ \ \ \ \ \ \ \ \ \ \ \ s := s' \\
& \ \ \ \ \ \ \ \ \ \ \ \ P \coloneqq P'[\read{i} := \val{v}] \\
& \ \ \ \ \ \ \ \ \mathsf{else} \\
& \ \ \ \ \ \ \ \ \ \ \ \ s := s' \\
& \ \ \ \ \ \ \ \ \ \ \ \ P := P' \\
& \ \ \ \ \mathsf{else \ if} \ q = o \in I' \ \mathsf{then} \\
& \ \ \ \ \ \ \ \ \mathsf{if} \ (\assign{o}{v}) \in P' \ \textit{for some $v$} \ \mathsf{then} \\
& \ \ \ \ \ \ \ \ \ \ \ \ s \coloneqq \In_o(v,s') \\
& \ \ \ \ \ \ \ \ \ \ \ \ P := P' \\
& \ \ \ \ \ \ \ \ \mathsf{else} \\ 
& \ \ \ \ \ \ \ \ \ \ \ \ s := s' \\
& \ \ \ \ \ \ \ \ \ \ \ \ P := P' \\
& \mathsf{return} \ \Dec(s)
\end{align*}%
\end{minipage}
}
\caption{Interaction of an \ipdl protocol $\Delta \vdash P : I \to O$ with a distinguisher $\Adv$.}
\label{fig:interaction}
\end{figure}

In Figure~\ref{fig:interaction}, the distinguisher interacts with the protocol through the specified number of rounds. The algorithm maintains a state variable $s$ and a protocol variable $P$, which we respectively initialize to the initial state $s_\star$ and the original protocol $P$, appropriately embedded in $\Delta'$. In each round, the protocol $P$ probabilistically evolves to a new protocol $P'$. Independently, the distinguisher probabilistically computes the type of interaction $q \in \{\bot\} \cup I' \cup O'$ together with a new state $s'$ according to $\Step(s)$. If $q = \bot$, in which case no interaction takes place, the state and the protocol are updated to $s'$ and $P'$. If $q = i$ for some $i \in O'$, we compute $\Out_i(s')$ to see if in the distinguisher's current state $s'$ the channel $i$ carries a value $v$. If this is the case, we update the state to $s'$ while computing a new protocol $P'[\read{i} := \val{v}]$. Otherwise we update the state and the protocol to $s'$ and $P'$. Finally, if $q = o$ for some $o \in I'$, we examine the protocol $P'$ to see if the output channel $o$ carries a value $v$. If this is the case, we compute a new distinguisher state $\In_o(v,s')$, while updating the protocol to $P'$. Otherwise we update the state and the protocol to $s'$ and $P'$. After completing the prescribed number of rounds, we obtain a decision Boolean $\Dec(s)$ based on the distinguisher's current state.

\emph{Note}: Strictly speaking, the interaction $\interaction{\Adv}{P}{\int{-}}$ is only a \emph{sub}-distribution on Booleans, since $\Step(s)$ may halt without a result. Since the probability of this happening will be negligible, this technical point is inessential. We are now ready to give the definition of computational indistinguishability.

\begin{definition}[Computational Indistinguishability]\label{def:comp_indist}
Let $\Sigma$ be an \ipdl signature and consider a family $\big\{\int{-}_\lambda\big\}_{\lambda \in \nat}$ of interpretations for $\Sigma$. Let $\big\{\Delta_\lambda \vdash P_\lambda : I_\lambda \to O_\lambda\big\}_{\lambda \in \nat}$ and $\big\{\Delta_\lambda \vdash Q_\lambda : I_\lambda \to O_\lambda\big\}_{\lambda \in \nat}$ be two protocol families with identical typing judgments. We say that $\{P_\lambda\}$ and $\{Q_\lambda\}$ are \emph{indistinguishable} under $\big\{\int{-}_\lambda\big\}$, written
\[\big\{\int{-}_\lambda\big\}_{\lambda \in \nat} \; \mathlarger{\mathlarger{\vDash}} \; \big\{\Delta_\lambda \vdash P_\lambda \approx Q_\lambda : I_\lambda \to O_\lambda\big\}_{\lambda \in \nat}\]
if for any polynomial $p(\lambda)$ and negligible function $\eta(\lambda)$, there exists a negligible function $\varepsilon(\lambda)$ with a natural number $N \in \nat$ such that for any $\lambda \geq N$ and any distinguisher $\Adv$ for protocols of type $\Delta_\lambda \vdash I_\lambda \to O_\lambda$ with respect to the interpretation $\int{-}_\lambda$, such that $|\Adv| \leq p(\lambda)$ and $\err(\Adv) \leq \eta(\lambda)$, we have
\[\Big|\mathsf{Pr}\big[\interaction{\Adv}{P_\lambda}{\int{-}_\lambda} = 1\big] - \mathsf{Pr}\big[\interaction{\Adv}{Q_\lambda}{\int{-}_\lambda} = 1\big]\Big| \leq \varepsilon(\lambda).\]
\end{definition}

\noindent Instead of comparing the probabilities that the decision Boolean is $1$, we could have likewise used $0$: the probability $\mathsf{Pr}[b = 0]$ that the decision Boolean $b$ is $0$ is only negligibly different from $1 - \mathsf{Pr}[b = 1]$, since the probability that the game ends without a decision Boolean is negligible. This follows from the fact that the error is negligible and the number of rounds is $\mathsf{O(poly(\lambda))}$.

\section{Soundness of Approximate Equality in IPDL}
We will say that the underlying exact \ipdl theory $\mathbb{T}_=$ is sound with respect to an interpretation $\int{-}$ if each of its constituent theories $\mathbb{T}_\mathsf{exp}, \mathbb{T}_\mathsf{dist}, \mathbb{T}_\mathsf{prot}$ is sound with respect to $\int{-}$. We will use the judgement $\int{-} \vDash \mathbb{T}_=$ to denote this. We now define what it means for an asymptotic theory to be sound with respect to a family of interpretations.

\begin{definition}
Fix an \ipdl signature $\Sigma$. An approximate axiom family $\big\{\Delta_\lambda \vdash P_\lambda \approx Q_\lambda : I_\lambda \to O_\lambda\big\}_{\lambda \in \nat}$ is \emph{sound} with respect to a family of interpretations $\big\{\int{-}_\lambda\big\}_{\lambda \in \nat}$ if $\big\{\int{-}_\lambda\big\}_{\lambda \in \nat} \; \mathlarger{\mathlarger{\vDash}} \; \big\{\Delta_\lambda \vdash P_\lambda \approx Q_\lambda : I_\lambda \to O_\lambda\big\}_{\lambda \in \nat}$.
\end{definition}

\noindent The asymptotic \ipdl theory $\mathbb{T}_\approx$ is said to be sound if each of its axioms is sound, and we will utilize the judgement $\big\{\int{-}_\lambda\big\}_{\lambda \in \nat} \vDash \mathbb{T}_\approx$ to denote this. Our goal in this section is to prove that this implies overall soundness:

\begin{theorem}[Soundness of asymptotic equality of protocols]\label{thm:main}
Assume
\begin{itemize}
\item an \ipdl signature $\Sigma$ with type symbols $\type_1,\ldots,\type_{|\Sigma_\mathsf{t}|}$,

\item two protocol families $\big\{\Delta_\lambda \vdash P_\lambda : I_\lambda \to O_\lambda\big\}_{\lambda \in \nat}$ and $\big\{\Delta_\lambda \vdash Q_\lambda : I_\lambda \to O_\lambda\big\}_{\lambda \in \nat}$ with identical typing judgments,

\item a PPT family of interpretations $\big\{\int{-}_\lambda\big\}_{\lambda \in \nat}$,

\item a strict \ipdl theory $\mathbb{T}_=$ such that for each $\lambda \in \nat$, we have $\int{-}_\lambda \vDash \mathbb{T}_=$, and

\item an asymptotic \ipdl theory $\mathbb{T}_\approx$ such that $\big\{\int{-}_\lambda\big\}_{\lambda \in \nat} \vDash \mathbb{T}_\approx$.
\end{itemize}
Then
\[ \mathbb{T}_=; \, \mathbb{T}_\approx \; \mathlarger{\mathlarger{\Rightarrow}} \; \big\{\Delta_\lambda \vdash P_\lambda \approx Q_\lambda : I_\lambda \to O_\lambda\big\}_{\lambda \in \nat}
\]  
implies
\[\big\{\int{-}_\lambda\big\}_{\lambda \in \nat} \; \mathlarger{\mathlarger{\vDash}} \; \big\{\Delta_\lambda \vdash P_\lambda \approx Q_\lambda : I_\lambda \to O_\lambda\big\}_{\lambda \in \nat}.\]
\end{theorem}

We begin by restructuring our proofs of approximate equality so that all invocations of the rule \textsc{embed} are carried out first, followed by applications of the rule \textsc{input-unused}, which are in turn followed by invocations of the rule \textsc{cong-comp-left}, and finally by applications of the rule \textsc{cong-new}. Crucially, a sequence of applications of the \textsc{cong-comp-left} rule with common contexts $Q_1, \ldots, Q_n$ can be collapsed into a single application with the common context $Q_1  \; || \; \ldots \; || \; Q_n$. An analogous observation holds for a sequence of applications of the \textsc{embed} rule with embeddings $\phi_1, \ldots, \phi_n$, which can be combined into a single application with the embedding $\phi_n \circ \ldots \circ \phi_1$. The new layered form of our approximate judgements is shown in Figures \ref{fig:protocols_lcongruence_approx} and \ref{fig:protocols_lequality_approx}.

\begin{figure*}
\begin{mathpar}
\fbox{$\Delta \vdash \lapproxcong{P}{Q}{I}{O}{l}{0}$}\\
\inferrule*[right=axiom]{\Delta \vdash P \approx Q : I \to O \ \axiom}{\Delta \vdash \lapproxcong{P}{Q}{I}{O}{0}{0}}\\
\fbox{$\Delta \vdash \lapproxcong{P}{Q}{I}{O}{l}{1}$}\\
\inferrule*[right=subsume]{\Delta \vdash \lapproxcong{P}{Q}{I}{O}{l}{0}}{\Delta \vdash \lapproxcong{P}{Q}{I}{O}{l}{1}}\and
\inferrule*[right=embed]{\phi : \Delta_1 \to \Delta_2 \\ \Delta_2 \vdash \lapproxcong{P}{Q}{I}{O}{l}{0}}{\Delta_1 \vdash \lapproxcong{\phi^\star(P)}{\phi^\star(Q)}{\phi^\star(I)}{\phi^\star(O)}{l}{1}}\\
\fbox{$\Delta \vdash \lapproxcong{P}{Q}{I}{O}{l}{2}$}\\
\inferrule*[right=subsume]{\Delta \vdash \lapproxcong{P}{Q}{I}{O}{l}{1}}{\Delta \vdash \lapproxcong{P}{Q}{I}{O}{l}{2}}\and
\inferrule*[right=input-unused]{i \notin I \cup O \\ \Delta \vdash \lapproxcong{P}{Q}{I}{O}{l}{2}}{\Delta \vdash \lapproxcong{P}{Q}{I \cup \{i\}}{O}{l}{2}}\\
\fbox{$\Delta \vdash \lapproxcong{P}{Q}{I}{O}{l}{3}$}\\
\inferrule*[right=subsume]{\Delta \vdash \lapproxcong{P}{Q}{I}{O}{l}{2}}{\Delta \vdash \lapproxcong{P}{Q}{I}{O}{l}{3}}\and
\inferrule*[right=cong-comp-left]{\Delta \vdash \lapproxcong{P}{P'}{I \cup O_2}{O_1}{l}{2} \\ \Delta \vdash Q : I \cup O_1 \to O_2}{\Delta \vdash \lapproxcong{\Par{P}{Q}}{\Par{P'}{Q}}{I}{O_1 \cup O_2}{l + \tmnorm{Q} + 3}{3}}\\
\fbox{$\Delta \vdash \lapproxcong{P}{Q}{I}{O}{l}{4}$}\\
\inferrule*[right=subsume]{\Delta \vdash \lapproxcong{P}{Q}{I}{O}{l}{3}}{\Delta \vdash \lapproxcong{P}{Q}{I}{O}{l}{4}}\and
\inferrule*[right=cong-new]{\Delta, o : \tau \vdash \lapproxcong{P}{P'}{I}{O \cup \{o\}}{l}{4}}{\Delta \vdash \lapproxcong{\big(\new{o}{\tau}{P}\big)}{\big(\new{o}{\tau}{P'}\big)}{I}{O}{l}{4}}\\
\fbox{$\Delta \vdash \lapproxcong{P}{Q}{I}{O}{l}{5}$}\\
\inferrule*[right=strict-subsume]{\Delta \vdash P = P' : I \to O \\ \Delta \vdash \lapproxcong{P'}{Q'}{I}{O}{l}{4} \\ \Delta \vdash Q' = Q : I \to O}{\Delta \vdash \lapproxcong{P}{Q}{I}{O}{l}{5}}
\end{mathpar}
\caption{Layered approximate congruence for \ipdl protocols.}
\label{fig:protocols_lcongruence_approx}
\end{figure*}

\begin{figure*}
\begin{mathpar}
\fbox{$\Delta \vdash \lapproxeq{P}{Q}{I}{O}{k}{l}{5}$}\\
\inferrule*[right=strict]{\Delta \vdash P = Q : I \to O}{\Delta \vdash \lapproxeq{P}{Q}{I}{O}{0}{0}{5}}\and
\inferrule*[right=approx-cong]{\Delta \vdash \lapproxcong{P}{Q}{I}{O}{l}{5}}{\Delta \vdash \lapproxeq{P}{Q}{I}{O}{1}{l}{5}}\and
\inferrule*[right=sym]{\Delta \vdash \lapproxeq{P_1}{P_2}{I}{O}{k}{l}{5}}{\Delta \vdash \lapproxeq{P_2}{P_1}{I}{O}{k}{l}{5}}\and
\inferrule*[right=trans]{\Delta \vdash \lapproxeq{P_1}{P_2}{I}{O}{k_1}{l_1}{5} \\ \Delta \vdash \lapproxeq{P_2}{P_3}{I}{O}{k_2}{l_2}{5}}{\Delta \vdash \lapproxeq{P_1}{P_3}{I}{O}{k_1 + k_2}{\max(l_1, l_2)}{5}}
\end{mathpar}
\caption{Layered approximate equality for \ipdl protocols.}
\label{fig:protocols_lequality_approx}
\end{figure*}

\begin{lemma}\label{lem:lapproxeq}
For any protocols $\Delta \vdash P : I \to O$ and $\Delta \vdash Q : I \to O$ we have $\Delta \vdash \approxeq{P}{Q}{I}{O}{k}{l}$ if and only if $\Delta \vdash \lapproxeq{P}{Q}{I}{O}{k}{l}{5}$.
\end{lemma}

\begin{proof}[Sketch]
For the right-to-left direction, $\Delta \vdash \lapproxcong{P}{Q}{I}{O}{l}{4}$ clearly implies $\Delta \vdash \approxcong{P}{Q}{I}{O}{l}$. Thus we have that $\Delta \vdash \lapproxcong{P}{Q}{I}{O}{l}{5}$ implies $\Delta \vdash \approxeq{P}{Q}{I}{O}{1}{l}$. Hence $\Delta \vdash \lapproxeq{P}{Q}{I}{O}{k}{l}{5}$ implies $\Delta \vdash \approxeq{P}{Q}{I}{O}{k}{l}$, as desired.

For the left-to-right direction, we first show that $\Delta \vdash \approxcong{P}{Q}{I}{O}{l}$ implies $\Delta \vdash \lapproxcong{P}{Q}{I}{O}{l}{5}$ by induction on the former derivation. That $\Delta \vdash \approxeq{P}{Q}{I}{O}{k}{l}$ implies $\Delta \vdash \lapproxeq{P}{Q}{I}{O}{k}{l}{5}$ now follows immediately.
\end{proof}

We can now prove a soundness theorem for approximate equality. Roughly speaking, if $P$ is approximately equal to $Q$, then the advantage that an adversary $\Adv$ has in distinguishing $P$ and $Q$ is a reasonable combination of the distinguishing advantages against each approximate axiom by an adversary whose computational resources are only slightly larger than those of the original adversary $\Adv$. We start by proving that strict equality of protocols implies perfect indistinguishability against any adversary (not just a resource-bounded one). 

%%%%%%%%%%%%%%%%%%%%%%%%%%%%%%%%%%%%%%%%%%%%%%%%%%%%%%%%%%%%%%%%%%%%%
%%%%%%%%%%%%%%%%%%%%%%%%%%%%%%%%%%%%%%%%%%%%%%%%%%%%%%%%%%%%%%%%%%%%%

\begin{lemma}[Soundness of approximate equality of protocols: Perfect indistinguishability]\label{lem:soundness_approximate_perfect}
For any \ipdl signature $\Sigma$, interpretation $\int{-}$ for $\Sigma$, strict \ipdl theory $\mathbb{T}_=$ such that $\int{-} \vDash \mathbb{T}_=$, derivation $\mathbb{T}_=; \, \Delta \vdash P = Q : I \to O$, and distinguisher $\Adv$ for protocols of type $\Delta \vdash I \to O$ under the interpretation $\int{-}$, we have
\[\Big|\mathsf{Pr}\big[\interaction{\Adv}{P}{\int{-}} = 1\big] - \mathsf{Pr}\big[\interaction{\Adv}{Q}{\int{-}} = 1\big]\Big| = 0.\]
\end{lemma}

\begin{proof}
Fix a distinguisher $\Adv$ as in Definition \ref{def:disting}. By assumption, we have a proof $\Delta \vdash P = Q : I \to O$, which means we also have a proof that $\Delta' \vdash \phi^\star(P) = \phi^\star(Q) : \phi^\star(I) \to \phi^\star(O)$. The soundness theorem for strict equality of protocols applied to this proof gives us a bisimulation $\sim$ such that $1[\phi^\star(P)] \sim 1[\phi^\star(Q)]$. Now let $\sim_\adv$ be a binary relation on sub-distributions on pairs where the first element is a distinguisher state and the second is a protocol of type $\Delta \vdash I \to O$, defined as follows:
\begin{itemize}
\item $(s,\eta) \sim_\adv (s,\varepsilon)$ if $s \in \St$ and $\eta \sim \varepsilon$, where we use a distribution in place of a protocol to indicate the obvious lifting to sub-distributions on pairs of the the aforementioned form, and
\item $1[\bot] \sim_\adv 1[\bot]$, where $\bot$ indicates that the security game between the distinguisher and the protocol halted without a decision Boolean.
\end{itemize}
Let $\mathcal{L}_{\sim_\adv}$ be the closure of $\sim_\adv$ under joint convex combinations. Explicitly, $\mathcal{L}_{\sim_\adv}$ is defined by
\[\Big(\sum_i c_i \, \eta_i\Big) \; \mathcal{L}_{\sim_\adv} \; \Big(\sum_i c_i \, \varepsilon_i\Big)\]
for coefficients $c_i > 0$ with $\sum_i c_i = 1$ and distributions $\eta_i \sim_\adv \varepsilon_i$. We now establish a loop invariant for the algorithm in Figure~\ref{fig:interaction}. Before starting the first round, the initial distributions are suitably related: by assumption, we have $1[\phi^\star(P)] \sim 1[\phi^\star(Q)]$, which means that
\[1\big[(s_\star,\phi^\star(P))\big] \; \mathcal{L}_{\sim_\adv} \; 1\big[(s_\star,\phi^\star(Q))\big]\]
as the two distributions are already related under $\sim_\adv$. Now assume that we have two sub-distributions related by $\mathcal{L}_{\sim_\adv}$. We prove that performing a single round yields sub-distributions that are again related by $\mathcal{L}_{\sim_\adv}$. It suffices to show this for the case $(s,\eta) \sim_\adv (s,\varepsilon)$, where $s \in \St$ and $\eta \sim \varepsilon$. We first compute the distributions $\eta' \coloneqq \eval{\eta}$ and $\varepsilon' \coloneqq \eval{\varepsilon}$. By definition of $\sim$ we have $\eta' \sim \varepsilon'$. Independently, we probabilistically compute the type of interaction to perform together with a new distinguisher state $s'$. If no interaction has been chosen, the resulting distributions are $(s',\eta')$ and $(s',\varepsilon')$. We have
\[(s',\eta') \; \mathcal{L}_{\sim_\adv} \; (s',\varepsilon')\]
as desired, as the two distributions are already related under $\sim_\adv$. If the interaction is an input on channel $i$, we compute $\Out_i(s')$ to see if in the distinguisher's current state $s'$ the channel $i$ carries a value. If this is not the case, the resulting distributions are $(s',\eta')$ and $(s',\varepsilon')$. Here we again have $(s',\eta') \; \mathcal{L}_{\sim_\adv} \; (s',\varepsilon')$, as desired. On the other hand, if the channel $i$ carries a value $v$, the resulting distributions are $\big(s',\eta'[\read{i} := \val{v}]\big)$ and $\big(s',\varepsilon'[\read{i} := \val{v}]\big)$. Now because $\eta' \sim \varepsilon'$, by definition of $\sim$ we have $\eta'[\read{i} := \val{v}] \sim \varepsilon'[\read{i} := \val{v}]$. Thus we have
\[\big(s',\eta'[\read{i} := \val{v}]\big) \; \mathcal{L}_{\sim_\adv} \; \big(s',\varepsilon'[\read{i} := \val{v}]\big)\]
as desired, as the two distributions are already related under $\sim_\adv$. Finally, if the interaction is a query for an output channel $o$, we recall that the valuation property of the bisimulation $\sim$ allows us to jointly partition the distributions $\eta' \sim \varepsilon'$ into a joint convex combination \[\eta' = \sum_i c_i \, \eta'_i \; \sim \, \sum_i c_i \, \varepsilon'_i = \varepsilon'\]
with $c_i > 0$ and $\sum_i c_i = 1$ such that
\begin{itemize}
\item the respective components $\eta'_i \sim \varepsilon'_i$ are again related, and
\item $\valueat{\eta'_i}{o} = v_\bot = \valueat{\varepsilon'_i}{o}$ for the same $v_\bot \in \{\bot\} \cup \int{\tau}$ where $o : \tau$ in $\Delta'$.
\end{itemize}
Therefore, it suffices to consider the respective components $\eta'_i \sim \varepsilon'_i$ with the same $v_\bot$. If $v_\bot$ is $\bot$, then the resulting distributions are $(s',\eta'_i)$ and $(s',\varepsilon'_i)$. Here we again have $(s',\eta'_i) \; \mathcal{L}_{\sim_\adv} \; (s',\varepsilon'_i)$, as desired. On the other hand, if $v_\bot$ is a value $v$, then the resulting distributions are $\big(\In_o(v,s'),\eta'_i\big)$ and $\big(\In_o(v,s'),\varepsilon'_i\big)$. Thus we have
\[\big(\In_o(v,s'),\eta'_i\big) \; \mathcal{L}_{\sim_\adv} \; \big(\In_o(v,s'),\varepsilon'_i\big)\]
as desired, as the two distributions are already related under $\sim_\adv$. This proves that after completing the required number of rounds, we end up with two sub-distributions related by $\mathcal{L}_{\sim_\adv}$. It is now easy to see that they induce the same sub-distribution on decision Booleans. It suffices to prove this for the case $(s,\eta) \sim_\adv (s,\varepsilon)$, where $s \in \St$ and $\eta \sim \varepsilon$. But the state $s$ is the same for both distributions, so the resulting distribution on decision Booleans is $1[\Dec(s)]$. This finishes the proof.
\end{proof}

%%%%%%%%%%%%%%%%%%%%%%%%%%%%%%%%%%%%%%%%%%%%%%%%%%%%%%%%%%%%%%%%%%%%%
%%%%%%%%%%%%%%%%%%%%%%%%%%%%%%%%%%%%%%%%%%%%%%%%%%%%%%%%%%%%%%%%%%%%%

We now establish soundness of approximate congruence of protocols as a sequence of lemmas, one for each level in Figure \ref{fig:protocols_lcongruence_approx}. We note that at levels $0$ -- $2$, the length $l$ of the derivation does not factor into the final distinguishing advantage because it is always $0$ by construction. If we wish to make the ambient strict theory $\mathbb{T}_=$ and the ambient approximate theory with axioms $\Delta^1 \vdash P^1 \approx Q^1 : I^1 \to O^1, \ldots, \Delta^n \vdash P^n \approx Q^n : I^n \to O^n$ explicit, we write the approximate congruence judgement as
\[\mathbb{T}_=; \, \Delta^1 \vdash P^1 \approx Q^1 : I^1 \to O^1, \ldots, \Delta^n \vdash P^n \approx Q^n : I^n \to O^n \; \mathlarger{\mathlarger{\Rightarrow}} \; \Delta \vdash \lapproxcong{P}{Q}{I}{O}{l}{i}\]
for each level $i = 0,\ldots,5$, and the approximate equality judgement as
\[\mathbb{T}_=; \, \Delta^1 \vdash P^1 \approx Q^1 : I^1 \to O^1, \ldots, \Delta^n \vdash P^n \approx Q^n : I^n \to O^n \; \mathlarger{\mathlarger{\Rightarrow}} \; \Delta \vdash \lapproxeq{P}{Q}{I}{O}{k}{l}{5}.\]

%%%%%%%%%%%%%%%%%%%%%%%%%%%%%%%%%%%%%%%%%%%%%%%%%%%%%%%%%%%%%%%%%%%%%
%%%%%%%%%%%%%%%%%%%%%%%%%%%%%%%%%%%%%%%%%%%%%%%%%%%%%%%%%%%%%%%%%%%%%

\begin{lemma}[Soundness of approximate congruence of protocols: Level 0]\label{lem:soundness_congruence_approximate_0}
For any \ipdl signature $\Sigma$, interpretation $\int{-}$ for $\Sigma$, strict \ipdl theory $\mathbb{T}_=$ such that $\int{-} \vDash \mathbb{T}_=$, and any
\begin{itemize}
\item approximate \ipdl theory with axioms $\Delta^1 \vdash P^1 \approx Q^1 : I^1 \to O^1, \ldots, \Delta^n \vdash P^n \approx Q^n : I^n \to O^n$,

\item derivation $\mathbb{T}_=; \, \Delta^1 \vdash P^1 \approx Q^1 : I^1 \to O^1, \ldots, \Delta^n \vdash P^n \approx Q^n : I^n \to O^n \; \mathlarger{\mathlarger{\Rightarrow}} \; \Delta \vdash \lapproxcong{P}{Q}{I}{O}{l}{0}$,

\item distinguisher $\Adv$ for protocols of type $\Delta \vdash I \to O$ under the interpretation $\int{-}$ for which there exist $K_\adv \in \nat$ and $\eta_\adv \in \rat_{\geq 0}$ such that $|\Adv| \leq K_\adv$ and $\err(\Adv) \leq \eta_\adv$,

\item bounds $\varepsilon^1,\ldots,\varepsilon^n \in \rat_{\geq 0}$ with the property that for any distinguisher $\Adv^i$ for protocols of type $\Delta^i \vdash I^i \to O^i$ with respect to the interpretation $\int{-}$ such that $|\Adv^i| \leq K_\adv$ and $\err(\Adv^i) \leq \eta_\adv$, we have
\[\Big|\mathsf{Pr}\big[\interaction{\Adv^i}{P^i}{\int{-}} = 1\big] - \mathsf{Pr}\big[\interaction{\Adv^i}{Q^i}{\int{-}} = 1\big]\Big| \leq \varepsilon^i,\]
\end{itemize}
we have
\[\Big|\mathsf{Pr}\big[\interaction{\Adv}{P}{\int{-}} = 1\big] - \mathsf{Pr}\big[\interaction{\Adv}{Q}{\int{-}} = 1\big]\Big| \leq \max(\varepsilon^1,\ldots,\varepsilon^n).\]
\end{lemma}

\begin{proof}
We proceed by induction on the derivation of $\cong_0$. The \textsc{axiom} rule is clear.
\end{proof}

%%%%%%%%%%%%%%%%%%%%%%%%%%%%%%%%%%%%%%%%%%%%%%%%%%%%%%%%%%%%%%%%%%%%%
%%%%%%%%%%%%%%%%%%%%%%%%%%%%%%%%%%%%%%%%%%%%%%%%%%%%%%%%%%%%%%%%%%%%%

\begin{lemma}[Soundness of approximate congruence of protocols: Level 1]\label{lem:soundness_cogruence_approximate_1}
For any \ipdl signature $\Sigma$, interpretation $\int{-}$ for $\Sigma$, strict \ipdl theory $\mathbb{T}_=$ such that $\int{-} \vDash \mathbb{T}_=$, and any
\begin{itemize}
\item approximate \ipdl theory with axioms $\Delta^1 \vdash P^1 \approx Q^1 : I^1 \to O^1, \ldots, \Delta^n \vdash P^n \approx Q^n : I^n \to O^n$,

\item derivation $\mathbb{T}_=; \, \Delta^1 \vdash P^1 \approx Q^1 : I^1 \to O^1, \ldots, \Delta^n \vdash P^n \approx Q^n : I^n \to O^n \; \mathlarger{\mathlarger{\Rightarrow}} \; \Delta \vdash \lapproxcong{P}{Q}{I}{O}{l}{1}$,

\item distinguisher $\Adv$ for protocols of type $\Delta \vdash I \to O$ under the interpretation $\int{-}$ for which there exist $K_\adv \in \nat$ and $\eta_\adv \in \rat_{\geq 0}$ such that $|\Adv| \leq K_\adv$ and $\err(\Adv) \leq \eta_\adv$,

\item bounds $\varepsilon^1,\ldots,\varepsilon^n \in \rat_{\geq 0}$ with the property that for any distinguisher $\Adv^i$ for protocols of type $\Delta^i \vdash I^i \to O^i$ with respect to the interpretation $\int{-}$ such that $|\Adv^i| \leq K_\adv$ and $\err(\Adv^i) \leq \eta_\adv$, we have
\[\Big|\mathsf{Pr}\big[\interaction{\Adv^i}{P^i}{\int{-}} = 1\big] - \mathsf{Pr}\big[\interaction{\Adv^i}{Q^i}{\int{-}} = 1\big]\Big| \leq \varepsilon^i,\]
\end{itemize}
we have
\[\Big|\mathsf{Pr}\big[\interaction{\Adv}{P}{\int{-}} = 1\big] - \mathsf{Pr}\big[\interaction{\Adv}{Q}{\int{-}} = 1\big]\Big| \leq \max(\varepsilon^1,\ldots,\varepsilon^n).\]
\end{lemma}

\begin{proof}
We proceed by induction on the derivation of $\cong_1$. The \textsc{subsume} rule follows immediately from the preceding lemma. In the case of the \textsc{embed} rule, let $\Adv \coloneqq \big(\Delta', I', O', \phi', \#_\round, \#_\tape, \Symb, \St, s_\star, \Step, \big\{\In_o\big\}_{o \, \in \, I'}, \big\{\Out_i\big\}_{i \, \in \, O'}, \Dec\big)$ be the original adversary for protocols of type $\Delta_1 \vdash \phi^\star(I) \to \phi^\star(O)$. Let
\[\Adv_\mathcal{R} \coloneqq \big(\Delta', I', O', \phi \circ \phi', \#_\round, \#_\tape, \Symb, \St, s_\star, \Step, \big\{\In_o\big\}_{o \, \in \, I'}, \big\{\Out_i\big\}_{i \, \in \, O'}, \Dec\big)\]
be the new adversary for protocols of type $\Delta_2 \vdash I \to O$. Clearly, $|\Adv_\mathcal{R}| \leq K_\adv$ and $\err(\Adv_\mathcal{R}) \leq \eta_\adv$, and
\begin{align*}
& \mathsf{Pr}\big[\interaction{\Adv_\mathcal{R}}{P}{\int{-}} = 1\big] = \mathsf{Pr}\big[\interaction{\Adv}{\phi^\star(P)}{\int{-}} = 1\big] \\
& \mathsf{Pr}\big[\interaction{\Adv_\mathcal{R}}{Q}{\int{-}} = 1\big] = \mathsf{Pr}\big[\interaction{\Adv}{\phi^\star(Q)}{\int{-}} = 1\big]
\end{align*}
To finish the proof, we invoke the preceding lemma with the premise $\Delta_2 \vdash \lapproxcong{P}{Q}{I}{O}{l}{0}$ and the new adversary $\Adv_\mathcal{R}$.
\end{proof}

%%%%%%%%%%%%%%%%%%%%%%%%%%%%%%%%%%%%%%%%%%%%%%%%%%%%%%%%%%%%%%%%%%%%%
%%%%%%%%%%%%%%%%%%%%%%%%%%%%%%%%%%%%%%%%%%%%%%%%%%%%%%%%%%%%%%%%%%%%%

\begin{lemma}[Soundness of approximate congruence of protocols: Level 2]\label{lem:soundness_congruence_approximate_2}
For any \ipdl signature $\Sigma$, interpretation $\int{-}$ for $\Sigma$, strict \ipdl theory $\mathbb{T}_=$ such that $\int{-} \vDash \mathbb{T}_=$, and any
\begin{itemize}
\item approximate \ipdl theory with axioms $\Delta^1 \vdash P^1 \approx Q^1 : I^1 \to O^1, \ldots, \Delta^n \vdash P^n \approx Q^n : I^n \to O^n$,

\item derivation $\mathbb{T}_=; \, \Delta^1 \vdash P^1 \approx Q^1 : I^1 \to O^1, \ldots, \Delta^n \vdash P^n \approx Q^n : I^n \to O^n \; \mathlarger{\mathlarger{\Rightarrow}} \; \Delta \vdash \lapproxcong{P}{Q}{I}{O}{l}{2}$,

\item distinguisher $\Adv$ for protocols of type $\Delta \vdash I \to O$ under the interpretation $\int{-}$ for which there exist $K_\adv \in \nat$ and $\eta_\adv \in \rat_{\geq 0}$ such that $|\Adv| \leq K_\adv$ and $\err(\Adv) \leq \eta_\adv$,

\item bounds $\varepsilon^1,\ldots,\varepsilon^n \in \rat_{\geq 0}$ with the property that for any distinguisher $\Adv^i$ for protocols of type $\Delta^i \vdash I^i \to O^i$ with respect to the interpretation $\int{-}$ such that $|\Adv^i| \leq K_\adv$ and $\err(\Adv^i) \leq \eta_\adv$, we have
\[\Big|\mathsf{Pr}\big[\interaction{\Adv^i}{P^i}{\int{-}} = 1\big] - \mathsf{Pr}\big[\interaction{\Adv^i}{Q^i}{\int{-}} = 1\big]\Big| \leq \varepsilon^i,\]
\end{itemize}
we have
\[\Big|\mathsf{Pr}\big[\interaction{\Adv}{P}{\int{-}} = 1\big] - \mathsf{Pr}\big[\interaction{\Adv}{Q}{\int{-}} = 1\big]\Big| \leq \max(\varepsilon^1,\ldots,\varepsilon^n).\]
\end{lemma}

\begin{proof}
We proceed by induction on the derivation of $\cong_2$. The \textsc{subsume} rule follows immediately from the preceding lemma. For the \textsc{input-unused} rule, let $\Adv \coloneqq \big(\Delta', I', O', \phi, \#_\round, \#_\tape, \Symb, \St, s_\star, \Step, \big\{\In_o\big\}_{o \, \in \, I'}, \big\{\Out_i\big\}_{i \, \in \, O'}, \Dec\big)$ be the original adversary for protocols of type $\Delta \vdash I \cup \{i\} \to O$. Then $\Adv_\mathcal{R} \coloneqq \Adv$ is also an adversary for protocols of type $\Delta \vdash I \to O$, and we have
\begin{align*}
& \mathsf{Pr}\big[\interaction{\Adv_\mathcal{R}}{P}{\int{-}} = 1\big] = \mathsf{Pr}\big[\interaction{\Adv}{P}{\int{-}} = 1\big] \\
& \mathsf{Pr}\big[\interaction{\Adv_\mathcal{R}}{Q}{\int{-}} = 1\big] = \mathsf{Pr}\big[\interaction{\Adv}{Q}{\int{-}} = 1\big]
\end{align*}
To finish the proof, we appeal to the inductive hypothesis for the premise $\Delta \vdash \lapproxcong{P}{Q}{I}{O}{l}{2}$ and the adversary $\Adv_\mathcal{R}$.
\end{proof}

%%%%%%%%%%%%%%%%%%%%%%%%%%%%%%%%%%%%%%%%%%%%%%%%%%%%%%%%%%%%%%%%%%%%%
%%%%%%%%%%%%%%%%%%%%%%%%%%%%%%%%%%%%%%%%%%%%%%%%%%%%%%%%%%%%%%%%%%%%%

\noindent To prove soundness of the approximate \textsc{comp-cong-left} rule, we need the following crucial lemma, the proof of which we defer to the end of this section.

\begin{lemma}[Absorption]\label{lem:absorption}
There exists a polynomial $\mathcal{P}(x,y,z) \geq y$ such that for any
\begin{itemize}
\item \ipdl signature $\Sigma$ with type symbols $\type_1,\ldots,\type_{|\Sigma_\mathsf{t}|}$,

\item interpretation $\int{-}$ for $\Sigma$ for which there exist $K_\sem \in \nat$ and $\eta_\sem \in \rat_{\geq 0}$ such that
\begin{itemize}
\item for all type symbols $\type$, $|\type| \leq K_\sem$,

\item for all function symbols $\func$, $\int{\func}$ is computable by a deterministic TM with symbols $\mathsf{0}, \mathsf{1}$ such that the number of states and the runtime are $\leq K_\sem$, and

\item for all distribution symbols $\dist$, $\int{\dist}$ is computable up to error $\eta_\sem$ by a probabilistic TM with symbols $\mathsf{0}, \mathsf{1}$ such that the number of states and the runtime are $\leq K_\sem$,
\end{itemize}

\item distinguisher $\Adv$ for protocols of type $\Delta \vdash I \to O_1 \cup O_2$ under the interpretation $\int{-}$ for which there exist $K_\adv \in \nat$ and $\eta_\adv \in \rat_{\geq 0}$ such that $|\Adv| \leq K_\adv$ and $\err(\Adv) \leq \eta_\adv$,

\item protocol $\Delta \vdash Q : I \cup O_1 \to O_2$,
\end{itemize}
we have a new distinguisher $\Adv_\mathcal{R}$ for protocols of type $\Delta \vdash I \cup O_2 \to O_1$ with
\[|\Adv_\mathcal{R}| \leq \mathcal{P}\big(K_\sem,K_\adv,\tmnorm{Q}(|\type_1|,\ldots,|\type_{|\Sigma_\mathsf{t}|}|)\big)\]
and $\err(\Adv_\mathcal{R}) \leq \max(\eta_\sem,\eta_\adv)$ such that for any protocol $\Delta \vdash P : I \cup O_2 \to O_1$ we have
\[\Big|\mathsf{Pr}\big[\interaction{\Adv}{\Par{P}{Q}}{\int{-}} = 1\big] - \mathsf{Pr}\big[\interaction{\Adv_\mathcal{R}}{P}{\int{-}} = 1\big]\Big| \leq \tmnorm{Q}(|\type_1|,\ldots,|\type_{|\Sigma_\mathsf{t}|}|) * \eta_\sem.\]
\end{lemma}

%%%%%%%%%%%%%%%%%%%%%%%%%%%%%%%%%%%%%%%%%%%%%%%%%%%%%%%%%%%%%%%%%%%%%
%%%%%%%%%%%%%%%%%%%%%%%%%%%%%%%%%%%%%%%%%%%%%%%%%%%%%%%%%%%%%%%%%%%%%

\begin{lemma}[Soundness of approximate equality of protocols: Level 3]\label{lem:soundness_congruence_approximate_3}
For any
\begin{itemize}
\item \ipdl signature $\Sigma$ with type symbols $\type_1,\ldots,\type_{|\Sigma_\mathsf{t}|}$,

\item interpretation $\int{-}$ for $\Sigma$ for which there exist $K_\sem \in \nat$ and $\eta_\sem \in \rat_{\geq 0}$ such that
\begin{itemize}
\item for all type symbols $\type$, $|\type| \leq K_\sem$,

\item for all function symbols $\func$, $\int{\func}$ is computable by a deterministic TM with symbols $\mathsf{0}, \mathsf{1}$ such that the number of states and the runtime are $\leq K_\sem$, and

\item for all distribution symbols $\dist$, $\int{\dist}$ is computable up to error $\eta_\sem$ by a probabilistic TM with symbols $\mathsf{0}, \mathsf{1}$ such that the number of states and the runtime are $\leq K_\sem$,
\end{itemize}

\item strict \ipdl theory $\mathbb{T}_=$ such that $\int{-} \vDash \mathbb{T}_=$,

\item approximate \ipdl theory with axioms $\Delta^1 \vdash P^1 \approx Q^1 : I^1 \to O^1, \ldots, \Delta^n \vdash P^n \approx Q^n : I^n \to O^n$,

\item derivation \[\mathbb{T}_=; \, \Delta^1 \vdash P^1 \approx Q^1 : I^1 \to O^1, \ldots, \Delta^n \vdash P^n \approx Q^n : I^n \to O^n \; \mathlarger{\mathlarger{\Rightarrow}} \; \Delta \vdash \lapproxcong{P}{Q}{I}{O}{l}{3},\]%

\item distinguisher $\Adv$ for protocols of type $\Delta \vdash I \to O$ under the interpretation $\int{-}$ for which there exist $K_\adv \in \nat$ and $\eta_\adv \in \rat_{\geq 0}$ such that $|\Adv| \leq K_\adv$ and $\err(\Adv) \leq \eta_\adv$,

\item bound $K_\length \in \nat$ such that $l(|\type_1|,\ldots,|\type_{|\Sigma_\mathsf{t}|}|) \leq K_\length$, and

\item bounds $\varepsilon^1,\ldots,\varepsilon^n \in \rat_{\geq 0}$ with the property that for any distinguisher $\Adv^i$ for protocols of type $\Delta^i \vdash I^i \to O^i$ with respect to the interpretation $\int{-}$ such that $|\Adv^i| \leq \mathcal{P}(K_\sem,K_\adv,K_\length)$ and $\err(\Adv^i) \leq \max(\eta_\sem,\eta_\adv)$, we have
\[\Big|\mathsf{Pr}\big[\interaction{\Adv^i}{P^i}{\int{-}} = 1\big] - \mathsf{Pr}\big[\interaction{\Adv^i}{Q^i}{\int{-}} = 1\big]\Big| \leq \varepsilon^i,\]
\end{itemize}
we have
\[\Big|\mathsf{Pr}\big[\interaction{\Adv}{P}{\int{-}} = 1\big] - \mathsf{Pr}\big[\interaction{\Adv}{Q}{\int{-}} = 1\big]\Big| \leq \max(\varepsilon^1,\ldots,\varepsilon^n) + 2 * K_\length * \eta_\sem.\]
\end{lemma}

\begin{proof}
We proceed by induction on the derivation of $\cong_3$. The \textsc{subsume} rule follows from Lemma \ref{lem:soundness_congruence_approximate_2} instantiated with $K_\adv \coloneqq \mathcal{P}(K_\sem,K_\adv,K_\length)$ and $\eta_\adv \coloneqq \max(\eta_\sem,\eta_\adv)$. We can do this because
\begin{align*}
& |\Adv| \leq K_\adv \leq \mathcal{P}(K_\sem,K_\adv,K_\length), \\
& \err(\Adv) \leq \eta_\adv \leq \max(\eta_\sem,\eta_\adv).
\end{align*}
Then we have
\begin{align*}
\Big|\mathsf{Pr}\big[\interaction{\Adv}{P}{\int{-}} = 1\big] - \mathsf{Pr}\big[\interaction{\Adv}{Q}{\int{-}} = 1\big]\Big| & \leq \max(\varepsilon^1,\ldots,\varepsilon^n) \leq \max(\varepsilon^1,\ldots,\varepsilon^n) + 2 * K_\length * \eta_\sem,
\end{align*}
where the first inequality is the bound we got from Lemma \ref{lem:soundness_congruence_approximate_2}.

For the \textsc{cong-comp-left} rule, we appeal to the absorption lemma to give us a new adversary $\Adv_\mathcal{R}$ for protocols of type $\Delta \vdash I \cup O_2 \to O_1$ with $|\Adv_\mathcal{R}| \leq \mathcal{P}\big(K_\sem,K_\adv,\tmnorm{Q}(|\type_1|,\ldots,|\type_{|\Sigma_\mathsf{t}|}|)\big)$ and $\err(\Adv_\mathcal{R}) \leq \max(\eta_\sem,\eta_\adv)$ such that
\begin{align*}
& \Big|\mathsf{Pr}\big[\interaction{\Adv}{\Par{P}{Q}}{\int{-}} = 1\big] - \mathsf{Pr}\big[\interaction{\Adv_\mathcal{R}}{P}{\int{-}} = 1\big]\Big| \leq \tmnorm{Q}(|\type_1|,\ldots,|\type_{|\Sigma_\mathsf{t}|}|) * \eta_\sem, \\
& \Big|\mathsf{Pr}\big[\interaction{\Adv}{\Par{P'}{Q}}{\int{-}} = 1\big] - \mathsf{Pr}\big[\interaction{\Adv_\mathcal{R}}{P'}{\int{-}} = 1\big]\Big| \leq \tmnorm{Q}(|\type_1|,\ldots,|\type_{|\Sigma_\mathsf{t}|}|) * \eta_\sem.
\end{align*}
We can now appeal to Lemma \ref{lem:soundness_congruence_approximate_2} for the premise $\Delta \vdash \lapproxeq{P}{P'}{I \cup O_2}{O_1}{k}{l}{2}$ and the adversary $\Adv_\mathcal{R}$, instantiated with $K_\adv \coloneqq \mathcal{P}(K_\sem,K_\adv,K_\length)$ and $\eta_\adv \coloneqq \max(\eta_\sem,\eta_\adv)$.
We can do this because
\[|\Adv_\mathcal{R}| \leq \mathcal{P}\big(K_\sem,K_\adv,\tmnorm{Q}(|\type_1|,\ldots,|\type_{|\Sigma_\mathsf{t}|}|)\big) \leq \mathcal{P}(K_\sem,K_\adv,K_\length),\]
where the second inequality follows from the assumption
\[\tmnorm{Q}(|\type_1|,\ldots,|\type_{|\Sigma_\mathsf{t}|}|) + 3 \leq K_\length.\]
We recall that $l$ does not appear in the above because it comes from level 2, and is thus necessarily $0$. Lemma \ref{lem:soundness_congruence_approximate_2} gives us the bound
\[\Big|\mathsf{Pr}\big[\interaction{\Adv_\mathcal{R}}{P}{\int{-}} = 1\big] - \mathsf{Pr}\big[\interaction{\Adv_\mathcal{R}}{P'}{\int{-}} = 1\big]\Big| \leq \max(\varepsilon^1,\ldots,\varepsilon^n).\]
Combining the three bounds
\begin{align*}
& \Big|\mathsf{Pr}\big[\interaction{\Adv}{\Par{P}{Q}}{\int{-}} = 1\big] - \mathsf{Pr}\big[\interaction{\Adv_\mathcal{R}}{P}{\int{-}} = 1\big]\Big| \leq \tmnorm{Q}(|\type_1|,\ldots,|\type_{|\Sigma_\mathsf{t}|}|) * \eta_\sem, \\
& \Big|\mathsf{Pr}\big[\interaction{\Adv_\mathcal{R}}{P}{\int{-}} = 1\big] - \mathsf{Pr}\big[\interaction{\Adv_\mathcal{R}}{P'}{\int{-}} = 1\big]\Big| \leq \max(\varepsilon^1,\ldots,\varepsilon^n), \\
& \Big|\mathsf{Pr}\big[\interaction{\Adv_\mathcal{R}}{P'}{\int{-}} = 1\big] - \mathsf{Pr}\big[\interaction{\Adv}{\Par{P'}{Q}}{\int{-}} = 1\big]\Big| \leq \tmnorm{Q}(|\type_1|,\ldots,|\type_{|\Sigma_\mathsf{t}|}|) * \eta_\sem
\end{align*}
yields the following:
\begin{align*}
\Big|\mathsf{Pr}\big[\interaction{\Adv}{\Par{P}{Q}}{\int{-}} = 1\big] - \mathsf{Pr}\big[\interaction{\Adv}{\Par{P'}{Q}}{\int{-}} = 1\big]\Big| & \leq \max(\varepsilon^1,\ldots,\varepsilon^n) + 2 * \tmnorm{Q}(|\type_1|,\ldots,|\type_{|\Sigma_\mathsf{t}|}|) * \eta_\sem \\ & \leq \max(\varepsilon^1,\ldots,\varepsilon^n) + 2 * K_\length * \eta_\sem.
\end{align*}
\end{proof}

%%%%%%%%%%%%%%%%%%%%%%%%%%%%%%%%%%%%%%%%%%%%%%%%%%%%%%%%%%%%%%%%%%%%%
%%%%%%%%%%%%%%%%%%%%%%%%%%%%%%%%%%%%%%%%%%%%%%%%%%%%%%%%%%%%%%%%%%%%%

\begin{lemma}[Soundness of approximate congruence of protocols: Level 4]\label{lem:soundness_congruence_approximate_4}
For any
\begin{itemize}
\item \ipdl signature $\Sigma$ with type symbols $\type_1,\ldots,\type_{|\Sigma_\mathsf{t}|}$,

\item interpretation $\int{-}$ for $\Sigma$ for which there exist $K_\sem \in \nat$ and $\eta_\sem \in \rat_{\geq 0}$ such that
\begin{itemize}
\item for all type symbols $\type$, $|\type| \leq K_\sem$,

\item for all function symbols $\func$, $\int{\func}$ is computable by a deterministic Turing Machine with symbols $\mathsf{0}, \mathsf{1}$ such that the number of states and the runtime are $\leq K_\sem$, and

\item for all distribution symbols $\dist$, $\int{\dist}_\lambda$ is computable up to an error $\eta_\sem$ by a probabilistic Turing Machine with symbols $\mathsf{0}, \mathsf{1}$ such that the number of states and the runtime are $\leq K_\sem$,
\end{itemize}

\item strict \ipdl theory $\mathbb{T}_=$ such that $\int{-} \vDash \mathbb{T}_=$,

\item approximate \ipdl theory with axioms $\Delta^1 \vdash P^1 \approx Q^1 : I^1 \to O^1, \ldots, \Delta^n \vdash P^n \approx Q^n : I^n \to O^n$,

\item derivation \[\mathbb{T}_=; \, \Delta^1 \vdash P^1 \approx Q^1 : I^1 \to O^1, \ldots, \Delta^n \vdash P^n \approx Q^n : I^n \to O^n \; \mathlarger{\mathlarger{\Rightarrow}} \; \Delta \vdash \lapproxcong{P}{Q}{I}{O}{l}{4},\]%

\item distinguisher $\Adv$ for protocols of type $\Delta \vdash I \to O$ under the interpretation $\int{-}$ for which there exist $K_\adv \in \nat$ and $\eta_\adv \in \rat_{\geq 0}$ such that $|\Adv| \leq K_\adv$ and $\err(\Adv) \leq \eta_\adv$,

\item bound $K_\length \in \nat$ such that $l(|\type_1|,\ldots,|\type_{|\Sigma_\mathsf{t}|}|) \leq K_\length$, and

\item bounds $\varepsilon^1,\ldots,\varepsilon^n \in \rat_{\geq 0}$ with the property that for any distinguisher $\Adv^i$ for protocols of type $\Delta^i \vdash I^i \to O^i$ with respect to the interpretation $\int{-}$ such that $|\Adv^i| \leq \mathcal{P}(K_\sem,K_\adv,K_\length)$ and $\err(\Adv^i) \leq \max(\eta_\sem,\eta_\adv)$, we have
\[\Big|\mathsf{Pr}\big[\interaction{\Adv^i}{P^i}{\int{-}} = 1\big] - \mathsf{Pr}\big[\interaction{\Adv^i}{Q^i}{\int{-}} = 1\big]\Big| \leq \varepsilon^i,\]
\end{itemize}
we have
\[\Big|\mathsf{Pr}\big[\interaction{\Adv}{P}{\int{-}} = 1\big] - \mathsf{Pr}\big[\interaction{\Adv}{Q}{\int{-}} = 1\big]\Big| \leq \max(\varepsilon^1,\ldots,\varepsilon^n) + 2 * K_\length * \eta_\sem.\]
\end{lemma}

\begin{proof}
We proceed by induction on the derivation of $\cong_4$. The \textsc{subsume} rule follows immediately from the preceding lemma. To prove the \textsc{cong-new} rule, let $\Adv \coloneqq \big(\Delta', I', O', \phi, \#_\round, \#_\tape, \Symb, \St, s_\star, \Step, \big\{\In_o\big\}_{o \, \in \, I'}, \big\{\Out_i\big\}_{i \, \in \, O'}, \Dec\big)$ be the original adversary for protocols of type $\Delta \vdash I \to O$. We define a new adversary $\Adv_\mathcal{R}$ for protocols of type $\Delta, o : \tau \vdash I \to O \cup \{o\}$ as follows:
\[\Adv_\mathcal{R} \coloneqq \big(\Delta'_\mathcal{R}, I'_\mathcal{R}, O'_\mathcal{R}, \phi_\mathcal{R}, \#_\round, \#_\tape, \Symb, \St, s_\star, \Step^\mathcal{R}, \big\{\In^\mathcal{R}_o\big\}_{o \, \in \, I'_\mathcal{R}}, \big\{\Out^\mathcal{R}_i\big\}_{i \, \in \, O'_\mathcal{R}}, \Dec\big).\]
In the above definition, we have the following, where we recall that each channel name stands for a de Bruijn index:
\begin{itemize}
\item $\Delta'_\mathcal{R} \coloneqq \Delta', o:\tau$ is the channel context $\Delta'$ extended with the type $\tau$,
\item $I'_\mathcal{R}$ is the set $\{o + 1 \, | \, o \in I'\}$ of de Bruijn indices in $I'$ increased by $1$,
\item $O'_\mathcal{R}$ is the set $\{i + 1 \, | \, i \in O'\}$ of de Bruijn indices in $O'$ increased by $1$,
\item $\phi_\mathcal{R} \coloneqq \phi \times \mathsf{id}_\tau$ from the context $\Delta',o:\tau$ to the context $\Delta,o:\tau$ is the extended channel embedding,
\item each $\In^\mathcal{R}_{o+1}$ is the Turing Machine $\In_o$ corresponding to the original index $o \in I'$,
\item each $\Out^\mathcal{R}_{i+1}$ is the Turing Machine $\Out_i$ corresponding to the original index $i \in O'$,
\item $\Step^\mathcal{R}$ is the Turing Machine $\Step$ with each index $o \in I'$ and $i \in O'$ renamed to $o+1$ and $i+1$, respectively.
\end{itemize}
Since all we did was rename channel indices, it is easy to see that we have
\begin{align*}
& \mathsf{Pr}\big[\interaction{\Adv_\mathcal{R}}{P}{\int{-}} = 1\big] = \mathsf{Pr}\big[\interaction{\Adv}{\big(\new{o}{\tau}{P}\big)}{\int{-}} = 1\big] \\
& \mathsf{Pr}\big[\interaction{\Adv_\mathcal{R}}{Q}{\int{-}} = 1\big] = \mathsf{Pr}\big[\interaction{\Adv}{\new{o}{\tau}{Q}}{\int{-}} = 1\big]
\end{align*}
In particular, the new adversary $\Adv_\mathcal{R}$ will never query for the newly exposed channel $o:\tau$, since this channel is hidden from the original adversary's view. To finish the proof, we appeal to the inductive hypothesis for the premise $\Delta, o : \tau \vdash \lapproxcong{P}{P'}{I}{O \cup \{o\}}{l}{4}$ and the adversary $\Adv_\mathcal{R}$.
\end{proof}

%%%%%%%%%%%%%%%%%%%%%%%%%%%%%%%%%%%%%%%%%%%%%%%%%%%%%%%%%%%%%%%%%%%%%
%%%%%%%%%%%%%%%%%%%%%%%%%%%%%%%%%%%%%%%%%%%%%%%%%%%%%%%%%%%%%%%%%%%%%

\begin{lemma}[Soundness of approximate congruence of protocols: Level 5]\label{lem:soundness_congruence_approximate_5}
For any
\begin{itemize}
\item \ipdl signature $\Sigma$ with type symbols $\type_1,\ldots,\type_{|\Sigma_\mathsf{t}|}$,

\item interpretation $\int{-}$ for $\Sigma$ for which there exist $K_\sem \in \nat$ and $\eta_\sem \in \rat_{\geq 0}$ such that
\begin{itemize}
\item for all type symbols $\type$, $|\type| \leq K_\sem$,

\item for all function symbols $\func$, $\int{\func}$ is computable by a deterministic Turing Machine with symbols $\mathsf{0}, \mathsf{1}$ such that the number of states and the runtime are $\leq K_\sem$, and

\item for all distribution symbols $\dist$, $\int{\dist}_\lambda$ is computable up to an error $\eta_\sem$ by a probabilistic Turing Machine with symbols $\mathsf{0}, \mathsf{1}$ such that the number of states and the runtime are $\leq K_\sem$,
\end{itemize}

\item strict \ipdl theory $\mathbb{T}_=$ such that $\int{-} \vDash \mathbb{T}_=$,

\item approximate \ipdl theory with axioms $\Delta^1 \vdash P^1 \approx Q^1 : I^1 \to O^1, \ldots, \Delta^n \vdash P^n \approx Q^n : I^n \to O^n$,

\item derivation \[\mathbb{T}_=; \, \Delta^1 \vdash P^1 \approx Q^1 : I^1 \to O^1, \ldots, \Delta^n \vdash P^n \approx Q^n : I^n \to O^n \; \mathlarger{\mathlarger{\Rightarrow}} \; \Delta \vdash \lapproxcong{P}{Q}{I}{O}{l}{5},\]%

\item distinguisher $\Adv$ for protocols of type $\Delta \vdash I \to O$ under the interpretation $\int{-}$ for which there exist $K_\adv \in \nat$ and $\eta_\adv \in \rat_{\geq 0}$ such that $|\Adv| \leq K_\adv$ and $\err(\Adv) \leq \eta_\adv$,

\item bound $K_\length \in \nat$ such that $l(|\type_1|,\ldots,|\type_{|\Sigma_\mathsf{t}|}|) \leq K_\length$, and

\item bounds $\varepsilon^1,\ldots,\varepsilon^n \in \rat_{\geq 0}$ with the property that for any distinguisher $\Adv^i$ for protocols of type $\Delta^i \vdash I^i \to O^i$ with respect to the interpretation $\int{-}$ such that $|\Adv^i| \leq \mathcal{P}(K_\sem,K_\adv,K_\length)$ and $\err(\Adv^i) \leq \max(\eta_\sem,\eta_\adv)$, we have
\[\Big|\mathsf{Pr}\big[\interaction{\Adv^i}{P^i}{\int{-}} = 1\big] - \mathsf{Pr}\big[\interaction{\Adv^i}{Q^i}{\int{-}} = 1\big]\Big| \leq \varepsilon^i,\]
\end{itemize}
we have
\[\Big|\mathsf{Pr}\big[\interaction{\Adv}{P}{\int{-}} = 1\big] - \mathsf{Pr}\big[\interaction{\Adv}{Q}{\int{-}} = 1\big]\Big| \leq \max(\varepsilon^1,\ldots,\varepsilon^n) + 2 * K_\length * \eta_\sem.\]
\end{lemma}

\begin{proof}
We proceed by induction on the derivation of $\cong_5$. To prove the \textsc{strict-subsume} rule, we invoke the preceding lemma with the premise $\Delta \vdash \lapproxcong{P'}{Q'}{I}{O}{l}{4}$ to give us the bound
\[\Big|\mathsf{Pr}\big[\interaction{\Adv}{P'}{\int{-}} = 1\big] - \mathsf{Pr}\big[\interaction{\Adv}{Q'}{\int{-}} = 1\big]\Big| \leq \max(\varepsilon^1,\ldots,\varepsilon^n) + 2 * K_\length * \eta_\sem.\]
Invoking Lemma \ref{lem:soundness_approximate_perfect} on the two premises $\Delta \vdash P = P' : I \to O$ and $\Delta \vdash Q' = Q : I \to O$ gives us the two respective bounds
\begin{align*}
& \Big|\mathsf{Pr}\big[\interaction{\Adv}{P}{\int{-}} = 1\big] - \mathsf{Pr}\big[\interaction{\Adv}{P'}{\int{-}} = 1\big]\Big| = 0, \\
& \Big|\mathsf{Pr}\big[\interaction{\Adv}{Q'}{\int{-}} = 1\big] - \mathsf{Pr}\big[\interaction{\Adv}{Q}{\int{-}} = 1\big]\Big| = 0.
\end{align*}
Combining the three bounds
\begin{align*}
& \Big|\mathsf{Pr}\big[\interaction{\Adv}{P}{\int{-}} = 1\big] - \mathsf{Pr}\big[\interaction{\Adv}{P'}{\int{-}} = 1\big]\Big| = 0, \\
&\Big|\mathsf{Pr}\big[\interaction{\Adv}{P'}{\int{-}} = 1\big] - \mathsf{Pr}\big[\interaction{\Adv}{Q'}{\int{-}} = 1\big]\Big| \leq \max(\varepsilon^1,\ldots,\varepsilon^n) + 2 * K_\length * \eta_\sem, \\
& \Big|\mathsf{Pr}\big[\interaction{\Adv}{Q'}{\int{-}} = 1\big] - \mathsf{Pr}\big[\interaction{\Adv}{Q}{\int{-}} = 1\big]\Big| = 0
\end{align*}
yields the following:
\begin{align*}
& \Big|\mathsf{Pr}\big[\interaction{\Adv}{P}{\int{-}} = 1\big] - \mathsf{Pr}\big[\interaction{\Adv}{Q}{\int{-}} = 1\big]\Big| \leq \max(\varepsilon^1,\ldots,\varepsilon^n) + 2 * K_\length * \eta_\sem.
\end{align*}
\end{proof}

%%%%%%%%%%%%%%%%%%%%%%%%%%%%%%%%%%%%%%%%%%%%%%%%%%%%%%%%%%%%%%%%%%%%%
%%%%%%%%%%%%%%%%%%%%%%%%%%%%%%%%%%%%%%%%%%%%%%%%%%%%%%%%%%%%%%%%%%%%%

\begin{lemma}[Soundness of approximate equality of protocols]\label{lem:soundness_equality_approximate}
For any
\begin{itemize}
\item \ipdl signature $\Sigma$ with type symbols $\type_1,\ldots,\type_{|\Sigma_\mathsf{t}|}$,

\item interpretation $\int{-}$ for $\Sigma$ for which there exist $K_\sem \in \nat$ and $\eta_\sem \in \rat_{\geq 0}$ such that
\begin{itemize}
\item for all type symbols $\type$, $|\type| \leq K_\sem$,

\item for all function symbols $\func$, $\int{\func}$ is computable by a deterministic Turing Machine with symbols $\mathsf{0}, \mathsf{1}$ such that the number of states and the runtime are $\leq K_\sem$, and

\item for all distribution symbols $\dist$, $\int{\dist}_\lambda$ is computable up to an error $\eta_\sem$ by a probabilistic Turing Machine with symbols $\mathsf{0}, \mathsf{1}$ such that the number of states and the runtime are $\leq K_\sem$,
\end{itemize}

\item strict \ipdl theory $\mathbb{T}_=$ such that $\int{-} \vDash \mathbb{T}_=$,

\item approximate \ipdl theory with axioms $\Delta^1 \vdash P^1 \approx Q^1 : I^1 \to O^1, \ldots, \Delta^n \vdash P^n \approx Q^n : I^n \to O^n$,

\item derivation \[\mathbb{T}_=; \, \Delta^1 \vdash P^1 \approx Q^1 : I^1 \to O^1, \ldots, \Delta^n \vdash P^n \approx Q^n : I^n \to O^n \; \mathlarger{\mathlarger{\Rightarrow}} \; \Delta \vdash \lapproxeq{P}{Q}{I}{O}{k}{l}{5},\]%

\item distinguisher $\Adv$ for protocols of type $\Delta \vdash I \to O$ under the interpretation $\int{-}$ for which there exist $K_\adv \in \nat$ and $\eta_\adv \in \rat_{\geq 0}$ such that $|\Adv| \leq K_\adv$ and $\err(\Adv) \leq \eta_\adv$,

\item bound $K_\length \in \nat$ such that $l(|\type_1|,\ldots,|\type_{|\Sigma_\mathsf{t}|}|) \leq K_\length$, and

\item bounds $\varepsilon^1,\ldots,\varepsilon^n \in \rat_{\geq 0}$ with the property that for any distinguisher $\Adv^i$ for protocols of type $\Delta^i \vdash I^i \to O^i$ with respect to the interpretation $\int{-}$ such that $|\Adv^i| \leq \mathcal{P}(K_\sem,K_\adv,K_\length)$ and $\err(\Adv^i) \leq \max(\eta_\sem,\eta_\adv)$, we have
\[\Big|\mathsf{Pr}\big[\interaction{\Adv^i}{P^i}{\int{-}} = 1\big] - \mathsf{Pr}\big[\interaction{\Adv^i}{Q^i}{\int{-}} = 1\big]\Big| \leq \varepsilon^i,\]
\end{itemize}
we have
\[\Big|\mathsf{Pr}\big[\interaction{\Adv}{P}{\int{-}} = 1\big] - \mathsf{Pr}\big[\interaction{\Adv}{Q}{\int{-}} = 1\big]\Big| \leq k * \big(\max(\varepsilon^1,\ldots,\varepsilon^n) + 2 * K_\length * \eta_\sem\big).\]
\end{lemma}

\begin{proof}
We proceed by induction on the derivation of $\approx_5$. The \textsc{strict} rule follows immediately from Lemma \ref{lem:soundness_approximate_perfect}. The \textsc{approx-cong} rule follows immediately from Lemma \ref{lem:soundness_equality_approximate}. The \textsc{sym} rule follows easily from the inductive hypothesis for the premise $\Delta \vdash \lapproxeq{P_1}{P_2}{I}{O}{k}{l}{5}$. For the \textsc{trans} rule, we use the inductive hypotheses for the two premises $\Delta \vdash \lapproxeq{P_1}{P_2}{I}{O}{k_1}{l_1}{5}$ and $\Delta \vdash \lapproxeq{P_2}{P_3}{I}{O}{k_2}{l_2}{5}$, both invoked with $K_\length$. We can do this because
\begin{align*}
& l_1(|\type_1|,\ldots,|\type_{|\Sigma_\mathsf{t}|}|) \leq \max\big(l_1(|\type_1|,\ldots,|\type_{|\Sigma_\mathsf{t}|}|), l_2(|\type_1|,\ldots,|\type_{|\Sigma_\mathsf{t}|}|)\big) \leq K_\length, \\
& l_2(|\type_1|,\ldots,|\type_{|\Sigma_\mathsf{t}|}|) \leq \max\big(l_1(|\type_1|,\ldots,|\type_{|\Sigma_\mathsf{t}|}|), l_2(|\type_1|,\ldots,|\type_{|\Sigma_\mathsf{t}|}|)\big) \leq K_\length.
\end{align*}
This gives us the two bounds
\begin{align*}
& \Big|\mathsf{Pr}\big[\interaction{\Adv}{P_1}{\int{-}} = 1\big] - \mathsf{Pr}\big[\interaction{\Adv}{P_2}{\int{-}} = 1\big]\Big| \leq k_1 * \big(\max(\varepsilon^1,\ldots,\varepsilon^n) + 2 * K_\length * \eta_\sem\big),\\
& \Big|\mathsf{Pr}\big[\interaction{\Adv}{P_2}{\int{-}} = 1\big] - \mathsf{Pr}\big[\interaction{\Adv}{P_3}{\int{-}} = 1\big]\Big| \leq k_2 * \big(\max(\varepsilon^1,\ldots,\varepsilon^n) + 2 * K_\length * \eta_\sem\big).
\end{align*}
Combining these yields the following:
\begin{align*}
& \Big|\mathsf{Pr}\big[\interaction{\Adv}{P_1}{\int{-}} = 1\big] - \mathsf{Pr}\big[\interaction{\Adv}{P_3}{\int{-}} = 1\big]\Big| \leq (k_1 + k_2) * \big(\max(\varepsilon^1,\ldots,\varepsilon^n) + 2 * K_\length * \eta_\sem\big).
\end{align*}
\end{proof}

%%%%%%%%%%%%%%%%%%%%%%%%%%%%%%%%%%%%%%%%%%%%%%%%%%%%%%%%%%%%%%%%%%%%%
%%%%%%%%%%%%%%%%%%%%%%%%%%%%%%%%%%%%%%%%%%%%%%%%%%%%%%%%%%%%%%%%%%%%%

\begin{corollary}[Soundness of approximate equality of protocols]\label{lem:soundness_approximate}
For any
\begin{itemize}
\item \ipdl signature $\Sigma$ with type symbols $\type_1,\ldots,\type_{|\Sigma_\mathsf{t}|}$,

\item interpretation $\int{-}$ for $\Sigma$ for which there exist $K_\sem \in \nat$ and $\eta_\sem \in \rat_{\geq 0}$ such that
\begin{itemize}
\item for all type symbols $\type$, $|\type| \leq K_\sem$,

\item for all function symbols $\func$, $\int{\func}$ is computable by a deterministic Turing Machine with symbols $\mathsf{0}, \mathsf{1}$ such that the number of states and the runtime are $\leq K_\sem$, and

\item for all distribution symbols $\dist$, $\int{\dist}_\lambda$ is computable up to an error $\eta_\sem$ by a probabilistic Turing Machine with symbols $\mathsf{0}, \mathsf{1}$ such that the number of states and the runtime are $\leq K_\sem$,
\end{itemize}

\item strict \ipdl theory $\mathbb{T}_=$ such that $\int{-} \vDash \mathbb{T}_=$,

\item approximate \ipdl theory with axioms $\Delta^1 \vdash P^1 \approx Q^1 : I^1 \to O^1, \ldots, \Delta^n \vdash P^n \approx Q^n : I^n \to O^n$,

\item derivation \[\mathbb{T}_=; \, \Delta^1 \vdash P^1 \approx Q^1 : I^1 \to O^1, \ldots, \Delta^n \vdash P^n \approx Q^n : I^n \to O^n \; \mathlarger{\mathlarger{\Rightarrow}} \; \Delta \vdash \approxeq{P}{Q}{I}{O}{k}{l},\]%

\item distinguisher $\Adv$ for protocols of type $\Delta \vdash I \to O$ under the interpretation $\int{-}$ for which there exist $K_\adv \in \nat$ and $\eta_\adv \in \rat_{\geq 0}$ such that $|\Adv| \leq K_\adv$ and $\err(\Adv) \leq \eta_\adv$,

\item bound $K_\length \in \nat$ such that $l(|\type_1|,\ldots,|\type_{|\Sigma_\mathsf{t}|}|) \leq K_\length$, and

\item bounds $\varepsilon^1,\ldots,\varepsilon^n \in \rat_{\geq 0}$ with the property that for any distinguisher $\Adv^i$ for protocols of type $\Delta^i \vdash I^i \to O^i$ with respect to the interpretation $\int{-}$ such that $|\Adv^i| \leq \mathcal{P}(K_\sem,K_\adv,K_\length)$ and $\err(\Adv^i) \leq \max(\eta_\sem,\eta_\adv)$, we have
\[\Big|\mathsf{Pr}\big[\interaction{\Adv^i}{P^i}{\int{-}} = 1\big] - \mathsf{Pr}\big[\interaction{\Adv^i}{Q^i}{\int{-}} = 1\big]\Big| \leq \varepsilon^i,\]
\end{itemize}
we have
\[\Big|\mathsf{Pr}\big[\interaction{\Adv}{P}{\int{-}} = 1\big] - \mathsf{Pr}\big[\interaction{\Adv}{Q}{\int{-}} = 1\big]\Big| \leq k * \big(\max(\varepsilon^1,\ldots,\varepsilon^n) + 2 * K_\length * \eta_\sem\big).\]
\end{corollary}

\begin{proof}
Follows immediately from Lemmas \ref{lem:lapproxeq} and \ref{lem:soundness_equality_approximate}.
\end{proof}

%%%%%%%%%%%%%%%%%%%%%%%%%%%%%%%%%%%%%%%%%%%%%%%%%%%%%%%%%%%%%%%%%%%%%
%%%%%%%%%%%%%%%%%%%%%%%%%%%%%%%%%%%%%%%%%%%%%%%%%%%%%%%%%%%%%%%%%%%%%

\noindent We are now ready to prove our main soundness theorem.

\begin{theorem}[Soundness of asymptotic equality of protocols]
Assume
\begin{itemize}
\item an \ipdl signature $\Sigma$ with type symbols $\type_1,\ldots,\type_{|\Sigma_\mathsf{t}|}$,

\item two protocol families $\big\{\Delta_\lambda \vdash P_\lambda : I_\lambda \to O_\lambda\big\}_{\lambda \in \nat}$ and $\big\{\Delta_\lambda \vdash Q_\lambda : I_\lambda \to O_\lambda\big\}_{\lambda \in \nat}$ with identical typing judgments,

\item a PPT family of interpretations $\big\{\int{-}_\lambda\big\}_{\lambda \in \nat}$,

\item a strict \ipdl theory $\mathbb{T}_=$ such that for each $\lambda \in \nat$, we have $\int{-}_\lambda \vDash \mathbb{T}_=$, and

\item an asymptotic \ipdl theory $\mathbb{T}_\approx$ such that $\big\{\int{-}_\lambda\big\}_{\lambda \in \nat} \vDash \mathbb{T}_\approx$.
\end{itemize}
Then
\[ \mathbb{T}_=; \, \mathbb{T}_\approx \; \mathlarger{\mathlarger{\Rightarrow}} \; \big\{\Delta_\lambda \vdash P_\lambda \approx Q_\lambda : I_\lambda \to O_\lambda\big\}_{\lambda \in \nat}
\]  
implies
\[\big\{\int{-}_\lambda\big\}_{\lambda \in \nat} \; \mathlarger{\mathlarger{\vDash}} \; \big\{\Delta_\lambda \vdash P_\lambda \approx Q_\lambda : I_\lambda \to O_\lambda\big\}_{\lambda \in \nat}.\]
\end{theorem}

\begin{proof}
Let $\big\{\Delta^1_\lambda \vdash P^1_\lambda \approx Q^1_\lambda : I^1_\lambda \to O^1_\lambda\big\}_{\lambda \in \nat}, \ldots, \big\{\Delta^n_\lambda \vdash P^n_\lambda \approx Q^n_\lambda : I^n_\lambda \to O^n_\lambda\big\}_{\lambda \in \nat}$ be the axioms comprising the theory $\mathbb{T}_\approx$. The top-level asymptotic equality judgement
\[ \mathbb{T}_=; \, \mathbb{T}_\approx \; \mathlarger{\mathlarger{\Rightarrow}} \; \big\{\Delta_\lambda \vdash P_\lambda \approx Q_\lambda : I_\lambda \to O_\lambda\big\}_{\lambda \in \nat}
\]
gives us functions $k = \mathsf{O}(\poly(\lambda))$ and $l = \mathsf{O}\big(\poly(\lambda,t_1,\ldots,t_{|\Sigma_\type|})\big)$ such that

\[\mathbb{T}_=; \, \Delta^1_\lambda \vdash P^1_\lambda \approx Q^1_\lambda : I^1_\lambda \to O^1_\lambda, \ldots, \Delta^n_\lambda \vdash P^n_\lambda \approx Q^n_\lambda : I^n_\lambda \to O^n_\lambda \; \mathlarger{\mathlarger{\Rightarrow}} \; \Delta_\lambda \vdash \approxeq{P_\lambda}{Q_\lambda}{I_\lambda}{O_\lambda}{k(\lambda)}{l(\lambda)} \tag{$\star$}\]%

\noindent In particular, there exists a polynomial $p_\width(\lambda)$ with an index $N_\width \in \nat$ such that $k(\lambda) \leq p_\width(\lambda)$ if $\lambda \geq N_\width$, and a polynomial $p_\length(\lambda,t_1,\ldots,t_{|\Sigma_\type|})$ with an index $N_\length \in \nat$ such that $l(\lambda,t_1,\ldots,t_{|\Sigma_\type|}) \leq p_\length(\lambda,t_1,\ldots,t_{|\Sigma_\type|})$ if $\lambda \geq N_\length$ and $t_i \geq N_\length$. Since the family $\big\{\int{-}_\lambda\big\}_{\lambda \in \nat}$ of interpretations is PPT, we have a polynomial $K_\sem(\lambda)$, a negligible function $\eta_\sem(\lambda)$, and a natural number $N_\sem \in \nat$ such that:

\begin{itemize}
\item[]
\begin{itemize}
\item \emph{for all type symbols $\type$, $|\type|_\lambda \leq K_\sem(\lambda)$ if $\lambda \geq N_\sem$,}

\item \emph{for all function symbols $\func$, $\int{\func}_\lambda$ is computable by a deterministic Turing Machine $\TM_\lambda$ with symbols $\mathsf{0}, \mathsf{1}$, and both the number of states and the runtime of $\TM_\lambda$ are $\leq K_\sem(\lambda)$ if $\lambda \geq N_\sem$, and}

\item \emph{for all distribution symbols $\dist$, $\int{\dist}_\lambda$ is computable up to an error $\eta_\sem(\lambda)$ by a probabilistic Turing Machine $\TM_\lambda$ with symbols $\mathsf{0}, \mathsf{1}$, and both the number of states and the runtime of $\TM_\lambda$ are $\leq K_\sem(\lambda)$ if $\lambda \geq N_\sem$.}
\end{itemize}
\end{itemize}

\noindent To prove
\[\big\{\int{-}_\lambda\big\}_{\lambda \in \nat} \; \mathlarger{\mathlarger{\vDash}} \; \big\{\Delta_\lambda \vdash P_\lambda \approx Q_\lambda : I_\lambda \to O_\lambda\big\}_{\lambda \in \nat},\]
we fix a polynomial $K_\adv(\lambda)$ and a negligible function $\eta_\adv(\lambda)$. Since $\mathbb{T}_\approx$ is sound under the family of interpretations $\big\{\int{-}_\lambda\big\}_{\lambda \in \nat}$, we have the computational indistinguishability assumption
\[\big\{\int{-}_\lambda\big\}_{\lambda \in \nat} \; \mathlarger{\mathlarger{\vDash}} \; \big\{\Delta^i_\lambda \vdash P^i_\lambda \approx Q^i_\lambda : I^i_\lambda \to O^i_\lambda\big\}_{\lambda \in \nat}.\]
Define
\[ K_\length(\lambda) \coloneqq p_\length\big(\lambda,K_\sem(\lambda) + N_\length,\ldots,K_\sem(\lambda) + N_\length\big),\]
and apply the computational indistinguishability assumption to the polynomial $\mathcal{P}\big(K_\sem(\lambda),K_\adv(\lambda),K_\length(\lambda)\big)$ and the negligible function $\max\big(\eta_\sem(\lambda),\eta_\adv(\lambda)\big)$. We get a negligible function $\varepsilon^i(\lambda)$ with a natural number $N^i \in \nat$ such that for any $\lambda \geq N^i$ and any distinguisher $\Adv^i$ for protocols of type $\Delta^i_\lambda \vdash I^i_\lambda \to O^i_\lambda$ under the interpretation $\int{-}_\lambda$ such that $|\Adv^i| \leq \mathcal{P}\big(K_\sem(\lambda),K_\adv(\lambda),K_\length(\lambda)\big)$ and $\err(\Adv^i) \leq \max\big(\eta_\sem(\lambda),\eta_\adv(\lambda)\big)$, we have
\[\Big|\mathsf{Pr}\big[\interaction{\Adv}{P^i_\lambda}{\int{-}_\lambda} = 1\big] - \mathsf{Pr}\big[\interaction{\Adv}{Q^i_\lambda}{\int{-}_\lambda} = 1\big]\Big| \leq \varepsilon^i(\lambda).\]%

\noindent We can now define our desired negligible function as
\[\varepsilon(\lambda) \coloneqq k(\lambda) * \big(\max(\varepsilon^1(\lambda),\ldots,\varepsilon^n(\lambda)) + 2 * K_\length(\lambda) * \eta_\sem(\lambda)\big).\]
The negligibility of $\varepsilon(\lambda)$ follows easily: if $\lambda \geq N_\width$ then
\[\varepsilon(\lambda) \leq p_\width(\lambda) * \big(\max(\varepsilon^1(\lambda),\ldots,\varepsilon^n(\lambda)) + 2 * K_\length(\lambda) * \eta_\sem(\lambda)\big),\]
so it suffices to show that this latter function is negligible. But this is immediate from the negligibility of each $\varepsilon^i(\lambda)$, the negligibility of $\eta_\sem(\lambda)$, and the fact that $p_\width(\lambda)$ and $K_\length(\lambda)$ are polynomials. Define
\[N \coloneqq \max\big(N_\length,N_\sem,N^1,\ldots,N^n\big).\]
Assume $\lambda \geq N$ and take any distinguisher $\Adv$ for protocols of type $\Delta_\lambda \vdash I_\lambda \to O_\lambda$ under the interpretation $\int{-}_\lambda$, such that $|\Adv| \leq K_\adv(\lambda)$ and $\err(\Adv) \leq \eta_\adv(\lambda)$. We aim to show that
\[\Big|\mathsf{Pr}\big[\interaction{\Adv}{P_\lambda}{\int{-}_\lambda} = 1\big] - \mathsf{Pr}\big[\interaction{\Adv}{Q_\lambda}{\int{-}_\lambda} = 1\big]\Big| \leq k(\lambda) * \big(\max(\varepsilon^1(\lambda),\ldots,\varepsilon^n(\lambda)) + 2 * K_\length(\lambda) * \eta_\sem(\lambda)\big).\]
But this is precisely the conclusion of Lemma \ref{lem:soundness_approximate} applied to the derivation $(\star)$. It thus suffices to prove the hypotheses of Lemma \ref{lem:soundness_approximate}. Among these, the only non-trivial assumption is
\[l\big(\lambda,|\type_1|,\ldots,|\type_{|\Sigma_\mathsf{t}|}|\big) \leq K_\length(\lambda).\]
We show this via the following sequence of inequalities:
\begin{align*}
l\big(\lambda,|\type_1|_\lambda,\ldots,|\type_{|\Sigma_\mathsf{t}|}|_\lambda\big) & \leq
l\big(\lambda,K_\sem(\lambda),\ldots,K_\sem(\lambda)\big) \\ & \leq
l\big(\lambda,K_\sem(\lambda) + N_\length,\ldots,K_\sem(\lambda) + N_\length\big) \\ & \leq
p_\length\big(\lambda,K_\sem(\lambda) + N_\length,\ldots,K_\sem(\lambda) + N_\length\big) \\ & = K_\length.
\end{align*}
The first inequality follows from the fact that the function $l(\lambda) : \nat^{|\Sigma_\type|} \to \nat$ is monotonically increasing in each argument and $|\type_i|_\lambda \leq K_\sem(\lambda)$ by assumption since $\lambda \geq N_\sem$. The second inequality is again monotonicity of $l(\lambda)$, and the third follows from the definition of $p_\length$ since $\lambda \geq N_\length$ and $K_\sem(\lambda) + N_\length \geq N_\length$.
\end{proof}

%%%%%%%%%%%%%%%%%%%%%%%%%%%%%%%%%%%%%%%%%%%%%%%%%%%%%%%%%%%%%%%%%%%%%
%%%%%%%%%%%%%%%%%%%%%%%%%%%%%%%%%%%%%%%%%%%%%%%%%%%%%%%%%%%%%%%%%%%%%

\noindent The remainder of this section is devoted to the proof of the absorption lemma that we promised earlier.

\begin{lemma}[Absorption]
There exists a polynomial $\mathcal{P}(x,y,z) \geq y$ such that for any
\begin{itemize}
\item \ipdl signature $\Sigma$ with type symbols $\type_1,\ldots,\type_{|\Sigma_\mathsf{t}|}$,

\item interpretation $\int{-}$ for $\Sigma$ for which there exist $K_\sem \in \nat$ and $\eta_\sem \in \rat_{\geq 0}$ such that
\begin{itemize}
\item for all type symbols $\type$, $|\type| \leq K_\sem$,

\item for all function symbols $\func$, $\int{\func}$ is computable by a deterministic TM with symbols $\mathsf{0}, \mathsf{1}$ such that the number of states and the runtime are $\leq K_\sem$, and

\item for all distribution symbols $\dist$, $\int{\dist}$ is computable up to error $\eta_\sem$ by a probabilistic TM with symbols $\mathsf{0}, \mathsf{1}$ such that the number of states and the runtime are $\leq K_\sem$,
\end{itemize}

\item distinguisher $\Adv$ for protocols of type $\Delta \vdash I \to O_1 \cup O_2$ under the interpretation $\int{-}$ for which there exist $K_\adv \in \nat$ and $\eta_\adv \in \rat_{\geq 0}$ such that $|\Adv| \leq K_\adv$ and $\err(\Adv) \leq \eta_\adv$,

\item protocol $\Delta \vdash Q : I \cup O_1 \to O_2$,
\end{itemize}
we have a new distinguisher $\Adv_\mathcal{R}$ for protocols of type $\Delta \vdash I \cup O_2 \to O_1$ with
\[|\Adv_\mathcal{R}| \leq \mathcal{P}\big(K_\sem,K_\adv,\tmnorm{Q}(|\type_1|,\ldots,|\type_{|\Sigma_\mathsf{t}|}|)\big)\]
and $\err(\Adv_\mathcal{R}) \leq \max(\eta_\sem,\eta_\adv)$ such that for any protocol $\Delta \vdash P : I \cup O_2 \to O_1$ we have
\[\Big|\mathsf{Pr}\big[\interaction{\Adv}{\Par{P}{Q}}{\int{-}} = 1\big] - \mathsf{Pr}\big[\interaction{\Adv_\mathcal{R}}{P}{\int{-}} = 1\big]\Big| \leq \tmnorm{Q}(|\type_1|,\ldots,|\type_{|\Sigma_\mathsf{t}|}|) * \eta_\sem.\]
\end{lemma}

\noindent To encode \ipdl protocols on a Turing Machine tape, we will make use of the following sets of symbols:
\begin{itemize}
\item \textsf{Punc} with symbols \textsf{``$\langle$''}, \textsf{``$\rangle$''}, \textsf{``(''}, \textsf{``)''}, \textsf{``$\{$''}, \textsf{``$\}$''}, \textsf{``[''}, \textsf{`]''}, \textsf{``$\_$''}, \textsf{``:''}, \textsf{``$\cdot$''}, \textsf{``;''}, \textsf{``$\to$''}, \textsf{``$\twoheadrightarrow$''}, \textsf{``$\leftarrow$''}, \textsf{``$\times$''}, \textsf{``$\coloneqq$''}, \textsf{``$\|$''},
\item \textsf{KeyWords} with symbols \textsf{``var''}, \textsf{``$\checkmark$''}, \textsf{``true''}, \textsf{``false''}, \textsf{``app''}, \textsf{``fst''}, \textsf{``snd''}, \textsf{``of''}, \textsf{``val''}, \textsf{``ret''}, \textsf{``samp''}, \textsf{``read''}, \textsf{``input-to-query''}, \textsf{``input-queried''}, \textsf{``input-not-to-query''}, \textsf{``if''}, \textsf{``then''}, \textsf{``else''}, \textsf{``0''}, \textsf{``new''}, \textsf{``in''}, \textsf{``wen''}.
\end{itemize}

\noindent We will also need a finite set of de Bruijn indices in lieu of channel and variable names. To derive an upper bound on how many indices we will need, we statically count the maximum depth of variable and channel declarations, giving us a \emph{variable-index bound} and a \emph{channel-index bound}, respectively. The variable-index bound for reactions is invariant under substitutions, embeddings, and input assignment.
\begin{align*}
\bnorm{\val{v}} & \coloneqq 0 \\
\bnorm{\ret{e}} & \coloneqq 0 \\
\bnorm{\samp{\dist}{e}} & \coloneqq 0 \\
\bnorm{\read{c}} & \coloneqq 0 \\
\bnorm{\ifte{e}{R_1}{R_2}} & \coloneqq \max(\bnorm{R_1}, \bnorm{R_2}) \\
\bnorm{x \leftarrow R; \ S} & \coloneqq \max(\bnorm{R}, \bnorm{S} + 1)
\end{align*}
The variable-index bound for protocols is invariant under embeddings and input assignment.
\begin{align*}
\bnorm{\zero} & \coloneqq 0 \\
\bnorm{\assign{o}{v}} & \coloneqq 0 \\
\bnorm{\assign{o}{R}} & \coloneqq \bnorm{R} \\
\bnorm{\Par{P}{Q}} & \coloneqq \max(\bnorm{P}, \bnorm{Q}) \\
\bnorm{\new{c}{\tau}{P}} & \coloneqq \bnorm{P}
\end{align*}
Likewise, the channel-index bound for protocols is invariant under embeddings and input assignment.
\begin{align*}
\nnorm{\zero} & \coloneqq 0 \\
\nnorm{\assign{o}{v}} & \coloneqq 0 \\
\nnorm{\assign{o}{R}} & \coloneqq 0 \\
\nnorm{\Par{P}{Q}} & \coloneqq \max(\nnorm{P}, \nnorm{Q}) \\
\nnorm{\new{c}{\tau}{P}} & \coloneqq \nnorm{P} + 1
\end{align*}
To avoid an infinite loop, an adversary executing the absorbed protocol will need to keep track of which channels have already been queried for a value. We store this information inside the protocol in the form of an annotation: for each channel read $\Read{c}{\tau}$, we denote whether the channel $c$ has already been queried for a value, if applicable:

\begin{syntax}
  \category[Query Annotations]{a}
	  \alternative{\textsf{input-to-query}}
		\alternative{\textsf{input-queried}}
		\alternative{\textsf{input-not-to-query}}

  \category[Query-Annotated Reactions]{R, S}
    \alternative{\ldots}
    \alternative{\QAnnRead{c}{\tau}{a}}
    \alternative{\ldots}

	\category[Query-Annotated Protocols]{P, Q}
	  \alternative{\ldots}	
	  \alternative{\assign{o}{R}}
	  \alternative{\ldots}
\end{syntax}

\noindent By erasing the annotations from a query-annotated reaction or protocol, we obtain the underlying (valued) \ipdl construct.

Given an ambient interpretation $\int{-}$ for the \ipdl signature $\Sigma$, we now show how to encode \ipdl constructs as a sequence of symbols on a Turing Machine tape. For types, the encoding $\tmenc{\tau}$ consists of the symbol \textsf{``$\cdot$''} repeated $|\tau|$-many times. For expressions, we have the encoding below, where + denotes string concatenation. We recall that each variable name is represented as a de Bruijn index, and is in particular a natural number.
\begin{align*}
\tmenc{v} & \coloneqq v \\
\tmenc{\Var{x}{\tau}} & \coloneqq \textsf{``(''} + \textsf{``var''} + x + \textsf{``:''} + \tmenc{\tau} + \textsf{``)''} \\
\tmenc{\checkmark} & \coloneqq \textsf{``(''} + \textsf{``$\checkmark$''} + \textsf{``)''} \\
\tmenc{\true} & \coloneqq \textsf{``(''} + \textsf{``true''} + \textsf{``)''} \\
\tmenc{\false} & \coloneqq \textsf{``(''} + \textsf{``false''} + \textsf{``)''} \\
\tmenc{\App{\func}{\sigma}{\tau}{e}} & \coloneqq \textsf{``(''} + \textsf{``app''} + \tmenc{\sigma} + \textsf{``$\to$''} + \tmenc{\tau} + \func + \tmenc{e} + \textsf{``)''}\\
\tmenc{(e_1, e_2)} & \coloneqq \tmenc{e_1} + \tmenc{e_2} \\
\tmenc{\fst_{\sigma \times \tau} \ e} & \coloneqq \textsf{``(''} + \textsf{``fst''} + \tmenc{\sigma} + \textsf{``$\times$''} + \tmenc{\tau} + \textsf{``of''} + \tmenc{e} + \textsf{)''} \\
\tmenc{\snd_{\sigma \times \tau} \ e} & \coloneqq \textsf{``(''} + \textsf{``snd''} + \tmenc{\sigma} + \textsf{``$\times$''} + \tmenc{\tau} + \textsf{``of''} + \tmenc{e} + \textsf{)''} \\
\end{align*}
The encoding $\tmenc{a}$ of an annotation $a = \textsf{input-to-query}, \textsf{input-queried},$ or $\textsf{input-not-to-query}$ is the single symbol \textsf{``input-to-query''}, \textsf{``input-queried''}, or \textsf{``input-not-to-query''}, respectively. For reactions, we have the following encoding, where we recall that each channel name is represented as a de Bruijn index, and is in particular a natural number.
\begin{align*}
\tmenc{\val{v}} & \coloneqq \textsf{``$\langle$''} + \textsf{``val''} + v + \textsf{``$\rangle$''} \\
\tmenc{\ret{e}} & \coloneqq \textsf{``(''} + \textsf{``ret''} + \tmenc{e} + \textsf{``)''} \\
\tmenc{\Samp{\dist}{\sigma}{\tau}{e}} & \coloneqq \textsf{``(''} + \textsf{``samp''} + \tmenc{\sigma} + \textsf{``$\twoheadrightarrow$''} + \tmenc{\tau} + \dist + \tmenc{e} + \textsf{``)''} \\
\tmenc{\QAnnRead{c}{\tau}{a}} & \coloneqq \textsf{``(''} + \textsf{``read''} + \tmenc{a} + c + \textsf{``:''} + \tmenc{\tau} + \textsf{``)''} \\
\tmenc{\ifte{e}{R_1}{R_2}} & \coloneqq \textsf{``(''} + \textsf{``if''} + \tmenc{e} + \textsf{``then''} + \tmenc{R_1} + \textsf{``else''} + \tmenc{R_2} + \textsf{``)''} \\
\tmenc{x : \sigma \leftarrow R; \ S} & \coloneqq \textsf{``$\{$''} + \textsf{``$\_$''} + \textsf{``:''} + \tmenc{\sigma} + \textsf{``$\leftarrow$''} + \tmenc{R} + \textsf{``;''} + \tmenc{S} + \textsf{``$\}$''}
\end{align*}
Finally, for protocols we have the encoding below.
\begin{align*}
\tmenc{\zero} & \coloneqq \textsf{``0''} \\
\tmenc{\assign{o}{v}} & \coloneqq \textsf{``[''} + o + \textsf{``$\coloneqq$''} + v + \textsf{``]''} \\
\tmenc{\assign{o}{R}} & \coloneqq \textsf{``(''} + o + \textsf{``$\coloneqq$''} + \textsf{``react''} + \tmenc{R} + \textsf{``)''} \\
\tmenc{\Par{P}{Q}} & \coloneqq \textsf{``(''} + \tmenc{P} + \textsf{``$\|$''} + \tmenc{Q} + \textsf{``)''} \\
\tmenc{\new{c}{\tau}{P}} & \coloneqq \textsf{``new''} + \textsf{``$\_$''} + \textsf{``:''} + \tmenc{\tau} + \textsf{``in''} + \tmenc{P} + \textsf{``wen''}
\end{align*}
To avoid having to shift the tape contents when executing \ipdl protocols on a Turing Machine tape, we will make use of the white-space symbol \textsf{`` ''}, which we consider as distinct from the symbol \emph{blank}. The former will be used as a placeholder so that our protocol encoding remains at a constant length throughout the execution. For this reason, we extend our notion of encoding to allow extra white-spaces around the encoding of a valued expression $e$ or a query-annotated reaction $R$ occurring inside a query-annotated protocol $P$.

Given a query-annotated \ipdl construct, its \emph{query bound} $\qnorm{-}$ is the number of occurrences of the annotation \textsf{input-to-query} inside the construct. Furthermore, given a channel set $C$, we define $\QAnn{R}{C}$ and $\QAnn{P}{C}$ to be the query-annotated version of $R$ and $P$ that annotates every $\mathsf{read}$ from a channel in $C$ with the annotation \textsf{input-to-query} and every $\mathsf{read}$ from a channel not in $C$ with the annotation \textsf{input-not-to-query}. Additionally, given a channel set $C$, we define the minimal set $\MinQIn{R}{C} \subseteq C$ of query inputs to the reaction $R$ as follows:
\begin{align*}
\MinQIn{\val{v}}{C} & \coloneqq \emptyset \\
\MinQIn{\ret{e}}{C} & \coloneqq \emptyset \\
\MinQIn{\samp{\dist}{e}}{C} & \coloneqq \emptyset \\
\MinQIn{\read{c}}{C} & \coloneqq \{c\} \text{ if } c \in C \\
\MinQIn{\read{i}}{C} & \coloneqq \{\emptyset\} \text{ if } i \notin C \\
\MinQIn{\ifte{e}{R_1}{R_2}}{C} & \coloneqq \MinQIn{R_1}{C} \, \bigcup \, \MinQIn{R_2}{C} \\
\MinQIn{x \leftarrow R; \ S}{C} & \coloneqq \MinQIn{R}{C} \, \bigcup \, \MinQIn{S}{C}
\end{align*}
Similarly, given a channel set $C$, we define the minimal set $\MinQIn{P}{C} \subseteq C$ of query inputs to the protocol $P$ as follows:
\begin{align*}
\MinQIn{\zero}{C} & \coloneqq \emptyset \\
\MinQIn{\assign{o}{v}}{C} & \coloneqq \emptyset \\
\MinQIn{\assign{o}{R}}{C} & \coloneqq \MinQIn{R}{C} \\
\MinQIn{\Par{P}{Q}}{C} & \coloneqq \MinQIn{P}{C} \, \bigcup \, \MinQIn{Q}{C} \\
\MinQIn{\new{c}{\tau}{P}}{C} & \coloneqq \MinQIn{P}{C}
\end{align*}

\noindent The typing of query-annotated \ipdl constructs has the form $\Delta; \ \Gamma \vdash R : I \to \tau \query C$ and $\Delta \vdash P : I \to O \query C$, where $C \subseteq I$ is the set of \emph{query inputs}. For reactions, reading from a query input must be annotated with either \textsf{input-to-query} or \textsf{input-queried}, and reading from any other channel must carry the annotation \textsf{input-not-to-query}.
\begin{mathpar}
\inferrule*{c : \tau \in \Delta \\ c \in C \\ a \in \{\textsf{input-to-query}, \textsf{input-queried}\}}{\Delta; \ \Gamma \vdash \QAnnRead{c}{\tau}{a} : I \to \tau \query C} \\ \and
\inferrule*{i : \tau \in \Delta \\ i \in I \, \setminus \, C \\ a \in \{\textsf{input-not-to-query}\}}{\Delta; \ \Gamma \vdash \QAnnRead{i}{\tau}{a} : I \to \tau \query C}\and
\end{mathpar}

\noindent For protocols, the interesting rules are shown below, where we propagate the set of query inputs throughout.
\begin{mathpar}
\inferrule*{o \notin I \\ o : \tau \in \Delta \\ \Delta; \ \cdot \vdash R : I \cup \{o\} \to \tau \query C}{\Delta \vdash \big(\assign{o}{R}\big) : I \to \{o\} \query C}\and
\inferrule*{\Delta \vdash P : I \cup O_2 \to O_1 \query C \\ \Delta \vdash Q : I \cup O_1 \to O_2 \query C}{\Delta \vdash \Par{P}{Q} : I \to O_1 \cup O_2 \query C} \\ \and
\inferrule*{\Delta, o : \tau \vdash P : I \to O \cup \{o\} \query C}{\Delta \vdash \big(\new{o}{\tau}{P}\big) : I \to O \query C}
\end{mathpar}

\begin{proof}
Let $O_1^\textsf{min} = \MinQIn{Q}{O_1}$ be the minimal set of query inputs to $Q$ from among $O_1$. In other words, $O_1^\textsf{min}$ contains precisely those channels of $O_1$ that $Q$ reads from. Then $\Delta \vdash \QAnn{Q}{O_1} : I \cup O_1^\textsf{min} \to Q_2 \query O_1^\textsf{min}$. The reason for replacing $O_1$ with $O_1^\textsf{min}$ is that we do not have a bound on the size of the former. The size of the latter is bounded by the query bound $\qnorm{Q}$, and this is in turn bounded by $\tmnorm{Q}(|\type_1|,\ldots,|\type_{|\Sigma_\mathsf{t}|}|)$. Let \textsf{ProtEncSymb} be the disjoint union of the sets
\begin{itemize}
\item $\{\textsf{`` ''}\}$,
\item \textsf{Punc},
\item \textsf{KeyWords},
\item \emph{the set $\Sigma_\func$ of function symbols declared in $\Sigma$},
\item \emph{the set $\Sigma_\dist$ of distribution symbols declared in $\Sigma$},
\item $\big\{n \; | \; 0 \leq n < \bnorm{Q}\big\}$,
\item $\big\{n \; | \; 0 \leq n < \nnorm{Q}\big\} \, \bigcup \, \big\{m + n \; | \; m \in \phi^\star(I \cup O_1^\textsf{min} \cup O_2) \textit{ and } 0 \leq n \leq \nnorm{Q}\big\}$.
\end{itemize}
Let $\Adv \coloneqq \big(\Delta', I', O', \phi, \#_\round, \#_\tape, \Symb, \St, s_\star, \Step, \big\{\In_o\big\}_{o \, \in \, I'}, \big\{\Out_i\big\}_{i \, \in \, O'}, \Dec\big)$ be the adversary for protocols of type $\Delta \vdash I \to O_1 \cup O_2$. We define a new adversary $\Adv_\mathcal{R}$ for protocols of type $\Delta \vdash I \cup O_2 \to O_1$ as follows:
\[\Adv_\mathcal{R} \coloneqq \big(\Delta', I'_\mathcal{R}, O'_\mathcal{R}, \phi, \#_\round^\mathcal{R}, \#_\tape^\mathcal{R}, \Symb^\mathcal{R}, \St^\mathcal{R}, s^\mathcal{R}_\star, \Step^\mathcal{R}, \big\{\In^\mathcal{R}_o\big\}_{o \, \in \, I'_\mathcal{R}}, \big\{\Out^\mathcal{R}_i\big\}_{i \, \in \, O'_\mathcal{R}}, \Dec^\mathcal{R}\big),\]
where
\begin{itemize}
\item $I'_\mathcal{R} \coloneqq (I' \, \setminus \, \phi^\star(O_2)) \cup (\phi^\star(O_1^\textsf{min}) \, \setminus \, I')$;

\item $O'_\mathcal{R} \coloneqq O' \cup \phi^\star(O_2)$;

\item $\#_\round^\mathcal{R} \coloneqq \#_\round * \tmnorm{Q}(|\type_1|,\ldots,|\type_{|\Sigma_\mathsf{t}|}|)^2 + \#_\round$;

\item $\#_\tape^\mathcal{R} \coloneqq \#_\tape + 1$;

\item $\Symb^\mathcal{R} \coloneqq \textsf{ProtEncSymb} \, \bigsqcup \ \{\textsf{``$\rightleftharpoons$''}, \textsf{``$\#$''}\} \, \bigsqcup \, \textsf{Symb}$;

\item $\St^\mathcal{R} \coloneqq \big\{\textsf{``(''} + b + s_\mathsf{prot} + \textsf{``$\rightleftharpoons$''} + s_\mathsf{adv} + \textsf{``)''}\big\}$ \emph{contains strings of the specified form, where $b \in \{0,1\}$, $s_\mathsf{adv} \in \St$ is a distinguisher state, and $s_\mathsf{prot}$ is a string of length $\tmnorm{Q}(|\type_1|,\ldots,|\type_{|\Sigma_\mathsf{t}|}|)$ with symbols drawn from} \textsf{ProtEncSymb}, \emph{that encodes a protocol $\phi^\star(Q')$, where} $\Delta \vdash Q' : I \cup O_1^\textsf{min} \to O_2 \query O_1^\textsf{min}$ \emph{is such that $\bnorm{Q'} \leq \bnorm{Q}$ and $\nnorm{Q'} \leq \nnorm{Q}$};

\item $s^\mathcal{R}_\star \coloneqq \textsf{``(''} + \textsf{``$1$''} + \tmenc{\phi^\star(\QAnn{Q}{O_1})} + \textsf{``$\rightleftharpoons$''} + s_\star + \textsf{``)''}$;

\item $\Dec^\mathcal{R}$ \emph{processes a state of the form} $\textsf{``(''} + b + s_\mathsf{prot} + \textsf{``$\rightleftharpoons$''} + s_\mathsf{adv} + \textsf{``)''}$ \emph{by applying $\Dec$ to $s_\mathsf{adv}$};

\item $\Out_i^\mathcal{R}$ \emph{for $i \in O'$ processes a state of the form} $\textsf{``(''} + b + s_\mathsf{prot} + \textsf{``$\rightleftharpoons$''} + s_\mathsf{adv} + \textsf{``)''}$ \emph{by applying $\Out_i$ to $s_\mathsf{adv}$};

\item $\Out_{o_2}^\mathcal{R}$ \emph{for $o_2 \in \phi^\star(O_2)$ processes a state of the form} $\textsf{``(''} + b + s_\mathsf{prot} + \textsf{``$\rightleftharpoons$''} + s_\mathsf{adv} + \textsf{``)''}$ \emph{by traversing through $s_\mathsf{prot}$ to locate the assignment to channel} $o_2$. \emph{If $o_2$ is assigned a value $v$, it returns $v$; otherwise it returns $\bot$};

\item $\In_{o_1}^\mathcal{R}$ \emph{for $o_1 \in I' \, \setminus \, \phi^\star(O_2) \, \setminus \, \phi^\star(O_1^\mathsf{min})$ processes a state of the form} $\textsf{``(''} + 0 + s_\mathsf{prot} + \textsf{``$\rightleftharpoons$''} + s_\mathsf{adv} + \textsf{``)''}$ \emph{by applying $\In_{o_1}$ to $s_\mathsf{adv}$};

\item $\In_{o_1}^\mathcal{R}$ \emph{for $o_1 \in I' \, \setminus \, \phi^\star(O_2) \, \setminus \, \phi^\star(O_1^\mathsf{min})$ leaves a state of the form} $\textsf{``(''} + 1 + s_\mathsf{prot} + \textsf{``$\rightleftharpoons$''} + s_\mathsf{adv} + \textsf{``)''}$ \emph{unchanged};

\item $\In_{o_1}^\mathcal{R}$ \emph{for $o_1 \in \phi^\star(O_1^\mathsf{min}) \, \setminus \, I'$ leaves a state of the form} $\textsf{``(''} + 0 + s_\mathsf{prot} + \textsf{``$\rightleftharpoons$''} + s_\mathsf{adv} + \textsf{``)''}$ \emph{unchanged};

\item $\In_{o_1}^\mathcal{R}$ \emph{for $o_1 \in \phi^\star(O_1^\mathsf{min}) \, \setminus \, I'$ processes a state of the form} $\textsf{``(''} + 1 + s_\mathsf{prot} + \textsf{``$\rightleftharpoons$''} + s_\mathsf{adv} + \textsf{``)''}$ \emph{by traversing through $s_\mathsf{prot}$ and replacing every $\mathsf{read}$ from the channel $o_1$ by the assigned input value $v$};

\item $\In_{o_1}^\mathcal{R}$ \emph{for $o_1 \in I' \, \setminus \, \phi^\star(O_2) \cap \phi^\star(O_1^\mathsf{min})$ processes a state of the form} $\textsf{``(''} + 0 + s_\mathsf{prot} + \textsf{``$\rightleftharpoons$''} + s_\mathsf{adv} + \textsf{``)''}$ \emph{by applying $\In_{o_1}$ to $s_\mathsf{adv}$};

\item $\In_{o_1}^\mathcal{R}$ \emph{for $o_1 \in I' \, \setminus \, \phi^\star(O_2) \cap \phi^\star(O_1^\mathsf{min})$ processes a state of the form} $\textsf{``(''} + 1 + s_\mathsf{prot} + \textsf{``$\rightleftharpoons$''} + s_\mathsf{adv} + \textsf{``)''}$ \emph{by traversing through $s_\mathsf{prot}$ and replacing every $\mathsf{read}$ from the channel $o_1$ by the assigned input value $v$};

\end{itemize}

%P(x,y,z) >= y (assumption)

%P(x,y,z) >= y + z (bound for I')

%P(x,y,z) >= y + z (bound for O')

%P(x,y,z) >= yz^2 (bound for #_round)

%P(x,y,z) >= y + 1 (bound for #_tape)

%P(x,y,z) >= 2z^2 + yz + 2y + 4z + |\Sigma_f| + |\Sigma_d| + |Punc| + |KeyWords| + 3

%P(x,y,z) >= y + z + 4 (bound for St)

%P(x,y,z) >= 3y + z + 7 (bound for Dec, runtime)
%P(x,y,z) >= y + 7 (bound for Dec, states)

%P(x,y,z) >= 5y + z + 9 (bound for Out if i in O', runtime)
%P(x,y,z) >= y + 9 (bound for Out if i in O', states)

%P(x,y,z) >= y + 2z + 6 (bound for Out if i in \phi^\star(O_2), runtime)
%P(x,y,z) >= 2z + 13 (bound for Out if i in \phi^\star(O_2), states)

%P(x,y,z) >= 7y + 2z + 12 (bound for In if o_1 in (I' \setminus \phi^\star(O_2) \setminus \phi^\star(O_1^min)), runtime)
%P(x,y,z) >= y + 9 (bound for In if o_1 in (I' \setminus \phi^\star(O_2) \setminus \phi^\star(O_1^min)), states)

%P(x,y,z) >= 2z^2 + 2y + 10z + 13 (bound for In if o_1 in (\phi^\star(O_1^min) \setminus I'), runtime)
%P(x,y,z) >= 5z + 24 (bound for In if o_1 in (\phi^\star(O_1^min) \setminus I'), states)

%P(x,y,z) >= 2z^2 + 7y + 10z + 13 (bound for In if o_1 in (I' \setminus \phi^\star(O_2) \cap \phi^\star(O_1^min)), runtime)
%P(x,y,z) >= y + 5z + 29 (bound for In if o_1 in (I' \setminus \phi^\star(O_2) \cap \phi^\star(O_1^min)), states)

%P(x,y,z) >= 2 + max(z + 1 + max(3 + z, 6 + z, 6 + z, 6 + z, 5 + x + 3z, 3 + 3z, 3 + 3z, 3 + 2z, 5 + x + 3z, 3 + 3z, ...), ...) (bound for Step, runtime)
%P(x,y,z) >= 2 + 20 + 4 + 3 + 3 + 3 + (6 + |\Sigma_f|x) + 9 + 9 + 3 + (6 + |\Sigma_d|x) + 15 + ... (bound for Step, states)

%Resulting bound: P(x,y,z) := yz^2 + 2z^2 + 5y + 7z + const
\end{proof}

%\section{Maude Formalization}
%\subsection{Normal Forms}
We work with protocols that start with a list of declarations of 
internal channels, using $\mathsf{new}$, followed by a parallel compositions of channel assignments. The reactions in these
assignments can be transformed into a list of binds of the form
$x : \tau \leftarrow read(c)$, called
bind-read reactions, followed by a reaction without binds.
The list of binds can be regarded as commutative, 
as two reactions with the same list of binds in different order
are equivalent due to the reaction equivalence rule \textsc{exch}.
Similarly, different order of declarations of internal channels gives
equivalent protocols, by using the protocol equivalence rule
\textsc{new-exch}. When writing equivalence proofs, we do not want to 
make the use of these rules explicit. Therefore, we introduce
normal forms of reactions and protocols. The normal form of
a reaction 
$\nf(L, R, O)$
consists of a commutative list $L$ of bind-read reactions,
a bind-free reaction $R$ 
and a chosen order $O$ of the names of the variables occuring in
the binds in $L$.
The latter will be used to determine how to turn the normal form 
of a reaction into a regular reaction. 
During equivalence proofs, we may obtain either arbitrary binds in
$L$ or reactions $R$ that are not bind-free.
This will be represented as a pre-normal-form, 
written $\preNf(L, R, O)$, which is a normal
form without restrictions on the occuring reactions. The general
strategy will be to transform pre-normal-forms to normal 
forms by rule applications.
The normal form of a 
protocol 
$\newNf(L, P, O)$
consists of a commutative list $L$ of declarations of internal
channels, a protocol $P$ that does not start with internal channel declarations
and again a designated order $O$ 
for the names of internal channels occuring in the declarations in $L$.
Since in both cases the lists are commutative, we can write
equivalence rules for normal forms where we assume that an
internal channel declaration or a bind-read reaction are first in the 
lists.

\bibliography{bib} 
\bibliographystyle{ieeetr}

\end{document}