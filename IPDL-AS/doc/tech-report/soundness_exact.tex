Soundness of equality at the expression level means that if we substitute the same valued expression for each free variable, the resulting closed expressions will compute to the same value:

\begin{definition}
An axiom $\Gamma \vdash e_1 = e_2 : \tau$ is \emph{sound} if for any valued substitution $\theta : \cdot \to \Gamma$ we have $\eval{\theta^\star(e_1)} = \eval{\theta^\star(e_2)}$.
\end{definition}

\noindent The ambient \ipdl theory $\mathbb{T}_\mathsf{exp}$ for expressions is said to be sound if each of its axioms is sound. It is straightforward to show that this implies overall soundness:
 
\begin{lemma}[Soundness of equality of expressions]
If the ambient \ipdl theory for expressions is sound, then for any expressions $\Gamma \vdash e_1 = e_2 : \tau$ and any valued substitution $\theta : \cdot \to \Gamma$ we have that $\eval{\theta^\star(e_1)} = \eval{\theta^\star(e_2)}$.
\end{lemma}

At the reaction level, two equal reactions should behave in a way that is perfectly indistinguishable by an external observer. We formally capture this notion of indistinguishability by a logical relation known as a \emph{bisimulation} -- a binary relation on distributions on reactions that satisfies certain closure properties, together with the crucial \emph{valuation property} that allows us to jointly partition two related final distributions so that any two corresponding components are again related and have the same \emph{value}: a final reaction $\Delta; \ \cdot \vdash R : I \to \tau$ is said to have value $v \in \int{\tau}$ if $R$ is of the form $\val{v}$; if $R$ is stuck we define the value to be $\bot$. Given a $v_\bot \in \{\bot\} \cup \int{\tau}$, we write $\valueof{R} = v_\bot$ to indicate that the value of $R$ is $v_\bot$, and lift this notation to distributions in the obvious way.

We emphasize that at the reaction level, we only require the valuation property to hold for those distributions that are \emph{final}, \emph{i.e.}, no reaction in the support steps.

\begin{definition}[Reaction bisimulation]
A \emph{reaction bisimulation} $\sim$ is a binary relation on distributions on reactions of type $\Delta; \ \cdot \vdash I \to \tau$ satisfying the following conditions:
\begin{itemize}
\item \emph{Closure under convex combinations}: For any distributions $\eta_1 \sim \varepsilon_1$ and $\eta_2 \sim \varepsilon_2$, and any coefficients $c_1, c_2 > 0$ with $c_1 + c_2 = 1$, we have $(c_1 * \eta_1 + c_2 * \eta_2) \sim (c_1 * \varepsilon_1 + c_2 * \varepsilon_2)$.

\item \emph{Closure under input assignment}: For any distributions $\eta \sim \varepsilon$, input channel $i \in I$ with $i : \tau \in \Delta$, and value $v \in \int{\tau}$, we have $\eta[\read{i} \coloneqq \val{v}] \sim \varepsilon[\read{i} \coloneqq \val{v}]$.

\item \emph{Closure under computation}: For any distributions $\eta \sim \varepsilon$, we have $(\eval{\eta}) \sim (\eval{\varepsilon})$.

\item \emph{Valuation property}: For any final distributions $\eta \sim \varepsilon$, there exists a joint convex combination \[\eta = \sum_i c_i * \eta_i \; \sim \, \sum_i c_i * \varepsilon_i = \varepsilon\]
with $c_i > 0$ and $\sum_i c_i = 1$ such that
\begin{itemize}
\item the respective components $\eta_i \sim \varepsilon_i$ are again related, and
\item $\valueof{\eta_i} = v_\bot = \valueof{\varepsilon_j}$ for the same $v_\bot \in \{\bot\} \cup \int{\tau}$ if and only if $i = j$.
\end{itemize}
\end{itemize}
\end{definition}

\noindent Crucially, we note that the joint convex combination in the valuation property is unique up to the order of the summands.

\begin{lemma}
We have the following: 
\begin{itemize}
\item The identity relation is a reaction bisimulation.
\item The inverse of a reaction bisimulation is a reaction bisimulation.
\item The composition of two reaction bisimulations is a reaction bisimulation.
\end{itemize}
\end{lemma}

\noindent We now describe one canonical way to construct reaction bisimulations:

\begin{definition}
Let $\sim$ be an arbitrary binary relation on distributions on reactions of type $\Delta; \ \cdot \vdash I \to \tau$. The \emph{expansion} $\lift{\sim}$ is the closure of $\sim$ under joint convex combinations. Explicitly, $\lift{\sim}$ is defined by
\[\Big(\sum_i c_i * \eta_i\Big) \; \lift{\sim} \; \Big(\sum_i c_i * \varepsilon_i\Big)\]
for coefficients $c_i > 0$ with $\sum_i c_i = 1$ and distributions $\eta_i \sim \varepsilon_i$.
\end{definition}

\begin{lemma}\label{lem:reaction_seed}
Let $\sim$ be a binary relation on distributions on reactions of type $\Delta; \ \cdot \vdash I \to \tau$ with the following properties:
\begin{itemize}
\item \emph{Closure under input assignment}: For any distributions $\eta \sim \varepsilon$, input channel $i \in I$ with $i : \tau \in \Delta$, and value $v \in \int{\tau}$, we have $\eta[\read{i} \coloneqq \val{v}] \sim \varepsilon[\read{i} \coloneqq \val{v}]$.

\item \emph{Expansion closure under computation}: For any distributions $\eta \sim \varepsilon$, we have $(\eval{\eta}) \; \lift{\sim} \; (\eval{ \varepsilon})$.

\item \emph{Valuation property}: For any final distributions $\eta \sim \varepsilon$, there exists a joint convex combination \[\eta = \sum_i c_i * \eta_i \; \sim \, \sum_i c_i * \varepsilon_i = \varepsilon\]
with $c_i > 0$ and $\sum_i c_i = 1$ such that
\begin{itemize}
\item the respective components $\eta_i \sim \varepsilon_i$ are again related, and
\item $\valueof{\eta_i} = v_\bot = \valueof{\varepsilon_j}$ for the same $v_\bot \in \{\bot\} \cup \int{\tau}$ if and only if $i = j$.
\end{itemize}
\end{itemize}
Then the expansion $\lift{\sim}$ is a reaction bisimulation.
\end{lemma}

\begin{example}
Fix two expressions $\cdot \vdash e_1 : \sigma$ and $\cdot \vdash e_2 : \sigma$ such that $\eval{e_1} = \eval{e_2}$. Then the relation $\sim$ defined by
\begin{itemize}
\item $1[R(x \coloneqq e_1)] \sim 1[R(x \coloneqq e_2)]$ for reaction $\Delta; \ x : \sigma \vdash R : I \to \tau$
\end{itemize}
satisfies the hypotheses of Lemma \ref{lem:reaction_seed} (and its lifting is hence a reaction bisimulation).
\end{example}

\noindent Having defined reaction bisimulations, we can now formally state what it means for reaction equality to be sound:

\begin{definition}
An axiom $\Gamma \vdash R_1 = R_2 : \tau$ is \emph{sound} if there is a reaction bisimulation $\sim$ such that for any valued substitution $\theta : \cdot \to \Gamma$ we have $1[\theta^\star(R_1)] \sim 1[\theta^\star(R_2)]$.
\end{definition}

\noindent The ambient \ipdl theory $\mathbb{T}_\mathsf{dist}$ for reactions is said to be sound if each of its axioms is sound. We now show that this implies overall soundness:

\begin{lemma}[Soundness of equality of reactions]
If the ambient \ipdl theories for expressions and reactions are sound, then for any reactions $\Delta; \ \Gamma \vdash R_1 = R_2 : I \to \tau$ there exists a reaction bisimulation $\sim$ such that for any valued substitution $\theta : \cdot \to \Gamma$ we have $1[\theta^\star(R_1)] \sim 1[\theta^\star(R_2)]$.
\end{lemma}

\begin{proof}
We first replace the exchange rule \textsc{exch} by the three rules \textsc{exch-samp-samp}, \textsc{exch-samp-read}, and \textsc{exch-read-read} in Figure \ref{fig:exch_alt}; it is easy to see that this new set of rules is equivalent to the original one. We now proceed by induction on the alternative set of rules for reaction equality. We will freely use a distribution in place of a value (rule \textsc{exch-samp-read}) or a reaction (rules \textsc{embed}, \textsc{cong-bind}) to indicate the obvious lifting of the corresponding construct to distributions on reactions.

\begin{itemize}
\item \textsc{refl}: Our desired bisimulation is the identity relation.
\item \textsc{sym}: Our desired bisimulation is the inverse of the bisimulation obtained from the premise.
\item \textsc{trans}: Our desired bisimulation is the composition of the two bisimulations obtained from the two premises.
\item \textsc{axiom}: Our desired bisimulation is precisely the bisimulation obtained from the soundness of the axiom.
\item \textsc{input-unused}: Our desired bisimulation is precisely the bisimulation obtained from the premise, seen as a bisimulation on distributions on reactions with the additional input $i$.
\item \textsc{subst}: Our desired bisimulation is precisely the bisimulation obtained from the premise.
\item \textsc{embed}: Let $\sim$ be the bisimulation obtained from the premise. Our desired bisimulation $\sim_\phi$ is defined by
\begin{itemize}
\item $\phi^\star(\eta) \sim_\phi \phi^\star(\varepsilon)$ if $\eta \sim \varepsilon$
\end{itemize}
\item \textsc{cong-ret}: Our desired bisimulation is the expansion of the relation $\sim$ defined by
\begin{itemize}
\item $1[\ret{e}] \sim 1[\ret{e'}]$ for expressions $\cdot \vdash e : \tau$ and $\cdot \vdash e' : \tau$ such that $\eval{e} = \eval{e'}$
\item $1[\val{v}] \sim 1[\val{v}]$ for value $v \in \int{\tau}$
\end{itemize}
\item \textsc{cong-samp}: Our desired bisimulation is the expansion of the relation $\sim$ defined by
\begin{itemize}
\item $1[\samp{\dist}{e}] \sim 1[\samp{\dist}{e'}]$ for expressions $\cdot \vdash e : \sigma$ and $\cdot \vdash e' : \sigma$ such that $\eval{e} = \eval{e'}$
\item $1[\val{v}] \sim 1[\val{v}]$ for value $v \in \int{\tau}$
\end{itemize}
\item \textsc{cong-if}: Let $\sim_1$ and $\sim_2$ be the two bisimulations obtained from the two premises. Our desired bisimulation is the expansion of the relation $\sim_\mathsf{if}$ defined by
\begin{itemize}
\item $1[\ifte{e}{R_1}{R_2}] \sim_\mathsf{if} \, 1[\ifte{e'}{R'_1}{R'_2}]$ for 
\begin{itemize}
\item expressions $\cdot \vdash e : \Bool$ and $\cdot \vdash e' : \Bool $ such that $\eval{e} = \eval{e'}$
\item reactions $\Delta; \ \cdot \vdash R_1 : I \to \tau$ and $\Delta; \ \cdot \vdash R'_1 : I \to \tau$ such that $1[R_1] \sim_1 1[R'_1]$
\item reactions $\Delta; \ \cdot \vdash R_2 : I \to \tau$ and $\Delta; \ \cdot \vdash R'_2 : I \to \tau$ such that $1[R_2] \sim_2 1[R'_2]$
\end{itemize}
\item $\eta_1 \sim_\mathsf{if} \eta'_1$ if $\eta_1 \sim_1 \eta_1'$
\item $\eta_2 \sim_\mathsf{if} \eta'_2$ if $\eta_2 \sim_2 \eta_2'$
\end{itemize}
\item \textsc{cong-bind}: Let $\sim_1$ and $\sim_2$ be the two bisimulations obtained from the two premises. Our desired bisimulation is the expansion of the relation $\sim_\mathsf{bind}$ defined by
\begin{itemize}
\item $\big(x \leftarrow \eta; \ S\big) \sim_\mathsf{bind} \big(x \leftarrow \eta'; \ S'\big)$ for
\begin{itemize}
\item distributions $\eta \sim_1 \eta'$
\item reactions $\Delta; \ x : \sigma \vdash S : I \to \tau$ and $\Delta; \ x : \sigma \vdash S' : I \to \tau$ such that for any value $v \in \int{\sigma}$ we have $1[S(x \coloneqq v)] \sim_2 1[S'(x \coloneqq v)]$
\end{itemize}
\item $\varepsilon \sim_\mathsf{bind} \varepsilon'$ if $\varepsilon \sim_2 \varepsilon'$
\end{itemize}
\item \textsc{ret-bind}: Our desired bisimulation is the expansion of the relation $\sim$ defined by
\begin{itemize}
\item $1[x \leftarrow \ret{e}; \ R] \sim 1[R(x \coloneqq e)]$ for expression $\cdot \vdash e : \sigma$ and reaction $\Delta; \ x : \sigma \vdash R : I \to \tau$
\item $1[R(x \coloneqq \eval{e})] \sim 1[R(x \coloneqq e)]$ for reaction $\Delta; \ x : \sigma \vdash R : I \to \tau$ and expression $\cdot \vdash e : \sigma$
\end{itemize}
\item \textsc{bind-ret}: Our desired bisimulation is the expansion of the relation $\sim$ defined by
\begin{itemize}
\item $1[x \leftarrow R; \ \ret{x}] \sim 1[R]$ for reaction $\Delta; \ \cdot \vdash R : I \to \tau$
\item $1[\val{v}] \sim 1[\val{v}]$ for value $v \in \int{\tau}$
\end{itemize}
\item \textsc{bind-bind}: Our desired bisimulation is the expansion of the relation $\sim$ defined by
\begin{itemize}
\item $1[x_2 \leftarrow (x_1 \leftarrow R_1; \ R_2); \ S] \sim 1[x_1 \leftarrow R_1; \ x_2 \leftarrow R_2; \ S]$ for
\begin{itemize}
\item reaction $\Delta; \ \cdot \vdash R_1 : I \to \sigma_1$
\item reaction $\Delta; \ x_1 : \sigma_1 \vdash R_2 : I \to \sigma_2$
\item reaction $\Delta; \ x_2 : \sigma_2 \vdash S : I \to \tau$
\end{itemize}
\item $1[x_2 \leftarrow R_2; \ S] \sim 1[x_2 \leftarrow R_2; \ S]$ for reactions $\Delta; \ \cdot \vdash R_2 : I \to \sigma_2$ and $\Delta; \ x_2 : \sigma_2 \vdash S : I \to \tau$
\item $1[S] \sim 1[S]$ for reaction $\Delta; \ \cdot \vdash S : I \to \tau$
\end{itemize}
\item \textsc{samp-pure}: Our desired bisimulation is the expansion of the relation $\sim$ defined by
\begin{itemize}
\item $1[x \leftarrow \samp{\dist}{e}; \ R] \sim 1[R]$ for expression $\cdot \vdash e : \rho$ and reaction $\Delta; \ \cdot \vdash R : I \to \tau$
\item $1[R] \sim 1[R]$ for reaction $\Delta; \ \cdot \vdash R : I \to \tau$
\end{itemize}
\item \textsc{read-det}: Our desired bisimulation is the expansion of the relation $\sim$ defined by
\begin{itemize}
\item $1[x \leftarrow \read{i}; \ y \leftarrow \read{i}; \ R] \sim 1[x \leftarrow \read{i}; \ R(y \coloneqq x)]$ for reaction $\Delta; \ x : \sigma, y : \sigma \vdash R : I \to \tau$
\item $1[x \leftarrow \val{v}; \ y \leftarrow \val{v}; \ R] \sim 1[x \leftarrow \val{v}; \ R(y \coloneqq x)]$ for
\begin{itemize}
\item reaction $\Delta; \ x : \sigma, y : \sigma \vdash R : I \to \tau$ 
\item value $v \in \int{\sigma}$
\end{itemize}
\item $1[R] \sim 1[R]$ for reaction $\Delta; \ \cdot \vdash R : I \to \tau$
\end{itemize}
\item \textsc{if-left}: Our desired bisimulation is the expansion of the relation $\sim$ defined by
\begin{itemize}
\item $1[\ifte{\true}{R_1}{R_2}] \sim 1[R_1]$ for reactions $\Delta; \ \cdot \vdash R_1 : I \to \tau$ and $\Delta; \ \cdot \vdash R_2 : I \to \tau$
\item $1[R_1] \sim 1[R_1]$ for reaction $\Delta; \ \cdot \vdash R_1 : I \to \tau$
\end{itemize}
\item \textsc{if-right}: Our desired bisimulation is the expansion of the relation $\sim$ defined by
\begin{itemize}
\item $1[\ifte{\false}{R_1}{R_2}] \sim 1[R_2]$ for reactions $\Delta; \ \cdot \vdash R_1 : I \to \tau$ and $\Delta; \ \cdot \vdash R_2 : I \to \tau$
\item $1[R_2] \sim 1[R_2]$ for reaction $\Delta; \ \cdot \vdash R_2 : I \to \tau$
\end{itemize}
\item \textsc{if-ext}: Our desired bisimulation is the expansion of the relation $\sim$ defined by
\begin{itemize}
\item $1[\ifte{e}{R(x \coloneqq \true)}{R(x \coloneqq \false)]} \sim 1[R(x \coloneqq e)]$ for
\begin{itemize}
\item reaction $\Delta; \ x : \Bool \vdash R : I \to \tau$
\item expression $\cdot \vdash e : \Bool$
\end{itemize}
\item $1[R(x \coloneqq \true)] \sim 1[R(x \coloneqq e)]$ for
\begin{itemize}
\item reaction $\Delta; \ x : \Bool \vdash R : I \to \tau$
\item expression $\cdot \vdash e : \Bool$ such that $e \Downarrow 1$
\end{itemize}
\item $1[R(x \coloneqq \false)] \sim 1[R(x \coloneqq e)]$ for
\begin{itemize}
\item reaction $\Delta; \ x : \Bool \vdash R : I \to \tau$
\item expression $\cdot \vdash e : \Bool$ such that $e \Downarrow 0$
\end{itemize}
\end{itemize}
\item \textsc{exch-samp-samp}: Our desired bisimulation is the expansion of the relation $\sim$ defined by
\begin{itemize}
\item $1[x_1 \leftarrow \samp{\dist_1}{e_1}; \ x_2 \leftarrow \samp{\dist_2}{e_2}; \ \ret{(x_1,x_2)}] \sim \\ 1[x_2 \leftarrow \samp{\dist_2}{e_2}; \ x_1 \leftarrow \samp{\dist_1}{e_1}; \ \ret{(x_1,x_2)}]$ for
\begin{itemize}
\item expressions $\cdot \vdash e_1 : \sigma_1$ and $\cdot \vdash e_2 : \sigma_2$
\end{itemize}
\item $1[\val{v_1 v_2}] \sim 1[\val{v_1 v_2}]$ for values $v_1 \in \int{\tau_1}$ and $v_2 \in \int{\tau_2}$
\end{itemize}
\item \textsc{exch-samp-read}: Our desired bisimulation is the expansion of the relation $\sim$ defined by
\begin{itemize}
\item $1[x_1 \leftarrow \samp{\dist}{e}; \ x_2 \leftarrow \read{i}; \ \ret{(x_1,x_2)}] \sim 1[x_2 \leftarrow \read{i}; \ x_1 \leftarrow \samp{\dist}{e}; \ \ret{(x_1,x_2)}]$ for
\begin{itemize}
\item expression $\cdot \vdash e : \sigma$
\end{itemize}
\item $1[x_1 \leftarrow \samp{\dist}{e}; \ x_2 \leftarrow \val{v_2}; \ \ret{(x_1,x_2)}] \sim 1[x_2 \leftarrow \val{v_2}; \ x_1 \leftarrow \samp{\dist}{e}; \ \ret{(x_1,x_2)}]$ for
\begin{itemize}
\item expression $\cdot \vdash e : \sigma$
\item value $v_2 \in \int{\tau_2}$
\end{itemize}
\item $\big(x_2 \leftarrow \read{i}; \ \ret{(\int{\dist}(\eval{e}),x_2)}\big) \sim 1[x_2 \leftarrow \read{i}; \ x_1 \leftarrow \samp{\dist}{e}; \ \ret{(x_1,x_2)}]$ for
\begin{itemize}
\item expression $\cdot \vdash e : \sigma$
\end{itemize}
\item $\big(x_2 \leftarrow \val{v_2}; \ \ret{(\int{\dist}(\eval{e}),x_2)}\big) \sim 1[x_2 \leftarrow \val{v_2}; \ x_1 \leftarrow \samp{\dist}{e}; \ \ret{(x_1,x_2)}]$ for
\begin{itemize}
\item expression $\cdot \vdash e : \sigma$
\item value $v_2 \in \int{\tau_2}$
\end{itemize}
\item $1[\val{v_1 v_2}] \sim 1[\val{v_1 v_2}]$ for values $v_1 \in \int{\tau_1}$ and $v_2 \in \int{\tau_2}$
\end{itemize}
\item \textsc{exch-read-read}: Our desired bisimulation is the expansion of the relation $\sim$ defined by
\begin{itemize}
\item $1[x_1 \leftarrow \read{i_1}; \ x_2 \leftarrow \read{i_2}; \ \ret{(x_1,x_2)}] \sim 1[x_2 \leftarrow \read{i_2}; \ x_1 \leftarrow \read{i_1}; \ \ret{(x_1,x_2)}]$
\item $1[x_1 \leftarrow \val{v_1}; \ x_2 \leftarrow \read{i_2}; \ \ret{(x_1,x_2)}] \sim 1[x_2 \leftarrow \read{i_2}; \ x_1 \leftarrow \val{v_1}; \ \ret{(x_1,x_2)}]$ for value $v_1 \in \int{\tau_1}$
\item $1[x_1 \leftarrow \read{i_1}; \ x_2 \leftarrow \val{v_2}; \ \ret{(x_1,x_2)}] \sim 1[x_2 \leftarrow \val{v_2}; \ x_1 \leftarrow \read{i_1}; \ \ret{(x_1,x_2)}]$ for value $v_2 \in \int{\tau_2}$
\item $1[x_1 \leftarrow \val{v_1}; \ x_2 \leftarrow \val{v_2}; \ \ret{(x_1,x_2)}] \sim 1[x_2 \leftarrow \val{v_2}; \ x_1 \leftarrow \val{v_1}; \ \ret{(x_1,x_2)}]$ for
\begin{itemize}
\item values $v_1 \in \int{\tau_1}$ and $v_2 \in \int{\tau_2}$
\end{itemize}
\item $1[x_2 \leftarrow \read{i_2}; \ \ret{(v_1,x_2)}] \sim 1[x_2 \leftarrow \read{i_2}; \ x_1 \leftarrow \val{v_1}; \ \ret{(x_1,x_2)}]$ for value $v_1 \in \int{\tau_1}$
\item $1[x_1 \leftarrow \read{i_1}; \ x_2 \leftarrow \val{v_2}; \ \ret{(x_1,x_2)}] \sim 1[x_1 \leftarrow \read{i_1}; \ \ret{(x_1,v_2)}]$ for value $v_2 \in \int{\tau_2}$
\item $1[x_2 \leftarrow \val{v_2}; \ \ret{(v_1,x_2)}] \sim 1[x_2 \leftarrow \val{v_2}; \ x_1 \leftarrow \val{v_1}; \ \ret{(x_1,x_2)}]$ for values $v_1 \in \int{\tau_1}$ and $v_2 \in \int{\tau_2}$
\item $1[x_1 \leftarrow \val{v_1}; \ x_2 \leftarrow \val{v_2}; \ \ret{(x_1,x_2)}] \sim 1[x_1 \leftarrow \val{v_1}; \ \ret{(x_1,v_2)}]$ for
values $v_1 \in \int{\tau_1}$ and $v_2 \in \int{\tau_2}$
\item $1[\val{v_1 v_2}] \sim 1[\val{v_1 v_2}]$ for values $v_1 \in \int{\tau_1}$ and $v_2 \in \int{\tau_2}$
\end{itemize}
\end{itemize}
\end{proof}

\begin{figure}
\begin{mathpar}
\inferrule*[right=exch-samp-samp]{\dist_1 : \sigma_1 \twoheadrightarrow \tau_1, \dist_2 : \sigma_2 \twoheadrightarrow \tau_2 \in \Sigma \\ \Gamma \vdash e_1 : \sigma_1 \\ \Gamma \vdash e_2 : \sigma_2}{\Delta; \ \Gamma \vdash \big(x_1 : \tau_1 \leftarrow \Samp{\dist_1}{\sigma_1}{\tau_1}{e_1}; \ x_2 : \tau_2 \leftarrow \Samp{\dist_2}{\sigma_2}{\tau_2}{e_2}; \ \ret{(x_1,x_2)}\big) = \hspace{52.3pt} \\ \big(x_2 : \tau_2 \leftarrow \Samp{\dist_2}{\sigma_2}{\tau_2}{e_2}; \ x_1 : \tau_1 \leftarrow \Samp{\dist_1}{\sigma_1}{\tau_1}{e_1}; \ \ret{(x_1,x_2)}\big) : I \to \tau_1 \times \tau_2\hspace{-39.8pt}}\and
\inferrule*[right=exch-samp-read]{\dist : \sigma \twoheadrightarrow \tau_1 \in \Sigma \\ \Gamma \vdash e : \sigma \\ i : \tau_2 \in \Delta \\ i \in I}{\Delta; \ \Gamma \vdash \big(x_1 : \tau_1 \leftarrow \Samp{\dist}{\sigma}{\tau_1}{e}; \ x_2 : \tau_2 \leftarrow \read{i}; \ \ret{(x_1,x_2)}\big) = \hspace{52.3pt} \\ \big(x_2 : \tau_2 \leftarrow \read{i}; \ x_1 : \tau_1 \leftarrow \Samp{\dist}{\sigma}{\tau_1}{e}; \ \ret{(x_1,x_2)}\big) : I \to \tau_1 \times \tau_2\hspace{-39.8pt}}\and
\inferrule*[right=exch-read-read]{i_1 : \tau_1, i_2 : \tau_2 \in \Delta \\ i_1, i_2 \in I}{\Delta; \ \Gamma \vdash \big(x_1 : \tau_1 \leftarrow \read{i_1}; \ x_2 : \tau_2 \leftarrow \read{i_2}; \ \ret{(x_1,x_2)}\big) = \hspace{52.3pt} \\ \big(x_2 : \tau_2 \leftarrow \read{i_2}; \ x_1 : \tau_1 \leftarrow \read{i_1}; \ \ret{(x_1,x_2)}\big) : I \to \tau_1 \times \tau_2 \hspace{-39.8pt}}
\end{mathpar}
\caption{Alternative formulation of the \textsc{exch} rule for reaction equality.}
\label{fig:exch_alt}
\end{figure}

At last we get to the protocol level. A protocol $\Delta \vdash P : I \to O$ is said to have value $v \in \int{\tau}$ on channel $o \in O$ with $o : \tau \in \Delta$ if $P$ contains the assignment $\assign{o}{v}$; otherwise we define the value to be $\bot$. We write $\valueat{P}{o} = v_\bot$ to indicate that the value of $P$ on $o$ is $v_\bot \in \{\bot\} \cup \int{\tau}$, and lift this notation to distributions in the obvious way. In the case when $\valueat{P}{o} = \bot$, $P$ contains the assignment $\assign{o}{R}$ for some reaction $\Delta; \ \cdot \vdash R : I \to \tau$. The value of $R$ will be called the \emph{local value} of $P$ at $o$ (this terminology indicates that the computation of the channel $o$ to $v$ has not yet been communicated to the rest of the protocol). We write $\localvalueat{P}{o} = v_\bot$ to indicate that the local value of $P$ on $o$ is $v_\bot \in \{\bot\} \cup \int{\tau}$ (therefore $\valueof{R} = v_\bot$), and lift this notation to distributions $\eta$ in the obvious way. Thus, $\localvalueat{\eta}{o} = v$ indicates that each protocol in the support of $\eta$ contains the assignment $o \coloneqq \val{v}$, whereas $\localvalueat{\eta}{o} = \bot$ indicates that each protocol in the support of $\eta$ carries a stuck reaction on $o$.

A protocol bisimulation is entirely analogous to a reaction bisimulation, except we require the valuation property to hold: \emph{i)} per output channel $o$, and \emph{ii)} for all distributions (not necessarily final).

\begin{definition}[Protocol bisimulation]
A \emph{protocol bisimulation} $\sim$ is a binary relation on distributions on protocols of type $\Delta \vdash I \to O$ satisfying the following conditions:
\begin{itemize}
\item \emph{Closure under convex combinations}: For any distributions $\eta_1 \sim \varepsilon_1$ and $\eta_2 \sim \varepsilon_2$, and any coefficients $c_1, c_2 > 0$ with $c_1 + c_2 = 1$, we have $(c_1 * \eta_1 + c_2 * \eta_2) \sim (c_1 * \varepsilon_1 + c_2 * \varepsilon_2)$.

\item \emph{Closure under input assignment}: For any distributions $\eta \sim \varepsilon$, input channel $i \in I$ with $i : \tau \in \Delta$, and value $v \in \int{\tau}$, we have $\eta[\read{i} \coloneqq \val{v}] \sim \varepsilon[\read{i} \coloneqq \val{v}]$.

\item \emph{Closure under computation}: For any distributions $\eta \sim \varepsilon$, we have $(\eval{\eta}) \sim (\eval{\varepsilon})$.

\item \emph{Valuation property}: For any output channel $o \in O$ with $o : \tau \in \Delta$ and any distributions $\eta \sim \varepsilon$, there exists a joint convex combination \[\eta = \sum_i c_i * \eta_i \; \sim \, \sum_i c_i * \varepsilon_i = \varepsilon\]
with $c_i > 0$ and $\sum_i c_i = 1$ such that
\begin{itemize}
\item the respective components $\eta_i \sim \varepsilon_i$ are again related, and
\item $\valueat{\eta_i}{o} = v_\bot = \valueat{\varepsilon_j}{o}$ for the same $v_\bot \in \{\bot\} \cup \int{\tau}$ if and only if $i = j$.
\end{itemize}
\end{itemize}
\end{definition}

\noindent We have the analogous results for bisimulations on the protocol level:

\begin{lemma}
We have the following: 
\begin{itemize}
\item The identity relation is a protocol bisimulation.
\item The inverse of a protocol bisimulation is a protocol bisimulation.
\item The composition of two protocol bisimulations is a protocol bisimulation.
\end{itemize}
\end{lemma}

\begin{definition}
Let $\sim$ be an arbitrary binary relation on distributions on protocols of type $\Delta \vdash I \to O$. The \emph{expansion} $\lift{\sim}$ is the closure of $\sim$ under joint convex combinations. Explicitly, $\lift{\sim}$ is defined by
\[\Big(\sum_i c_i * \eta_i\Big) \; \lift{\sim} \; \Big(\sum_i c_i * \varepsilon_i\Big)\]
for coefficients $c_i > 0$ with $\sum_i c_i = 1$ and distributions $\eta_i \sim \varepsilon_i$.
\end{definition}

\begin{lemma}\label{lem:protocol_seed}
Let $\sim$ be a binary relation on distributions on protocols of type $\Delta \vdash I \to O$ with the following properties:
\begin{itemize}
\item \emph{Closure under input assignment}: For any distributions $\eta \sim \varepsilon$, input channel $i \in I$ with $i : \tau \in \Delta$, and value $v \in \int{\tau}$, we have $\eta[\read{i} \coloneqq \val{v}] \sim \varepsilon[\read{i} \coloneqq \val{v}]$.

\item \emph{Expansion closure under computation}: For any distributions $\eta \sim \varepsilon$, we have $(\eval{\eta}) \; \lift{\sim} \; (\eval{ \varepsilon})$.

\item \emph{Valuation property}: For any output channel $o \in O$ with $o : \tau \in \Delta$ and any distributions $\eta \sim \varepsilon$, there exists a joint convex combination \[\eta = \sum_i c_i * \eta_i \; \sim \, \sum_i c_i * \varepsilon_i = \varepsilon\]
with $c_i > 0$ and $\sum_i c_i = 1$ such that
\begin{itemize}
\item the respective components $\eta_i \sim \varepsilon_i$ are again related, and
\item $\valueat{\eta_i}{o} = v_\bot = \valueat{\varepsilon_j}{o}$ for the same $v_\bot \in \{\bot\} \cup \int{\tau}$ if and only if $i = j$.
\end{itemize}
\end{itemize}
Then the expansion $\lift{\sim}$ is a protocol bisimulation.
\end{lemma}

\noindent We now formally state what it means for exact protocol equality to be sound:

\begin{definition}
An axiom $\Delta \vdash P_1 = P_2 : I \to O$ is \emph{sound} if there is a protocol bisimulation $\sim$ such that $1[P_1] \sim 1[P_2]$.
\end{definition}

\noindent The ambient exact \ipdl theory $\mathbb{T}_\mathsf{prot}$ for protocols is said to be sound if each of its axioms is sound. We now show that this implies overall soundness for exact equality:

\begin{lemma}[Soundness of exact equality of protocols]
If the ambient (exact) \ipdl theories for expressions, reactions, and protocols are sound, then for any protocols $\Delta \vdash P_1 = P_2 : I \to O$ there exists a protocol bisimulation $\sim$ such that $1[P_1] \sim 1[P_2]$.
\end{lemma}

\begin{proof}
We first replace the rules \textsc{fold-if-left} and \textsc{fold-if-right} by the equivalent formulation in Figure \ref{fig:fold_if_alt}. We now proceed by induction on this alternative set of rules for exact protocol equality. We will freely use a distribution in place of a reaction (rule \textsc{cong-react}) or a protocol (rules \textsc{embed}, \textsc{absorb-left}) to indicate the obvious lifting of the corresponding construct to distributions on protocols.

\begin{itemize}
\item \textsc{refl}: Our desired bisimulation is the identity relation.
\item \textsc{sym}: Our desired bisimulation is the inverse of the bisimulation obtained from the premise.
\item \textsc{trans}: Our desired bisimulation is the composition of the two bisimulations obtained from the two premises.
\item \textsc{axiom}: Our desired bisimulation is precisely the bisimulation obtained from the soundness of the axiom.
\item \textsc{input-unused}: Our desired bisimulation is precisely the bisimulation obtained from the premise, seen as a bisimulation on distributions on protocols with the additional input $i$.
\item \textsc{embed}: Let $\sim$ be the bisimulation obtained from the premise. Our desired bisimulation $\sim_\phi$ is defined by
\begin{itemize}
\item $\phi^\star(\eta) \sim_\phi \phi^\star(\varepsilon)$ if $\eta \sim \varepsilon$
\end{itemize}
\item \textsc{cong-react}: Let $\sim$ be the reaction bisimulation obtained from the premise. Our desired bisimulation is the expansion of the relation $\sim_{\mathsf{react}}$ defined by
\begin{itemize}
\item $(o \coloneqq \eta) \sim_{\mathsf{react}} (o \coloneqq \eta')$ for distributions $\eta \sim \eta'$
\item $1[o \coloneqq v] \sim_{\mathsf{react}} 1[o \coloneqq v]$ for  value $v \in \int{\tau}$
\end{itemize}
\item \textsc{cong-comp-left}: Let $\sim$ be the bisimulation obtained from the premise. The expansion of the relation $\sim_{\mathsf{par}}$ defined by
\begin{itemize}
\item $(\Par{\eta}{Q}) \sim_{\mathsf{par}} (\Par{\eta'}{Q})$ for $\eta \sim \eta'$ and protocol $\Delta \vdash Q : I \cup O_1 \to O_2$
\end{itemize}
is the natural candidate for our desired bisimulation. Proving that this is indeed a bisimulation requires a fair amount of work (see Lemma \ref{lem:composability_exact}).
\item \textsc{cong-new}: Let $\sim$ be the bisimulation obtained from the premise. Our desired bisimulation $\sim_{\mathsf{new}}$ is defined by
\begin{itemize}
\item $\big(\new{o}{\tau}{\eta}\big) \sim_{\mathsf{new}} \big(\new{o}{\tau}{\eta'}\big)$ if $\eta \sim \eta'$
\end{itemize}
\item \textsc{comp-comm}: Our desired bisimulation is the expansion of the relation $\sim$ defined by
\begin{itemize}
\item $1[\Par{P_1}{P_2}] \sim 1[\Par{P_2}{P_1}]$ for protocols $\Delta \vdash P_1 : I \cup O_2 \to O_1$ and $\Delta \vdash P_2 : I \cup O_1 \to O_2$
\end{itemize}
\item \textsc{comp-assoc}: Our desired bisimulation is the expansion of the relation $\sim$ defined by
\begin{itemize}
\item $1\big[\Par{(\Par{P_1}{P_2})}{P_3}\big] \sim 1\big[\Par{P_1}{(\Par{P_2}{P_3})}\big]$ for
\begin{itemize}
\item protocol $\Delta \vdash P_1 : I \cup O_2 \cup O_3 \to O_1$
\item protocol $\Delta \vdash P_2 : I \cup O_1 \cup O_3 \to O_2$
\item protocol $\Delta \vdash P_3 : I \cup O_1 \cup O_2 \to O_3$
\end{itemize}
\end{itemize}
\item \textsc{new-exch}: The desired bisimulation is the expansion of the relation $\sim$ defined by
\begin{itemize}
\item $1[\new{o_1}{\tau_1}{\new{o_2}{\tau_2}{P}}] \sim 1[\new{o_2}{\tau_2}{\new{o_1}{\tau_1}{P}}]$ for
\begin{itemize}
\item protocol $\Delta, o_1 : \tau_1, o_2 : \tau_2 \vdash P : I \to O \cup \{o_1,o_2\}$
\end{itemize}
\end{itemize}
\item \textsc{comp-new}: Our desired bisimulation is the expansion of the relation $\sim$ defined by
\begin{itemize}
\item $1[\Par{P}{(\new{o}{\tau}{Q}\big})] \sim 1[\new{o}{\tau}{(\Par{P}{Q})}]$ for
\begin{itemize}
\item protocol $\Delta \vdash P : I \cup O_2 \to O_1$
\item protocol $\Delta, o : \tau \vdash Q : I \cup O_1 \to O_2 \cup \{o\}$
\end{itemize}
\end{itemize}
\item \textsc{absorb-left}: Our desired bisimulation is the expansion of the relation $\sim$ defined by
\begin{itemize}
\item $1[\Par{P}{Q}] \sim 1[P]$ for protocols $\Delta \vdash P : I \to O$ and $\Delta \vdash Q : I \cup O \to \emptyset$
\end{itemize}
\item \textsc{diverge}: Our desired bisimulation is the expansion of the relation $\sim$ defined by
\begin{itemize}
\item $1[\assign{o}{x \leftarrow \read{o}; \ R}] \sim 1[\assign{o}{\read{o}}]$ for reaction $\Delta; \ \cdot \vdash R : I \to \tau$
\end{itemize}
\item \textsc{fold-if-left}: Our desired bisimulation is the expansion of the relation $\sim$ defined by
\begin{itemize}
\item $1[\new{l}{\tau}{\Par{\assign{o}{x \leftarrow \read{b}; \ \ifte{x}{\read{l}}}{S_2}}{\assign{l}{x \leftarrow \read{b}; \ S_1}}}] \sim \\ 1[\assign{o}{x \leftarrow \read{b}; \ \ifte{x}{S_1}{S_2}}]$ for
\begin{itemize}
\item reaction $\Delta; \ \cdot \vdash S_1 : I \to \tau$
\item reaction $\Delta; \ \cdot \vdash S_2 : I \to \tau$
\end{itemize}
\item $1[\new{l}{\tau}{\Par{\assign{o}{x \leftarrow \val{v}; \ \ifte{x}{\read{l}}}{S_2}}{\assign{l}{x \leftarrow \val{v}; \ S_1}}}] \sim \\ 1[\assign{o}{x \leftarrow \val{v}; \ \ifte{x}{S_1}{S_2}}]$ for
\begin{itemize}
\item value $v \in \{0,1\}$
\item reaction $\Delta; \ \cdot \vdash S_1 : I \to \tau$
\item reaction $\Delta; \ \cdot \vdash S_2 : I \to \tau$
\end{itemize}
\item $1[\new{l}{\tau}{\Par{\assign{o}{\read{l}}}{\assign{l}{S_1}}}] \sim 1[\assign{o}{S_1}]$ for reaction $\Delta; \ \cdot \vdash S_1 : I \to \tau$
\item $1[\new{l}{\tau}{\Par{\assign{o}{S_2}}{\assign{l}{S_1}}}] \sim 1[\assign{o}{S_2}]$ for reactions $\Delta; \ \cdot \vdash S_1 : I \to \tau$ and $\Delta; \ \cdot \vdash S_2 : I \to \tau$
\item $1[\new{l}{\tau}{\Par{\assign{o}{v_2}}{\assign{l}{S_1}}}] \sim 1[\assign{o}{v_2}]$ for reaction $\Delta; \ \cdot \vdash S_1 : I \to \tau$ and value $v_2 \in \int{\tau}$
\item $1[\new{l}{\tau}{\Par{\assign{o}{S_2}}{\assign{l}{v_1}}}] \sim 1[\assign{o}{S_2}]$ for value $v_1 \in \int{\tau}$ and reaction $\Delta; \ \cdot \vdash S_2 : I \to \tau$
\item $1[\new{l}{\tau}{\Par{\assign{o}{v_2}}{\assign{l}{v_1}}}] \sim 1[\assign{o}{v_2}]$ for values $v_1,v_2 \in\int{\tau}$
\end{itemize}
\item \textsc{fold-if-right}: Our desired bisimulation is the expansion of the relation $\sim$ defined by
\begin{itemize}
\item $1[\new{r}{\tau}{\Par{\assign{o}{x \leftarrow \read{b}; \ \ifte{x}{S_1}{\read{r}}}}{\assign{r}{x \leftarrow \read{b}; \ S_2}}}] \sim \\ 1[\assign{o}{x \leftarrow \read{b}; \ \ifte{x}{S_1}{S_2}}]$ for
\begin{itemize}
\item reaction $\Delta; \ \cdot \vdash S_1 : I \to \tau$
\item reaction $\Delta; \ \cdot \vdash S_2 : I \to \tau$
\end{itemize}
\item $1[\new{r}{\tau}{\Par{\assign{o}{x \leftarrow \val{v}; \ \ifte{x}{S_1}{\read{r}}}}{\assign{r}{x \leftarrow \val{v}; \ S_2}}}] \sim \\ 1[\assign{o}{x \leftarrow \val{v}; \ \ifte{x}{S_1}{S_2}}]$ for
\begin{itemize}
\item value $v \in \{0,1\}$
\item reaction $\Delta; \ \cdot \vdash S_1 : I \to \tau$
\item reaction $\Delta; \ \cdot \vdash S_2 : I \to \tau$
\end{itemize}
\item $1[\new{r}{\tau}{\Par{\assign{o}{\read{r}}}{\assign{r}{S_2}}}] \sim 1[\assign{o}{S_2}]$ for reaction $\Delta; \ \cdot \vdash S_2 : I \to \tau$
\item $1[\new{r}{\tau}{\Par{\assign{o}{S_1}}{\assign{r}{S_2}}}] \sim 1[\assign{o}{S_1}]$ for reactions $\Delta; \ \cdot \vdash S_1 : I \to \tau$ and $\Delta; \ \cdot \vdash S_2 : I \to \tau$
\item $1[\new{r}{\tau}{\Par{\assign{o}{S_1}}{\assign{r}{v_2}}}] \sim 1[\assign{o}{S_1}]$ for reaction $\Delta; \ \cdot \vdash S_1 : I \to \tau$ and value $v_2 \in \int{\tau}$
\item $1[\new{r}{\tau}{\Par{\assign{o}{v_1}}{\assign{r}{S_2}}}] \sim 1[\assign{o}{v_1}]$ for value $v_1 \in \int{\tau}$ and reaction $\Delta; \ \cdot \vdash S_2 : I \to \tau$
\item $1[\new{r}{\tau}{\Par{\assign{o}{v_1}}{\assign{r}{v_2}}}] \sim 1[\assign{o}{v_1}]$ for values $v_1, v_2 \in \int{\tau}$
\end{itemize}
\item \textsc{fold-bind}: Our desired bisimulation is the expansion of the relation $\sim$ defined by
\begin{itemize}
\item $1[\new{c}{\sigma}{\Par{\assign{o}{x \leftarrow \read{c};} \ S}{\assign{c}{R}}}] \sim 1[\assign{o}{x \leftarrow R; \ S}]$ for
\begin{itemize}
\item reaction $\Delta; \ \cdot \vdash R : I \to \sigma$
\item reaction $\Delta; \ x : \sigma \vdash S : I \to \tau$
\end{itemize}
\item $1[\new{c}{\sigma}{\Par{\assign{o}{S}}{\assign{c}{u}}}] \sim 1[\assign{o}{S}]$ for value $u \in \int{\sigma}$ and reaction $\Delta; \ \cdot \vdash S : I \to \tau$
\item $1[\new{c}{\sigma}{\Par{\assign{o}{v}}{\assign{c}{u}}}] \sim 1[\assign{o}{u}]$ for values $u \in \int{\sigma}$ and $v \in \int{\tau}$
\end{itemize}
\item \textsc{subst}: Let $\sim$ be the reaction bisimulation obtained from the premise
\[\Delta; \ \cdot \vdash \big(x_1 \leftarrow R_1; \ x'_1 \leftarrow R_1; \ \ret{(x_1,x'_1)}\big) = \big(x_1 \leftarrow R_1; \ \ret{(x_1,x_1)}\big) : I \to \tau_1 \times \tau_1\]
Our desired bisimulation is the expansion of the relation $\sim_\mathsf{subst}$ defined by
\begin{itemize}
\item $\big(\Par{\assign{o_1}{\eta}}{\assign{o_2}{x_1 \leftarrow \read{o_1}; \ R_2}}\big) \sim_\mathsf{subst} \big(\Par{\assign{o_1}{\eta}}{\assign{o_2}{x_1 \leftarrow \eta; \ R_2}}\big)$ for
\begin{itemize}
\item distribution $\eta$ on reactions $\Delta; \ \cdot \vdash R_1 : I \to \tau_1$
\item reaction $\Delta; \ \cdot \vdash R_1 : I \to \tau_1$ such that $\eval{R_1} = \eval{\eta}$
\item reaction $\Delta; \ x_1 : \tau_1 \vdash R_2 : I \to \tau_2$
\end{itemize}
such that $1[x_1 \leftarrow R_1; \ x_1' \leftarrow R_1; \ \ret{(x_1,x_1')}] \sim 1[x_1 \leftarrow R_1; \ \ret{(x_1,x_1)}]$
\item $1[\Par{\assign{o_1}{v_1}}{\assign{o_2}{R_2}}] \sim_\mathsf{subst} 1[\Par{\assign{o_1}{v_1}}{\assign{o_2}{R_2}}]$ for value $v_1 \in \int{\tau_1}$ and reaction $\Delta; \ \cdot \vdash R_2 : I \to \tau_2$
\item $1[\Par{\assign{o_1}{v_1}}{\assign{o_2}{v_2}}] \sim_\mathsf{subst} 1[\Par{\assign{o_1}{v_1}}{\assign{o_2}{v_2}}]$ for values $v_1 \in \int{\tau_1}$ and $v_2 \in \int{\tau_2}$
\end{itemize}
\item \textsc{drop}: Let $\sim$ be the reaction bisimulation obtained from the premise
\[\Delta; \ \cdot \vdash \big(x_1 \leftarrow R_1; \ R_2\big) = R_2 : I \to \tau_2\]
Our desired bisimulation is the expansion of the relation $\sim_{\mathsf{drop}}$ defined by
\begin{itemize}
\item $\big(\Par{\assign{o_1}{\eta_1}}{\assign{o_2}{x_1 \leftarrow \read{o_1}; \ R_2}}\big) \sim_{\mathsf{drop}} \big(\Par{\assign{o_1}{\eta_1}}{\assign{o_2}{\eta_2}}\big)$ for
\begin{itemize}
\item distribution $\eta_1$ on reactions $\Delta; \ \cdot \vdash R_1 : I \to \tau_1$
\item reaction $\Delta; \ \cdot \vdash R_1 : I \to \tau_1$ such that either
\begin{itemize}
\item[\emph{i)}] $\eval{R_1} = \eval{\eta_1}$, or
\item[\emph{ii)}] $\eval{R_1} = c_1 (\eval{\eta_1}) + c_2 \mu$ for some distribution $\mu$ and some $c_1,c_2 > 0$ with $c_1 + c_2 = 1$
\end{itemize}
\item distribution $\eta_2$ on reactions $\Delta; \ \cdot \vdash R_2 : I \to \tau_2$
\item reaction $\Delta; \ \cdot \vdash R_2 : I \to \tau_2$ such that $\eval{R_2} = \eval{\eta_2}$
\end{itemize}
such that $1[x_1 \leftarrow R_1; \ R_2] \sim 1[R_2]$
\item $1[\Par{\assign{o_1}{v_1}}{\assign{o_2}{R_2}}] \sim_{\mathsf{drop}} 1[\Par{\assign{o_1}{v_1}}{\assign{o_2}{R_2}}]$ for value $v_1 \in \int{\tau_1}$ and reaction $\Delta; \ \cdot \vdash R_2 : I \to \tau_2$
\item $1[\Par{\assign{o_1}{v_1}}{\assign{o_2}{v_2}}] \sim_{\mathsf{drop}} 1[\Par{\assign{o_1}{v_1}}{\assign{o_2}{v_2}}]$ for values $v_1 \in \int{\tau_1}$ and $v_2 \in \int{\tau_2}$
\end{itemize}
\end{itemize}
\end{proof}

\begin{figure*}
\begin{mathpar}
\inferrule*[right=fold-if-left]{b \neq o \\ b \in I \\ b : \Bool, o : \tau \in \Delta \\ \Delta; \ \cdot \vdash S_1 : I \to \tau \\ \Delta; \ \cdot \vdash S_2 : I \to \tau}{\Delta \vdash \big(\new{l}{\tau}{\Par{\assign{o}{x \leftarrow \read{b}; \ \ifte{x}{{\color{red} \read{l}}}{S_2}}}{{\color{red} \assign{l}{x \leftarrow \read{b}; \ S_1}}}}\big) = \\ \big(\assign{o}{x \leftarrow \read{b}; \ \ifte{x}{{\color{red} S_1}}{S_2}}\big) : I \ \setminus \ \{o\} \to \{o\}\hspace{-23pt}}\and
\inferrule*[right=fold-if-right]{b \neq o \\ b \in I \\ b : \Bool, o : \tau \in \Delta \\ \Delta; \ \cdot \vdash S_1 : I \to \tau \\ \Delta; \ \cdot \vdash S_2 : I \to \tau}{\Delta \vdash \big(\new{r}{\tau}{\Par{\assign{o}{x \leftarrow \read{b}; \ \ifte{x}{S_1}{{\color{red} \read{r}}}}}{{\color{red} \assign{r}{x \leftarrow \read{b}; \ S_2}}}}\big) = \\ \big(\assign{o}{x \leftarrow \read{b}; \ \ifte{x}{S_1}{{\color{red} S_2}}}\big) : I \ \setminus \ \{o\} \to \{o\}\hspace{-21pt}}
\end{mathpar}
\caption{Alternative formulation of the \textsc{fold-if-left} and \textsc{fold-if-right} rules.}
\label{fig:fold_if_alt}
\end{figure*}

\noindent The remainder of this section is devoted to proving the following lemma:

\begin{lemma}[Composability for bisimulations]\label{lem:composability_exact}
Let $\sim$ be a bisimulation on protocols of type $\Delta \vdash I \cup O_2 \to O_1$. Then the expansion of the relation $\sim_{\mathsf{par}}$ defined by
\begin{itemize}
\item $(\Par{\eta}{Q}) \sim_{\mathsf{par}} (\Par{\eta'}{Q})$ for $\eta \sim \eta'$ and protocol $\Delta \vdash Q : I \cup O_1 \to O_2$
\end{itemize}
is again a protocol bisimulation.
\end{lemma}

The one property hard to verify is expansion closure under computation: \emph{for any protocol $\Delta \vdash Q : I \cup O_1 \to O_2$ and any distributions $\eta \sim \eta'$, we have $\eval{\big(\Par{\eta}{Q}\big)} \;\; \lift{\sim_\mathsf{par}} \;\; \eval{\big(\Par{\eta'}{Q}\big)}$}. The difficulty arises from the global nature of the protocol semantics: in the composition $\Par{P}{Q}$, a step of the form $P \outstep{o}{v} P'$ changes the protocol $Q$ (specifically to $Q[\assign{\read{o}}{\val{v}}]$). This makes it hard to express the computation of $\Par{P}{Q}$ in terms of the computation of $P$, because in the course of the latter we are simultaneously probabilistically updating $Q$.

We solve this problem by defining an alternate \emph{local} form of operational semantics for \ipdl protocols, where we can use local values (that is, an assignment of the form $\assign{o}{\val{v}}$ rather than $\assign{o}{v}$) to compute in $P$ without having to update $Q$ at the same time. The relation $P \pull{o}{v} Q$ (``\emph{pull}'') for protocols formalizes this notion.
\begin{mathpar}
\inferrule*[right=pull-react]{ }{\big(\assign{o}{\val{v}}\big) \pull{o}{v} \big(\assign{o}{\val{v}}\big)}\and
\inferrule*[right=pull-comp-left]{P \pull{o}{v} P'}{\Par{P}{Q} \pull{o}{v} \Par{P'}{Q[\assign{\read{o}}{\val{v}}]}}\and
\inferrule*[right=pull-comp-right]{Q \pull{o}{v} Q'}{\Par{P}{Q} \pull{o}{v} \Par{P[\assign{\read{o}}{\val{v}}]}{Q'}}\and
\inferrule*[right=pull-new]{P \pull{o}{v} P' \\ o \neq c}{\big(\new{c}{\tau}{P}\big) \pull{o}{v} \big(\new{c}{\tau}{P'}\big)}\\\\
\end{mathpar}
The only difference between $P \pull{o}{v} Q$ and $P \outstep{o}{v} Q$ is that the latter turns the assignment $\assign{o}{\val{v}}$ into $\assign{o}{v}$. We extract this simple extra step into the dual relation $P \localassign{o}{v} Q$ (``\emph{local assign}''), which, crucially, does not involve any manipulation of $\mathsf{read}$s.
\begin{mathpar}
\inferrule*[right=loc-assign-react]{ }{\big(\assign{o}{\val{v}}\big) \localassign{o}{v} \big(\assign{o}{v}\big)}\and
\inferrule*[right=loc-assign-comp-left]{P \localassign{o}{v} P'}{\big(\Par{P}{Q}\big) \localassign{o}{v} \big(\Par{P'}{Q}\big)}\and
\inferrule*[right=loc-assign-comp-right]{Q \localassign{o}{v} Q'}{\big(\Par{P}{Q}\big) \localassign{o}{v} \big(\Par{P}{Q'}\big)}\and
\inferrule*[right=loc-assign-new]{P \localassign{o}{v} P' \\ o \neq c}{\big(\new{c}{\tau}{P}\big) \localassign{o}{v} \big(\new{c}{\tau}{P'}\big)}
\end{mathpar}
The big-step form $P \Rightarrow \eta$ of our local operational semantics strings together a sequence of internal and pull steps.
\begin{mathpar}
\inferrule*{}{ }\and
\inferrule*{ }{P \Rightarrow 1[P]}\and
\inferrule*{P \pull{o}{v} Q \\ Q \Rightarrow \eta}{P \Rightarrow \eta}\and \\
\inferrule*{P \to \sum_i c_i * 1[P_i] \\ P_i \Rightarrow \eta_i}{P \Rightarrow \sum_i c_i * \eta_i}
\end{mathpar}
To bridge the gap between the local and the global versions of our operational semantics, we use the big-step form $P \downmapsto Q$, which strings together a sequence of local assign steps \emph{that coincide with output steps}.
\begin{mathpar}
\inferrule*{ }{P \downmapsto P}\and
\inferrule*{P \localassign{o}{v} P' \\ P \outstep{o}{v} P' \\ P' \downmapsto Q}{P \downmapsto Q}
\end{mathpar}
So if $P \downmapsto Q$ then $Q$ is a computation of $P$ that has been obtained chiefly by performing local assign steps.

\begin{lemma}[Lifting]
For any protocol $\Delta \vdash P : I \to O$ we have the following:
\begin{itemize}
\item If $P \localassign{o_1}{v_1} Q$ and $Q \outstep{o_2}{v_2} Q'$ then there is $P'$ such that $P \outstep{o_2}{v_2} P'$ and $P' \localassign{o_1}{v_1} Q'$.
\item If $P \localassign{o}{v} P'$ and $P' \to \eta'$ then there is $\eta$ such that $P \to \eta$ and $\eta \localassign{o}{v} \eta'$.
\item If $P \localassign{o_1}{v_1} Q$ and $P \outstep{o_1}{v_1} Q$, and
$Q \localassign{o_2}{v_2} Q'$ and $Q \outstep{o_2}{v_2} Q'$, then there is $P'$ such that \\ $P \localassign{o_2}{v_2} P'$ and $P \outstep{o_2}{v_2} P'$, and $P' \localassign{o_1}{v_1} Q'$ and $P' \outstep{o_1}{v_1} Q'$.
\item If $P \downmapsto Q$ and $Q \outstep{o}{v} Q'$ then there is $P'$ such that $P \outstep{o}{v} P'$ and $P' \downmapsto Q'$.
\item If $P \downmapsto P'$ and $P' \to \eta'$ then there is $\eta$ such that $P \to \eta$ and $\eta \downmapsto \eta'$.
\end{itemize}
\end{lemma}

\noindent In the above lemma, we lift the relations $\localassign{o}{v}$ and $\downmapsto$ to distributions in the natural way.

\begin{lemma}[Approximation]
For any protocol $\Delta \vdash P : I \to O$ we have the following:
\begin{itemize}
\item If $P \pull{o}{v} P'$ then there is $Q$ such that $P \outstep{o}{v} Q$ and $P' \outstep{o}{v} Q$.
\end{itemize}
\end{lemma}

\begin{lemma}[Factoring]
For any protocol $\Delta \vdash P : I \to O$ we have the following:
\begin{itemize}
\item If $P \outstep{o}{v} Q$ then there is $P'$ such that $P \pull{o}{v} P'$ and $P' \localassign{o}{v} Q$ and $P' \outstep{o}{v} Q$.
\end{itemize}
\end{lemma}

\begin{lemma}[Correctness]
For any protocol $\Delta \vdash P : I \to O$ we have the following:
\begin{itemize}
\item If $P \Rightarrow \eta$ then $\eval{P} = \eval{\eta}$.
\item If $P \downmapsto Q$ then $\eval{P} = \eval{Q}$.
\end{itemize}
\end{lemma}



\noindent The following lemma expresses the computation of $\Par{P}{Q}$ in terms of the local computation of $P$.

\begin{lemma}[Composability for local semantics]
For protocols $\Delta \vdash P : I \cup O_2 \to O_1$ and $\Delta \vdash Q : I \cup O_1 \to O_2$, if $P \Rightarrow \eta$ then $\eval{(\Par{P}{Q})} = \eval{\big(\Par{\eta}{Q}\big)}$.
\end{lemma}

\begin{lemma}[Termination]
For any protocol $\Delta \vdash P : I \to O$ there are distributions $\eta$ and $\varepsilon$ such that $P \Rightarrow \eta$ and $\eta \downmapsto \varepsilon$ and $\varepsilon$ is final.
\end{lemma}

\begin{proof}[Sketch]
We generalize the statement. \emph{Given any $n \in \nat$, any protocol $\Delta \vdash P : I \to O$, and any protocol $Q$ such that $P \downmapsto Q$ and $\snorm{Q} = n$, there are distributions $\eta$ and $\varepsilon$ such that $P \Rightarrow \eta$ and $\eta \downmapsto \varepsilon$ and $\varepsilon$ is final.} We prove this statement by induction on the structure bound $n$ using the lifting and factoring lemmas.
\end{proof}

\noindent We now have all the preliminaries necessary to prove that $\sim_\mathsf{par}$ enjoys expansion closure under computation.

\begin{proof}
We proceed by induction on the structure bound of the protocol $Q$:

\begin{claim}[Expansion closure under computation]
Given any $n \in \nat$, any protocol $\Delta \vdash Q : I \cup O_1 \to O_2$ such that $\snorm{Q} = n$, and any distributions $\eta \sim \eta'$ on protocols of type $\Delta \vdash I \cup O_2 \to O_1$, we have $\eval{\big(\Par{\eta}{Q}\big)} \;\; \lift{\sim_\mathsf{par}} \;\; \eval{\big(\Par{\eta'}{Q}\big)}$.
\end{claim}

In the remainder of this section we will work with a fixed $n \in \nat$. Since the set $O_1$ is finite, we can start off by successively applying the valuation property of the bisimulation $\sim$ to $\eta \sim \eta'$ for each output channel $o \in O_1$, until we end up with the special case when $\eta$ and $\eta'$ have the same value on each $o$. In other words, it suffices to prove the following:

\begin{claim}
Given any protocol $\Delta \vdash Q : I \cup O_1 \to O_2$ such that $\snorm{Q} = n$, and any distributions $\eta \sim \eta'$ on protocols of type $\Delta \vdash I \cup O_2 \to O_1$ such that
\begin{itemize}
\item for each channel $o \in O_1$ with $o : \tau \in \Delta$ we have $\valueat{\eta}{o} = v_\bot = \valueat{\eta'}{o}$ for some $v_\bot \in \{\bot\} \cup \int{\tau}$,
\end{itemize}
we have $\eval{\big(\Par{\eta}{Q}\big)} \;\; \lift{\sim_\mathsf{par}} \;\; \eval{\big(\Par{\eta'}{Q}\big)}$.
\end{claim}

By performing internal and pull steps on the distributions $\eta$ and $\eta'$, we can approximate their computations without changing any channel valuations. The resulting distributions will not be necessarily related by $\sim$ but that's okay: their computations will again be related, as these coincide with the computations of $\eta$ and $\eta'$, respectively. Specifically, by the correctness, composability, and termination lemmas for our local semantics it suffices to prove the following:

\begin{claim}
Given any protocol $\Delta \vdash Q : I \cup O_1 \to O_2$ such that $\snorm{Q} = n$, any distributions $\eta$ and $\eta'$ on protocols of type $\Delta \vdash I \cup O_2 \to O_1$ such that
\begin{itemize}
\item for each channel $o \in O_1$ with $o : \tau \in \Delta$ we have $\valueat{\eta}{o} = v_\bot = \valueat{\eta'}{o}$ for some $v_\bot \in \{\bot\} \cup \int{\tau}$,
\end{itemize}
and any final distributions $\varepsilon \sim \varepsilon'$ such that $\eta \downmapsto \varepsilon$ and $\eta' \downmapsto \varepsilon'$, we have $\eval{\big(\Par{\eta}{Q}\big)} \;\; \lift{\sim_\mathsf{par}} \;\; \eval{\big(\Par{\eta'}{Q}\big)}$.
\end{claim}

\noindent We now establish an analogue of the valuation property for the local semantics of protocols:

\begin{claim}[Local valuation property]
For any channel $o \in O_1$ with $o : \tau \in \Delta$, any distributions $\eta$ and $\eta'$ on protocols of type $\Delta \vdash I \cup O_2 \to O_1$ such that $\valueat{\eta}{o} = \bot = \valueat{\eta'}{o}$, and any final distributions $\varepsilon \sim \varepsilon'$ such that $\eta \downmapsto \varepsilon$ and $\eta' \downmapsto \varepsilon'$, there exist convex combinations \[\eta = \sum_i c_i * \eta_i, \;\; \eta' = \sum_i c_i * \eta'_i, \;\; \varepsilon = \sum_i c_i * \varepsilon_i, \;\; \varepsilon' = \sum_i c_i * \varepsilon'_i\]
with $c_i > 0$ and $\sum_i c_i = 1$ such that
\begin{itemize}
\item $\eta_i \downmapsto \varepsilon_i$ and $\eta'_i \downmapsto \varepsilon'_i$,
\item the respective components $\varepsilon_i \sim \varepsilon_i'$ are again related, and
\item $\localvalueat{\eta_i}{o} = v_\bot = \localvalueat{\eta'_j}{o}$ for the same $v_\bot \in \{\bot\} \cup \int{\tau}$ if and only if $i = j$.
\end{itemize}
\end{claim}

\noindent The local valuation property follows easily from the valuation property of the bisimulation $\sim$ and the fact that the relation $P \downmapsto Q$ turns local values on $o \in O$ into global ones: if $\localvalueat{P}{o} = v_\bot$ and $Q$ is final then $\valueat{Q}{o} = v_\bot$.

We can now carry out the analogous argument from earlier but for local valuation: since the set $O_1$ of output channels is finite, we can successively apply the local valuation property to $\eta$, $\eta'$ for each output channel $o$ where $\valueat{\eta}{o} = \bot = \valueat{\eta'}{o}$, until we end up with the special case when $\eta$ and $\eta'$ have the same local value on each such $o$. In other words, it suffices to prove the following:

\begin{claim}
Given any protocol $\Delta \vdash Q : I \cup O_1 \to O_2$ such that $\snorm{Q} = n$, any distributions $\eta$ and $\eta'$ on protocols of type $\Delta \vdash I \cup O_2 \to O_1$ such that
\begin{itemize}
\item for each channel $o \in O_1$ with $o : \tau \in \Delta$ we have $\valueat{\eta}{o} = v_\bot = \valueat{\eta'}{o}$ for some $v_\bot \in \{\bot\} \cup \int{\tau}$, and
\begin{itemize}
\item in the case that $\valueat{\eta}{o} = \bot = \valueat{\eta'}{o}$, we have $\localvalueat{\eta}{o} = v_\bot = \localvalueat{\eta'}{o}$ for some $v_\bot \in \{\bot\} \cup \int{\tau}$,
\end{itemize}
\end{itemize}
and any final distributions $\varepsilon \sim \varepsilon'$ such that $\eta \downmapsto \varepsilon$ and $\eta' \downmapsto \varepsilon'$, we have $\eval{\big(\Par{\eta}{Q}\big)} \;\; \lift{\sim_\mathsf{par}} \;\; \eval{\big(\Par{\eta'}{Q}\big)}$.
\end{claim}

To prove the latest claim, we let $\eta$ and $\eta'$ step simultaneously on the distinct output channels $o_1,\ldots,o_n \in O_1$ as follows, where the lifting lemma guarantees that the order in which we execute these steps is immaterial:
\begin{itemize}
\item $\eta = \mu_0 \localassign{o_1}{v_1} \mu_1 \localassign{o_2}{v_2} \ldots \localassign{o_{n-1}}{v_{n-1}} \mu_{n-1} \localassign{o_n}{v_n} \mu_n = \varepsilon$,
\item $\eta = \mu_0 \outstep{o_1}{v_1} \mu_1 \outstep{o_2}{v_2} \ldots \outstep{o_{n-1}}{v_{n-1}} \mu_{n-1} \outstep{o_n}{v_n} \mu_n = \varepsilon$, and
\item $\eta' = \mu'_0 \localassign{o_1}{v_1} \mu'_1 \localassign{o_2}{v_2} \ldots \localassign{o_{n-1}}{v_{n-1}} \mu'_{n-1} \localassign{o_n}{v_n} \mu'_n = \varepsilon'$,
\item $\eta' = \mu'_0 \outstep{o_1}{v_1} \mu'_1 \outstep{o_2}{v_2} \ldots \outstep{o_{n-1}}{v_{n-1}} \mu'_{n-1} \outstep{o_n}{v_n} \mu'_n = \varepsilon'$.
\end{itemize}

The valuation assumptions on $\eta$ and $\eta'$, and consequently on $\mu_i$ and $\mu'_i$, guarantee that the corresponding steps $\mu_i \outstep{o_{i+1}}{v_{i+1}} \mu_{i+1}$ and $\mu'_i \outstep{o_{i+1}}{v_{i+1}} \mu'_{i+1}$ exert the same effect on the common context, thereby yielding the same sequence $Q = Q_0, \ldots, Q_n$ of updates: we have
\begin{align*}
\Par{\mu_i}{Q_i} & \outstep{o_{i+1}}{v_{i+1}} \Par{\mu_{i+1}}{Q_i[\read{o_{i+1}} \coloneqq \val{v_{i+1}}]} \\
\Par{\mu'_i}{Q_i} & \outstep{o_{i+1}}{v_{i+1}} \Par{\mu'_{i+1}}{Q_i[\read{o_{i+1}} \coloneqq \val{v_{i+1}}]}
\end{align*}
and hence $Q_{i+1} \coloneqq Q_i[\read{o_{i+1}} \coloneqq \val{v_{i+1}}]$. It thus suffices to prove the following: 

\begin{claim}
Given any protocol $\Delta \vdash Q : I \cup O_1 \to O_2$ with $\snorm{Q} = n$, and any final distributions $\eta \sim \eta'$, we have $\eval{\big(\Par{\eta}{Q}\big)} \;\; \lift{\sim_\mathsf{par}} \;\; \eval{\big(\Par{\eta'}{Q}\big)}$.
\end{claim}

At last we have the opportunity to use the induction hypothesis: either $Q$ is itself final, in which case the claim follows at once from the definition of $\sim_\mathsf{par}$, or $Q$ takes a step using one of the stepping relations $\to$ and $\outstep{o}{v}$, in which case its structure bound becomes strictly smaller and the induction hypothesis applies.
\end{proof}