In this section we define an operational semantics for \ipdl expressions, reactions, and protocols. This semantics will validate the \emph{exact} fragment of our equational logic and prove perfect observational equivalence. To give semantics to user-defined symbols, we define interpretations:

\begin{definition}[Interpretation]
An interpretation $\int{-}$ for a signature $\Sigma$ associates to:
\begin{itemize}
\item each type symbol $\type$ a subset $\subseteq \{0,1\}^{|\type|}$ of bitstrings of a fixed length $|\type| \geq 0$;
\item each function symbol $\func : \sigma \to \tau$ a function $\int{\func}$ from $\int{\sigma}$ to $\int{\tau}$;
\item each distribution symbol $\dist : \sigma \twoheadrightarrow \tau$ a function $\int{\dist}$ from $\int{\sigma}$ to \emph{distributions} on $\int{\tau}$.
\end{itemize}
\end{definition}
\noindent In the above, we naturally lift the interpretation $\int{-}$ to all types by setting
\begin{align*}
\int{\one} & \coloneqq \{()\} \\
\int{\Bool} & \coloneqq \{0,1\} \\
\int{\tau \times \sigma} & \coloneqq \big\{v_1 v_2 \, | \, v_1 \in \int{\tau}, v_2 \in \int{\sigma}\big\}
\end{align*}
where $()$ denotes the empty bitstring and $v_1 v_2$ stands for the concatenation of bitstrings $v_1$ and $v_2$. We similarly define the length $|\sigma|$ of a type $\sigma$ in the obvious way, letting $|\one| \coloneqq 0$, $|\Bool| \coloneqq 1$, and $|\tau \times \sigma| \coloneqq |\tau| + |\sigma|$.

To handle partial computations, we augment the syntax of \ipdl expressions, reactions, and protocols to contain intermediate bitstring values $v \in \{0,1\}^\star$:\smallskip

\begin{syntax}
\category[Valued Expressions]{e}
\alternative{{\color{red} v}} \alternative{\dots}
\category[Valued Reactions]{R, S}
\alternative{{\color{red} \val{v}}} \alternative{\dots}
\category[Valued Protocols]{P, Q}
\alternative{{\color{red} \assign{o}{v}}} \alternative{\dots}
\end{syntax}\smallskip

\noindent We extend the notion of a Turing Machine bound to valued \ipdl constructs as follows, where $|v|$ denotes the length of the bitstring $v$:
\begin{align*}
\tmnorm{v} & \coloneqq |v| \\
\tmnorm{\val{v}} & \coloneqq |v| + 3 \\
\tmnorm{\assign{o}{v}} & \coloneqq |v| + 4
\end{align*}

\noindent Given an ambient interpretation $\int{-}$ for the signature $\Sigma$, we can type the valued counterpart of \ipdl constructs as expected: in addition to the regular typing rules, we have

\begin{mathpar}
\inferrule*{v \in \int{\tau}}{\Gamma \vdash v : \tau}\and
\inferrule*{v \in \int{\tau}}{\Delta; \ \Gamma \vdash \val{v} : I \to \tau}\and
\inferrule*{o : \tau \in \Delta \\ o \notin I \\ v \in \int{\tau}}{\Delta \vdash \big(\assign{o}{v}\big) : I \to \{o\}}
\end{mathpar}

\subsection{Small-Step Operational Semantics of IPDL}

The small-step semantics $e \to e'$ for expressions is standard, see Figure \ref{fig:expressions_semantics}. Pairing is given by bitstring concatenation (rule \textsc{pair}), and the projections $\fst_{\sigma \times \tau}$ and $\snd_{\sigma \times \tau}$ unambiguously split the pair according to $|\sigma|$ and $|\tau|$, respectively (rules \textsc{fst-eval} and \textsc{snd-eval}).

Reactions have a straightforward small-step semantics of the form $R \to \eta$, where $\eta$ is a probability distribution over reactions. Figure \ref{fig:reactions_semantics} shows the rules, where we write $1[R]$ for the distribution with unit mass at the reaction $R$, and freely use a distribution in place of a value (rule \textsc{samp-eval}) or a reaction (rule \textsc{bind-react}) to indicate the obvious lifting of the corresponding construct to distributions on reactions. All distributions are implicitly finitely supported. Crucially, there is no semantic rule for stepping the reaction $\read{c}$; we model communication via semantics for protocols, which substitute all instances of $\mathsf{read}$ for values.

We give semantics to protocols via two main small-step rules, see Figure \ref{fig:protocols_semantics}, where we analogously write $1[P]$ for the distribution with unit mass at the protocol $P$, and freely use a distribution in place of a reaction (in rule \textsc{step-react}) or a protocol (rules \textsc{step-comp-left}, \textsc{step-comp-right}, and \textsc{step-new}) to indicate the obvious lifting of the corresponding construct to distributions on protocols.

First we have the \emph{output} relation $P \outstep{o}{v} Q$, which is enabled when the reaction for channel $o$ in $P$ terminates, resulting in value $v$ (rule \textsc{out-val}). When this happens, the value of $o$ is broadcast through the protocol context enveloping $P$ (rules \textsc{out-comp-left}, \textsc{out-comp-left}, and \textsc{out-new}), resulting in each $\read{o}$ command in other reactions to be substituted with $\val{v}$. We note that the value of $o$ is not broadcast above the $\mathsf{new}$ quantifier when the local channel introduced is equal to $o$.

Next we have the \emph{internal stepping} relation $P \to \eta$, specified similarly to the small-step relation for reactions. The rule \textsc{step-react} lifts the stepping relation for $R$ to the stepping relation for $\assign{o}{R}$, while the three rules \textsc{step-comp-left}, \textsc{step-comp-right}, \textsc{step-new} simply propagate the stepping relation through parallel composition and the $\mathsf{new}$ quantifier. The last rule \textsc{out-new-hide} links the output relation with the stepping relation: whenever $P$ steps to $P'$, resulting in the output $\assign{o}{v}$, we have that $\new{o}{\tau}{P}$ steps with unit mass to $\new{o}{\tau}{P'}$.

\begin{figure}
\begin{mathpar}
\fbox{$e \to e'$}\\
\inferrule*[right=check]{ }{\checkmark \to ()}\and
\inferrule*[right=true]{ }{\true \to 1}\and
\inferrule*[right=false]{ }{\false \to 0}\and
\inferrule*[right=app-cong]{e \to e'}{\func \ e \to \func \ e'}\and
\inferrule*[right=app-eval]{ }{\func \ v \to \int{\func}(v)}\and
\inferrule*[right=pair-cong-fst]{e_1 \to e'_1}{(e_1,e_2) \to (e'_1,e_2)}\and
\inferrule*[right=pair-cong-snd]{e_2 \to e'_2}{(e_1,e_2) \to (e_1,e'_2)}\and
\inferrule*[right=pair-eval]{ }{(v_1,v_2) \to v_1 v_2}\and
\inferrule*[right=fst-cong]{e \to e'}{\fst_{\tau_1 \times \tau_2} \ e \to \fst_{\tau_1 \times \tau_2} \ e'}\and
\inferrule*[right=fst-eval]{v_1 \in \{0,1\}^{|\tau_1|} \\ v_2 \in \{0,1\}^{|\tau_2|}}{\fst_{\tau_1 \times \tau_2} \ v_1 v_2 \to v_1}\and
\inferrule*[right=snd-cong]{e \to e'}{\snd_{\tau_1 \times \tau_2} \ e \to \snd_{\tau_1 \times \tau_2} \ e'}\and
\inferrule*[right=snd-eval]{v_1 \in \{0,1\}^{|\tau_1|} \\ v_2 \in \{0,1\}^{|\tau_2|}}{\snd_{\tau_1 \times \tau_2} \ v_1 v_2 \to v_2}
\end{mathpar}
\caption{Small-step operational semantics for $\ipdl$ expressions.} 
\label{fig:expressions_semantics}
\end{figure}

\begin{figure}
\begin{mathpar}
\fbox{$R \to \eta$}\\
\inferrule*[right=ret-step]{e \to e'}{\ret{e} \to 1[\ret{e'}]}\and
\inferrule*[right=ret-eval]{ }{\ret{v} \to 1[\val{v}]}\and
\inferrule*[right=samp-step]{e \to e'}{\samp{\dist}{e} \to 1[\samp{\dist}{e'}]}\and
\inferrule*[right=samp-eval]{ }{\samp{\dist}{v} \to \val{\int{\dist}(v)}}\and
\inferrule*[right=if-step]{e \to e'}{\big(\ifte{e}{R_1}{R_2}\big) \to 1[\ifte{e'}{R_1}{R_2}]}\and
\inferrule*[right=if-true]{ }{\big(\ifte{1}{R_1}{R_2}\big) \to 1[R_1]}\and
\inferrule*[right=if-false]{ }{\big(\ifte{0}{R_1}{R_2}\big) \to 1[R_2]}\and
\inferrule*[right=bind-val]{ }{\big(x : \sigma \leftarrow \val{v}; \ S\big) \to 1\big[S[\assign{x}{v}]\big]}\and
\inferrule*[right=bind-react]{R \to \eta}{\big(x : \sigma \leftarrow R; \ S\big) \to {\big(x : \sigma \leftarrow \eta; \ S\big)}}
\end{mathpar}
\caption{Small-step operational semantics for $\ipdl$ reactions.} 
\label{fig:reactions_semantics}
\end{figure}

\begin{figure}
\begin{mathpar}
\fbox{$P \outstep{o}{v} Q$}\\
\inferrule*[right=out-react]{ }{\big(\assign{o}{\val{v}}\big) \outstep{o}{v} \big(\assign{o}{v}\big)}\and
\inferrule*[right=out-comp-left]{P \outstep{o}{v} P'}{\Par{P}{Q} \outstep{o}{v} \Par{P'}{Q[\assign{\read{o}}{\val{v}}]}}\and
\inferrule*[right=out-comp-right]{Q \outstep{o}{v} Q'}{\Par{P}{Q} \outstep{o}{v} \Par{P[\assign{\read{o}}{\val{v}}]}{Q'}}\and
\inferrule*[right=out-new]{P \outstep{o}{v} P' \\ o \neq c}{\big(\new{c}{\tau}{P}\big) \outstep{o}{v} \big(\new{c}{\tau}{P'}\big)}\\\\
\fbox{$P \to \eta$}\\
\inferrule*[right=step-react]{R \to \eta}{\big(\assign{o}{R}\big) \to \big(\assign{o}{\eta}\big)}\and
\inferrule*[right=step-comp-left]{P \to \eta}{\Par{P}{Q} \to \Par{\eta}{Q}}\and
\inferrule*[right=step-comp-right]{Q \to \eta}{\Par{P}{Q} \to \Par{P}{\eta}}\and
\inferrule*[right=step-new]{P \to \eta}{\big(\new{c}{\tau}{P}\big) \to \big(\new{c}{\tau}{\eta}\big)}\and
\inferrule*[right=out-new-hide]{P \outstep{c}{v} P'}{\big(\new{c}{\tau}{P}\big) \to 1[\new{c}{\tau}{P'}]}
\end{mathpar}
\caption{Small-step operational semantics for \ipdl protocols.}
\label{fig:protocols_semantics}
\end{figure}

\subsection{Big-Step Operational Semantics of IPDL}

The big-step semantics for expressions $e \Downarrow v$, see Figure \ref{fig:expressions_big_step}, performs a sequence of small steps to compute $e$ to a value. The big-step semantics for reactions $R \Downarrow \eta$, see Figure \ref{fig:reactions_big_step}, performs as many steps as possible in an attempt to compute $R$, resulting in a distribution $\eta$ on reactions. A reaction that cannot step any further is \emph{final}. We can syntactically describe final reactions as those that have either yielded a value or have an unresolved read in the leading position (\emph{i.e.}, are \emph{stuck}). The big-step operational semantics for protocols $P \Downarrow \eta$, see Figure \ref{fig:protocols_big_step}, performs as many output and internal steps as possible in an attempt to compute $P$, resulting in a distribution $\eta$ on protocols. A protocol that cannot step any further using either of the two stepping relations is \emph{final}. We can syntactically describe final protocols as those where every channel, including the internal ones, carries either a value or a reaction that is stuck.

\begin{figure}
\begin{mathpar}
\fbox{$e \Downarrow v$}\\
\inferrule*{ }{v \Downarrow v}\and
\inferrule*{e \to e' \\ e' \Downarrow v}{e \Downarrow v}
\end{mathpar}
\caption{Big-step operational semantics for \ipdl expressions.}
\label{fig:expressions_big_step}
\end{figure}

\begin{figure}
\begin{mathpar}
\fbox{$R \ \stuck$}\\
\inferrule*{ }{\big(\read{c}\big) \ \stuck}\and
\inferrule*{R \ \stuck}{\big(x : \sigma \leftarrow R; \ S\big) \ \stuck}\\
\fbox{$R \ \final$}\\
\inferrule*{ }{\big(\val{v}\big) \ \final}\and
\inferrule*{R \ \stuck}{R \ \final}\\
\fbox{$R \Downarrow \eta$}\\
\inferrule*{}{ }\and
\inferrule*{R \ \final}{R \Downarrow 1[R]}\and \\
\inferrule*{R \to \sum_i c_i * 1[R_i] \\ R_i \Downarrow \eta_i}{R \Downarrow \sum_i c_i * \eta_i}\and
\end{mathpar}
\caption{Big-step operational semantics for \ipdl reactions.}
\label{fig:reactions_big_step}
\end{figure}

\begin{figure}
\begin{mathpar}
\fbox{$P \ \final$}\\
\inferrule*{ }{\zero \ \final}\and
\inferrule*{ }{\big(\assign{o}{v}\big) \ \final}\and
\inferrule*{R \ \stuck}{\big(\assign{o}{R}\big) \ \final}\and
\inferrule*{P \ \final \\ Q \ \final}{\Par{P}{Q} \ \final}\and
\inferrule*{P \ \final}{\big(\new{o}{\tau}{P}\big) \ \final}\\\\
\fbox{$P \Downarrow \eta$}\\
\inferrule*{}{ }\and
\inferrule*{P \ \final}{P \Downarrow 1[P]}\and
\inferrule*{P \outstep{o}{v} Q \\ Q \Downarrow \eta}{P \Downarrow \eta}\and \\
\inferrule*{P \to \sum_i c_i * 1[P_i] \\ P_i \Downarrow \eta_i}{P \Downarrow \sum_i c_i * \eta_i}
\end{mathpar}
\caption{Big-step operational semantics for \ipdl protocols.}
\label{fig:protocols_big_step}
\end{figure}

Note that while the small-step semantics for reactions is sequential, both output and internal step relations for protocols are non-deterministic. Indeed, any two channels in a protocol may output in any order. Ordinarily, this presents a problem for reasoning about cryptography, since non-deterministic choice may present a security leak. However, our language introduces \emph{no} way to exploit this extra non-determinism, essentially due to the $\mathsf{read}$ commands in reactions being blocking. This is formalized by a number of \emph{confluence} results for \textsf{IPDL}:

\begin{lemma}[Confluence for expressions]
For any expression $\Gamma \vdash e : \tau$ we have the following:
\begin{itemize}
\item If $e \to e_1$ and $e \to e_2$ then either $e_1 = e_2$ or there is $e'$ such that $e_1 \to e'$ and $e_2 \to e'$.
\item If $e \to e'$ and $e \Downarrow v$ then $e' \Downarrow v$.
\end{itemize}
\end{lemma}

\begin{lemma}[Confluence for reactions]
For any reaction $\Delta; \ \Gamma \vdash R : I \to \tau$ we have the following:
\begin{itemize}
\item We have $R[\read{i} \coloneqq \val{v}][\read{i} \coloneqq \val{v}] = R[\read{i} \coloneqq \val{v}]$.
\item Let $i_1 \neq i_2$. Then $R[\read{i_1} \coloneqq \val{v_1}][\read{i_2} \coloneqq \val{v_2}] = R[\read{i_2} \coloneqq \val{v_2}][\read{i_1} \coloneqq \val{v_1}]$.
\item If $R \to \eta$ then $R[\read{i} \coloneqq \val{v}] \to \eta[\read{i} \coloneqq \val{v}]$.
\end{itemize}
\end{lemma}

\begin{lemma}[Confluence for protocols, part I]
For any protocol $\Delta \vdash P : I \to O$ we have the following:
\begin{itemize}
\item We have $P[\read{i} \coloneqq \val{v}][\read{i} \coloneqq \val{v}] = P[\read{i} \coloneqq \val{v}]$.
\item Let $i_1 \neq i_2$. Then $P[\read{i_1} \coloneqq \val{v_1}][\read{i_2} \coloneqq \val{v_2}] = P[\read{i_2} \coloneqq \val{v_2}][\read{i_1} \coloneqq \val{v_1}]$.
\item Let $o_1 \neq o_2$. If $P \outstep{o_1}{v_1} Q$ then $P[\read{o_2} \coloneqq \val{v_2}] \outstep{o_1}{v_1} Q[\read{o_2} \coloneqq \val{v_2}]$.
\item If $P \to \eta$ then $P[\read{o} \coloneqq \val{v}] \to \eta[\read{o} \coloneqq \val{v}]$.
\end{itemize}
\end{lemma}

\begin{lemma}[Confluence for protocols, part II]
For any protocol $\Delta \vdash P : I \to O$ we have the following:
\begin{itemize}
\item Let $o_1 \neq o_2$. If $P \outstep{o_1}{v_1} Q$ and $P \outstep{o_2}{v_2} P'$ then there is $Q'$ such that $Q \outstep{o_2}{v_2} Q'$ and $P' \outstep{o_1}{v_1} Q'$.
\item If $P \outstep{o}{v} P'$ and $P \to \eta$ then there is $\eta'$ such that $\eta \outstep{o}{v} \eta'$ and $P' \to \eta'$.
\item If $P \to \eta_1$ and $P \to \eta_2$ then either $\eta_1 = \eta_2$ or there is $\eta$ such that $\eta_1 \to \eta$ and $\eta_2 \to \eta$.
\end{itemize}
\end{lemma}

\noindent In the above lemma, we lift the two protocol stepping relations $\outstep{o}{v}$ and $\to$ to distributions in the natural way.

\begin{lemma}[Confluence for protocols, part III]
For any protocol $\Delta \vdash P : I \to O$ we have the following:
\begin{itemize}
\item If $P \outstep{o}{v} Q$ and $P \Downarrow \varepsilon$ then $Q \Downarrow \varepsilon$.
\item If $P \to \eta$ and $P \Downarrow \varepsilon$ then $\eta \Downarrow \varepsilon$.
\end{itemize}
\end{lemma}

\noindent To guarantee termination (for protocols especially), we introduce the notion of a \emph{structure bound}. The structure bound for expressions, defined below, is invariant under substitutions.
\begin{align*}
\snorm{v} & \coloneqq 0 \\
\snorm{\checkmark} & \coloneqq 1 \\
\snorm{\true} & \coloneqq 1 \\
\snorm{\false} & \coloneqq 1 \\
\snorm{\func \ e} & \coloneqq \snorm{e} + 1 \\
\snorm{(e_1,e_2)} & \coloneqq \snorm{e_1} + \snorm{e_2} + 1 \\
\snorm{\fst \ e} & \coloneqq \snorm{e} + 1 \\
\snorm{\snd \ e} & \coloneqq \snorm{e} + 1
\end{align*}
\noindent The structure bound for reactions, defined below, is invariant under substitutions, embeddings, and input assignment.
\begin{align*}
\snorm{\val{v}} & \coloneqq 0 \\
\snorm{\ret{e}} & \coloneqq \snorm{e} + 1 \\
\snorm{\samp{\dist}{e}} & \coloneqq \snorm{e} + 1 \\
\snorm{\read{c}} & \coloneqq 0 \\
\snorm{\ifte{e}{R_1}{R_2}} & \coloneqq \snorm{e} + \mathsf{max} \, \big(\snorm{R_1}, \snorm{R_2}\big) + 1 \\
\snorm{x \leftarrow R; \ S} & \coloneqq \snorm{R} + \snorm{S} + 1
\end{align*}
\noindent The structure bound for protocols, defined below, is invariant under embeddings and input assignment.
\begin{align*}
\snorm{\zero} & \coloneqq 0 \\
\snorm{\assign{o}{v}} & \coloneqq 0 \\
\snorm{\assign{o}{R}} & \coloneqq \snorm{R} + 1 \\
\snorm{\Par{P}{Q}} & \coloneqq \snorm{P} + \snorm{Q} \\
\snorm{\new{c}{\tau}{P}} & \coloneqq \snorm{P}
\end{align*}

\begin{lemma}[Progress for expressions]
For any expression $\cdot \vdash e : \tau$ we have the following:
\begin{itemize}
\item If $e \to e'$ then $\snorm{e'} < \snorm{e}$.
\end{itemize}
\end{lemma}

\begin{lemma}[Progress for reactions]
For any reaction $\Delta; \ \cdot \vdash R : I \to \tau$ we have the following:
\begin{itemize}
\item If $R \to \sum_i c_i * 1[R_i]$ with $c_i \neq 0$, then $\snorm{R_i} < \snorm{R}$.
\end{itemize}
\end{lemma}

\begin{lemma}[Progress for protocols]
For any protocol $\Delta \vdash P : I \to O$ we have the following:
\begin{itemize}
\item If $P \outstep{o}{v} Q$ then $\snorm{Q} < \snorm{P}$.
\item If $P \to \sum_i c_i * 1[P_i]$ with $c_i \neq 0$, then $\snorm{P_i} < \snorm{P}$.
\end{itemize}
\end{lemma}

\noindent The confluence and progress lemmas together imply that our semantics is indeed well-defined:

\begin{corollary}[Determinism of $\Downarrow$ for expressions]
For any expression $\Gamma \vdash e : \tau$, there exists a unique bitstring $v$ such that $e \Downarrow v$. We will denote $v$ by $\eval{e}$.
\end{corollary}

\begin{corollary}[Determinism of $\Downarrow$ for reactions]
For any reaction $\Delta; \ \cdot \vdash R : I \to \tau$, there exists a unique distribution $\eta$ on reactions such that $R \Downarrow \eta$. We will denote $\eta$ by $\eval{R}$.
\end{corollary}

\begin{corollary}[Determinism of $\Downarrow$ for protocols]
For any protocol $\Delta \vdash P : I \to O$, there exists a unique distribution $\eta$ on protocols such that $P \Downarrow \eta$. We will denote $\eta$ by $\eval{P}$.
\end{corollary}