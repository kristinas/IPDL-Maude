\documentclass{article}

\usepackage{hyperref}
\usepackage{listings}
\lstset{
    literate={~} {$\sim$}{1}
}

\usepackage{ amssymb }

\title{Soundess of Derived Rules}
\author{Kristina Sojakova \and Mihai Codescu}
\date{}

\begin{document}
\maketitle\footnote{This project is funded through NGI Zero Core (\url{https://nlnet.nl/core}), a fund established by NLnet (\url{https://nlnet.nl}) with financial support from the European Commission's Next Generation Internet (\url{https://ngi.eu}) program. Learn more at the NLnet project page (\url{https://nlnet.nl/project/IPDL-II}).}

\section{Core Rules}

\subsection{Expressions}

The typing predicate is
\begin{lstlisting}
 op typeOf 
  : Signature TypeContext IPDLExpression  
  -> IPDLType .
\end{lstlisting}

\noindent and the configuration has the form

\begin{lstlisting}
 op expConfig 
  : Signature TypeContext IPDLType IPDLExpression 
  -> ExpConfig [ctor] .
\end{lstlisting}

\noindent which means that the expression equality judgement
$$\Gamma \vDash e_1 = e_2 : T$$
translates to
\begin{lstlisting}
 expConfig(Sigma, Gamma, T, e1)
 =>
 expConfig(Sigma, Gamma, T, e2) 
\end{lstlisting}

Equality rules are below. 
 REFL, TRANS hold by the properties of rewriting in Maude.
 AXIOM is not needed as a rule, because we already write an axiom as
\begin{lstlisting}
 expConfig(Sigma, Gamma, T, e1)
 =>
 expConfig(Sigma, Gamma, T, e2) 
\end{lstlisting} 

\begin{itemize}
\item[SYM]: \begin{lstlisting}
 crl [exp-sym] :
 expConfig(Sigma, Gamma, T, e2)
 =>
 expConfig(Sigma, Gamma, T, e1)
 if 
 expConfig(Sigma, Gamma, T, e1)
 =>
 expConfig(Sigma, Gamma, T, e2)
 /\ typeOf(Sigma, Gamma, e1) == T .  
\end{lstlisting}

\item[SUBST] :
\begin{lstlisting}
 crl [exp-subst] :
 expConfig(Sigma, Gamma1, T, e1')
 => 
 expConfig(Sigma, Gamma1, T, applySubst(e2, theta))
 if 
 expConfig(Sigma, Gamma2, T, e1)
 => 
 expConfig(Sigma, Gamma2, T, e2)
 /\ 
 typeOf(Sigma, Gamma2, e1) == T
 /\
 e1' == applySubst(e1, theta^)
\end{lstlisting}

\item[APP-CONG]: 
\begin{lstlisting} 
 crl[app-cong] :
 expConfig(Sigma (f : T1 -> T2), Gamma, T2, ap f e1)
 => 
 expConfig(Sigma (f : T1 -> T2), Gamma, T2, ap f e2)
 if 
 expConfig(Sigma (f : T1 -> T2), Gamma, T1, e1)
 => 
 expConfig(Sigma (f : T1 -> T2), Gamma, T1, e2) .
\end{lstlisting}
\item[PAIR-CONG]:
 \begin{lstlisting}
 crl [pair-cong] :
 expConfig(Sigma, Gamma, T1 * T2, pair(M1, M2))
 => 
 expConfig(Sigma, Gamma, T1 * T2, pair(M3, M4))
 if 
 expConfig(Sigma, Gamma, T1, M1)
 => 
 expConfig(Sigma, Gamma, T1, M3)
 /\ 
 expConfig(Sigma, Gamma, T2, M2)
 => 
 expConfig(Sigma, Gamma, T2, M4)
 .
 \end{lstlisting}
\item[FST-CONG]:
 \begin{lstlisting}
 crl [fst-cong] :
 expConfig(Sigma, Gamma, T1, fst(T1, T2, M1))
 => 
 expConfig(Sigma, Gamma, T1, fst(T1, T2, M2)) 
 if
 expConfig(Sigma, Gamma, T1 * T2, M1)
 => 
 expConfig(Sigma, Gamma, T1 * T2, M2)
 .
\end{lstlisting} 
\item[SND-CONG]:
 \begin{lstlisting} 
 crl [snd-cong] :
 expConfig(Sigma, Gamma, T2, snd(T1, T2, M1))
 => 
 expConfig(Sigma, Gamma, T2, snd(T1, T2, M2)) 
 if
 expConfig(Sigma, Gamma, T1 * T2, M1)
 => 
 expConfig(Sigma, Gamma, T1 * T2, M2)
 .
 \end{lstlisting}
\item[UNIT-EXT]:
\begin{lstlisting}
 crl [unit-ext] : 
 expConfig(Sigma, Gamma, unit, M1)
 =>
 expConfig(Sigma, Gamma, unit, ())
 if typeOf(Sigma, Gamma, M1) == unit .
\end{lstlisting} 
\item[FST-PAIR]:
\begin{lstlisting}
 crl [fst-pair] :
 expConfig(Sigma, Gamma, T1 * T2, fst pair(M1, M2))
 =>
 expConfig(Sigma, Gamma, T1, M1) 
 if typeOf(Sigma, Gamma, M1) == T1 
 /\ typeOf(Sigma, Gamma, M2) == T2
  .
\end{lstlisting}
\item[SND-PAIR]:
\begin{lstlisting}
 crl [snd-pair] :
 expConfig(Sigma, Gamma, T1 * T2, snd pair(M1, M2))
 =>
 expConfig(Sigma, Gamma, T2, M2) 
  if typeOf(Sigma, Gamma, M1) == T1 
 /\ typeOf(Sigma, Gamma, M2) == T2
  .
\end{lstlisting}
\item[PAIR-EXT]:
\begin{lstlisting}
 crl [pair-ext] :
 expConfig(Sigma, Gamma, T1 * T2, pair(fst M, snd M))
 =>
 expConfig(Sigma, Gamma, T1 * T2, M) 
 if typeOf(Sigma, Gamma, M) == T1 * T2 
  .
\end{lstlisting}
\end{itemize}

Comments:
\begin{itemize}
\item if we do not annotate the projections with their types, we would have to write
\begin{lstlisting}
 crl [fst-cong] :
 expConfig(Sigma, Gamma, T1, fst(M1))
 => 
 expConfig(Sigma, Gamma, T1, fst(M2)) 
 if
 expConfig(Sigma, Gamma, T1 * T2, M1)
 => 
 expConfig(Sigma, Gamma, T1 * T2, M2) 
 [nonexec]
 .
\end{lstlisting}
\noindent and specify a type for T2 when applying the rule, which is
inconvenient.
\end{itemize}

\subsection{Reactions}

The typing predicate is
\begin{lstlisting}
  op typeOf 
  : Signature ChannelContext TypeContext 
    Set{CNameBound} Set{BoolTerm} Reaction 
  -> IPDLType 
\end{lstlisting}
\noindent where the first set argument is the set of inputs 
and the second set argument is the set of assumptions on indices,
for families of protocols, and hypotheses of the type $N + 1$ is honest.

The configuration has the form
\begin{lstlisting}
op rConfig 
 : Signature ChannelContext TypeContext
   Reaction Set{CNameBound} Set{BoolTerm} 
   IPDLType 
 -> ReactionConfig [ctor] .
\end{lstlisting}

\noindent which means that the reaction equality judgement
$$\Delta ; \Gamma \vDash R_1 = R_2 : I \rightarrow T$$
translates to
\begin{lstlisting}
 rConfig(Sigma, Delta, Gamma, R1, I, A, T)
 =>
 rConfig(Sigma, Delta, Gamma, R2, I, A, T) 
\end{lstlisting}
\noindent where $A$ is the set of assumptions (new argument).

Equality rules are below. Again,
 REFL, TRANS, SUBST hold by the properties of rewriting in Maude and
 AXIOM is not needed as a rule. 
 To avoid name clashes, these rules have their names in 
 lowercaps.
 
\begin{itemize} 
\item[SYM]:
\begin{lstlisting}

 crl [sym] : 
 rConfig(Sigma, Delta, Gamma, R2, I, A, T) 
 => 
 rConfig(Sigma, Delta, Gamma, R1, I, A, T)
 if rConfig(Sigma, Delta, Gamma, R1, I, A, T) 
    =>
   rConfig(Sigma, Delta, Gamma, R2, I, A, T)
 [nonexec] .
 \end{lstlisting} 
 
\item[INPUT-UNUSED] :
\begin{lstlisting}
 
 crl [input-unused] :
 rConfig(Sigma, Delta, Gamma, R1, (I, chn c), A, T)
 =>
 rConfig(Sigma, Delta, Gamma, R2, (I, chn c), A, T)
 if
 rConfig(Sigma, Delta, Gamma, R1, I, A, T) 
 =>
 rConfig(Sigma, Delta, Gamma, R1, I, A, T) .
\end{lstlisting} 
 
\item[EMBED]:
\begin{lstlisting}
 
 crl [embed] :   
  rConfig(Sigma, Delta1, Gamma, R1', I', A, T)
  =>
  rConfig(Sigma, Delta1, Gamma, 
          embedReaction(R2, phi), 
          I', A, T)
  if
  rConfig(Sigma, Delta2, Gamma, R1, I, A, T)
  =>
  rConfig(Sigma, Delta2, Gamma, R2, I, A, T)
  /\
  I' == embedIO(I, phi) 
  /\
  R1' == embedReaction(R1, phi)
  [nonexec]
  .
  \end{lstlisting}
  \noindent where 
  \texttt{phi : Delta1 -> Delta2} is an embedding,  
  \texttt{embedIO(I, phi)} stands for $\phi^*(I)$ and
  \texttt{embedReaction(R, phi)} stands for $\phi^*(R)$.
\item[CONG-RET]:
\begin{lstlisting} 

 crl [cong-ret] : 
 rConfig(Sigma, Delta, Gamma, return M1, I, A, T) 
 => 
 rConfig(Sigma, Delta, Gamma, return M2, I, A, T)
 if
 expConfig(Sigma, Gamma, T, M1) 
 => 
 expConfig(Sigma, Gamma, T, M2) 
 .       
 
  \end{lstlisting} 
\item[CONG-SAMP]:
\begin{lstlisting}
 
  crl [cong-samp] :
  rConfig(Sigma (d : T1 ->> T2), Delta, Gamma,  
          samp (d < M1 >), I, A, T) 
  => 
  rConfig(Sigma (d : T1 ->> T2), Delta, Gamma, 
          samp (d < M2 >), I, A, T) 
  if 
  expConfig(Sigma (d : T1 ->> T2), Gamma, T1, M1) 
  => 
  expConfig(Sigma (d : T1 ->> T2), Gamma, T1, M2) 
  . 
  
   \end{lstlisting} 
\item[CONG-IF]:
\begin{lstlisting} 
 
  crl [cong-if] : 
  rConfig(Sigma, Delta, Gamma,   
          if M1 then R1 else R2, I, A, T)
  => 
  rConfig(Sigma, Delta, Gamma, 
          if M2 then R3 else R4, I, A, T)
  if
  rConfig(Sigma, Delta, Gamma, R1, I, A, T) 
  => 
  rConfig(Sigma, Delta, Gamma, R3, I, A, T)
  /\
  rConfig(Sigma, Delta, Gamma, R2, I, A, T) 
  => 
  rConfig(Sigma, Delta, Gamma, R4, I, A, T) 
  /\
  expConfig(Sigma, Gamma, bool, M1)
  => 
  expConfig(Sigma, Gamma, bool, M2) .   
  
  \end{lstlisting} 
\item[CONG-BIND]:
\begin{lstlisting}             
          
 crl [cong-bind] :
     rConfig(Sigma, Delta, Gamma, 
             x : T1 <- R1 ; R2, 
             I , A, T2) 
     => 
     rConfig(Sigma, Delta, Gamma, 
             x : T1 <- R3 ; R4, 
             I, A, T2) 
     if
     rConfig(Sigma, Delta, Gamma, R1, I, A, T1)  
     => 
     rConfig(Sigma, Delta, Gamma, R3, I, A, T1) 
     /\
     rConfig(Sigma, Delta, Gamma (x : T1), 
             R2, I, A, T2)
     => 
     rConfig(Sigma, Delta, Gamma (x : T1), 
             R4, I, A, T2) .
     .
 \end{lstlisting} 
\item[SAMP-PURE]:
\begin{lstlisting}
 crl [samp-pure] : 
     rConfig(Sigma, Delta, Gamma, 
             x : T1 <- samp D ; R, 
             I, A, T2)
     => 
     rConfig(Sigma, Delta, Gamma,  
             R, 
             I, A, T2)
 if typeOf(Sigma, Gamma, D) == T1 
 /\ typeOf(Sigma, Delta, Gamma, I, A, R) == T2 
 . 
  \end{lstlisting} 
\item[READ-DET]:
\begin{lstlisting}

 crl [read-det] : 
     rConfig(Sigma, Delta, Gamma, 
                    x : T1 <- read i ; 
                    y : T1 <- read i ; 
                    R , I, A, T2) 
     => 
     rConfig(Sigma, Delta, Gamma, 
                    x : T1 <- read i ; 
                    (R [y / x]), I, A, T2)
 if  isElemB(i, I, A)  
 /\ elem (chn i) T1 Delta A
 /\  typeOf(Sigma, Delta, Gamma (x : T1) (y : T1), 
            I, A, R) == T2 
 .
 
  \end{lstlisting}
  
 \noindent where 
 \texttt{isElemB(i, I, A)} checks that \texttt{i} is in \texttt{I}
  and 
 \texttt{elem (chn i) T1 Delta A} checks that 
 \texttt{(i : T1)} is in \texttt{Delta}.
 Both methods take into account that \texttt{i} may appear in 
 \texttt{I} and \texttt{Delta} as a part of a family.  
\item[IF-LEFT]:
\begin{lstlisting}
                             
 crl [if-left] : 
     rConfig(Sigma, Delta, Gamma, 
             if True then R1 else R2, I, A, T) 
     =>  
     rConfig(Sigma, Delta, Gamma, R1, I, A, T)
 if
     typeOf(Sigma, Delta, Gamma, I, A, R1) == T
     /\
     typeOf(Sigma, Delta, Gamma, I, A, R2) == T 
 .
  \end{lstlisting} 
\item[IF-RIGHT]:
\begin{lstlisting}

 crl [if-right] : 
     rConfig(Sigma, Delta, Gamma, 
             if False then R1 else R2, I, A, T) 
     =>  
     rConfig(Sigma, Delta, Gamma, R2, I, A, T)
 if
     typeOf(Sigma, Delta, Gamma, I, A, R1) == T 
     /\
     typeOf(Sigma, Delta, Gamma, I, A, R2) ==  T 
 . 

 \end{lstlisting} 
\item[IF-EXT]:
We would write the rule as
\begin{lstlisting}
  crl [if-ext] : 
     rConfig(Sigma, Delta, Gamma, R [b / M], I, A, T)
     => 
     rConfig(Sigma, Delta, Gamma, 
       if M then (R [b / True]) else (R [b / False]), 
       I, A, T)
     if
     typeOf(Sigma, Gamma, M) == bool
     /\
     typeOf(Sigma, Delta, Gamma (b : bool), 
            I, A, R) == T . 
\end{lstlisting}            
\noindent but Maude cannot handle these kind of rules.
What we can write is a version where \texttt{M} is a variable:
\begin{lstlisting}
   crl [if-intro-ext] :
     rConfig(Sigma, Delta, Gamma (q : bool), 
             R, I, A, T) 
     => 
     rConfig(Sigma, Delta, Gamma (q : bool),
             if q then (R[q / True]) 
                  else (R[q / False]), I, A, T) 
   if typeOf(Sigma, Delta, Gamma (q : bool), 
             I, A, R) == T                   
  .  
 \end{lstlisting} 
\noindent When we apply the rule we might need to mention which \texttt{q} 
should be used, as there could be more than one in  \texttt{Gamma}. 
We would not be able to write the rule in the reverse direction, but we
can use this rule under a \texttt{sym}.
\item[RET-BIND]:
\begin{lstlisting}           

 crl [ret-bind] : 
     rConfig(Sigma, Delta, Gamma, 
             x : T1 <- return M ; R , I , A, T2) 
     => 
     rConfig(Sigma, Delta, Gamma, 
             R [x / M], I, A, T2) 
     if
     typeOf(Sigma, Gamma, M) == T1
     /\
     typeOf(Sigma, Delta, Gamma (x : T1),  
            I, A, R) == T2 .

 \end{lstlisting} 
\item[BIND-RET]:
\begin{lstlisting}            

 crl [bind-ret] : 
     rConfig(Sigma, Delta, Gamma, 
             x : T <- R ; return x, I , A, T) 
     => 
     rConfig(Sigma, Delta, Gamma, R, I, A, T)
     if 
     typeOf(Sigma, Delta, Gamma, I, A, R) == T . 

 \end{lstlisting} 
\item[BIND-BIND]:
\begin{lstlisting}     

 crl [bind-bind] : 
     rConfig(Sigma, Delta, Gamma, 
            x2 : T2 <- (x1 : T1 <- R1 ; 
                        R2) ; 
                       R3, I, A, T3)
     => 
     rConfig(Sigma, Delta, Gamma, 
            x1 : T1 <- R1 ; 
                       (x2 : T2 <- R2 ; 
                        R3), I, A, T3)
 if
     typeOf(Sigma, Delta, Gamma, I, A, R1) == T1
     /\
     typeOf(Sigma, Delta, Gamma (x1 : T1), 
            I, A, R2) == T2
     /\ 
     typeOf(Sigma, Delta, Gamma (x2 : T2), 
            I, A, R3) == T3 
 .              

 \end{lstlisting} 
\item[EXCH]:
\begin{lstlisting}

 crl [exchange] : 
     rConfig(Sigma, Delta, Gamma, 
             x1 : T1 <- R1 ; 
             x2 : T2 <- R2 ; 
             R, I, A, T3)
     => 
     rConfig(Sigma, Delta, Gamma, 
             x2 : T2 <- R2 ; 
             x1 : T1 <- R1 ; 
             R, I, A, T3)
 if
     typeOf(Sigma, Delta, Gamma, I, A, R1) == T1
     /\
     typeOf(Sigma, Delta, Gamma, I, A, R2) == T2
     /\
     typeOf(Sigma, Delta, Gamma (x1 : T1) (x2 : T2), 
            I, A, R) == T3 
 .

\end{lstlisting} 
\end{itemize}

\subsection{Protocols}

The typing predicate is
\begin{lstlisting}
  op typeOf 
   : Signature ChannelContext 
     Set{CNameBound} Set{BoolTerm} Protocol 
   -> Bool 
\end{lstlisting}
\noindent where the first set argument is the set of inputs 
and the second set argument is the set of assumptions on indices.

The configuration has the form
\begin{lstlisting}
 op pConfig 
  : Signature ChannelContext Protocol 
    Set{CNameBound} Set{CNameBound} Set{BoolTerm} -> 
    ProtocolConfig [ctor] .
\end{lstlisting}

\noindent which means that the protocol equality judgement
$$\Delta \vDash P_1 = P_2 : I \rightarrow O$$
translates to
\begin{lstlisting}
 pConfig(Sigma, Delta, P1, I, O, A)
 =>
 pConfig(Sigma, Delta, P2, I, O, A) 
\end{lstlisting}
\noindent where $A$ is the set of assumptions (new argument).

Equality rules are below. Again,
 REFL, TRANS, SUBST hold by the properties of rewriting in Maude and
 AXIOM is not needed as a rule. Moreover COMP-ASSOC and COMP-COMM are not needed,
 as we have defined the parallel composition as a commutative and associative 
 operator. This also means that having both a -LEFT and a -RIGHT version for 
 ABSORB, FOLD-IF and CONG-COMP is not needed, and we should keep just one.
 
\begin{itemize}
\item[SYM]:
 \begin{lstlisting}
 crl [SYM] :
     pConfig(Sigma, Delta2, P2, I, O2, A) 
     => 
     pConfig(Sigma, Delta1, P1, I, O1, A)
     if
     pConfig(Sigma, Delta1, P1, I, O1, A)
     => 
     pConfig(Sigma, Delta2, P2, I, O2, A) 
     /\ Delta1 equiv Delta2 
     /\ O1 equiv O2     
     [nonexec] .
     
  \end{lstlisting}
  \noindent where the \texttt{equiv} relations hold if the arguments
  are equal modulo splitting. Splitting of families of protocols means
  that e.g. the family
  \texttt{F[< X < Y < N + 2 ]}
  is equivalent with
   \texttt{F[< X < Y < N + 1 ]}
   and 
    \texttt{F[< X < Y = N + 1 ]}. 
  
\item[INPUT-UNUSED] :
 \begin{lstlisting}
  crl [INPUT-UNUSED] :
   pConfig(Sigma, Delta, P1, (I, chn c), O, A)
   =>
   pConfig(Sigma, Delta, P2, (I, chn c), O, A)
   if
   pConfig(Sigma, Delta, P1, I, O, A)
   =>
   pConfig(Sigma, Delta, P2, I, O, A) .
 \end{lstlisting}  
  
\item[CONG-REACT]:
\begin{lstlisting}    
     
 crl [CONG-REACT] : 
     pConfig(Sigma, Delta, cn ::= R, I, bn, A) 
     =>
     pConfig(Sigma, Delta, cn ::= R', I, bn, A)
     if
     rConfig(Sigma, Delta, emptyTypeContext, R, 
             (I, chn cn), A, 
             typeInCtx(chn cn, A, Delta))
     =>
     rConfig(Sigma, Delta, emptyTypeContext, R', 
             (I, chn cn), A, T) 
     /\ T == typeInCtx(chn cn, A, Delta)
     /\ not (isElemB(chn cn, I, A))
     .
  \end{lstlisting} 
  \noindent where \texttt{typeInCtx(chn cn, A, Delta)} gives us the 
  the type of \texttt{cn} in \texttt{Delta}, possibly by looking at the 
  family that \texttt{cn} is a member of, and
  we also test that \texttt{chn cn} is not an input channel or member of
  an input family.
\item[CONG-COMP-LEFT]:
\begin{lstlisting}    

 crl [CONG-COMP-LEFT] :
     pConfig(Sigma, Delta1, P1 || Q, I, O, A) 
     => 
     pConfig(Sigma, Delta2, P2 || Q, I,  
             union(getOutputs(P2), getOutputs(Q)), A)
     if
     pConfig(Sigma, Delta1, P1, 
             union(I, getOutputs(Q)), 
             getOutputs(P1), A)
     => 
     pConfig(Sigma, Delta2, P2, I1, O2, A)
     /\ O2 == getOutputs(P2) 
     /\ I1 == union(I, getOutputs(Q))
     /\ typeOf(Sigma, Delta2, 
               union(I, getOutputs(P2)), 
               A, Q)
     /\ Delta1 equiv Delta2
     /\ O equiv 
        (union(getOutputs(P2), getOutputs(Q)))
 .
 \end{lstlisting} 
 \noindent where we must allow splitting and 
 \texttt{getOutputs(P)} gives us the outputs of the protocol \texttt{P}. 
 Note that Maude won't let us write \texttt{getOutputs(P2)} after the 
 \texttt{=>} sign. If we were to do that, it would look for an exact 
 syntactic match and it would fail.
\item[CONG-COMP-RIGHT]:
\begin{lstlisting}
  crl [CONG-COMP-RIGHT] :
     pConfig(Sigma, Delta1, Q || P1, 
             I, O, A) 
     => 
     pConfig(Sigma, Delta2, Q || P2, 
             I, 
             union(getOutputs(P2), getOutputs(Q)), 
             A)
     if
     pConfig(Sigma, Delta1, P1, 
             union(I, getOutputs(Q)), 
             getOutputs(P1), A)
     => 
     pConfig(Sigma, Delta2, P2, I1, O2, A)
     /\ typeOf(Sigma, Delta1, 
              union(I, getOutputs(P1)), A, Q) 
     /\ I1 == union(I, getOutputs(Q))    
     /\ O2 == getOutputs(P1)      
     /\ O == union(getOutputs(P1), getOutputs(Q)).
   
    \end{lstlisting} 
\item[CONG-NEW]:
\begin{lstlisting}
   
 crl [CONG-NEW] : 
     pConfig(Sigma, Delta1, 
             new cn : T in P1, I, O1, A)
     => 
     pConfig(Sigma, 
             removeEntry ((chn cn) :: T) Delta2, 
             new cn : T in P2, I, getOutputs(P2), A)
     if
     pConfig(Sigma, ((chn cn) :: T) Delta1,  
             P1, I, insert(chn cn, O1), A)
     => 
     pConfig(Sigma, Delta2,  P2, I, O2, A) 
     /\ O2 == insert(chn cn, getOutputs(P2))
     /\ Delta2 equiv (((chn cn) :: T) Delta1)
     /\ insert(chn cn, O1) equiv O2
     .
    \end{lstlisting} 
   \noindent where \texttt{removeEntry} deletes a channel from a 
   channel context. 
\item[NEW-EXCH]:
\begin{lstlisting}   
    
  
  crl [NEW-EXCH] :
     pConfig(Sigma, Delta, 
              new cn1 : T1 in 
                new cn2 : T2 in P, I, O, A) 
     => 
     pConfig(Sigma, Delta, 
               new cn2 : T2 in 
                new cn1 : T1 in P, I, O, A)
     if
     typeOf(Sigma, Delta (chn cn1 :: T1) 
                         (chn cn2 :: T2), 
            I, A, P) /\
     getOutputs(P) == insert(chn cn1, 
                             insert(chn cn2, O)) .
                             
     \end{lstlisting} 
\item[COMP-NEW]:
\begin{lstlisting}                             

 crl [COMP-NEW] :
     pConfig(Sigma, Delta, 
             P || (new cn : T in Q), I, O, A)
     =>  
     pConfig(Sigma, Delta, 
             new cn : T in (P || Q), I, O, A)
     if
     typeOf(Sigma, Delta (chn cn :: T), 
            union(I, getOutputs(P)), A, Q)
     /\
     typeOf(Sigma, Delta, 
            union(I ,(getOutputs(Q) \ (chn cn))), 
            A, P) 
 .
 
     \end{lstlisting} 
\item[ABSORB-LEFT]:
\begin{lstlisting} 

  crl [ABSORB-LEFT] : 
     pConfig(Sigma, Delta, P1 || P2, I, O, A) =>
     pConfig(Sigma, Delta, P1, I, O, A) 
     if
     typeOf(Sigma, Delta, I, A, P1) 
     /\
     typeOf(Sigma, Delta, union(I, O), A, P2) 
     /\
     getOutputs(P1) == O
     /\ 
     getOutputs(P2) == empty
     .
      \end{lstlisting} 
\item[ABSORB-RIGHT]:
\begin{lstlisting}            
 
 crl [ABSORB-RIGHT] : 
     pConfig(Sigma, Delta, P1 || P2, I, O, A) =>
     pConfig(Sigma, Delta, P2, I, O, A) 
     if
     typeOf(Sigma, Delta, I, A, P2) 
     /\
     typeOf(Sigma, Delta, union(I, O), A, P1) 
     /\
     getOutputs(P2) == O
     /\ 
     getOutputs(P1) == empty 
 .
 
     \end{lstlisting} 
     
%\noindent In the rule for families I have replaced the type checking of \texttt{P1} with a syntactic condition that \texttt{P1} doesn't read 
%from P2, so that we don't run type checking for large protocols unnecessarily often. I think I should do the same here.     
\item[DIVERGE]:
\begin{lstlisting}  
 
 crl [DIVERGE] :
     pConfig(Sigma, Delta, 
             cn ::= x : T <- read cn ; R, 
             I, chn cn, A)
     =>
     pConfig(Sigma, Delta, 
             cn ::= read cn, I, chn cn, A)
     if
     typeOf(Sigma, Delta, emptyTypeContext, 
            insert(chn cn, I), A, R)
     == 
     typeInCtx(chn cn, A, Delta)
     /\ occurs (chn cn) Delta A
      .

     \end{lstlisting} 
\item[FOLD-IF-RIGHT]:
\begin{lstlisting}  

 crl [FOLD-IF-RIGHT] :
     pConfig(Sigma, Delta,
             new cn1 : T in 
              ((cn2 ::= b : bool <- R ;
                        if b then S1 
                             else read cn1)
                ||
                (cn1 ::= S2)
               )
            ,I, O, A)
     => 
     pConfig(Sigma, Delta,
             cn2 ::= b : bool <- R ;
                   if b then S1 else S2
             , I, O, A)
     if
     typeOf(Sigma, Delta, emptyTypeContext, 
            I, A, R) 
     == 
     bool
     /\
     typeOf(Sigma, Delta, emptyTypeContext, 
            insert(chn cn2, I), A, S1) 
     == 
     T
     /\
     typeOf(Sigma, Delta, emptyTypeContext, 
            insert(chn cn2, I), A, S2) == T
     /\
     O == chn cn2 
     /\
     elem (chn cn2) T Delta A .

     \end{lstlisting} 
\item[FOLD-IF-LEFT]:
\begin{lstlisting}  
    
  crl [FOLD-IF-LEFT] :
     pConfig(Sigma, Delta,
             new cn2 : T in 
               ((cn1 ::= b : bool <- R ;
                         if b then read cn2 
                              else S2)
                 ||
                 (cn2 ::= S1))
            ,I, O, A)
     => 
     pConfig(Sigma, Delta,
             cn1 ::= b : bool <- R ;
                   if b then S1 else S2
             , I, O, A)
     if
     typeOf(Sigma, Delta, emptyTypeContext,   
            I, A, R) 
     == 
     bool
     /\
     typeOf(Sigma, Delta, emptyTypeContext,
            insert(chn cn1, I), A, S1) == T
     /\
     typeOf(Sigma, Delta, emptyTypeContext, 
            insert(chn cn1, I), A, S2) == T
     /\
     O == chn cn1 
     /\
     elem (chn cn1) T Delta A .
     
 
     \end{lstlisting} 
\item[FOLD-BIND]:
\begin{lstlisting}      

  crl [FOLD-BIND] :
     pConfig(Sigma, Delta, 
             new c : T in 
               ((o ::= x : T <- read c ; S)
                ||
                (c ::= R)),
             I, O, A)
     =>
     pConfig(Sigma, Delta,
             o ::= x : T <- R ; S,
             I, O, A)
    if 
       typeOf(Sigma, Delta, x : T,
              (I, chn c), 
              A, S) 
       == 
       typeInCtx(chn o, A, Delta)             
    /\ typeOf(Sigma, Delta, emptyTypeContext,
              I, A, R)
       == 
       T          
    /\ O == chn o .
   
  
     \end{lstlisting} 
\item[SUBSUME]:
\begin{lstlisting}    
    
 crl [SUBSUME] :
    pConfig(Sigma, Delta, 
            (cn1 ::= x0 : T0 <- read i ; R1) || 
            (cn2 ::= x0 : T0 <- read i ;
                     x1 : T1 <- read cn1 ; 
                     R2) 
           , I, O, A)
    => 
    pConfig(Sigma, Delta,
            (cn1 ::= x0 : T0 <- read i ; R1) || 
            (cn2 ::= x1 : T1 <- read cn1 ; R2)  
           , I, O, A)
   if typeOf(Sigma, Delta, x1 : T1, 
             insert(chn cn1, insert(chn cn2, I)), 
             A, R2) == 
      typeInCtx(chn cn2, A, Delta) 
   /\ O == insert(chn cn1, insert(chn cn2, empty)) 
   /\ elem (chn cn1) T1 Delta A . 
   
   
     \end{lstlisting} 
  \noindent This rule is actually derivable.   
\item[DROP]:
\begin{lstlisting}  
   
 crl [DROP] : 
   pConfig(Sigma, Delta,
           (cn1 ::= R1) || 
           (cn2 ::= x1 : T1 <- read cn1 ; R2)
          ,I, O, A)
   => 
   pConfig(Sigma, Delta,
           (cn1 ::= R1) || (cn2 ::= R2)
          ,I, O, A)
   if rConfig(Sigma, Delta, emptyTypeContext, 
              x1 : T1 <- R1 ; R2
             , insert(chn cn1, insert(chn cn2, I)), 
             A, typeInCtx(chn cn2, A, Delta)) 
      => 
      rConfig(Sigma, Delta, emptyTypeContext, 
              R2
             ,I', A, T2) /\
      T2 == typeInCtx(chn cn2, A, Delta) /\    
      I' == insert(chn cn1, insert(chn cn2, I)) /\
      O == insert(chn cn1, insert(chn cn2, empty)) /\
      typeOf(Sigma, Delta, emptyTypeContext, 
             insert(chn cn1, insert(chn cn2, I)), 
             A, R2) 
      == 
      typeInCtx(chn cn2, A, Delta) /\ 
      elem (chn cn1) T1 Delta A
      [nonexec] .
      
      
     \end{lstlisting} 
\item[SUBST]:
\begin{lstlisting}  
  
   crl [SUBST] : 
     pConfig(Sigma, Delta,
             (cn1 ::= R1) 
             || 
             (cn2 ::= x1 : T1 <- read cn1 ; 
                      R2),
             I, O, A)  
     => 
     pConfig(Sigma, Delta,
             (cn1 ::= R1) 
             || 
             (cn2 ::= x1 : T1 <- R1 ; R2),
             I, O, A) 
     if
     rConfig(Sigma, Delta, emptyTypeContext,
             x1 : T1 <- R1 ; 
             x2 : T1 <- R1 ; 
             return pair(x1, x2),
             insert(chn cn1, insert(chn cn2, I)), A, 
             T1 * T1 )
     => 
     rConfig(Sigma, Delta, emptyTypeContext,
             x1 : T1 <- R1 ;  return pair(x1, x1),
             I', A, T1 * T1 ) /\      
     O == insert(chn cn1, chn cn2) /\ 
     I' == insert(chn cn1, insert(chn cn2, I)) /\
     elem (chn cn1) T1 Delta A
     [nonexec] .
   \end{lstlisting}
\end{itemize}

\section{Derived Rules}

\subsection{Plain Protocols}

Here we only have derived rules at the reaction level.
The rule names should be changed.

\begin{itemize}

\item[SAME-REACTION-IF]
\begin{lstlisting}
   crl [same-reaction-if] : 
     rConfig(Sigma, Delta, Gamma, 
             if M then R else R, 
             I, A, T)
     => 
     rConfig(Sigma, Delta, Gamma, 
             R, I, A, T) 
  if typeOf(Sigma, Delta, Gamma, 
            I, A, R) == T  
  /\ typeOf(Sigma, Gamma, M) == bool 
  . 
   \end{lstlisting} 
   
Proof:
Assume x : bool is a variable that doesn't occur in R. Then
\begin{lstlisting}
 
 R 
 = (by def. of _[_/_]) 
 R [x / M] 
 => (by if-ext)
 if M then R[x/True] else R[x/False]
 = (by def. of _[_/_] )
 if M then R else R 
 
\end{lstlisting}   
   
\item[CONG-BRANCH-REFL]:
\begin{lstlisting}  
  crl [cong-branch-refl] : 
     rConfig(Sigma, Delta, Gamma, 
             if M then R1 else R2, I, A, T)
     => 
     rConfig(Sigma, Delta, Gamma, 
             if M then R3 else R4, I, A, T)
     if
     typeOf(Sigma, Gamma, M) == bool 
     /\
     rConfig(Sigma, Delta, Gamma, R1, I, A, T) => 
     rConfig(Sigma, Delta, Gamma, R3, I, A, T)
     /\
     rConfig(Sigma, Delta, Gamma, R2, I, A, T) => 
     rConfig(Sigma, Delta, Gamma, R4, I, A, T) . 
  
    \end{lstlisting}
    This holds immediately by \texttt{CONG-IF} and taking the 
    rewrite for \texttt{M} as the one that leaves it as it is.
    I added this rule when I did not have expression equality
    and I think we could still leave it for convenience.

\item[IF-OVER-BIND]:
\begin{lstlisting}  
 
 crl [if-over-bind] : 
    rConfig(Sigma, Delta, Gamma, 
             x : T1 <- if M then R1 else R2 ; 
             R , I, A, T) 
    =>
    rConfig(Sigma, Delta, Gamma, 
             if M then (x : T1 <- R1 ; R) 
                  else (x : T1 <- R2 ; R) , 
            I, A, T) 
 if typeOf(Sigma, Delta, Gamma, 
           I, A, R1) == T1  /\
    typeOf(Sigma, Delta, Gamma, 
           I, A, R2) == T1  /\
    typeOf(Sigma, Delta, Gamma (x : T1), 
           I, A, R) == T /\
    typeOf(Sigma, Gamma, M) == bool
   .
   
\end{lstlisting} 

Proof:
\begin{lstlisting}
 x : T1 <- if M then R1 else R2 ; 
 R 
 => (by if-ext)
 if M 
  then x : T1 <- if True then R1 
                         else R2 ; 
       R
  else x : T1 <- if False then R1 
                         else R2 ; 
       R     
 => (by cong-branch-refl{
         cong-bind{if-left, idle}, 
         cong-bind{if-right, idle}
        })
 if M then x : T1 <- R1 ; R
      else x : T1 <- R2 ; R
\end{lstlisting}   
  
\item[BIND-OVER-IF]:
\begin{lstlisting}    
      
 crl [bind-over-if] : 
     rConfig(Sigma, Delta, Gamma, 
       if M then (x : T1 <- R1 ; R) 
            else (x : T1 <- R1 ; S), 
       I, A, T)
     =>        
     rConfig(Sigma, Delta, Gamma, 
        x : T1 <- R1 ; 
        if M then R else S, I, A, T)
 if
     typeOf(Sigma, Delta, Gamma, 
           I, A, R1) == T1 /\
     typeOf(Sigma, Delta, Gamma (x : T1), 
            I, A, R) == T /\
     typeOf(Sigma, Delta, Gamma (x : T1), 
            I, A, S) == T /\
     typeOf(Sigma, Gamma, M) == bool 
 .
 
\end{lstlisting}  

Proof:
\begin{lstlisting}
 x : T1 <- R1 ;
 if M then R else S
 => (by if-ext)
 if M 
  then x : T1 <- R1 ; 
       if True then R else S
  else x : T1 <- R1 ; 
       if False then R else S
 => (by cong-branch-refl{
          cong-bind{idle, if-left}, 
          cong-bind{idle, if-right}
        })
 if M then x : T1 <- R1 ; R
      else x : T1 <- R1 ; S
\end{lstlisting} 


\item[IF-OVER-BIND-SAME]:
\begin{lstlisting}  
           
 crl [if-over-bind-same] :
    rConfig(Sigma, Delta, Gamma, 
     x : T1 <- if M then R1 else R2 ; 
     if M then R3 else R4, 
     I, A, T)
    =>
    rConfig(Sigma, Delta, Gamma, 
    if M then (x : T1 <- R1 ; R3) 
         else (x : T1 <- R2 ; R4) , 
    I, A, T)                   
 if typeOf(Sigma, Delta, Gamma, 
           I, A, R1) 
    == T1  /\
    typeOf(Sigma, Delta, Gamma, 
           I, A, R2) 
    == T1  /\
    typeOf(Sigma, Delta, Gamma (x : T1), 
           I, A, R3) 
    == T /\
    typeOf(Sigma, Delta, Gamma (x : T1), 
           I, A, R4) 
    == T /\
    typeOf(Sigma, Gamma, M) == bool 
 .

\end{lstlisting}  

Proof:

 \begin{lstlisting}
  x : T1 <- if M then R1 else R2 ; 
  if M then R3 else R4
  => (by if-ext)
  if M then 
    x : T1 <- if True then R1 else R2 ; 
    if True then R3 else R4
  else
    x : T1 <- if False then R1 else R2 ; 
     if False then R3 else R4
 => (by cong-branch-refl{
          cong-bind{if-left, if-left}, 
          cong-bind{if-right, if-right}
        })
  if M then 
     x : T1 <- R1 ; R3
       else
     x : T1 <- R2 ; R4       
 \end{lstlisting}

\item[IF-OVER-BIND-SAME-2]:
\begin{lstlisting}                    
 crl [if-over-bind-same-2] :
     rConfig(Sigma, Delta, Gamma, 
      x : T1 <- 
       if M1 
        then if M2 then R1 else R2
        else if M2 then R3 else R4 ;
      if M1 
         then if M2 then S1 else S2
         else if M2 then S3 else S4,
      I, A, T)
    => 
    rConfig(Sigma, Delta, Gamma, 
      if M1 
        then if M2 then (x : T1 <- R1 ; S1) 
                   else (x : T1 <- R2 ; S2)
        else if M2 then (x : T1 <- R3 ; S3) 
                   else (x : T1 <- R4 ; S4),          
      I, A, T) 
   if typeOf(Sigma, Gamma, M1) == bool
   /\ typeOf(Sigma, Gamma, M2) == bool 
   /\ typeOf(Sigma, Delta, Gamma, 
             I, A, R1) == T1
   /\ typeOf(Sigma, Delta, Gamma, 
             I, A, R2) == T1
   /\ typeOf(Sigma, Delta, Gamma, 
             I, A, R3) == T1     
   /\ typeOf(Sigma, Delta, Gamma, 
             I, A, R4) == T1  
   /\ typeOf(Sigma, Delta, Gamma (x : T1), 
             I, A, S1) == T                                
   /\ typeOf(Sigma, Delta, Gamma (x : T1), 
             I, A, S2) == T
   /\ typeOf(Sigma, Delta, Gamma (x : T1), 
             I, A, S3) == T
   /\ typeOf(Sigma, Delta, Gamma (x : T1), 
             I, A, S4) == T 
  .                                                  

\end{lstlisting}    
  
Proof:

\begin{lstlisting}
 x : T1 <- 
   if M1 
    then if M2 then R1 else R2
    else if M2 then R3 else R4 ;
 if M1 
    then if M2 then S1 else S2
    else if M2 then S3 else S4
 => (by if-ext for M2)
 if M2 
 then  
  x : T1 <- 
   if M1 
    then if True then R1 else R2
    else if True then R3 else R4 ;
 if M1 
    then if True then S1 else S2
    else if True then S3 else S4
 else
 x : T1 <- 
   if M1 
    then if False then R1 else R2
    else if False then R3 else R4 ;
 if M1 
    then if False then S1 else S2
    else if False then S3 else S4
 => (by if-left, 
        if-right 
     under the right congruence rules)
 if M2 then
  x : T1 <- if M1 then R1 else R3 ;
  if M1 then S1 else S3
 else
  x : T1 <- if M1 then R2 else R4 ;
  if M1 then S2 else S4
 => (by if-ext for M1)
 if M1 
 then    
 if M2 then
  x : T1 <- if True then R1 else R3 ;
  if True then S1 else S3
 else
  x : T1 <- if True then R2 else R4 ;
  if True then S2 else S4
 else
 if M2 then
  x : T1 <- if False then R1 else R3 ;
  if False then S1 else S3
 else
  x : T1 <- if False then R2 else R4 ;
  if False then S2 else S4 
 => (by if-left, 
        if-right 
     under the right congruence rules)
 if M1 
        then if M2 then (x : T1 <- R1 ; S1) 
                   else (x : T1 <- R2 ; S2)
        else if M2 then (x : T1 <- R3 ; S3) 
                   else (x : T1 <- R4 ; S4)               
\end{lstlisting}  
  
\item[ALPHA]:
\begin{lstlisting}  
 
 var vx vy : Qid . 
        
 crl [alpha] :
     rConfig(Sigma, Delta, Gamma, 
             vx : T1 <- R1 ; R2 , 
             I, A, T2 )
     =>
     rConfig(Sigma, Delta, Gamma, 
              vy : T1 <- R1 ; 
              (R2 [vx / vy]), 
              I, A, T2)
 if typeOf(Sigma, Delta, Gamma, 
           I, A, R1) == T1 /\
    typeOf(Sigma, Delta, Gamma (vx : T1), 
           I, A, R2) == T2 [nonexec] .

\end{lstlisting} 

Follows by \texttt{sym\{bind-ret\} then bind-bind then ret-bind}:
\begin{lstlisting}
load ../src/strategies
mod ALPHA-SOUND is
 including APPROX-EQUALITY . 
 
 *** constants without definitions
 *** will be interpreted as any value of that type 
 
 op Sigma : -> Signature .
 op Delta : -> ChannelContext .
 op Gamma : -> TypeContext .
 ops vx vy : -> Qid .
 ops R1 R2 : -> Reaction .
 op I : -> Set{CNameBound} .
 op A : -> Set{BoolTerm} .
 ops T1 T2 : -> Type .
 
 ***  I need this because I want
 ***  R2 to typecheck if the type context
 ***  has more than Gamma and (vx : T1)
 var Gamma' : TypeContext .
 
 *** assumptions 
 eq typeOf(Sigma, Delta, Gamma, I, A, R1) = T1 .
 eq typeOf(Sigma, Delta, 
           Gamma (vx : T1) Gamma', 
           I, A, R2) = T2 .
 
endm

srew [1] 
 rConfig(Sigma, Delta, Gamma, 
         vx : T1 <- R1 ; R2,
         I, A, T2)
 using sym[R1:Reaction <- 
            vx : T1 <- (vy : T1 <- R1 ; 
                        return vy) ; 
                        R2
          ]
       {cong-bind{bind-ret, idle}}  
     ; bind-bind  
     ; cong-bind{idle, ret-bind}         
 . 
 
 *** we get
 *** result ReactionConfig: 
 *** rConfig(Sigma, Delta, Gamma, 
 ***         vy : T1 <- R1 ; (R2[vx / vy]), 
 ***         I, A, T2)
 
\end{lstlisting}   
  
%\item[SAMP-FREE]:
%\begin{lstlisting} 
%               
% crl [samp-free] : 
%     rConfig(Sigma, Delta, emptyTypeContext,
%             x1 : T1 <- R1 ; 
%             x2 : T1 <- R1 ; 
%             return pair(x1, x2),
%             I, A, T1 * T1)
%     => 
%     rConfig(Sigma, Delta, emptyTypeContext,
%             x1 : T1 <- R1 ;  
%             return pair(x1, x1),
%             I, A, T1 * T1)
% if isSampFree(R1) 
% /\ typeOf(Sigma, Delta, emptyTypeContext,
%          I, A, R1) == T1
%     .
%     
%     *** check recursively whether samp occurs
% op isSampFree : Reaction -> Bool .
% eq isSampFree(return M) = true .
% eq isSampFree(samp D) = false .
% eq isSampFree(read c) = true .
% eq isSampFree(if M then R1 else R2) = 
%    isSampFree(R1) and isSampFree(R2) .
% eq isSampFree(i : T <- R1 ; R2) = 
%    isSampFree(R1) and isSampFree(R2) .
% eq isSampFree(nf(BL, R)) = isSampFree(R) .
% eq isSampFree(preNF(BL, R)) = isSampFree(R) .  
% 
% 
%\end{lstlisting}    
  
\item[SAMP-OVER-IF]:
\begin{lstlisting}    
     
 crl [samp-over-if] :
    rConfig(Sigma, Delta, Gamma, 
            x : T1 <- samp D ; 
            if M then R1 else R2, 
            I, A, T) 
    => 
    rConfig(Sigma, Delta, Gamma, 
            if M then x : T1 <- samp D ; 
                      R1
                 else x : T1 <- samp D ; 
                      R2, 
            I, A, T)
 if typeOf(Sigma, Delta, Gamma (x : T1), 
           I, A, R1) == T
 /\ typeOf(Sigma, Delta, Gamma (x : T1), 
           I, A, R2) == T          
 /\ typeOf(Sigma, Gamma, D) == T1                       
    . 
\end{lstlisting}

Proof:

\begin{lstlisting}
 x : T1 <- samp D ; 
 if M then R1 else R2
 => (by if-ext)
 if M then 
   x : T1 <- samp D ; 
   if True then R1 else R2
 else
  x : T1 <- samp D ; 
  if False then R1 else R2
 => (by cong-branch-refl{
         cong-bind{idle, if-left}, 
         cong-bind{idle, if-right}
        })
 if M then x : T1 <- samp D ; 
                      R1
      else x : T1 <- samp D ; 
                      R2  
\end{lstlisting}

\end{itemize}

\section{Normal Forms}

\subsection{Reactions}

We introduce normal forms of reactions to avoid the use of the rule EXCH and
CONG-BIND. 
The main idea is that instead of writing 
\begin{lstlisting}

x1 : T1 <- read C1 ; 
...
xN : TN <- read CN ;
R

\end{lstlisting}

\noindent we turn the binds into a commutative list
 \begin{lstlisting}
nf(
(x1 : T1 <- read C1)
...
(xN : TN <- read CN) ,
R
)
\end{lstlisting}
\noindent and thus we can select any of them to apply reaction equality rules.
The reaction \verb+R+ is bind free. We can relax this restriction and also the one that all binds read from channels, and then we obtain a pre-normal form instead. 

The normal form of a reaction can be computed with a function \verb+computeNF+, and we can also
assume a selection among the reactions that are equivalent modulo their normal
form that allows us to pick a certain order for the list of binds.
This amounts to using a rule
\begin{lstlisting}
crl[select-plain]:
rConfig(Sigma, Delta, Gamma,
        nf(BRL, R), I, A, T)
=> 
rConfig(Sigma, Delta, Gamma,
        R', I, A, T)        
if computeNF(R') == nf(BRL, R)
[nonexec]        
\end{lstlisting}
\noindent in one direction and 
\begin{lstlisting}
rl[compute-nf]:
rConfig(Sigma, Delta, Gamma,
        R, I, A, T)
=> 
rConfig(Sigma, Delta, Gamma,
        computeNF(R), I, A, T)        
\end{lstlisting}
\noindent in the other. These rules are sound by definition.

\begin{itemize}
 
\item[alpha-nf]:
\begin{lstlisting}
 crl [alpha-nf] : 
    rConfig(Sigma, Delta, Gamma, 
            nf((vx : T1 <- R1) BRL,
               R2
              ),
            I, A, T2    
           )
     => 
     rConfig(Sigma, Delta, Gamma, 
            nf((vy : T1 <- R1) BRL,
               R2 [vx / vy] 
              ),
            I, A, T2    
           )   
 if typeOf(Sigma, Delta, Gamma, 
           I, A, R1) == T1 /\
    typeOf(Sigma, Delta, 
           addDeclarations BRL (Gamma (vx : T1)), 
           I, A, R2) == T2          
      [nonexec]        
 .
 \end{lstlisting}
 
 We start with \verb+nf((vx : T1 <- R1) BRL, R2)+.
we can turn this into a plain reaction \verb+R'+
 by selecting the order of binds where
\verb+vx+ comes last. We can then define a Maude strategy 
\begin{lstlisting}
strat S @ ReactionConfig .
sd S := 
 alpha[vy:Qid <- vy] 
 or-else 
 cong-bind{idle, S}
.
 \end{lstlisting}
By applying it recursively, we leave all binds in \verb+BRL+ unchanged. When we reach
\verb+vx : T1 <- R1 ; R2+
we notice that the conditions of the ALPHA rule hold and we can do the renaming
\verb+vy : T1 <- R1 ; R2[vx / vy]+.
The result of applying \verb+S+ to \verb+R'+ 
is then a reaction \verb+R''+ that starts with the binds in \verb+BRL+ 
and ends with \verb+vy : T1 <- R1 ; R2[vx / vy]+.
The normal form of \verb+R''+ is precisely 
\verb+nf((vy : T1 <- R1) BRL, R2[vx / vy])+.
 
\item[cong-nf]:
\begin{lstlisting} 
 
 crl [cong-nf] :
     rConfig(Sigma, Delta, Gamma, 
              nf(BRL , R1), I, A, T)
     => 
     rConfig(Sigma, Delta, Gamma, 
              nf(BRL , R2), I, A, T)
     if 
     rConfig(Sigma, Delta, 
             addDeclarations BRL Gamma, 
             R1, I, A, T) 
     => 
     rConfig(Sigma, Delta, Gamma', 
              R2, I, A, T) 
     /\  Gamma' == addDeclarations BRL Gamma .    
  \end{lstlisting} 
  
We start with \verb+nf(BRL, R1)+ and by assumption we know that
\begin{lstlisting} 
rConfig(Sigma, Delta, 
             addDeclarations BRL Gamma, 
             R1, I, A, T) 
     => 
     rConfig(Sigma, Delta, Gamma', 
              R2, I, A, T) 
  \end{lstlisting}              
by a rewrite that we call \verb+rew+.              
We can turn \verb+nf(BRL, R1)+ into a plain reaction \verb+R'+ 
by selecting any order of binds. 
We then define a Maude strategy 
\begin{lstlisting} 
strat S @ ReactionConfig .
sd S := 
 cong-bind{idle, rew}
 or-else
 cong-bind{idle, S}
.
  \end{lstlisting}
By applying it recursively, we leave all binds in \verb+BRL+ 
unchanged and when we reach
\verb+R1+ we can rewrite it to \verb+R2+ using \verb+rew+,
as \verb+cong-bind+ adds all declarations in \verb+BRL+ to
\verb+Gamma+ by repeated application. 
The result of applying \verb+S+ to \verb+R'+ is a reaction 
\verb+R''+ that starts with the binds in \verb+BRL+
and ends with \verb+R2+.
The normal form of \verb+R''+ is precisely 
\verb+nf(BRL, R2)+.  
 
\item[read-det-pre]:
\begin{lstlisting}     
     
 crl [read-det-pre] :
     rConfig(Sigma, Delta, Gamma, 
               preNF( (x : T1 <- read i)  
                      (y : T1 <- read i) BL , 
                      R ), 
            I, A, T2) 
     =>
     rConfig(Sigma, Delta, Gamma, 
               preNF( (x : T1 <- read i) BL , 
                      R [y / x] ), 
             I, A, T2) 
 if isElemB(i, I, A)  /\ 
    elem (toBound i) T1 Delta A /\
    typeOf(Sigma, Delta, 
           addDeclarations BL 
             (Gamma (x : T1) (y : T1)), 
           I, A, R) 
    == T2 .
   \end{lstlisting} 
   
We start with the reaction
\begin{lstlisting} 
 preNF( (x : T1 <- read i)(y : T1 <- read i) BL , R )
\end{lstlisting} 
and select its plain reaction equivalent \verb+R'+ 
that starts with the binds in BL
and ends with
 \begin{lstlisting}
   x : T1 <- read i ; 
   y : T1 <- read i ; 
   R
 \end{lstlisting} 

We then define a Maude strategy 
\begin{lstlisting} 
strat S @ ReactionConfig .
sd S := 
 cong-bind{idle, read-det-pre}
 or-else
 cong-bind{idle, S}
.
  \end{lstlisting}
By applying it recursively, we leave all binds in \verb+BL+ 
unchanged until we reach
the left-hand side of the rule \verb+read-det+.
This rule has the 
same conditions as \verb+read-det-pre+, and we know these hold by assumption,
since the declarations in \verb+BL+ are added to \verb+Gamma+ by repeatedly applying \verb+cong-bind+.
We can apply \verb+read-det+ to get
 \begin{lstlisting}
   x : T1 <- read i ;  
   (R[y / x])
 \end{lstlisting}
The result of applying the strategy to  \verb+R'+ is then 
a reaction \verb+R''+ that starts with the binds in BL
and ends with
 \begin{lstlisting}
   x : T1 <- read i ;  
   (R[y / x])
 \end{lstlisting}
 Its pre-normal form is precisely 
 \begin{lstlisting}
 preNF( (x : T1 <- read i) BL , 
                      R [y / x] )
 \end{lstlisting}
 and we obtain it by calling \verb+computeNF(R'')+ and applying
 \verb+nf2Pre+ to the result if needed.
 
\item[read-det-nf]:
\begin{lstlisting}
  crl [read-det-nf] :
     rConfig(Sigma, Delta, Gamma, 
               nf( (x : T1 <- read i) 
                   (y : T1 <- read i) BRL , 
                   R), 
             I, A, T2) 
     =>
     rConfig(Sigma, Delta, Gamma, 
               nf( (x : T1 <- read i) BRL , 
                    R [y / x]), 
             I, A, T2) 
 if isElemB(i, I, A)  /\ 
    elem (toBound i) T1 Delta A /\
    typeOf(Sigma, Delta, 
           addDeclarations BRL 
            (Gamma (x : T1) (y : T1)), 
           I, A, R) 
    == T2 . 
   \end{lstlisting} 
   
 Same proof as above, only use \verb+read-det+ in the strategy and
 turn the final plain reaction to a normal form instead of a 
 pre-normal form.  
 
\item[bind-ret-pre]:
\begin{lstlisting}
                
 crl [bind-ret-pre] : 
     rConfig(Sigma, Delta, Gamma, 
               preNF( (x : T1 <~ R1) BL , 
                      return x ), 
             I, A, T1) 
     =>
     rConfig(Sigma, Delta, Gamma, 
               preNF( BL , R1 ), 
             I, A, T1) 
 if typeOf(Sigma, Delta, Gamma, 
           I, A, R1) 
    == T1 .  
  
     \end{lstlisting} 
     
We start with 
\begin{lstlisting}
preNF( (x : T1 <~ R1) BL , 
       return x )           
\end{lstlisting}        
We can turn it into a plain reaction \verb+R'+ by selecting the order of binds
that starts with \verb+BL+ and ends with \verb+x : T1 <- R1+ . 
We then define a Maude strategy 
\begin{lstlisting}
strat S @ ReactionConfig .
sd S := 
 cong-bind{idle, bind-ret}
 or-else
 cong-bind{idle, S}
.
     \end{lstlisting}
By applying it recursively, we leave all binds in \verb+BL+ unchanged and 
when we reach
\verb+x : T <- R1 ; return x+ we rewrite it to \verb+R1+ 
using \verb+bind-ret+.
The result of applying \verb+S+ to \verb+R'+ is a reaction \verb+R''+ 
that starts with the binds in \verb+BL+ 
and ends with \verb+R1+.
The normal form of \verb+R''+ is precisely 
\verb+preNF(BL, R1)+.     
 
\item[read2Binds]:
\begin{lstlisting}       
 crl [read2Binds] : 
    rConfig(Sigma, Delta, Gamma, 
             preNF(BL (x : T1 <~ read i), 
                   R ), 
           I, A, T) 
    =>
    rConfig(Sigma, Delta, Gamma, 
             preNF(BL (x : T1 <- read i), 
                   R ), 
            I, A, T) 
    if isElemB(i, I, A) and 
       elem (chn i) T1 Delta A .
    \end{lstlisting} 
    
 Both reactions have the same plain forms.   
 
\item[pre2Nf]:
\begin{lstlisting}  
 crl [pre2Nf] : preNF(BRL, R ) => nf(BRL, R) 
 if R : BindFreeReaction . 
    \end{lstlisting} 
    
 Both reactions have the same plain forms.  
 The condition that \verb+R+ should be bind free and
 the requirement that \verb+BRL+ is a list of read binds
 ensures that the normal form is well-formed.
 
\item[nf2Pre]:
\begin{lstlisting}
 rl [nf2Pre] : nf(BRL, R) => preNF(BRL, R) .
     \end{lstlisting} 
 
 Both reactions have the same plain forms. 
 
\item[merge-pre]:
\begin{lstlisting}
 crl [merge-pre] :
     rConfig(Sigma, Delta, Gamma, 
             preNF(BL (x : T1 <~ R1) , 
                   R2 ), 
             I, A, T2)
     =>
     rConfig(Sigma, Delta, Gamma, 
              preNF(BL , 
                    x : T1 <- R1 ; R2 ), 
             I, A, T2) 
   if typeOf(Sigma, Delta, Gamma, 
             I, A, R1) 
      == T1
   /\ typeOf(Sigma, Delta, 
             addDeclarations BL (Gamma (x : T1)), 
             I, A, R2) 
      == T2 .
       \end{lstlisting} 
       
 Both reactions have the same plain forms.      
  
\item[bind2R-pre-reverse]:
\begin{lstlisting}  .    
 crl [bind2R-pre-reverse] :
     rConfig(Sigma, Delta, Gamma, 
             preNF(BL , 
                   x : T1 <- read i ; R2 ), 
             I, A, T2)
     =>
     rConfig(Sigma, Delta, Gamma, 
             preNF(BL (x : T1 <- read i) , 
                   R2 ),
             I, A, T2) 
 if isElemB(i, I, A)  
 /\ elem (chn i) T1 Delta A 
 /\ typeOf(Sigma, Delta, 
           addDeclarations BL (Gamma (x : T1)), 
           I, A, R2) 
    == T2 .  
          \end{lstlisting} 
          
  Follows by the previous rule and symmetry.        
 
\item[ret-bind-pre]:
\begin{lstlisting}  .  
 crl [ret-bind-pre] :
     rConfig(Sigma, Delta, Gamma, 
              preNF((x : T1 <~ (return M)) BL, 
                    R ), 
             I, A, T2) 
     =>
     rConfig(Sigma, Delta, Gamma, 
             preNF(BL, 
                   R [x / M] ), 
             I, A, T2) 
 if typeOf(Sigma, Gamma, M) == T1 
 /\ typeOf(Sigma, Delta, 
           addDeclarations BL (Gamma (x : T1)), 
           I, A, R) 
    == T2 
          \end{lstlisting} 
 

 Start with 
  \begin{lstlisting}
 preNF((x : T1 <~ (return M)) BL, R )
 \end{lstlisting} 
 \noindent and select its 
 plain form that starts with the binds in BL and ends with \verb+x : T1 <- return M ; R+.
We then define a Maude strategy 
\begin{lstlisting}
strat S @ ReactionConfig .
sd S := 
 cong-bind{idle, ret-bind}
 or-else
 cong-bind{idle, S}
.
     \end{lstlisting}
By applying it recursively, we leave all binds in \verb+BL+ unchanged and 
when we reach
\verb+x : T <- return M ; R+ we rewrite it to \verb+R[x / M]+ 
using \verb+ret-bind+. 
The normal form of the resulting reaction is \verb+preNF(BL, R [x / M] )+, possibly applying \verb+nf2Pre+ if BL has only
read binds and \verb+R [x / M]+ is bind free.
 
\item[bind-bind-pre]:
\begin{lstlisting}  . 
 crl [bind-bind-pre] :
     rConfig(Sigma, Delta, Gamma, 
              preNF((x2 : T2 <~ nf(BRL, R2)) BL, 
                     R1), 
            I, A, T1)
     =>  
     rConfig(Sigma, Delta, Gamma, 
              preNF(BRL (x2 : T2 <~ R2) BL, 
                    R1), 
             I, A, T1) 
 if typeOf(Sigma, Delta, 
            addDeclarations BRL Gamma, 
           I, A, R2) 
    == T2
 /\ typeOf(Sigma, Delta, 
           addDeclarations BL (Gamma (x2 : T2)), 
           I, A, R1) 
    == T1 .  
          \end{lstlisting} 
          
We start with \begin{lstlisting}
 preNF( (x2 : T2 <~ nf(BRL, R2)) BL, R1)
 \end{lstlisting} \noindent and 
select the plain representation \verb+P+ that starts with the binds in BL and ends with
\verb+ x2 : T2 <- R' ; R1+ where \verb+R'+ is any plain representation of \verb+nf(BRL, R2)+.
We define two Maude strategies. The first one will extract the binds in \verb+BRL+ from the inner reaction and lift them to the outer level:
\begin{lstlisting}
 start S1 @ ReactionConfig .
 sd S1 :=
  bind-bind 
  or-else 
  cong-bind{idle, S1}
.  
\end{lstlisting}
\noindent while the other will leave unchanged the outer binds:
\begin{lstlisting}
strat S2 @ ReactionConfig .
sd S2 := 
 S1
 or-else
 cong-bind{idle, S2}
.
     \end{lstlisting}
When applying \verb+S2+ to \verb+P+ we obtain a reaction \verb+P'+
that has first the binds in \verb+BL+, then the ones in \verb+BRL+, and finally
\verb+x2 : T2 <- R2 ; R1+. The pre-normal form of \verb+P'+ is
\begin{lstlisting}
 preNF(BRL (x2 : T2 <~ R2) BL, R1)
\end{lstlisting}
 
\item[bind-bind-pre-pre]:
\begin{lstlisting}  .
  crl [bind-bind-pre-pre] :
     rConfig(Sigma, Delta, Gamma, 
              preNF((x2 : T2 <~ preNF(BRL, R2)) BL, 
                    R1), 
             I, A, T1)
     =>  
     rConfig(Sigma, Delta, Gamma, 
              preNF(BRL (x2 : T2 <~ R2) BL, 
                    R1), 
             I, A, T1) 
 if typeOf(Sigma, Delta, 
            addDeclarations BRL Gamma,
            I, A, R2) 
    == T2
 /\ typeOf(Sigma, Delta, 
            addDeclarations BL (Gamma (x2 : T2)), 
            I, A, R1) 
    == T1 . 
\end{lstlisting}
\end{itemize}

The proof is similar to the one above, namely the same strategies are used, and the only thing that changes is that we start with an inner pre-normal form.

\subsection{Protocols}

We introduce normal forms of protocols to avoid the use of the rule NEW-EXCH. 
The main idea is that instead of writing 
\begin{lstlisting}

new cn1 : T1 in 
new cn2 : T2 in
...
new cnN : TN in
P

\end{lstlisting}

\noindent we turn the hidden channels into a commutative list
 \begin{lstlisting}
newNF(
< C1 : T1 >
...
< CN : TN > ,
P
)
\end{lstlisting}
\noindent and thus we can select any of them to apply protocol equality rules.

The normal form of a protocol can be computed 
with a function \verb+new2NF+, and we can also
assume a selection among the protocols that are equivalent modulo their normal
form that allows us to pick a certain order for the list of hidden channels.
This amounts to using a rule
\begin{lstlisting}
 crl [select-plain] :
     pConfig(Sigma, Delta, 
             newNF(ltq, P1), 
             I, O, A)
     =>
     pConfig(Sigma, Delta, 
              P, 
             I, O, A) 
 if new2NF(P) ==  newNF(ltq, P1)           
 .
\end{lstlisting}
\noindent in one direction, that chooses the plain form of a protocol in normal form given by the alphabetical order of names of hidden channels, and 
\begin{lstlisting}  
 rl [sugar-newNF] :
    pConfig(Sigma, Delta,  
             P, 
             I, O, A)
    => 
    pConfig(Sigma, Delta, 
             new2NF(P), 
             I, O, A) 
 . 
 \end{lstlisting}
\noindent in the other. These rules are sound by definition.

\begin{itemize}

\item[delete-empty-newNF]
 \begin{lstlisting}
 rl [delete-empty-newNF] :
     pConfig(Sigma, Delta, 
             newNF(emptyTypedCNameList, P), 
             I, O, A)
     => 
     pConfig(Sigma, Delta, 
              P, 
             I, O, A) 
 . 
  \end{lstlisting}
  
  The empty list of hidden channels doesn't add anything to the normal form
  of \verb+P+, so both \verb+P+ and \verb+newNF(emptyTypedCNameList, P)+ have
  the same plain representations.

\item[CONG-NEW-NF]
 \begin{lstlisting}
 
 crl [CONG-NEW-NF] : 
    pConfig(Sigma, Delta1, 
             newNF(ltq, P1), 
            I, O1, A) 
    => 
    pConfig(Sigma, 
             diff 
              Delta2 
              (addChannels ltq emptyChannelCtx), 
            newNF(ltq, P2), 
            I, O2 \ (chansInList ltq), A)
    if
    pConfig(Sigma, addChannels ltq Delta1, 
             P1, 
             I, union(chansInList ltq, O1), A)
    =>
    pConfig(Sigma, Delta2, 
             P2, 
            I, O2, A) 
    /\ *** the channels in ltq have not changed
    diff 
     (addChannels ltq emptyChannelCtx) 
     Delta2 
    == emptyChannelCtx
    /\
    O2 == getOutputs(P2)
    /\ 
    (addChannels ltq Delta1) equiv Delta2 
    /\ 
    O2 equiv (chansInList ltq, O1)
  .
 \end{lstlisting}
 
 Let \verb+rew+ be the rewrite in the condition of the rule. 
 We start with the protocol \verb+newNF(ltq, P1)+ and select any 
 plain representation of it \verb+Q1+.
 We define the following Maude strategy:
 \begin{lstlisting}
  strat S @ ProtocolConfig .
  sd S :=
     rew
     or-else CONG-NEW{S}
  .
 \end{lstlisting}
 The strategy adds arbitrarily many hidden channels to the current context
 and then applies \verb+rew+. The result of applying \verb+S+ to \verb+Q1+
 is a protocol \verb+Q2+ that has the same hidden channels as \verb+Q1+ followed
 by \verb+P2+. By taking its normal form we obtain precisely 
 \verb+newNF(ltq, P2)+.

\item[absorb-new-nf]
 \begin{lstlisting}
 crl [absorb-new-nf] :
     pConfig(Sigma, Delta, 
              newNF(< c : T > ltq, 
                    P || (c ::= R)
              ), 
             I, O, A) 
     => 
     pConfig(Sigma, Delta, 
              newNF(ltq, P), 
             I, O, A) 
 if typeOf(Sigma, 
            addChannels ltq 
              (Delta (chn c :: T)), 
            emptyTypeContext, 
            (chn c, (I, getOutputs(P))), 
            A, R) 
    == T 
 /\ typeOf(Sigma, addChannels ltq Delta, 
          I, A, P)       
 /\ getOutputs(newNF2New(newNF(ltq, P))) 
    == O .
    \end{lstlisting}

We start with 
\verb+newNF(< c : T > ltq, P || (c ::= R))+  
and let $Q1$ be its plain representation that starts with
the hidden channels in \verb+ltq+ then with the hidden channel \verb+c+ and
the protocol \verb+P || (c ::= R)+.
We define the following Maude strategy:
\begin{lstlisting}
strat S @ ProtocolConfig .
sd S := 
 (COMP-NEW-2 ; ABSORB-LEFT)
 or-else
 CONG-NEW{S}
.
\end{lstlisting}

 The strategy adds arbitrarily many hidden channels to the current context
 and then applies \verb+COMP-NEW-2+ to turn
 \verb+new c : T in (P || c ::= R)+ into
 \verb+P || (new c : T in c ::= R)+. The assumptions of 
 \verb+absorb-new-nf+ ensure that\verb+P+ type checks in the absence of
 \verb+c+ from the channel context and that 
 \verb+new c : T in c ::= R+ type checks with the outputs of \verb+P+
 as inputs, so we can apply \verb+ABSORB-LEFT+ to eliminate
 \verb+new c : T in c ::= R+. The result of applying 
 \verb+S+ to \verb+Q1+ is a protocol \verb+Q2+ that starts with 
 the hidden channels in \verb+ltq+ and ends with \verb+P+.
 The normal form of \verb+Q1+ is precisely \verb+newNF(ltq, P)+.
 
\item[fold-bind-new-nf]
 \begin{lstlisting}
           
  crl [fold-bind-new-nf] :
    pConfig(Sigma, Delta, 
             newNF(< c : T > ltq, 
                   P || 
                   ( c ::= R ) || 
                   ( o ::= nf((x : T <- read c) BRL, 
                              S) 
                   )
                  ),
            I, O, A)
    => 
    pConfig(Sigma, Delta, 
             newNF(ltq, 
                   P || 
                   (o ::= preNF((x : T <~ R) BRL, 
                                S))
                   ),
            I, O, A) 
    if typeOf(Sigma, 
              addChannels ltq 
                 ( Delta ( (chn c):: T) ) , 
                emptyTypeContext, 
              (chn o, 
                (chn c, (I, getOutputs(P)) ) 
              ), 
              A, R) 
       == T 
    /\ typeOf(Sigma, addChannels ltq Delta, 
              addDeclarations 
                ((x : T <- read c) BRL) 
                emptyTypeContext,
              (chn o, 
                (I, getOutputs(P)) ), A, S
       ) 
       == 
       typeInCtx(chn o, A, addChannels ltq Delta)   
    /\ typeOf(Sigma, addChannels ltq Delta, 
          (I, chn o), A, P)     
  .
    \end{lstlisting}
    
We start with 
 \begin{lstlisting}  
newNF(< c : T > ltq, 
      P || ( c ::= R ) || 
     ( o ::= nf((x : T <- read c) BRL, S) )
)    
 \end{lstlisting}
\noindent and we select the plain representation \verb+Q1+ that starts with the
hidden channels in \verb+ltq+ and ends with 
\begin{lstlisting}
new c : T in  
  P || 
  ( c ::= R ) || 
  ( o ::= nf((x : T <- read c) BRL, S) )
\end{lstlisting}

We define the following Maude strategy
\begin{lstlisting}
strat S @ ProtocolConfig . 
sd S := 
 COMP-NEW-2 ; CONG-COMP-RIGHT{FOLD-BIND}
 or-else
 CONG-NEW{S}
.
\end{lstlisting}
The strategy \verb+S+ leaves the hidden channels in \verb+ltq+ unchanged,
then the rule \verb+COMP-NEW-2+ rewrites
\begin{lstlisting}
new c : T in  
    P || 
    ( c ::= R ) || 
    ( o ::= nf((x : T <- read c) BRL, S) )
\end{lstlisting}
to
\begin{lstlisting} 
  P || new c : T in 
         ( c ::= R ) || 
         ( o ::= nf((x : T <- read c) BRL, S) )
\end{lstlisting} 
\noindent Then \verb+CONG-COMP-RIGHT{FOLD-BIND}+ leaves P unchanged (by \verb+CONG-COMP-RIGHT+) and rewrites
\begin{lstlisting}
 new c : T in 
   ( c ::= R ) || 
   ( o ::= nf((x : T <- read c) BRL, S)
\end{lstlisting}
to
\begin{lstlisting} 
 o ::= preNF((x : T <~ R) BRL, S)
\end{lstlisting} (by \verb+FOLD-BIND+).
The result of applying \verb+S+ to \verb+Q1+ is then a protocol \verb+Q2+ that 
starts with the hidden channels in \verb+ltq+ and ends with
\begin{lstlisting} 
 P || (o ::= preNF((x : T <~ R) BRL, S))
 \end{lstlisting}. The normal form of 
\verb+Q2+ is the protocol in the right hand side of the rule \verb+fold-bind-new-nf+.

 
\item[fold-bind-new-nf-0]
 \begin{lstlisting}  
  
  crl [fold-bind-new-nf-0] :
    pConfig(Sigma, Delta, 
             newNF(< c : T > ltq, 
                   (c ::= R) || 
                   (o ::= nf((x : T <- read c) BRL, 
                             S))
                  ),
            I, O, A)
    => 
    pConfig(Sigma, Delta, 
             newNF(ltq, 
                    (o ::= preNF((x : T <~ R) BRL, 
                                 S))
                  ),
            I, O, A) 
    if typeOf(Sigma, 
              addChannels ltq 
                (Delta ((chn c):: T)), 
              emptyTypeContext, 
             (chn o, (chn c, I )), 
             A, R) 
        == T  /\ typeOf(Sigma, addChannels ltq Delta, 
              addDeclarations 
                ((x : T <- read c) BRL) 
                emptyTypeContext,
              (chn o, I), A, S
       ) 
       == 
       typeInCtx(chn o, A, addChannels ltq Delta) 
       .
       
    \end{lstlisting}
    
   The proof is similar to the one above, the only difference is that
   the strategy \verb+S+ doesn't apply \verb+CONG-COMP-RIGHT+ anymore:´
\begin{lstlisting}
strat S @ ProtocolConfig . 
sd S := 
 COMP-NEW-2 ; FOLD-BIND
 or-else
 CONG-NEW{S}
.
\end{lstlisting}  

\item[fold-bind-new-prenf]
 \begin{lstlisting}       
            
  crl [fold-bind-new-prenf] :
    pConfig(Sigma, Delta, 
             newNF(< c : T > ltq,
                   P || 
                   (c ::= R) || 
                   (o ::= preNF((x : T <- read c) 
                                BRL, 
                                 S))
                   ),
            I, O, A)
    => 
    pConfig(Sigma, Delta, 
             newNF(ltq, 
                   P || 
                   (o ::= preNF((x : T <~ R) BRL, 
                                S))
                  ),
            I, O, A) 
 if typeOf(Sigma, 
           addChannels ltq 
             (Delta ((chn c) :: T)), 
             emptyTypeContext, 
             (chn o, (chn c, 
               union(I, getOutputs(P)))), 
             A, R) 
    == T .        
                \end{lstlisting}
                
   The proof is the same as for \verb+fold-bind-new-nf+.             

\item[COMP-NEW-newNF]
 \begin{lstlisting}  
            
 crl [COMP-NEW-newNF] :
     pConfig(Sigma, Delta, 
              P || newNF(ltq, Q), 
            I, O, A)
     =>   
     pConfig(Sigma, Delta, 
              newNF(ltq, P || Q), 
             I, O, A)
 if typeOf(Sigma, 
             addChannels ltq Delta, 
             union(I, getOutputs(P)), 
             A, Q)
 /\ typeOf(Sigma, Delta, 
           union(I, getOutputs(newNF(ltq, Q))), 
            A, P) .
   \end{lstlisting}
   
  Start with \verb+ newNF(ltq, P || Q)+ and consider
  the plain representation \verb+Q1+ that starts with any order of the
  hidden channels in \verb+ltq+ and ends with \verb+P || Q+.
  
  We define the following Maude strategies
  \begin{lstlisting}
  strat S @ ProtocolConfig.
  sd S := 
     COMP-NEW
     or-else 
     CONG-NEW{S}
  \end{lstlisting}
  By applying \verb+S!+ to \verb+Q1+ (using the \verb+!+ strategy operator
   that applies a strategy as many times as possible), we obtain
   the protocol \verb+P || Q2+, where \verb+Q2+ starts with the hidden channels
   in \verb+ltq+ and ends with \verb+Q+. The normal form of \verb+Q2+ is
   \verb+newNF(ltq, Q)+, so we can plug it next to \verb+P+ using
   \verb+CONG-COMP-RIGHT+ to obtain \verb+P || newNF(ltq, Q)+.
   The proof is completed by applying the \verb+SYM+ rule to the proof above. 
    

\item[COMP-NEW-newNF-inside-new]
 \begin{lstlisting} 
 
 crl [COMP-NEW-newNF-inside-new] : 
     pConfig(Sigma, Delta, 
              newNF(ltq1, 
                     P || newNF(ltq, Q)), 
            I, O, A)
     =>   
     pConfig(Sigma, Delta, 
               newNF(ltq1 ltq, 
                      P || Q), 
             I, O, A)
 if typeOf(Sigma, addChannels ltq Delta, 
            (I, getOutputs(P)), 
            A, Q)
 /\ typeOf(Sigma, Delta, 
            ( I, getOutputs(newNF(ltq, QL)) ), 
           A, P) .
   \end{lstlisting}
   
   The proof is similar to the one above, but we restrict the number 
   of applications of \verb+S+ to the length of \verb+ltq+, then using
   \verb+CONG-NEW+ for the hidden channels in \verb+ltq1+.

\item[DROP-nf]
 \begin{lstlisting}
   crl [DROP-nf] :
    pConfig(Sigma, Delta,
            (cn1 ::= nf(emptyBRList, samp Dist)) || 
            (cn2 ::= nf( (x : T1 <- read cn1) BRL , 
                         R2) ),
            I, O, A) 
    =>           
     pConfig(Sigma, Delta,
            (cn1 ::= nf(emptyBRList, samp Dist)) || 
            (cn2 ::= nf( BRL , R2) ),
            I, O, A) 
    if 
     typeOf(Sigma, Delta, 
             addDeclarations BRL (x : T1), 
             (chn cn1, (chn cn2, I)), 
             A, R2) 
     == 
     typeInCtx(chn cn2, A, Delta)
     /\
     elem (chn cn1) T1 Delta A .  
       \end{lstlisting}

 We start with 
 \begin{lstlisting}
  (cn1 ::= nf(emptyBRList, samp Dist)) || 
  (cn2 ::= nf( (x : T1 <- read cn1) BRL , R2) )
 \end{lstlisting} 
 \noindent and we replace the reactions assigned to the 
 channels with \verb+samp Dist+ and the reaction that starts
 with the binds in \verb+x : T1 <- read cn1+ and ends with 
 the binds in \verb+BRL+ followed by \verb+R2+ (we call this reaction
 \verb+R'+), respectively.
 Working in reverse: by the rule \verb+samp-pure+,
 we can rewrite the reaction \verb+ x : T1 <- samp Dist ; R' +  
 to \verb+R'+. Thus, by the rule \verb+DROP+, we can rewrite the 
 protocol 
 \begin{lstlisting}
 cn1 ::= samp Dist || cn2 ::= x : T1 <- read cn1 ; R'
 \end{lstlisting}
 to \verb+cn1 ::= samp Dist || cn2 ::= R'+. The proof ends by replacing
 the reactions assigned to the channels \verb+cn1+ and \verb+cn2+ with 
 their normal forms.
 
 
\item[DROP-pre-nf]
 \begin{lstlisting} 
                   
crl [DROP-pre-nf] :
  pConfig(Sigma, Delta,
          (cn1 ::= samp Dist) || 
          (cn2 ::= preNF( (x : T1 <- read cn1) BRL , 
                           R2) ),
         I, O, A) 
  =>           
 pConfig(Sigma, Delta,
         (cn1 ::= samp Dist) || 
         (cn2 ::= preNF( BRL , R2) ),
         I, O, A) 
 if typeOf(Sigma, Delta, 
           addDeclarations BRL (x : T1), 
           (chn cn1, (chn cn2, I)), A, R2) 
    == 
    typeInCtx(chn cn2, A, Delta)
 /\ elem (chn cn1) T1 Delta A .   
       \end{lstlisting}
       
 The proof is identical to the one above, except
 we turn the reactions to their pre-normal form at the end.      

\item[DROP-SUBSUME-channels]
 \begin{lstlisting} 
        
    crl [DROP-SUBSUME-channels] :
    pConfig(Sigma, Delta,
            (cn1 ::= nf(BRL, samp Dist)) || 
            (cn2 ::= nf( (x : T1 <- read cn1) BRL' , 
                         R2) ),
            I, O, A) 
    =>           
     pConfig(Sigma, Delta,
            (cn1 ::= nf(BRL, samp Dist)) || 
            (cn2 ::= nf(BRL BRL' , R2) ),
            I, O, A) 
    if 
     typeOf(Sigma, Delta, 
             addDeclarations BRL' (x : T1), 
             (chn cn1, (chn cn2, I)), 
             A, R2) 
     == 
     typeInCtx(chn cn2, A, Delta)
     /\
     elem (chn cn1) T1 Delta A .  
       \end{lstlisting}
 The proof is similar to the one of \verb+DROP-nf+ but before
 using the \verb+DROP+ rule we apply the reverse of the 
 \verb+SUBSUME+ rule to duplicate the binds in \verb+BRL+ in 
 the reaction assigned to the channel \verb+cn2+. Each time a bind
 is added, we use the \verb+EXCH+ rule to move the read from \verb+cn1+
 in front. 

\item[DROP-SUBSUME-channels-pre]
 \begin{lstlisting}

 crl [DROP-SUBSUME-channels-pre] :
 pConfig(Sigma, Delta,
         (cn1 ::= nf(BRL, samp Dist)) || 
         (cn2 ::= preNF( (x : T1 <- read cn1) BRL' , 
                          R2) ),
         I, O, A) 
 =>           
 pConfig(Sigma, Delta,
         (cn1 ::= nf(BRL, samp Dist)) || 
         (cn2 ::= preNF(BRL BRL' , R2) ),
         I, O, A) 
 if typeOf(Sigma, Delta,  
           addDeclarations BRL' (x : T1), 
           (chn cn1, (chn cn2, I)), A, R2) 
    == 
    typeInCtx(chn cn2, A, Delta)
 /\ elem (chn cn1) T1 Delta A . 
       \end{lstlisting}

 The proof is identical to the previous one, except at the end we
 compute the pre-normal form.

\item[SUBST-nf]
 \begin{lstlisting}           
  crl [SUBST-nf] : 
     pConfig(Sigma, Delta,
             (cn1 ::= R1) || 
             (cn2 ::= nf( (x1 : T1 <- read cn1) 
                          BRL , 
                          R2) ),
             I, O, A)  
     => 
     pConfig(Sigma, Delta,
             (cn1 ::= R1) || 
             (cn2 ::= preNF( (x1 : T1 <~ R1)
                             BRL , 
                             R2)),
             I, O, A) 
     if isSampFree(R1) /\
        O == insert(chn cn1, chn cn2) /\ 
     typeOf(Sigma, Delta, emptyTypeContext, 
             (chn cn1,  (chn cn2, I)), A, R1) 
     == T1 /\
     typeOf(Sigma, Delta, 
            addDeclarations BRL (x1 : T1), 
             (chn cn1,  (chn cn2, I)), A, R2) 
     == typeInCtx(chn cn2, A, Delta)
     /\
     elem (chn cn1) T1 Delta A . 
       \end{lstlisting}
       
 We start by showing that if a reaction is samp-free, the 
 condition 
 \begin{lstlisting}
 rConfig(Sigma, Delta, emptyTypeContext,
             x1 : T1 <- R1 ; 
             x2 : T1 <- R1 ; 
             return pair(x1, x2),
             insert(chn cn1, insert(chn cn2, I)), A, 
             T1 * T1 )
     => 
     rConfig(Sigma, Delta, emptyTypeContext,
             x1 : T1 <- R1 ;  return pair(x1, x1),
             I', A, T1 * T1 )
 \end{lstlisting}
 \noindent from the assumptions of the rule \verb+SUBST+
 holds.
 To do that we will prove a stronger statement, namely that 
 if a reaction is samp-free, then 
 \begin{lstlisting}
 rConfig(Sigma, Delta, emptyTypeContext,
             x1 : T1 <- R1 ; 
             x2 : T1 <- R1 ; 
             S(x1, x2),
             insert(chn cn1, insert(chn cn2, I)), A, 
             T1 * T1 )
     => 
     rConfig(Sigma, Delta, emptyTypeContext,
             x1 : T1 <- R1 ; S(x, x),
             I', A, T1 * T1 )
 \end{lstlisting}
 \noindent
 holds for any reaction \verb+S(x, y)+.
 
 We proceed by structural induction.

 Case \verb+R1 = return M+:
 we start with
 \begin{lstlisting}
 x1 : T1 <- return M ; 
 x2 : T1 <- return M ; 
 S(x1, x2)
 \end{lstlisting}
 By applying \verb+ret-bind+ two times, we get \verb+S(M, M)+, and this
 is also what we get by applying \verb+ret-bind+ to \verb+x1 : T1 <- return M ;  S(x1, x1)+.
 Case \verb+R = read c+:
 we start with 
  \begin{lstlisting}
 x1 : T1 <- read c ; 
 x2 : T1 <- read c ; 
 S(x1, x2)
 \end{lstlisting}
 \noindent and we can apply \verb+read-det+ to obtain
   \begin{lstlisting}
 x1 : T1 <- read c ; 
 S(x1, x1)
 \end{lstlisting}
 
 Case \verb+ R = if M then R1 else R2+, with \verb+R1, R2+ samp-free:
 we start with 
   \begin{lstlisting}
 x1 : T1 <- if M then R1 else R2 ; 
 x2 : T1 <- if M then R1 else R2 ; 
 S(x1, x2)
 \end{lstlisting} 
 and we notice that we can write it as
   \begin{lstlisting}
  if M then 
      x1 : T1 <- R1 ;
      x2 : T1 <- R1 ; 
      S(x1, x2)  
       else
      x1 : T1 <- R2 ;
      x2 : T1 <- R2 ; 
      S(x1, x2) 
 \end{lstlisting}
 by applying to this latter reaction two times the derived rule 
 \verb+if-over-bind-same+ followed by \verb+same-reaction-if+.
 We can now use the inductive hypothesis for \verb+R1+ and \verb+R2+ to get
   \begin{lstlisting}
  if M then 
      x1 : T1 <- R1 ;
      S(x1, x1)  
       else
      x1 : T1 <- R2 ; 
      S(x1, x1) 
 \end{lstlisting}
 But this is also what we get if we apply \verb+if-over-bind+ to
 the right hand side reaction
    \begin{lstlisting}
  x1 : T1 <- if M then R1 else R2 ;
  S(x1, x1)  
   \end{lstlisting}

 
 Case \verb+R = (a : t <- R1) ; R2(a)+, with \verb+R1, R2(a)+ samp-free:
 

We start with
  \begin{lstlisting}
   x : T <-  a : t <- R1 ; R2(a) ;
   y : T <-  a : t <- R1 ; R2(a) ;
   S(x, y)
  \end{lstlisting}

By bind-bind and exchange we get

  \begin{lstlisting}
   a : t <- R1 ; 
   a : t <- R1 ; 
   x : T <-  R2(a) ;
   y : T <-  R2(a) ;
   S(x, y)
  \end{lstlisting}

By the induction hypothesis for \verb+R2(a)+

  \begin{lstlisting}
   a : t <- R1 ; 
   a : t <- R1 ; 
   x : T <-  R2(a) ;
   S(x, x)
  \end{lstlisting}
  
By the induction hypothesis for \verb+R1+

 \begin{lstlisting}
   a : t <- R1 ; 
   x : T <-  R2(a) ;
   S(x, x)
  \end{lstlisting})

which is what we get from the right hand side as well by 
applying \verb+bind-bind+.
 
 We now proceed to showing that \verb+SUBST-nf+ is sound. 
 We start with 
    \begin{lstlisting}
   (cn1 ::= R1) || 
             (cn2 ::= nf( (x1 : T1 <- read cn1) 
                          BRL , 
                          R2) )
   \end{lstlisting}
   and we can choose the plain form of 
   the reaction assigned to \verb+cn2+ that starts with
   \verb+x1 : T1 <- read cn1+ and ends with the 
   binds in \verb+BRL+ followed by \verb+R2+. Let \verb+Q+ denote this last
   fragment.
   Since \verb+R1+ is samp-free, we know that the assumptions of
   the \verb+SUBST+ rule hold, using the first statement we proved, 
   and we get 
       \begin{lstlisting}
   (cn1 ::= R1) || 
             (cn2 ::= x1 : T1 <- R1 ; Q )
   \end{lstlisting}                   
   \noindent The normal form of
   \verb+x1 : T1 <- R1 ; Q+
   is precisely
   \begin{lstlisting}
    preNF( (x1 : T1 <~ R1) BRL , R2)
   \end{lstlisting} .

\item[SUBST-nf-read]
 \begin{lstlisting} 
  
  crl [SUBST-nf-read] : 
     pConfig(Sigma, Delta,
             (cn1 ::= nf((x2 : T1 <- read cn), 
                         return x2)) || 
             (cn2 ::= nf( (x1 : T1 <- read cn1) 
                          BRL , 
                          R2) ),
             I, O, A)  
     => 
     pConfig(Sigma, Delta,
             (cn1 ::= nf((x2 : T1 <- read cn), 
                         return x2)) || 
             (cn2 ::= nf((x2 : T1 <- read cn) 
                         BRL , 
                         R2 [x1 / x2])),
             I, O, A) 
     if
     isElemB(cn, I, A) /\
     O ==  (chn cn1, chn cn2) /\ 
     typeOf(Sigma, Delta, 
            addDeclarations BRL (x1 : T1), 
              (chn cn1,  (chn cn2, I)), A, R2) 
     == 
     typeInCtx(chn cn2, A, Delta)
             /\
     elem (chn cn1) T1 Delta A 
             /\
     elem (chn cn) T1 Delta A           
     .   

\end{lstlisting}  

 Follows immediately from soundness of \verb+SUBST-nf+, for the particular
 case of \verb+R1 = read cn+ and applying the substitution strategy.

\item[subst-diverge]
 \begin{lstlisting}     
  crl [subst-diverge] : 
    pConfig(Sigma, Delta, 
              (cn1 ::= nf(x1 : T1 <- read cn1, 
                          return x1))
              || 
              (cn2 ::= nf( (x2 : T1 <- read cn1) 
                           BRL , 
                           R2)),
              I, O, A)    
    =>
    pConfig(Sigma, Delta, 
              (cn1 ::= nf(x1 : T1 <- read cn1, 
                          return x1))
              || 
              (cn2 ::= nf(x3 : T2 <- read cn2, 
                          return x3)),
              I, O, A) 
   if        
    O == (chn cn1, chn cn2)
    /\
    elem (chn cn1) T1 Delta A
    /\
    elem (chn cn2) T2 Delta A
    /\
    typeOf(Sigma, Delta, 
           addDeclarations BRL (x2 : T1), 
           (chn cn1,  (chn cn2, I)), A, R2) 
     == 
     typeInCtx(chn cn2, A, Delta)   
           [nonexec] 
  .      
       \end{lstlisting}     
  
   Follows immediately from soundness of \verb+SUBST-nf+ for the particular
 case of \verb+R1 = read cn1+, followed by application of the 
 \verb+DIVERGE+ rule.

\item[moveReadInnerNf]
 \begin{lstlisting}     
  crl [moveReadInnerNf] :
    pConfig(Sigma, Delta,
             cn1 ::= nf((x : T <- read cn2) 
                        BRL , 
                        R1) ,
             I, O, A)  
     => 
       pConfig(Sigma, Delta,
             cn1 ::= preNF(BRL , 
                           x : T <- read cn2 ; 
                           R1) ,
             I, O, A)    
  if elem (chn cn2) T Delta A 
  /\ typeOf(Sigma, Delta, 
            addDeclarations BRL (x : T), 
            (chn cn1, I), A, R1)
  == typeInCtx(chn cn1, A, Delta) .  
    \end{lstlisting}
    
  The two protocols have the same plain forms.  

\item[moveReadInnerPreNf]
 \begin{lstlisting} 
  
    crl [moveReadInnerPreNf] :
    pConfig(Sigma, Delta,
             cn1 ::= preNF((x : T <- read cn2) 
                           BRL , 
                           R1) ,
             I, O, A)  
     => 
       pConfig(Sigma, Delta,
             cn1 ::= preNF(BRL , 
                           x : T <- read cn2 ; 
                           R1) ,
             I, O, A)  
    if elem (chn cn2) T Delta A 
    /\ typeOf(Sigma, Delta, 
               addDeclarations BRL (x : T), 
             (chn cn1, I), A, R1) 
       == typeInCtx(chn cn1, A, Delta) .
       \end{lstlisting}
       
    The two protocols have the same plain forms.  

\end{itemize}

%Note: fold-bind-new, comp-new-nf-left, comp-new-nf-right not used!
%Note: lift-inner-new-nf can be replaced by comp-new-newNF
%Note: moveBindInPre/Nf can be done with moveReadInnerPreNf/Nf
%Note: drop-nf should always use nf(nil, samp Dist)
%Note: bind-red-2-pre renamed to bind-ret-pre


\end{document}