\newcommand{\flip}{\mathsf{flip}}
\newcommand{\id}{\mathsf{id}}
\newcommand{\adv}{\mathsf{adv}}
\newcommand{\Alice}{\mathsf{Alice}}
\newcommand{\In}{\mathsf{In}}
\newcommand{\Ctxt}{\mathsf{Ctxt}}
\newcommand{\Key}{\mathsf{Key}}
\newcommand{\LeakCtxt}{\mathsf{LeakCtxt}}
\subsection{Maude}

Maude \cite{DBLP:conf/maude/2007} is a high-level declarative
language and 
a high-performance logical framework supporting both equational and rewriting logic computation for a wide range of applications. 
Maude features several kinds of modules:
\begin{itemize}
\item \emph{functional modules}, which are theories (with an initial
model semantics) in 
membership equational logic that allow definitions of 
data types and operations on them, via multiple sorts, subsort
relations between them, equations between terms, and
assertions of membership of a term to a sort,
\item \emph{system modules}, which are theories in rewriting logic
that extend functional modules with
definitions of rewrite rules, representing transitions between states, and
\item \emph{strategy modules}, which control 
the way the rewriting rules are applied, by means of strategy combinators,
such as concatenation, iterations and others.
\end{itemize}

We now present the features of the Maude language that we make use of
in formalizing \textsf{IPDL}.
Maude functional modules are introduced with the syntax
\code{fmod NAME is ... endfm}. In a functional module we can declare
sorts, using the keyword \code{sort}, state that two sorts are
in the subsort relation, written \code{subsort s1 < s2}, declare
operations on the sorts, using \code{op f : s1  ...  sk -> s} for 
an operation \code{f} with argument sorts \code{s1 ... sk} and 
result sort \code{s}. Moreover, operations
may have attributes, written in square brackets after their declarations,
like \code{comm} for commutativity or \code{assoc} for associativity.
In Maude, terms are rewritten to a normal form modulo the declared 
attributes and the equations of defined operations.
More precisely, 
equations are used as equational rules: instances of the left-hand side
pattern that match subterms of a term are replaces with the 
corresponding instances of the right-hand side.
The process is called term rewriting and the result of simplifying a term by complete application of equational rules is called its normal form. 
We can control which operations will appear in these ground
forms by adding the attribute \code{ctor} to them. An operation that
is not a constructor of a sort is regarded as defined. 
Equations are introduced by the syntax 
\code{eq t1 = t2}, where \code{t1} and \code{t2} are terms
of sorts related via subsorting. 
We can assert sort membership using the syntax 
\code{t : s}
where \code{t} is a term and \code{s} is a sort.
Conditional equations are written
\code{ceq t = t' if C1 $\land$ ... $\land$ Cn} 
where \code{Ci} is either an equation or a membership.
We may declare variables using the keyword \code{var}.
Functional modules are assumed to satisfy the executability requirements of confluence,
termination, and sort-decreasingness, see details
in \cite{DBLP:conf/maude/2007}.
The semantics of functional models is given in terms of the initial
model whose elements are ground equivalence classes of terms modulo
equations.

Rewriting logic extends equational logic by introducing the notion of rewrites corresponding to transitions between states.
Unlike equations, rewrites are not symmetric. Maude system modules
are introduced with the syntax \code{mod NAME is ... endm}. Rules
are declared with the syntax 
\code{rl [label] : t1 => t2}. 
Conditional rules are written with the keyword
\code{crl [label] : t => t' if C1 $\land$ ... $\land$ Cn} 
and their conditions
\code{Ci} may be equations, memberships or rewrites. 
Rewrites are not expected to be terminating, confluent or deterministic.
Rewrites denote transitions between the elements of the initial model
of the functional part of a system module.

Maude strategy modules are introduced with the syntax
\code{smod NAME is ... endsm}. In addition to declarations allowed in
system modules, we can have strategy declarations and definitions.
The main strategy combinators are \code{;} for concatenation of strategy 
expressions, \code{|} for alternative, \code{*} for iteration of an
expression zero or more times, \code{idle} for the strategy giving as result its argument, \code{fail} for the strategy that gives no result,
\code{s1 ? s2 : s3} for the strategy that attempts to run the
strategy \code{s1} then, if the run is successful, it runs \code{s2},
otherwise it runs \code{s3}.
Several other derived constructions are also
supported, \emph{e.g.}, \code{try s} for \code{s ? idle : idle} and \code{s1 or-else s2} for \code{s1 ? idle : s2}. The match and rewrite operator 
\code{matchrew} restricts the application of a strategy to a specific
subterm of the subject term, see details in \cite{DBLP:conf/maude/2007}.
Strategies are declared as \code{strat NAME : s1 ... sk @ s .},
where \code{s1 ... sk} are the sorts of the arguments of the strategy
and \code{s} is the subject sort to which the strategy is applied.
The syntax for definitions is \code{sd NAME(v1,..., vk) := Exp .} where
\code{vi} are variables of sort \code{si} and \code{Exp} is a strategy expression.


Maude supports module imports, using the keyword \code{protecting},
which means that no new elements of an imported sorts may be 
added and 
no identification between elements of an imported sorts via
equations are allowed. 
Two more importation modes are supported, but we do not make use 
of them.

Maude provides several predefined data types. We will use Booleans, natural
numbers, lists, sets and maps.
  
\subsection{Syntax}

We start with a sort \code{Type} for data types, together with 
constants \code{unit} and \code{bool} of sort \code{Type} and a
binary product on the sort \code{Type}. Expressions are built over
signatures, which are implemented as commutative lists of symbols,
where a function or distribution symbol pairs the symbol name with 
its arity. Signatures are valid if they don't contain multiple
occurences of same symbol name. Expressions are then implemented
as a sort \code{Expression} that includes as a subsort the identifiers,
which are provided by the default Maude sort \code{Qid},
such that we can use them for variable names. There are constructors
for \code{True}, \code{False} and \code{()}. Application is 
represented as \code{ap f e} where \code{ap} is a constructor, \code{f}
is an identifier standing for the name of the function symbol and
\code{e} is an expression. Moreover we have constructors for
pairs and projections on first and second component of a pair. 
Type contexts are implemented again as commutative lists of typed
variables, written \code{x : T}, where \code{x} is an identifier and
\code{T} is a type. Expression typing is 
implemented as a predicate 
\code{typeOf : Signature TypeContext Expression -> Bool}, while
we let Maude handle expression equality by only adding the expression
equality rules \textsc{fast-pair}, \textsc{snd-pair} and 
\textsc{pair-ext} as axioms, \emph{e.g.},
\code{eq fst pair(E1, E2) = E1 .} where \code{E1 E2 : Expression}.

Channel sets are simply sets of identifiers, standing for channel
names. Channel contexts are commutative lists of typed channel
names, written \code{c :: T}.

Reactions are introduced by the following constructors of the sort
\code{Reaction}, following the grammar for reactions. If \code{e} is an expression, \code{return e} is a reaction. If \code{d} is an identifier, standing for the name of a
distribution symbol, and \code{e} is an expression, \code{samp d < e >} 
is a reaction. If \code{c} is a channel name, \code{read c} is a reaction.
Moreover, we write \code{if e then R1 else R2} for branching, if \code{e} is an expression and \code{R1, R2} are reactions, and \code{x : T <- R1 ; R2} for binding, when \code{x} is an identifier, \code{T} is a type and 
\code{R1, R2} are reactions.
Typing of reactions $\Delta; \ \Gamma \vdash R : I \to \tau$ is given by a function
\code{typeOf : Signature ChannelContext TypeContext          
Set\{ChannelName\} Reaction -> Type }, 
with the meaning that
we compute the type of a reaction in the context given by a signature, a channel context, a type context and a set of inputs, 
i.e. $\code{Delta}; \ \code{Gamma} \vdash \code{R} : \code{I} \to \code{T}$ 
if and only if 
\code{typeOf(Sigma, Delta, Gamma, I, R) = T}, where \code{Sigma} is 
current signature. 
Maude allows us to write the typing
judgements in a very similar way to their original formulation, \emph{e.g.}, the typing rule for binding is written as
\begin{lstlisting}

 ceq typeOf(Sigma, Delta, Gamma, I, x : T1 <- R1 ; R2) = 
     typeOf(Sigma, Delta, Gamma (x : T1), I, R2)
     if typeOf(Sigma, Delta, Gamma, I, R1) == T1 .

\end{lstlisting}

Protocols also follow the grammar for protocols, using the following
constructors for the sort \code{Protocol}. For the empty protocol we write
\code{emptyProtocol}. If \code{c} is a channel name and \code{R} is a 
reaction, \code{c ::= R} is a protocol. If \code{P1, P2} are protocols,
so is \code{P1 || P2}. Finally. \code{new c : T in P} is a protocol,
if \code{c} is a channel name, \code{T} is a type and \code{P} is a protocol. Typing of protocols $\Delta \vdash P : I \to O$ is implemented as a predicate
\code{typeOf : Signature ChannelContext Set{ChannelName} Protocol -> Bool}, with the meaning that the protocol typechecks 
in the context given by a signature, a channel context and a set of inputs.
Note that since the set of outputs can be computed from a protocol, we
do not add it as a parameter of the type checking predicate, so we will have that
$\code{Delta} \vdash \code{P} : \code{I} \to \code{getOutputs(P)}$ 
if and only if
\code{typeOf(Sigma, Delta, I, P)},
where \code{Sigma} is the current signature and
\code{getOutputs} computes the outputs of \code{P}.
The implementation splits the typechecking into a check that the inputs are valid w.r.t. \code{Delta} and a recursive function that does the rest of typechecking:

\begin{lstlisting}
eq typeOf(Sigma, Delta, I, A, P) = 
    validChanSet I Delta A 
    and 
    typeOfAux(Sigma, Delta, I, A, P)
\end{lstlisting}

For example, the typing rule for \code{new} checks that \code{c} is new and that \code{P} typechecks
when extending \code{Delta} with the typed channel \code{ c :: T}:
\begin{lstlisting}
 eq typeOfAux(Sigma, Delta, I, new c : T in P) =
         not occurs c Delta
     and typeOfAux(Sigma, Delta (c :: T), I, P) .   
\end{lstlisting}


\subsection{Exact equality}



At the reaction level, exact equality is given with axioms of the form
$\Delta; \ \Gamma \vdash R_1 = R_2 : I \to \tau$. 
Let us consider the following example:

$\code{(A :: bool) (B :: bool)}; \code{empty} \vdash$

$\code{x : bool <- return True  ; if x then read A else read B} = \code{read A} : \code{\{A, B\}} \to \code{bool}$. 

The proof of this is obtained by applying the \textsc{trans} axiom to

$\code{(A :: bool) (B :: bool)}; \code{empty} \vdash$

$\code{x : bool <- return True  ; if x then read A else read B} = 
$

$\code{if True then read A else read B} : \code{\{A, B\}} \to \code{bool}$

\noindent that we prove by \textsc{ret-bind} and

$\code{(A :: bool) (B :: bool)}; \code{empty} \vdash \code{if True then read A else read B} = \code{read A} : \code{\{A, B\}} \to \code{bool}$

\noindent that we prove by \textsc{if-left}. 

From a practical point of view, it is inconvenient to write this proof in this way, because we have to make explicit all intermediate steps, 
and this is tedious and error-prone. Instead, we will work with a
transition system. Its states are \emph{configurations} containing the context, i.e. the current signature \code{Sigma, Delta, Gamma, I, T},
and the current reaction: \code{rConfig(Sigma, Delta, Gamma, R, I, T)}.
The transitions in the system are determined by rewrite rules, 
which are obtained by orienting the axioms of the exact equality calculus
from left to right. Since we can apply the \textsc{sym} axiom,
the choice of direction is not important.
For example,
the \textsc{if-left} axiom becomes
\begin{lstlisting}
crl [if-left] : 
     rConfig(Sigma, Delta, Gamma, if True then R1 else R2, I, A, T) 
     =>  
     rConfig(Sigma, Delta, Gamma, R1, I, A, T)
 if
     typeOf(Sigma, Delta, Gamma, I, A, R1) == T
     /\
     typeOf(Sigma, Delta, Gamma, I, A, R2) == T 
 .
\end{lstlisting}

We also employ the Maude strategy language to conveniently write 
application of the \textsc{trans} axiom as rule composition, denoted \code{;}. The proof in the example above becomes
\begin{lstlisting}
srew 
 rConfig(emptySig, (A :: bool) (B :: bool), emptyTypeCtx, 
         x : bool <- return True  ; if x then read A else read B, 
         (A, B), bool)
using ret-bind ; if-left .
\end{lstlisting}
\noindent and Maude returns the following result
\begin{lstlisting}
Solution 1
rewrites: 28 in 0ms cpu (0ms real) (~ rewrites/second)
result ReactionConfig: 
 rConfig(emptySig, (A :: bool) (B :: bool), emptyTypeCtx, 
         read A, (A , B), bool)
\end{lstlisting}

If the condition of a rule is a rewrite, we will need to explicitly
provide a sub-proof for that step as well.
For example, the rule \textsc{cong-bind} is
\begin{lstlisting}
 crl [cong-bind] :
     rConfig(Sigma, Delta, Gamma, x : T1 <- R1 ; R2, I , A, T2) 
     => 
     rConfig(Sigma, Delta, Gamma, x : T1 <- R3 ; R4, I, A, T2) 
     if
     rConfig(Sigma, Delta, Gamma, R1, I, A, T1)  
     => 
     rConfig(Sigma, Delta, Gamma, R3, I, A, T1) 
     /\
     rConfig(Sigma, Delta, Gamma (x : T1), R2, I, A, T2)
     => 
     rConfig(Sigma, Delta, Gamma (x : T1), R4, I, A, T2) .
\end{lstlisting}
\noindent and we can apply it to rewrite the reaction
\code{x : bool <- if True then read A else read B ; return x}
to
\code{x : bool <- read A ; return x}
by writing
\code{cong-bind\{if-left, idle\}}.

The \textsc{if-ext} axiom has the particularity that 
it establishes an equality between reactions 
where a variable has been substituted with a term.
Maude cannot apply this rule, because it cannot do the matching.
For this reason, we have omitted this rule and replaced it with
several rules that we can prove using \textsc{if-ext}. We have also
introduced an alpha-renaming rule for convenience, as we can also derive it
from the exact equality axioms.

The same principle is applied for exact equality of protocols.
This time we rewrite protocol configurations, of the form
\code{pConfig(Sigma, Delta, P, I, O)}.
The rules of exact equality for protocols may make use of
exact equality of reactions. For example, the \textsc{cong-react} rule
is
\begin{lstlisting}
 crl [CONG-REACT] : 
     pConfig(Sigma, Delta (c :: T), c ::= R, I, c) 
     =>
     pConfig(Sigma, Delta (c :: T), c ::= R', I, c)
     if
     rConfig(Sigma, Delta (c :: T), emptyTypeCtx, R, 
             insert(c, I), T)
     =>
     rConfig(Sigma, Delta (c :: T), emptyTypeCtx, R', I', T) 
     /\ I' == insert(c, I)
     /\ not c in I .
\end{lstlisting}

\subsection{Normal Forms}

We work with protocols that start with a list of declarations of 
internal channels, using $\code{new}$, followed by a parallel compositions of channel assignments. The reactions in these
assignments can be transformed into a list of binds of the form
\code{x : T <- read c},
called bind-read reactions, followed by a reaction without binds.
The list of binds can be regarded as commutative, 
as two reactions with the same list of binds in different order
are equivalent due to the reaction equivalence rule \textsc{exch}.
Similarly, different order of declarations of internal channels gives
equivalent protocols, by using the protocol equivalence rule
\textsc{new-exch}. When writing equivalence proofs, we do not want to 
make the use of these rules explicit. Instead, we want to be able to apply
the rules as though we could freely consider a certain declaration of an 
internal channel or a certain bind read reaction as the first.

Therefore, we introduce normal forms of reactions and protocols. 
For reactions, normal forms 
\code{nf(L, R, O)}
consist of a commutative list \code{L} of bind-read reactions,
a bind-free reaction \code{R}
and a chosen order \code{O} of the names of the variables occuring in
the binds in \code{L}, given as a list of names.
The latter will be used to determine how to turn the normal form 
of a reaction into a regular reaction.
For example, the normal form of
\begin{lstlisting}
 'd : bool <- read 'ce ;
 'm0 : bool <- read 'in0 ; 
 'm1 : bool <- read 'in1 ;
 'k0 : bool <- read 'key0 ;
 'k1 : bool <- read 'key1 ;
  if 'd then return 'k0 else return 'k1
\end{lstlisting}
is
\begin{lstlisting}
 nf(
  ('d : bool <- read 'ce)
  ('m0 : bool <- read 'in0) 
  ('m1 : bool <- read 'in1)
  ('k0 : bool <- read 'key0)
  ('k1 : bool <- read 'key1),
  if 'd then return 'k0 else return 'k1,
  'd :: 'm0 :: 'm1 :: 'k0 :: 'k1
 )
\end{lstlisting}
 
During equivalence proofs, we may obtain 
in a normal form \code{nf(L, R, O)}
either arbitrary binds in \code{L} 
(\emph{e.g.}, by substituting a read from a channel with the reaction
assigned to that channel)
or reactions \code{R} that are not bind-free.
This will be represented as a pre-normal-form, 
written 
\code{preNF(L, R, O)}, 
which is a normal form without restrictions on the occuring reactions.
If \code{L} contains a bind that is not a read bind, we will write it
as \code{x1 : T1 <$\sim$ R1}. 
The general strategy will be to transform pre-normal-forms 
\code{preNF(L, R, O)}
to normal forms using the following steps:
\begin{itemize}
\item if \code{x1 : T1 <$\sim$ R1} is in \code{L} and \code{R1} is of the form
\code{nf(L2, R2, O2)}, move the inner binds from \code{L1} at the level of \code{L}, and simplify the reaction of \code{x1} to \code{R2}.
\item if  \code{x1 : T1 <$\sim$ R1} is in \code{L} and \code{R1} is bind-free,
rewrite the entire reaction as 
\code{preNF(L', x1 : T1 <- R1 ; R, O')}, where \code{L'} and \code{O'}
are obtained by removing \code{x1 : T1 <$\sim$ R1} and \code{x1} from \code{L} and \code{O}, respectively.
\item apply reaction-level axioms to \code{R} to bring it in the form
\code{L' ; R'}, where \code{L'} is a list of bind reads and \code{R'} is
bind-free, then move \code{L'} at the outer level of \code{L}.
\end{itemize}    

At the level of protocols, normal forms  
\code{newNf(L, P, O)}
consist of a commutative list \code{L} of declarations of internal
channels, a protocol \code{P} that does not start with internal channel declarations
and again a designated order \code{O} 
for the names of internal channels occuring in the declarations in 
\code{L}.
For example, the normal form of
\begin{lstlisting}
new 'ce : bool in 
new 'key0 : bool in 
new 'key1 : bool in 
new 'flip : bool in 
new 'choice : bool in
 (
 ('ce ::= 'f : bool <- read 'flip ;
          'c : bool <- read 'choice ;
          if 'f then 
           (if 'c then return False else return True)
          else 
           (if 'c then return True else return False)
 )
 || ('msgenc0 ::= 'd : bool <- read 'ce ;
                  'm0 : bool <- read 'in0 ; 
                  'm1 : bool <- read 'in1 ;
                  'k0 : bool <- read 'key0 ;
                  'k1 : bool <- read 'key1 ;
                  if 'd then return 'k0 else return 'k1)
  || ('key0 ::= return True) 
  || ('key1 ::= return False)
  || ('flip ::= return True) 
  || ('choice ::= return False) 
)
\end{lstlisting}
\noindent is
\begin{lstlisting}
newNF(
 ('ce : bool) ('key0 : bool) ('key1 : bool)
 ('flip : bool) ('choice : bool),
 
 ('ce ::= 'f : bool <- read 'flip ;
          'c : bool <- read 'choice ;
          if 'f then 
           (if 'c then return False else return True)
          else 
           (if 'c then return True else return False)
 )
 || ('msgenc0 ::= 'd : bool <- read 'ce ;
                  'm0 : bool <- read 'in0 ; 
                  'm1 : bool <- read 'in1 ;
                  'k0 : bool <- read 'key0 ;
                  'k1 : bool <- read 'key1 ;
                  if 'd then return 'k0 else return 'k1)
  || ('key0 ::= return True) 
  || ('key1 ::= return False)
  || ('flip ::= return True) 
  || ('choice ::= return False),
  
  'ce :: 'key0 :: 'key1 :: 'flip :: 'choice
)
\end{lstlisting}

\subsection{Families of protocols}

Families of protocols provide a convenient abbreviation for 
semantically related protocols \code{P[0]...P[n]}, where 
the value of \code{n} is typically not known. The semantical relation
translates in the protocols being assigned similar reactions.
We illustrate the syntax with the help of an example:
\begin{lstlisting}
(family 'SumCommit 'i (bound (n + 2)) ::= 
        (when ('i =T= 0) --> nf(emptyBRList, return False, emptyCNameList)) ;; 
        (when ('i =T= (n + 2)) --> 
               nf( ('x : bool <- read ('SumCommit [ n + 1 ])) 
                   ('f : bool <- read 'LastCommit) ,
                   return (ap 'xor pair('x, 'f)) ,
                   'x :: 'f :: emptyCNameList )
        )          ;;
        (otherwise --> nf(('x : bool <- read ('SumCommit ['i -- 1])) 
                          ('c : bool <- read ('Commit ['i -- 1])),
                          return (ap 'xor pair('x, 'c)),
                          'x :: 'c :: emptyCNameList )
        )                  
    )
\end{lstlisting}    
Here \code{'i} is an index 
variable ranging between \code{0} and \code{n + 2}.
The reaction assigned to the protocol \code{'SumCommit['i]} is given
with alternatives. 
We allow the bound to be a natural number, an identifier denoting a 
natural number or an expression involving natural numbers and
identifiers. 
We represent this as a sort \code{NatTerm} that is a super-sort of
\code{Qid} and \code{Nat} together with 
addition, deletion (written \code{--}) and multiplication on that sort,
extending in the expected way the corresponding operations on natural
numbers. The conditions occuring in the alternatives are of sort
\code{BoolTerm}, and can be comparisons between \code{NatTerm}s
(\code{=T=, <T, <=T}),
user-defined predicates (\code{apply 'p t}) where \code{'p} is 
the name of the predicate and \code{t} is a \code{NatTerm} or negation
of a \code{BoolTerm}.

In this new setting, we may introduce a channel directly or via a family
of protocols. 
Channel names, which were identifiers so far, must be extended to
indexed identifiers. We implement them as a sort
\code{ChannelName} which includes as a sub-sort the sort of identifiers
and  has a constructor \code{\_[\_] : Qid List\{NatTerm\} -> ChannelName},
and thus \code{'c['i 'j]} is an example of a channel name. 
Taking this into account, we
have also extended channel sets and channel contexts to keep track of the 
bound of a family of protocols. Channel sets are implemented as
sets of bounded channel names, which are written \code{c @ l }, where \code{c} is an identifier and 
\code{l} is a list of bounds for the family named
\code{c}. We use an empty list for a regular protocol, with no indices. Channel contexts are commutative lists of 
typed bounded channel names, written 
\code{ c @ l :: T}, where \code{T} is a type.

We allow families of protocols with two indices as well. 
We write \code{family 'F ('i 'j) ((bound m) (uniformBound n))}
for a family indexed by \code{'i} ranging from \code{0} to
\code{m} and by \code{'j} ranging from \code{0} to
\code{n}. We also allow the second bound to vary for each \code{'i},
but we did not use this in the case studies so far.

The equality calculus must be adapted to the new notation. We
have introduced rules that apply the core equality rules over the new
notation, with the meaning that the rules are applied in parallel for
each index.
We need to record the assumptions made about indices, so we extend
the protocol configuration with a new component of sort
\code{Set\{BoolTerm\}}.
 Moreover, we have a rule for induction proofs. 
We present here the variant for one index, 
as the one for two indices is similar.
The goal is to rewrite
\code{ P || family C q (bound nt1) ::= cases} 
to 
\code{ P || family C q (bound nt1) ::= cases'} 
by induction on the index
\begin{lstlisting}
crl [INDUCTION-when-one] :
     pConfig(Sigma, Delta, 
       P || (family C q (bound nt1) ::= cases), I, O, A)
     =>
     pConfig(Sigma, Delta , 
       P || (family C q (bound nt1) ::= cases'), I, O, A)
\end{lstlisting} 
We start with the base case and we must provide a proof that
if we assign \code{C[0]} its corresponding case from \code{cases}
with \code{q = 0}, we can rewrite the resulting protocol to the protocol
that we get by assigning \code{C[0]} its corresponding case from \code{cases'} with \code{q = 0}. We record the assumption \code{q = 0} in
the set of index assumptions 
\code{A}. We also need to update the current outputs, by removing the outputs of the family \code{C} and adding \code{C[0]}:
      
\begin{lstlisting}
     if 
     pConfig(Sigma, Delta , 
       P || (projectIndex (family C q (bound nt1) ::= cases) 0 A empty ), I, 
       insert( C[0] @ nil, O \ (C @ nt1)), 
       insert(q =T= 0, A)
       ) 
     => 
     pConfig(Sigma, Delta , P2 , I, O', A')
     /\
     O' == insert( C[0] @ nil, O \ (C @ nt1))
     /\
     A' ==  insert(q =T= 0, A) 
     /\
     P2 == P || (projectIndex (family C q (bound nt1) ::= cases') 0 A empty)
\end{lstlisting}
The induction step assumes that we have successfully proven the 
property up to index \code{'k}, so now we can
make use of \code{family C q (bound 'k) ::= cases'} when proving the
property for index \code{'k + 1}, where \code{'k} is arbitrary.
We record in A the assumption that \code{'k + 1} must be in bounds. 
We also need to update the current outputs, by removing the outputs of the family \code{C} and adding \code{C[k + 1]} and the outputs of the family \code{C} with the new bound \code{k}:
\begin{lstlisting}     
     /\
     pConfig(Sigma, Delta , 
       P || (family C q (bound 'k) ::= cases') || 
       (projectIndex (family C q (bound nt1) ::= cases) ('k ++ 1) A empty), I, 
       insert(C @ 'k, insert(C['k ++ 1] @ nil, O \ (C @ nt1))), 
       insert('k ++ 1 <=T nt1, A)
       )
       =>
     pConfig(Sigma, Delta, P3, I, O'', A'')
     /\
     O'' = insert(C @ 'k, insert(C['k ++ 1] @ nil, O \ (C @ nt1)))
     /\
     A'' == insert('k ++ 1 <=T nt1, A) 
     /\
     P3 == (
             P || (family C q (bound 'k) ::= cases') ||
             (projectIndex (family C q (bound nt1) ::= cases') ('k ++ 1) A empty)
           )
       [nonexec] . 
\end{lstlisting}

A strategy will call the induction rule using
\begin{lstlisting}
INDUCTION-when-one[
         C:Qid <- Q, 
         cases':Cases <- 'the cases that we want to get after the induction proof'
        ]
        { 'proof of induction base',
          'proof of induction step'        
        }
\end{lstlisting}

\subsection{Strategies}

We now come to the strategies that will appear in proofs. It is possible that some of them will make use of other substrategies, but as these will not be in use, we refrain from including them here. To ease presentation,
we group strategies by the main core rule that is applied. They may 
have several forms to be applied in different contexts  
\emph{e.g.}, channels, families, groups of families, cases.

The rules \textsc{refl}, \textsc{trans}, \textsc{axiom} and
\textsc{embed} are not explicitly applied, as they are implied by
the properties of rewrite relation in Maude. 
The rule \textsc{sym} does not require the use of a strategy, and
in order to apply it we must specify explicitly the protocol that we 
rewrite from. More precisely, if the current protocol is \texttt{P}, 
we write \texttt{SYM[P1:Protocol <- P']\{proof\}} where 
\texttt{proof} is an
exact equality proof rewriting \texttt{P'} to \texttt{P}. 
The rule
\textsc{input-unused} is embedded in the application of other rules,
in the way the conditions on inputs are given.
The rules \textsc{cong-react}, \textsc{cong-new} and \textsc{cong-comp}
are applied inside the strategy definition, and their usage is not visible
to the user. The rules \textsc{cong-comm} and
\textsc{cong-assoc} are not applied explicitly, as
it suffices to specify the parallel composition in Maude as a commutative
and associative operation.

\subsubsection{SUBST}

Here we have the largest number of variations, because we need rules for
substituting a channel in a family, a family in a family taking into 
account whether they have one or two indices and so on. Since the number
of parameters varies, we cannot have a meta-strategy that tries all
possible variants. In the future we plan to generate the extra
arguments from the context where the rule applies, and thus reduce the arguments of all strategies to the name of the channel/family that gets
substituted and the the name of the channel/family where the 
substitution takes place. Thus, we will be able to introduce a
meta-strategy that greatly simplifies substitutions.

We have the following substitution strategies:
\begin{itemize}
 \item  \texttt{substNF(C1, C2)} substitutes the channel \texttt{C1} in \texttt{C2}. Both channels must be in normal form. The strategy also gets the pre-normal form resulting from the substitution to a normal form, as described above.
   \item  \texttt{substNFRead(C1,C2)} is a simpler particular case of substitution when the channel \texttt{C} that we substitute reads from another channel. Both channels must be in normal form.
   \item  \texttt{smart-subst-nf(C1, C2)} tries to apply \texttt{substNFRead} and if that fails, applies \texttt{substNF}. 
In the future we plan to plug all substitution strategies under this
meta-strategy.
   \item  \texttt{substNFFamiliesOne(C1, C2, R)} substitutes in the family \texttt{C2} with one index a read from \texttt{C1[i]} for some index \texttt{i} with the corresponding reaction \texttt{R} assigned in the family \texttt{C1} to the index \texttt{i}. The family \texttt{C2} must be in normal form.
   \item  \texttt{applyCaseDistSubst(q1, q2, q3, q4, pr)} works under the assumption that we have two groups, \texttt{q1, q2}, and \texttt{q1} is defined with cases. The rule does a \texttt{substNFRead} equivalent for the families \texttt{q3, q4} with two indices that come from the first branch of \texttt{q1} and from \texttt{q2}. The protocol \texttt{pr} is then used in a \textsc{sym} proof to redo the grouping.
   \item  \texttt{substChannelFamilyOne(C1, C2)} substitute a channel in a family with one index.
   \item  \texttt{applySubstChannelBranch(C1, q2)} substitutes the channel 
   \texttt{C1} in the family \texttt{C2}, in the first branch of a group defined with cases.
   \item  \texttt{applyCaseDistBranch2(q1, q2)} substitutes a channel in a family on the left branch of the right branch of a group.
   \item  \texttt{applyBranch2SubstRev(q1, q2, nt, x, T, R)} applies a reverse substitution on the left branch of a family. The parameters \texttt{nt, x, T, R} are the index, the name of the bind variable, the type and the reaction of the channel that is reversely substituted. 
   \item  \texttt{applySubstRevFamily(Q, C2, T)} does a reverse substitution on a branch of a family with cases. The parameter \texttt{T} is the type of the reaction that is reversely substituted.
   \item  \texttt{substNFReadFamilyOneChannel(C1, C2)} is a \texttt{substNFRead} equivalent for a family with one index and a channel.
   \item  \texttt{substNFReadFamilyTwoChannel(C1, C2)} is a \texttt{substNFRead} equivalent for a  family with two indices and a channel.
   \item  \texttt{substRevFamilyChannel(Q, C, nt, T)} does the reverse substitution of a family \texttt{Q} in a channel \texttt{C}. The parameters \texttt{nt, T} are the index and the type of the channel that is reversely substituted.
   \item  \texttt{substNFFamilyOneChannel(C1, C2, R)} is the \texttt{substNF} equivalent for a family with one index and a channel.
   \item  \texttt{applySubstNFLeft(q1, q2, R)} applies \texttt{substNFFamiliesOne} on the left branch of a family defined with cases.
\end{itemize}  

\subsubsection{DROP}

\begin{itemize}
\item \texttt{applyDropNF(C1, C2)} applies the normal form version of \textsc{drop}.
\item \texttt{applyDropPreNF(C1, C2)} applies the pre-normal form
version of \textsc{drop}.
\end{itemize}

\subsubsection{ABSORB}

\begin{itemize}
\item \texttt{absorbChannel(C)} applies the new-normal-form version of
\textsc{absorb} for the channel \texttt{C}.
\item \texttt{absorbFamily(Q)} applies the new-normal-form version
of \textsc{absorb} for the family \texttt{Q}.
\item \texttt{applyAbsorbReverse(P)} applies the reverse of the 
\textsc{absorb} rule for the protocol \texttt{P}.
\item \texttt{addNewFamilyToGroup(P, Q1, Q2)} adds the family \texttt{Q2},
introduced by the protocol \texttt{P}, to the group \texttt{Q1}.
\item \texttt{applyCaseDistAbsorb(q1, q2, q3, pr)} operates
under the assumption that the current protocol is of the form
\texttt{family q2 ::= P || family q1 ::= when cond1 --> P1 ;; otherwise --> P2} and applies the \textsc{absorb} rule on the protocol
\texttt{P || P1} then it reconstructs the original shape of the
current protocol.
\end{itemize}

\subsubsection{FOLD}

\begin{itemize}
\item \texttt{foldNF(C1, C2)} applies the normal form version of the 
\textsc{fold} rule, when the channel \texttt{C1} gets folded in the 
channel \texttt{C2}.
\item \texttt{foldNFPre(C1, C2)} applies the pre-normal form version of the 
\textsc{fold} rule, when the channel \texttt{C1} gets folded in the 
channel \texttt{C2}.
\item \texttt{foldNFFamily(Q1, Q2)} applies the normal form version 
of the 
\textsc{fold} rule, when the family \texttt{Q1} gets folded in the 
family \texttt{Q2}.
\end{itemize}

\subsubsection{READ-INSIDE-IF}

This is a rule derived from \textsc{if-ext} and allows us to rewrite
\texttt{x : T1 <- read i ; if M then R1 else R2} to 
\texttt{if M then x : T1 <- read i ; R1 else x : T1 <- read i ; R2}.

\begin{itemize}
\item \texttt{applyReadInsideIfPre(C)} applies 
\textsc{read-inside-if} to a protocol in new-normal-form.
\end{itemize}

\subsubsection{Purely syntactic transformations}

Under this heading we group a number of strategy that change
only the shape of a protocol. The rules that apply are
derivable from the core rules.

\begin{itemize}
\item \texttt{applyAddToGroup(Q1, Q2)} moves the family 
\texttt{Q2} inside the group \texttt{Q1}.
\item \texttt{changeOrder(C, ql)} changes the specified order
of reads in a normal or pre-normal form. \texttt{C} is the name of 
the channel that is assigned the (pre-)normal form and
\texttt{ql} is the new order, given as a list of variable names.
\item \texttt{applyReorderNF(Q, ql)}: on the first branch of the family
\texttt{Q}, changes the order in the normal form as specified in 
\texttt{ql}.
\item \texttt{nf2PreNF(C)} turn the normal form assigned to the 
channel \texttt{C} to a pre-normal form
\item \texttt{applyGroupFamilies(Q1, Q2)} composes the families 
\texttt{Q1, Q2} to a new family \texttt{'Comp[Q1 Q2]} that assigns to
each index \texttt{i} the protocol
\texttt{(Q1[i] ::= R1) || (Q2[i] ::= R2} where \texttt{R1, R2} are
the reactions assigned to the index \texttt{i} by \texttt{Q1, Q2}.
This transformation is needed for some induction proofs.
\item \texttt{applyUngroupFamilies(Q1, Q2)} is the reverse transformation
of the previous strategy.
\item \texttt{moveProtocolUnderNewNF} if the current protocol is the parallel composition of a protocol \texttt{P} with a new-normal-form, move
\texttt{P} inside the new-normal-form.
\item \texttt{applyDeleteEmptyNF(Q)} if the new-normal-form assigned to the
group \texttt{Q} has no new declarations, keeps only its protocol.
\item \texttt{applyDropName(Q)} removes the group name \texttt{Q}.
\item \texttt{applyCombine(Q)} if the group \texttt{Q} is defined using cases, removes the group name and moves the cases inside the families of the group.
\item \texttt{applyAlphaNFPr(C, QL)} does an alpha-renaming of the bind
variables of a normal form assigned to the channel \texttt{C}.
 \texttt{QL} specifies the
renaming.
\item \texttt{applyBranch2Alpha(q1, QL)} does an alpha-renaming on the otherwise branch of a family \texttt{q1}. \texttt{QL} specifies the
renaming.
\item \texttt{moveBindInPre(C, Q)} if the channel \texttt{C} is assigned
a normal form, move the bind assigned to the variable \texttt{Q} in the
reaction of the normal form, and turn the normal form to a pre-normal form.
\item \texttt{applyBranch2MoveReads(q1, ql)} moves the reads 
specified by the list \texttt{ql} from bind list of a normal form to the reaction of the normal form on the left branch of a family 
\texttt{q1}.
\item \texttt{moveReadsToRFamily(C,cnl)} move the reads specified in \texttt{cnl} from the bind list of a normal form to the reaction of the normal form assigned to the family \texttt{C}.

\end{itemize}  

\subsection{A simple example}

In our simplest example, the \ipdl \emph{Hello World} analogue, Alice receives a Boolean message, encodes it by xor-ing it with a randomly generated Boolean, and leaks the encoding to the adversary. We show that this is equal to leaking a randomly generated ciphertext.

Formally, our signature consists of two symbols: $\oplus : \Bool \times \Bool \to \Bool$ for the Boolean sum, and $\flip : \one \twoheadrightarrow \Bool$ for the uniform distribution on Booleans. We write $x \oplus y$ in place of $\oplus \ (x,y)$.

\subsubsection{The Assumptions}
Our single axiom states that $\flip$ is invariant under xor-ing with a fixed Boolean:
\begin{itemize}
\item $\cdot; \ x : \Bool \vdash (y \leftarrow \flip; \ \ret{x \oplus y}) = \samp{\flip} : \emptyset \to \Bool$
\end{itemize}
This is indeed the case if (and only if) $\flip$ is uniform.

\subsubsection{The Ideal Functionality}
Upon receiving the input message, the ideal functionality generates a random ciphertext on an internal channel $\Ctxt$ and leaks its value to the adversary:
\begin{itemize}
\item $\Ctxt \coloneqq m \leftarrow \In; \samp{\flip}$ 
\item $\LeakCtxt^\id_\adv \coloneqq \read{\Ctxt}$
\end{itemize}

\subsubsection{The Real Protocol}
In the real protocol, Alice generates a random Boolean key on an internal channel $\Key$, constructs the ciphertext by xor-ing the input message with the key, and leaks the resulting ciphertext to the adversary:
\begin{itemize}
\item $\Key \coloneqq \samp{\flip}$
\item $\Ctxt \coloneqq m \leftarrow \In; \ k \leftarrow \Key; \ \ret{m \oplus k}$
\item $\LeakCtxt^\Alice_\adv \coloneqq \read{\Ctxt}$
\end{itemize}

\subsubsection{The Simulator}
The simulator mediates between the two leakage channels $\LeakCtxt^\id_\adv$ and $\LeakCtxt^\Alice_\adv$ by forwarding the former to the latter:
\begin{itemize}
\item $\LeakCtxt^\Alice_\adv \coloneqq \read{\LeakCtxt^\id_\adv}$
\end{itemize}

\subsubsection{Real = Ideal + Sim}
On the left-hand side of the above equality we have the real protocol. On the right-hand side, we have the composition of the ideal functionality with the simulator, followed by the hiding of the channel $\LeakCtxt^\id_\adv$. The two protocols now have identical inputs (the channel $\In$) as well as outputs (the channel $\LeakCtxt^\Alice_\adv$).

We now simplify both protocols so that they have the same internal structure. On the left-hand side, we fold the internal channel
\begin{itemize}
\item $\Key \coloneqq \samp{\flip}$
\end{itemize}
into the channel
\begin{itemize}
\item $\Ctxt \coloneqq m \leftarrow \In; \ k \leftarrow \Key; \ \ret{m \oplus k}$,
\end{itemize}
yielding
\begin{itemize}
\item $\Ctxt \coloneqq m \leftarrow \In; \ k \leftarrow \flip; \ \ret{m \oplus k}$
\end{itemize}
and on the right-hand side we fold the internal channel
\begin{itemize}
\item $\LeakCtxt^\id_\adv \coloneqq \read{\Ctxt}$
\end{itemize}
into the channel
\begin{itemize}
\item $\LeakCtxt^\Alice_\adv \coloneqq \read{\LeakCtxt^\id_\adv}$,
\end{itemize}
yielding
\begin{itemize}
\item $\LeakCtxt^\Alice_\adv \coloneqq \read{\Ctxt}$.
\end{itemize}
The two protocols now both have an internal channel $\Ctxt$ and an output channel $\LeakCtxt^\Alice_\adv$.

To finish the proof, we fold the internal channel into the output channel in both protocols, yielding the two single-reaction protocols
\begin{itemize}
\item $\LeakCtxt^\Alice_\adv \coloneqq m \leftarrow \In; \ k \leftarrow \flip; \ \ret{m \oplus k}$, and
\item $\LeakCtxt^\Alice_\adv \coloneqq m \leftarrow \In; \ \samp{\flip}$
\end{itemize}
The equality between these two now follows immediately from our axiom, and we are done.

\subsubsection{Maude implementation}

Assume we work in the file \code{helloWorld.maude} placed in the 
\code{lib} folder of the IPDL-Maude repository. We start by importing
the strategies and starting a new module that extends \code{APPROX-EQUALITY}, which provides both exact and approximate equality
\begin{lstlisting}
load ../src/strategies

mod HELLO-WORLD is
 protecting APPROX-EQUALITY .

\end{lstlisting}

We have to define the signature.
Our example works with booleans, 
so we will not introduce new datatypes. 
When needed, we define them as new constants of sort \code{Type}.

We must introduce a function symbol for $\oplus$ and 
a distribution symbol for $\flip$:
\begin{lstlisting}
op xorF : -> SigElem .
eq xorF = 'xor : (bool * bool) ~> bool .

op flipF : -> SigElem .
eq flipF = 'flip : unit ~>> bool .

op sig : -> Signature .
eq sig = xorF flipF .
\end{lstlisting}
We then write the protocol resulting from composing the ideal functionality
with the simulator followed by hiding the channel 
\code{'LeakCtxt\_id\_adv}
and the protocol \code{real}:
\begin{lstlisting}
 op idealPlusSim : -> Protocol .
 eq idealPlusSim = 
    new 'Ctxt : bool in
    new 'LeakCtxt_id_adv : bool in
    (
    ('LeakCtxt_id_adv ::= 
       nf(('c : bool <- read 'Ctxt),
          return 'c,
          'c :: emptyCNameList )
    )  
    ||
    ('LeakCtxt_Alice_adv ::= 
       nf('c : bool <- read 'LeakCtxt_id_adv,
          return 'c, 
          'c :: emptyCNameList)
    )
    ||
    ('Ctxt ::= 
       nf( 'm : bool <- read 'In,
           samp ('flip < () >),
           'm :: emptyCNameList)
    )
    )
 .
 
  op real : -> Protocol .
  eq real = 
    new 'Key : bool in
    new 'Ctxt : bool in
    (
     ('Key ::= samp ('flip < () >))
     || 
     ('Ctxt ::= 
       nf( ('m : bool <- read 'In)
            'k : bool <- read 'Key,
            return (ap 'xor pair('m, 'k)),
           'm :: 'k :: emptyCNameList
               )
     )
     ||
     ('LeakCtxt_Alice_adv ::= 
       nf( 'c : bool <- read 'Ctxt ,
           return 'c,
          'c :: emptyCNameList )  
     )
    ) 
         
  .
\end{lstlisting}

We then close the \code{HELLO-WORLD} module and
open a new module, \code{EXECUTE}, importing \code{HELLO-WORLD} and
\code{STRATEGIES}, where we add the assumptions and typically strategies
for using them in proofs
\begin{lstlisting}
smod EXECUTE is
  protecting STRATEGIES .
  protecting HELLO-WORLD .
\end{lstlisting}

The assumption is introduced at the level of reactions
\begin{lstlisting}
rl [assumption] :
    rConfig(Sigma, Delta, Gamma (x : bool),
            y : bool <- samp flip ; 
            return (ap 'xor pair(x, y)), I, A, bool
            )
     => 
     rConfig(Sigma, Delta , Gamma (x : bool),
             samp flip, I, A, bool)
  .  
\end{lstlisting}
\noindent and we will want to apply it in the 
main reaction of a pre-normal form, so we need the following strategy:
\begin{lstlisting}
strat applyAssumption : ChannelName @ ProtocolConfig .
  sd applyAssumption(cn) := 
   match pConf s.t. startsWithNew pConf
    ? CONG-NEW-NF{applyAssumption(cn)}
    : matchrew pConf s.t. pConfig(Sigma, Delta, P, I, O, A) := pConf by pConf 
       using CONG-REACT[o:ChannelName <- cn]
                { cong-pre-nf{assumption} ; 
                  try (pre2Nf)}   
  .
\end{lstlisting}
\noindent which states, read in reverse order, that we apply 
\textsc{cong-pre-nf} with the assumption as a sub-proof 
to the reaction \code{R} that is assigned to the channel \code{cn}
(thus requiring to apply \textsc{cong-react}) 
that is inside a new normal form, and then we must apply
\code{cong-new-nf}. Moreover, \code{pre2Nf} converts the
pre-normal form to a normal form, as now we no longer have binds in the
reaction of the pre-normal form.

We now get to writing the proof. The initial configuration is
\begin{lstlisting}
pConfig(sig, 
        ('In @ nil :: bool) 
        ('LeakCtxt_Alice_adv @ nil :: bool), 
        idealPlusSim, 
        'In @ nil, 
        getOutputs(idealPlusSim), 
        empty)
\end{lstlisting}
\noindent where 
\begin{itemize}
 \item \code{sig} is the signature defined above,
 \item the channel context contains the input channel \code{'In}
       and the output channel \code{'LeakCtxt\_Alice\_adv}. Both are of type
       \code{bool} and they have no indices,
 \item the protocol that we want to rewrite,
 \item the set of its inputs, here we have just one input channel,      
 \item the set of its outputs, which can be computed, so it is more
       convenient to call the function \code{getOutputs} than to
       enumerate all inputs,
 \item since we have no indices, the set of assumptions about them is empty.      
\end{itemize}

We first turn the protocol in new normal form, then we do the two folds:
\begin{lstlisting}
srew [1] 
pConfig(sig, 
         ('In @ nil :: bool) 
         ('LeakCtxt_Alice_adv @ nil :: bool), 
         idealPlusSim, 
         'In @ nil, 
         getOutputs(idealPlusSim), 
         empty)
using 
       sugar-newNF
     ; foldNF('LeakCtxt_id_adv, 'LeakCtxt_Alice_adv) 
     ; foldNF('Ctxt, 'LeakCtxt_Alice_adv)    
.

\end{lstlisting}

Similarly, for \code{real}, we turn the protocol in new normal form,
we do the folds and then we apply the assumption:
\begin{lstlisting}
srew [1] 
 pConfig(sig, 
         ('In @ nil :: bool) 
         ('LeakCtxt_Alice_adv @ nil :: bool), 
         real, 
         'In @ nil, 
         getOutputs(real), 
         empty)
using 
       sugar-newNF
     ; foldNF('Key, 'Ctxt)
     ; foldNF('Ctxt, 'LeakCtxt_Alice_adv)
     ; applyAssumption('LeakCtxt_Alice_adv)
. 
\end{lstlisting}

In both cases we get the same result, so we now can combine the two proofs in one, using \textsc{sym}:
\begin{lstlisting}
srew [1] 
 pConfig(sig, 
         ('In @ nil :: bool) 
         ('LeakCtxt_Alice_adv @ nil :: bool), 
         real, 
         'In @ nil, 
         getOutputs(real), 
         empty)
using 
       sugar-newNF
     ; foldNF('Key, 'Ctxt)
     ; foldNF('Ctxt, 'LeakCtxt_Alice_adv)
     ; applyAssumption('LeakCtxt_Alice_adv)
     ; SYM[P1:Protocol <- idealPlusSim]{
       sugar-newNF
     ; foldNF('LeakCtxt_id_adv, 'LeakCtxt_Alice_adv) 
     ; foldNF('Ctxt, 'LeakCtxt_Alice_adv)  
     }
.  
\end{lstlisting}

When running the proof in Maude, we get \code{idealPlusSim} in the 
protocol in the resulting \code{pConfig}, as expected. Maude also reports
on the number of rewrites and the time spent doing the proof:´
\begin{lstlisting} 
$  maude lib/helloWorld.maude 
		     \||||||||||||||||||/
		   --- Welcome to Maude ---
		     /||||||||||||||||||\
	    Maude 3.2.1 built: Feb 21 2022 18:21:17
	     Copyright 1997-2022 SRI International
		   Tue Mar 14 04:52:45 2023
==========================================
srewrite [1] in EXECUTE : pConfig(sig, ('In @ nil :: bool)
    'LeakCtxt_Alice_adv @ nil :: bool, real, 'In @ nil, getOutputs(real),
    empty) using sugar-newNF ; foldNF('Key, 'Ctxt) ; foldNF('Ctxt,
    'LeakCtxt_Alice_adv) ; applyAssumption('LeakCtxt_Alice_adv) ; SYM[
    P1:Protocol <- idealPlusSim]{sugar-newNF ; foldNF('LeakCtxt_id_adv,
    'LeakCtxt_Alice_adv) ; foldNF('Ctxt, 'LeakCtxt_Alice_adv)} .

Solution 1
rewrites: 516 in 0ms cpu (1ms real) (~ rewrites/second)
result ProtocolConfig: pConfig(('xor : bool * bool ~> bool) 'flip : unit ~>>
    bool, ('In @ nil :: bool) 'LeakCtxt_Alice_adv @ nil :: bool, new 'Ctxt :
    bool in new 'LeakCtxt_id_adv : bool in ('Ctxt ::= nf('m : bool <- read
    'In, samp ('flip < () >), 'm :: emptyCNameList)) || ('LeakCtxt_Alice_adv
    ::= nf('c : bool <- read 'LeakCtxt_id_adv, return 'c, 'c ::
    emptyCNameList)) || 'LeakCtxt_id_adv ::= nf('c : bool <- read 'Ctxt,
    return 'c, 'c :: emptyCNameList), 'In @ nil, 'LeakCtxt_Alice_adv @ nil,
    empty)

\end{lstlisting}

\subsection{Approximate equality}

We start by defining wrappers for width and length
\begin{lstlisting}  
  sort Width .
  sort Length .
  
  op width_ : Nat -> Width [ctor] .
  op length_ : Nat -> Length [ctor] .
\end{lstlisting}  
\noindent and the measure functions  
\begin{lstlisting}
  op |_| : Protocol -> Nat .
  op |_| : Reaction -> Nat .
  op |_| : Expression -> Nat .
\end{lstlisting}

The configurations \code{aConfig(Sigma, Delta, P, I, O, A, w, l)}
for approximate equality
extend protocol configurations with fields for width and length.
The main idea is that we set these at 0 at the start of a proof,
and we keep track of their modifications with the help of the 
approximate equality rules.
The \textsc{strict} rule allows us to apply exact equality calculus
without modifying the width and the length
\begin{lstlisting}
 crl [STRICT] :
   aConfig(Sigma, Delta, P, I, O, A, w, l) 
   =>
   aConfig(Sigma, Delta, Q, I, O, A, w, l)
 if
  pConfig(Sigma, Delta, P, I, O, A)
  => 
  pConfig(Sigma, Delta, Q, I, O, A)
 . 
\end{lstlisting}
The \textsc{trans} rule
\begin{lstlisting}
crl [TRANS] :
  aConfig(Sigma, Delta, P, I, O, A, width nw, length nl)
  => 
  aConfig(Sigma, Delta, P2, I, O, A, width (nw + nw1 + nw2), length (nl + defMax(nl1, nl2)))
 if
  aConfig(Sigma, Delta, P, I, O, A, width 0, length 0)
  => 
  aConfig(Sigma, Delta, P1, I, O, A, width nw1, length nl1)
  /\ 
  aConfig(Sigma, Delta, P1, I, O, A, width 0, length 0)
  => 
  aConfig(Sigma, Delta, P2, I, O, A, width nw2, length nl2)
 . 
\end{lstlisting}
\noindent adds to the current value of width the values
computed in the sub-proofs and to the current value of length
the maximum of the two lengths computed in the subproofs.

Approximate equality axioms combine the rules \textsc{axiom} and
\textsc{input-unused}: if we rewrite a \code{aConfig} whose width is 
\code{w} and whose length is \code{l},
we get a \code{aConfig} whose width is \code{w + 1} and
whose length is \code{l + | I \ I'| } where \code{I} is the set of 
inputs of the configuration and \code{I'} is the set of inputs used 
by the protocol we want to rewrite with the axiom.
 
Strategies for applying an approximate equality axiom are of the form
\begin{lstlisting}
 strat S : Param @ ApproxEqConfig .  
 sd S(X) := 
    match aConf s.t. aConfStartsWithNew aConf
    ? CONG-NEW-NF-APPROX{S(X)} 
    : matchrew aConf s.t. 
       aConfig(Sigma, Delta, P, I, O, A, width w, length l) := aConf 
      by aConf 
      using
      CONG-COMP-APPROX{
         axiom(X)
      } 
    .
\end{lstlisting}

Proofs will typically consist of a composition, using \textsc{trans} rule,
of several \textsc{strict} steps with a number of applications of the axioms, using their corresponding strategies.

%%%\subsection{Case studies}