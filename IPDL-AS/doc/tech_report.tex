\documentclass[11pt,hidelinks]{article}

\usepackage{amsmath, amssymb, amsthm, stmaryrd}
\usepackage{url}
\usepackage{tikz}
\usepackage[override]{cmtt}
\usepackage{textcomp}
\usepackage{cryptocode}
\usepackage{mathtools}
\usepackage{listings}
\usepackage{xcolor}
\usepackage{graphics}
\usepackage{float}
\usepackage{xspace}
\usepackage{graphicx}
\usepackage{mathpartir}
\usepackage[margin=0.45in]{geometry}
\usepackage[colorlinks]{hyperref}
\usepackage{pl-syntax}
\usepackage{mathabx}
\usepackage{relsize}
\hypersetup{linkcolor=blue,filecolor=blue,citecolor=blue,urlcolor=blue}

\newcommand{\nat}{\mathbb{N}}
\newcommand{\ipdl}{\textsf{IPDL} }
\newcommand{\type}{\mathsf{t}}
\newcommand{\func}{\mathsf{f}}
\newcommand{\dist}{\mathsf{d}}
\newcommand{\one}{\mathsf{1}}
\newcommand{\Bool}{\mathsf{Bool}}
\newcommand{\true}{\mathsf{true}}
\newcommand{\false}{\mathsf{false}}
\newcommand{\fst}{\mathsf{fst}}
\newcommand{\snd}{\mathsf{snd}}
\newcommand{\ret}[1]{\mathsf{ret} \ #1}
\newcommand{\samp}[1]{\mathsf{samp} \ #1}
\renewcommand{\read}[1]{\mathsf{read} \ #1}
\newcommand{\ifte}[3]{\mathsf{if} \ #1 \ \mathsf{then} \ #2 \ \mathsf{else} \ #3}
\newcommand{\zero}{\mathsf{0}}
\newcommand{\assign}[2]{#1 \coloneqq #2}
\newcommand{\Par}[2]{#1 \; || \; #2}
\newcommand{\new}[3]{\mathsf{new} \ #1 : #2 \ \mathsf{in} \ #3}
\newcommand{\axiom}{\mathsf{axiom}}
\renewcommand{\approxeq}[6]{#1 \approx #2 : #3 \to #4 \ \mathsf{width} \ #5 \ \mathsf{length} \ #6}
\newcommand{\sem}[1]{\llbracket #1 \rrbracket}
\newcommand{\val}[1]{\mathsf{val} \ #1}
\newcommand{\outstep}[2]{\xmapsto{#1 \, \coloneqq \, #2}}
\newcommand{\stuck}{\mathsf{stuck}}
\newcommand{\final}{\mathsf{final}}
%\renewcommand{\norm}[1]{\Vert #1 \Vert}
%\renewcommand{\eval}[1]{#1 \! \Downarrow}
%\newcommand{\lift}[1]{\mathcal{L}(#1)}
%\newcommand{\pull}[2]{\stackrel{\hookleftarrow}{#1 \, \coloneqq \, #2}}
\newcommand{\newNf}{\mathsf{newNF}}
\newcommand{\preNf}{\mathsf{preNF}}
\newcommand{\nf}{\mathsf{NF}}

\newtheorem{lemma}{Lemma}
\newtheorem{theorem}{Theorem}						
\newtheorem{definition}{Definition}	
\newtheorem{corollary}{Corollary}	
\newtheorem{example}{Example}

% maude macros

\newcommand{\code}[1]{\texttt{#1}}
\lstset{
  basicstyle=\ttfamily,
  columns=fullflexible,
  literate={~} {$\sim$}{1}
}

\begin{document}
\title{A Core Calculus for Equational Proofs of Distributed Cryptographic Protocols: Technical Report}
\author{Kristina Sojakova \and Mihai Codescu \and Joshua Gancher}

\maketitle

\section*{\small Acknowledgement}
This project was funded through the NGI Assure Fund, a fund established by NLnet with financial support from the European Commission's Next Generation Internet programme, under the aegis of DG Communications Networks, Content and Technology under grant agreement No. 957073.

\section{Syntax of \ipdl}
\newcommand{\Sim}{\mathsf{Sim}}

\ipdl is built from three layers: \emph{protocols} are networks of
mutually interacting \emph{reactions}, which are simple monadic programs probabilistically computing an \emph{expression}. In the context of a protocol, a reaction operates on a unique \emph{channel} and may read from other channels, thereby utilizing computations coming from other reactions. The syntax and judgements of \ipdl are outlined in Figures \ref{fig:syntax}, \ref{fig:judgements}, respectively, and are parameterized by a user-defined \emph{signature} $\Sigma$:

\begin{definition}[Signature]
An \ipdl signature $\Sigma$ is a finite collection of:
\begin{itemize}
\item type constants $\type$;
\item function symbols $\func : \sigma \rightarrow \tau$; and
\item distribution symbols $\dist : \sigma \twoheadrightarrow \tau$.
\end{itemize}
\end{definition}

We have a minimal set of data types, including the unit type $\one$, Booleans, products, as well as arbitrary type symbols $\mathsf{t}$, drawn from the signature $\Sigma$. Expressions are used for non-probabilistic computations, and are standard. All values in \ipdl are bitstrings of a length given by data types, so we annotate the operations $\fst_{\tau \times \sigma}$ and $\snd_{\tau \times \sigma}$ with the type of the pair to determine the index to split the pair into two; for readability we omit this subscript whenever appropriate. All function symbols $\func$ and distribution symbols $\dist$ must be declared in the signature $\Sigma$. Substitutions $\theta : \Gamma_1 \to \Gamma_2$ between type contexts are standard.

As mentioned above, reactions are monadic programs which may return expressions, perform probabilistic sampling, read from channels, branch on a value of type $\Bool$, and sequentially compose. Protocols in \ipdl are given by a simple but expressive syntax: channel assignment $\assign{o}{R}$ assigns the reaction $R$ to channel $o$; parallel composition $\Par{P}{Q}$ allows $P$ and $Q$ to freely interact concurrently; and channel generation $\new{o}{\tau}{P}$ creates a new, internal channel for use in $P$. \emph{Embeddings} $\phi : \Delta_1 \to \Delta_2$ between channel contexts are injective, type-preserving mappings that specify how to rename channels in $\Delta_2$ to fit in the larger context $\Delta_1$.

Formally, references $\Var{x}{\tau}$ to variables and $\Read{c}{\tau}$ to channels include a typing annotation. This will come in handy later when we encode an \ipdl construct as a sequence of symbols on a Turing Machine tape; knowing the type $\tau$ will allow us to allocate the correct number of bits for the variable $x$ or the channel $c$. In almost all other instances, we simply write $x$ and $\read{c}$. Similarly, we often write $\func \ e$ instead of $\App{\func}{\sigma}{\tau}{e}$ and $\samp{\dist}{e}$ instead of $\Samp{\dist}{\sigma}{\tau}{e}$. For a constant $\func : \one \rightarrow \tau$, we write $\func$ in place of $\func \ \checkmark$, and for a constant $\dist : \one \twoheadrightarrow \tau$, we write $\dist$ instead of $\dist \ \checkmark$. We also frequently omit the type of the bound variable in a sequential composition. Finally, we might omit the typing subscript in expressions $\fst_{\sigma \times \tau}$ and $\snd_{\sigma \times \tau}$ if the types can be inferred from the context or are irrelevant.

%and write $x \leftarrow \read{c}; \ R$ and $x \leftarrow \samp{d}; \ R$ simply as $x \leftarrow c; \ R$ and $x \leftarrow d; \ R$ whenever appropriate.

%When binding the result of a reaction $R$ of type $\sigma_1 \times \sigma_2$, the syntactic sugar $(x_1,x_2) \leftarrow R; \ S(x_1,x_2)$ will stand for $x \leftarrow R; \ S(\fst \ x, \snd \ x)$. When binding the result of a reaction $R$ of type $\sigma_1 \times (\sigma_2 \times \sigma_3)$, the syntactic sugar $\big(x_1,(x_2,x_3)\big) \leftarrow R; \ S(x_1,x_2,x_3)$ will stand for $x \leftarrow R; \ S\big(\fst \ x, \fst \ (\snd \ x), \snd \ (\snd \ x)\big)$.

\begin{figure}[ht]
\begin{syntax}

  \abstractCategory[Variables]{x, y, z}+
  \abstractCategory[Channels]{i, o, c}
	
	\category[Channel Sets]{I, O}
    \alternative{\{c_1, \ldots, c_n\}}

  \category[Data Types]{\tau, \sigma}
    \alternative{\type}
		\alternative{\one}
    \alternative{\Bool}
    \alternative{\tau_1 \times \tau_2}

  \category[Expressions]{e}
    \alternative{\Var{x}{\tau}}
    \alternative{\checkmark}
	  \alternative{\true}
	  \alternative{\false}		
	  \alternative{\App{\func}{\sigma}{\tau}{e}} 
	  \alternative{(e_1,e_2)}    
	  \alternative{\fst_{\sigma \times \tau} \ e}
		\alternative{\snd_{\sigma \times \tau} \ e}		

  \category[Reactions]{R, S}
    \alternative{\ret{e}}
    \alternative{\Samp{\dist}{\sigma}{\tau}{e}}
    \alternative{\Read{c}{\tau}}
    \alternative{\ifte{e}{R_1}{R_2}}
    \alternative{x : \sigma \leftarrow R; \ S}         

	\category[Protocols]{P, Q}
	  \alternative{\zero}	
	  \alternative{\assign{o}{R}}
	  \alternative{\Par{P}{Q}}
	  \alternative{\new{o}{\tau}{P}}
		
  \category[Type Contexts]{\Gamma}
    \alternative{\cdot}
    \alternative{\Gamma, x : \tau}

  \category[Channel Contexts]{\Delta}
    \alternative{\cdot}
    \alternative{\Delta, c : \tau}
\end{syntax}
\caption{Syntax of \textsf{IPDL}.}
\label{fig:syntax}
\end{figure}

\begin{figure}[ht]
\begin{syntax}
  \abstractCategory[Expression Typing]{\Gamma \vdash e : \tau}
  \abstractCategory[Reaction Typing]{\Delta; \ \Gamma \vdash R : I \to \tau}
  \abstractCategory[Protocol Typing]{\Delta \vdash P : I \to O} \\

  \abstractCategory[Substitutions]{\theta : \Gamma_1 \to \Gamma_2}
  \abstractCategory[Embeddings]{\phi : \Delta_1 \to \Delta_2} \\

	\abstractCategory[Expression Equality]{\Gamma \vdash e_1 = e_2 : \tau}
  \abstractCategory[Reaction Equality]{\Delta; \ \Gamma \vdash R_1 = R_2 : I \to \tau}
  \abstractCategory[Protocol Equality (Strict)]{\Delta \vdash P_1 = P_2 : I \to O}
\end{syntax}
\caption{Judgements of the exact fragment of \textsf{IPDL}.}
\label{fig:judgements}
\end{figure}

\subsection{Typing}
We restrict our attention to well-typed \ipdl constructs. In addition to respecting data types, the typing judgments guarantee that all reads from channels in reactions are in scope, and that all channels are assigned at most one reaction in protocols. The typing $\Gamma \vdash e : \tau$ for expressions is standard, see Figure \ref{fig:expressions_typing}. Figure \ref{fig:reactions_typing} shows the typing rules for reactions. Intuitively, $\Delta; \ \Gamma \vdash R : I \to \tau$ holds when $R$ uses variables in $\Gamma$, reads from channels in $I$ typed according to $\Delta$, and returns a value of type $\tau$. Figure \ref{fig:protocols_typing} gives the typing rules for protocols: $\Delta \vdash P : I \to O$ holds when $P$ uses inputs in $I$ to assign reactions to the channels in $O$, all typed according to $\Delta$.

Channel assignment $\assign{o}{R}$ has the type $I \to \{o\}$ 
when $R$ is well-typed with an empty variable context, making use of inputs from $I$ as well as of $o$. We allow $R$ to read from its own output $o$ to express divergence: the protocol $\assign{o}{\read{o}}$ cannot reduce, which is useful for (conditionally) deactivating certain outputs. The typing rule for parallel composition $\Par{P}{Q}$ states that $P$ may use the outputs of $Q$ as inputs while defining its own outputs, and vice versa. Importantly, the typing rules ensure that the outputs of $P$ and $Q$ are disjoint so that each channel carries a unique reaction. Finally, the rule for channel generation allows a protocol to select a fresh channel name $o$, assign it a type $\tau$, and use it for internal computation and communication. Protocol typing plays a crucial role for modeling security. 
Simulation-based security in \ipdl is modeled by the existence of a \emph{simulator}  with an appropriate typing judgment, $\Delta \vdash \Sim : I \to O$. Restricting the behavior of $\Sim$ to only use inputs along $I$ is necessary to rule out trivial results (\emph{e.g.}, $\Sim$ simply copies a secret from the specification).

\begin{figure}
\begin{mathpar}
\fbox{$\Gamma \vdash e : \tau$}\\
\inferrule*{x : \tau \in \Gamma}{\Gamma \vdash \Var{x}{\tau} : \tau}\and
\inferrule*{ }{\Gamma \vdash \checkmark : \one}\and
\inferrule*{ }{\Gamma \vdash \true : \Bool}\and
\inferrule*{ }{\Gamma \vdash \false : \Bool}\and
\inferrule*{\func : \sigma \rightarrow \tau \in \Sigma \\ \Gamma \vdash e : \sigma}{\Gamma \vdash \App{\func}{\sigma}{\tau}{e} : \tau}\and
\inferrule*{\Gamma \vdash e_1 : \tau_1 \\ \Gamma \vdash e_2 : \tau_2}{\Gamma \vdash (e_1,e_2) : \tau_1 \times \tau_2}\and
\inferrule*{\Gamma \vdash e : \sigma \times \tau}{\Gamma \vdash \fst_{\sigma \times \tau} \ e : \sigma}\and
\inferrule*{\Gamma \vdash e : \sigma \times \tau}{\Gamma \vdash \snd_{\sigma \times \tau} \ e : \tau}
\end{mathpar}
\caption{Typing for \ipdl expressions.}
\label{fig:expressions_typing}
\end{figure}

\begin{figure*}
\begin{mathpar}
\fbox{$\Delta; \ \Gamma \vdash R : I \to \tau$}\\
\inferrule*{\Gamma \vdash e : \tau}{\Delta; \ \Gamma \vdash \ret{e} : I \to \tau}\and
\inferrule*{\dist : \sigma \twoheadrightarrow \tau \in \Sigma \\ \Gamma \vdash e : \sigma}{\Delta; \ \Gamma \vdash \Samp{\dist}{\sigma}{\tau}{e} : I \to \tau}\and
\inferrule*{i : \tau \in \Delta \\ i \in I}{\Delta; \ \Gamma \vdash \Read{i}{\tau} : I \to \tau}\and
\inferrule*{\Gamma \vdash e : \Bool \\ \Delta; \ \Gamma \vdash R_1 : I \to \tau \\ \Delta; \ \Gamma \vdash R_2 : I \to \tau}{\Delta; \
\Gamma \vdash \ifte{e}{R_1}{R_2} : I \to \tau}\and
\inferrule*{\Delta; \ \Gamma \vdash R : I \to \sigma \\ \Delta; \ \Gamma, x : \sigma \vdash S : I \to \tau}{\Delta; \ \Gamma \vdash (x : \sigma \leftarrow R; \ S) : I \to \tau}
\end{mathpar}
\caption{Typing for \ipdl reactions.}
\label{fig:reactions_typing}
\end{figure*}

\begin{figure*}
\begin{mathpar}
\fbox{$\Delta \vdash P : I \to O$}\\
\inferrule*{ }{\Delta \vdash \zero : I \to \emptyset}\and
\inferrule*{o \notin I \\ o : \tau \in \Delta \\ \Delta; \ \cdot \vdash R : I \cup \{o\} \to \tau}{\Delta \vdash \big(\assign{o}{R}\big) : I \to \{o\}}\and
\inferrule*{\Delta \vdash P : I \cup O_2 \to O_1 \\ \Delta \vdash    Q : I \cup O_1 \to O_2}{\Delta \vdash \Par{P}{Q} : I \to O_1 \cup O_2}\and
\inferrule*{\Delta, o : \tau \vdash P : I \to O \cup \{o\}}{\Delta \vdash \big(\new{o}{\tau}{P}\big) : I \to O}
\end{mathpar}
\caption{Typing for \ipdl protocols.}
\label{fig:protocols_typing}
\end{figure*}

\subsection{Equational Logic}
We now present the equational logic of \textsf{IPDL}. As mentioned above, the logic is divided into \emph{exact} rules that establish semantic equality between protocols, and \emph{approximate} rules that are used to discharge computational indistinguishability assumptions. 

\subsubsection{Exact Equality}
The majority of the reasoning in \ipdl is done using exact equalities. At the expression level, we assume an ambient finite set $\mathbb{T}_\mathsf{exp}$ of axioms of the form $\Gamma \vdash e_1 = e_2 : \tau$, where $\Gamma \vdash e_1 : \tau$ and $\Gamma \vdash e_2 : \tau$. The rules for the equality of expressions are standard, see Figure \ref{fig:expressions_equality}.

At the reaction level, we similarly assume an ambient finite set $\mathbb{T}_\mathsf{dist}$ of axioms of the form $\Gamma \vdash R_1 = R_2 : \tau$, where $\cdot \ ; \ \Gamma \vdash R_1 : \emptyset \to \tau$ and $\cdot \ ; \ \Gamma \vdash R_2 : \emptyset \to \tau$. The absence of any input channels means that the reactions $R_1$ and $R_2$ are in fact \emph{distributions}, represented using monadic syntax. We will therefore refer to axioms of this form as distribution-level axioms. The rules for reaction equality, shown in Figures \ref{fig:reactions_equality_1} and \ref{fig:reactions_equality_2}, ensure in particular that reactions form a \emph{commutative monad}: we have \[\big(x \leftarrow R_1; \ y \leftarrow R_2; \ S(x,y)\big) = \big(y \leftarrow R_2; \ x \leftarrow R_1; \ S(x,y)\big)\] whenever $R_2$ does not depend on $x$. All expected equalities for commutative monads hold for reactions, including the usual monad laws and congruence of equality under monadic bind. The \textsc{samp-pure} rule allows us to drop an unused sampling, and the \textsc{read-det} rule allows us to replace two reads from the same channel by a single one. The rules \textsc{if-left}, \textsc{if-right}, and \textsc{if-ext} allow us to manipulate conditionals.

At the protocol level, we again assume an ambient finite set $\mathbb{T}_\mathsf{prot}$ of axioms of the form $\Delta \vdash P_1 = P_2 : I \to O$, where $\Delta \vdash P_1 : I \to O$ and $\Delta \vdash P_2 : I \to O$. We use these axioms to specify user-defined functional assumptions, \emph{e.g.}, the correctness of decryption. Exact protocol equalities allow us to reason about communication between subprotocols and functional correctness, and to simplify intermediate computations. We will see later that exact equality implies the existence of a \emph{bisimulation} on protocols, which in turn implies perfect computational indistinguishability against an arbitrary distinguisher. The rules for the exact equality of protocols are in Figures~\ref{fig:protocols_equality_strict_1}, \ref{fig:protocols_equality_strict_2}; we now describe them informally.

The \textsc{comp-new} rule allows us to permute parallel composition and the creation of a new channel, and the same as \emph{scope
extrusion} in process calculi~\cite{picalc}. The \textsc{absorb-left} 
rule allows us to discard a component in a parallel composition if it has no outputs; this allows us to eliminate internal channels once they are no longer used. The \textsc{diverge} rule allows us to simplify diverging reactions: if a channel reads from itself and continues as an arbitrary reaction $R$, then we can safely discard $R$ as we will never reach it in the first place. The three (un)folding rules \textsc{fold-if-left}, \textsc{fold-if-right}, and \textsc{fold-bind} allow us to simplify composite reactions by bringing their
components into the protocol level as separate internal channels. The rule \textsc{subsume} states that channel dependency is transitive: if we depend on $o_1$, and $o_1$ in turn depends on $o_0$, then we also depend on $o_0$, and this dependency can be made explicit. The \textsc{subst} rule allows us to inline certain reactions into $\mathsf{read}$ commands. Inlining $\assign{o_1}{R_1}$ into $\assign{o_2}{x \leftarrow \read{o_1}; \ R_2}$ is sound provided $R_1$ is \emph{duplicable}: observing two independent results of evaluating $R_1$ is equivalent to observing the same result twice. This side condition is easily discharged whenever $R_1$ does not contain probabilistic
sampling. Finally, the \textsc{drop} rule allows dropping unused reads from channels in certain situations. Due to timing dependencies among channels, we only allow dropping reads from the channel $\assign{o_1}{R_1}$ in the context of $\assign{o_2}{\_ \leftarrow \read{o_1};\ R_2}$ when we have that $(\_ \leftarrow R_1; \ R_2) = R_2$. This side condition is met in particular whenever all reads present in $R_1$ are also present in $R_2$.

\begin{figure*}
\begin{mathpar}
\fbox{$\Gamma \vdash e_1 = e_2 : \tau$}\\
\inferrule*[right=refl]{\Gamma \vdash e : \tau}{\Gamma \vdash e = e : \tau}\and
\inferrule*[right=sym]{\Gamma \vdash e_1 = e_2 : \tau}{\Gamma \vdash e_2 = e_1 : \tau}\and
\inferrule*[right=trans]{\Gamma \vdash e_1 = e_2 : \tau \\ \Gamma \vdash e_2 = e_3 : \tau}{\Gamma \vdash e_1 = e_3 : \tau}\and
\inferrule*[right=axiom]{\Gamma \vdash e_1 = e_2 : \tau \ \axiom}{\Gamma \vdash e_1 = e_2 : \tau}\and
\inferrule*[right=subst]{\theta : \Gamma_1 \to \Gamma_2 \\ \Gamma_2 \vdash e_1 = e_2 : \tau}{\Gamma_1 \vdash \theta^\star(e_1) = \theta^\star(e_2) : \tau}\and
\inferrule*[right=app-cong]{\func : \sigma \rightarrow \tau \in \Sigma \\ \Gamma \vdash e = e' : \sigma}{\Gamma \vdash \App{\func}{\sigma}{\tau}{e} = \App{\func}{\sigma}{\tau}{e'} : \tau}\and
\inferrule*[right=pair-cong]{\Gamma \vdash e_1 = e_1' : \tau_1 \\ \Gamma \vdash e_2 = e_2' : \tau_2}{\Gamma \vdash (e_1,e_2) = (e_1',e_2') : \tau_1 \times \tau_2}\and
\inferrule*[right=fst-cong]{\Gamma \vdash e = e' : \sigma \times \tau}{\Gamma \vdash \fst_{\sigma \times \tau} \ e = \fst_{\sigma \times \tau} \ e' : \sigma}\and
\inferrule*[right=snd-cong]{\Gamma \vdash e = e' : \sigma \times \tau}{\Gamma \vdash \snd_{\sigma \times \tau} \ e = \snd_{\sigma \times \tau} \ e' : \tau}\and
\inferrule*[right=fst-pair]{\Gamma \vdash e_1 : \tau_1 \\ \Gamma \vdash e_2 : \tau_2}{\Gamma \vdash \fst_{\tau_1 \times \tau_2} \ (e_1, e_2) = e_1 : \tau_1}\and
\inferrule*[right=snd-pair]{\Gamma \vdash e_1 : \tau_1 \\ \Gamma \vdash e_2 : \tau_2}{\Gamma \vdash \snd_{\tau_1 \times \tau_2} \ (e_1, e_2) = e_2 : \tau_2}\and
\inferrule*[right=pair-ext]{\Gamma \vdash e : \sigma \times \tau}{\Gamma \vdash e = \big(\fst_{\sigma \times \tau} \ e, \ \snd_{\sigma \times \tau} \ e\big) : \sigma \times \tau}\and
\inferrule*[right=unit-ext]{\Gamma \vdash e : \one}{\Gamma \vdash e = \checkmark : \one}
\end{mathpar}
\caption{Equality for \ipdl expressions.}
\label{fig:expressions_equality}
\end{figure*}

\begin{figure*}
\begin{mathpar}
\fbox{$\Delta; \ \Gamma \vdash R_1 = R_2 : I \to \tau$}\\
\inferrule*[right=refl]{\Delta; \ \Gamma \vdash R : I \to \tau}{\Delta; \ \Gamma \vdash R = R : I \to \tau}\and
\inferrule*[right=sym]{\Delta; \ \Gamma \vdash R_1 = R_2 : I \to \tau}{\Delta; \ \Gamma \vdash R_2 = R_1 : I \to \tau}\and
\inferrule*[right=trans]{\Delta; \ \Gamma \vdash R_1 = R_2 : I \to \tau \\ \Delta; \ \Gamma \vdash R_2 = R_3 : I \to \tau}{\Delta; \ \Gamma \vdash R_1 = R_3 : I \to \tau}\and
\inferrule*[right=axiom]{\Gamma \vdash R_1 = R_2 : \tau \ \axiom}{\cdot \ ; \ \Gamma \vdash R_1 = R_2 : \emptyset \to \tau}\and
\inferrule*[right=input-unused]{i \notin I \\ \Delta; \ \Gamma \vdash R_1 = R_2 : I \to \tau}{\Delta; \ \Gamma \vdash R_1 = R_2 : I \cup \{i\} \to \tau}\and
\inferrule*[right=embed]{\phi : \Delta_1 \to \Delta_2 \\ \Delta_2 \vdash R_1 = R_2 : I \to \tau}{\Delta_1 \vdash \phi^\star(R_1) = \phi^\star(R_2) : \phi^\star(I) \to \tau}\and
\inferrule*[right=subst]{\theta : \Gamma_1 \to \Gamma_2 \\ \Delta; \ \Gamma_2 \vdash R_1 = R_2 : I \to \tau}{\Delta; \ \Gamma_1 \vdash \theta^\star(R_1) = \theta^\star(R_2) : I \to \tau}\and
\inferrule*[right=cong-ret]{\Gamma \vdash e = e' : \tau}{\Delta; \ \Gamma \vdash \ret{e} = \ret{e'} : I \to \tau}\and
\inferrule*[right=cong-samp]{\dist : \sigma \twoheadrightarrow \tau \in \Sigma \\ \Gamma \vdash e = e' : \sigma}{\Delta; \ \Gamma \vdash \Samp{\dist}{\sigma}{\tau}{e} = \Samp{\dist}{\sigma}{\tau}{e'} : I \to \tau}\and
\inferrule*[right=cong-if]{\Gamma \vdash e = e' : \Bool \\ \Delta; \ \Gamma \vdash R_1 = R_1' : I \to \tau \\ \Delta; \ \Gamma \vdash R_2 = R_2' : I \to \tau}{\Delta; \ \Gamma \vdash \big(\ifte{e}{R_1}{R_2}\big) = \big(\ifte{e'}{R_1'}{R_2'}\big) : I \to \tau}\and
\inferrule*[right=cong-bind]{\Delta; \ \Gamma \vdash R = R' : I \to \sigma \\ \Delta; \ \Gamma, x : \sigma \vdash S = S' : I \to \tau}{\Delta; \ \Gamma \vdash (x : \sigma \leftarrow R; \ S) = (x : \sigma \leftarrow R'; \ S') : I \to \tau}
\end{mathpar}
\caption{Equality for \ipdl reactions. Additional rules are given in
Figure~\ref{fig:reactions_equality_2}.}
\label{fig:reactions_equality_1}
\end{figure*}

\begin{figure*}
\begin{mathpar}
\fbox{$\Delta; \ \Gamma \vdash R_1 = R_2 : I \to \tau$}\\
\inferrule*[right=ret-bind]{\Gamma \vdash e : \sigma \\ \Delta; \ \Gamma, x : \sigma \vdash R : I \to \tau}{\Delta; \ \Gamma \vdash (x : \sigma \leftarrow \ret{e}; \ R) = R[\assign{x}{e}] : I \to \tau}\and
\inferrule*[right=bind-ret]{\Delta; \ \Gamma \vdash R : I \to \tau}{\Delta; \ \Gamma \vdash (x : \tau \leftarrow R; \ \ret{x}) = R : I \to \tau}\and
\inferrule*[right=bind-bind]{\Delta; \ \Gamma \vdash R_1 : I \to \sigma_1 \\ \Delta; \ \Gamma, x_1 : \sigma_1 \vdash R_2 : I \to \sigma_2 \\ \Delta; \ \Gamma, x_2 : \sigma_2 \vdash S : I \to \tau}{\Delta; \ \Gamma \vdash \big(x_2 : \sigma_2 \leftarrow (x_1 : \sigma_1 \leftarrow R_1; \ R_2); \ S\big) = \big(x_1 : \sigma_1 \leftarrow R_1; \ x_2 : \sigma_2 \leftarrow  R_2; \ S\big) : I \to \tau}\and
\inferrule*[right=exch]{\Delta; \ \Gamma \vdash R_1 : I \to \sigma_1 \\ \Delta; \ \Gamma \vdash R_2 : I \to \sigma_2 \\ \Delta; \ \Gamma, x_1 : \sigma_1, x_2 : \sigma_2 \vdash S : I \to \tau}{\Delta; \ \Gamma \vdash \big(x_1 : \sigma_1 \leftarrow R_1; \ x_2 : \sigma_2 \leftarrow R_2; \ S\big) = \big(x_2 : \sigma_2 \leftarrow R_2; \ x_1 : \sigma_1 \leftarrow R_1; \ S\big) : I \to \tau}\and
\inferrule*[right=samp-pure]{\dist : \rho \twoheadrightarrow \sigma \in \Sigma \\ \Gamma \vdash e : \rho \\ \Delta; \ \Gamma \vdash R : I \to \tau}{\Delta; \ \Gamma \vdash (x : \sigma \leftarrow \Samp{\dist}{\rho}{\sigma}{e}; \ R) = R : I \to \tau}\and
\inferrule*[right=read-det]{i : \sigma \in \Delta \\ i \in I \\ \Delta; \ \Gamma, x : \sigma, y : \sigma \vdash R : I \to \tau}{\Delta; \ \Gamma \vdash \big(x : \sigma \leftarrow \read{i}; \ y : \sigma \leftarrow \read{i}; \ R\big) = \big(x : \sigma \leftarrow \read{i}; \ R[\assign{y}{x}]\big) : I \to \tau}\and
\inferrule*[right=if-left]{\Delta; \ \Gamma \vdash R_1 : I \to \tau \\ \Delta; \ \Gamma \vdash R_2 : I \to \tau}{\Delta; \ \Gamma \vdash \big(\ifte{\true}{R_1}{R_2}\big) = R_1 : I \to \tau}\and
\inferrule*[right=if-right]{\Delta; \ \Gamma \vdash R_1 : I \to \tau \\ \Delta; \ \Gamma \vdash R_2 : I \to \tau}{\Delta; \ \Gamma \vdash \big(\ifte{\false}{R_1}{R_2}\big) = R_2 : I \to \tau}\and
\inferrule*[right=if-ext]{\Delta; \ \Gamma, x : \Bool \vdash R : I \to \tau \\ \Gamma \vdash e : \Bool}{\Delta; \ \Gamma \vdash \big(\ifte{e}{R[\assign{x}{\true}]}{R[\assign{x}{\false}]}\big) = R[\assign{x}{e}] : I \to \tau}
\end{mathpar}
\caption{Equality for \ipdl reactions.}
\label{fig:reactions_equality_2}
\end{figure*}

\begin{figure*}
\begin{mathpar}
\fbox{$\Delta \vdash P_1 = P_2 : I \to O$}\\
\inferrule*[right=refl]{\Delta \vdash P : I \to O}{\Delta \vdash P = P : I \to O}\and
\inferrule*[right=sym]{\Delta \vdash P_1 = P_2 : I \to O}{\Delta \vdash P_2 = P_1 : I \to O}\and
\inferrule*[right=trans]{\Delta \vdash P_1 = P_2 : I \to O \\ \Delta \vdash P_2 = P_3 : I \to O}{\Delta \vdash P_1 = P_3 : I \to O}\and
\inferrule*[right=axiom]{\Delta \vdash P_1 = P_2 : I \to O \ \axiom}{\Delta \vdash P_1 = P_2 : I \to O}\and
\inferrule*[right=input-unused]{i \notin I \cup O \\ \Delta \vdash P_1 = P_2 : I \to O}{\Delta \vdash P_1 = P_2 : I \cup \{i\} \to O}\and
\inferrule*[right=embed]{\phi : \Delta_1 \to \Delta_2 \\ \Delta_2 \vdash P_1 = P_2 : I \to O}{\Delta_1 \vdash \phi^\star(P_1) = \phi^\star(P_2) : \phi^\star(I) \to \phi^\star(O)}\and
\inferrule*[right=cong-react]{o \notin I \\ o : \tau \in \Delta \\ \Delta; \ \cdot \vdash R = R' : I \cup \{o\} \to \tau}{\Delta \vdash \big(\assign{o}{R}\big) = \big(\assign{o}{R'}\big) : I \to \{o\}}\and
\inferrule*[right=cong-comp-left]{\Delta \vdash P = P' : I \cup O_2 \to O_1 \\ \Delta \vdash Q : I \cup O_1 \to O_2}{\Delta \vdash \Par{P}{Q} = \Par{P'}{Q} : I \to O_1 \cup O_2}\and
\inferrule*[right=cong-new]{\Delta, o : \tau \vdash P = P' : I \to O \cup \{o\}}{\Delta \vdash \big(\new{o}{\tau}{P}\big) = \big(\new{o}{\tau}{P'}\big) : I \to O}\and
\inferrule*[right=comp-comm]{\Delta \vdash P_1 : I \cup O_2 \to O_1 \\ \Delta \vdash P_2 : I \cup O_1 \to O_2}{\Delta \vdash \Par{P_1}{P_2} = \Par{P_2}{P_1} : I \to O_1 \cup O_2}\and
\inferrule*[right=comp-assoc]{\Delta \vdash P_1 : I \cup O_2 \cup O_3 \to O_1 \\ \Delta \vdash P_2 : I \cup O_1 \cup O_3 \to O_2 \\ \Delta \vdash P_3 : I \cup O_1 \cup O_2 \to O_3}{\Delta \vdash \Par{(\Par{P_1}{P_2})}{P_3} = \Par{P_1}{(\Par{P_2}{P_3})} : I \to O_1 \cup O_2 \cup O_3}\and
\inferrule*[right=new-exch]{\Delta, o_1 : \tau_1, o_2 : \tau_2 \vdash P : I \to O \cup \{o_1,o_2\}}{\Delta \vdash \big(\new{o_1}{\tau_1}{\new{o_2}{\tau_2}{P}}\big) = \big(\new{o_2}{\tau_2}{\new{o_1}{\tau_1}{P}}\big) : I \to O}\and
\inferrule*[right=comp-new]{\Delta \vdash P : I \cup O_2 \to O_1 \\ \Delta, o : \tau \vdash Q : I \cup O_1 \to O_2 \cup \{o\}}{\Delta \vdash \Par{P}{\big(\new{o}{\tau}{Q}\big)} = \new{o}{\tau}{(\Par{P}{Q})} : I \to O_1 \cup O_2}\and
\inferrule*[right=absorb-left]{\Delta \vdash P : I \to O \\ \Delta \vdash Q : I \cup O \to \emptyset}{\Delta \vdash \Par{P}{Q} = P : I \to O}
\end{mathpar}
\caption{Exact equality for \ipdl protocols. Additional rules are given in Figure~\ref{fig:protocols_equality_strict_2}.}
\label{fig:protocols_equality_strict_1}
\end{figure*}

\begin{figure*}
\begin{mathpar}
\fbox{$\Delta \vdash P = Q : I \to O$}\\
\inferrule*[right=diverge]{o : \tau \in \Delta \\ \Delta; \ \cdot \vdash R : I \to \tau}{\Delta \vdash \big(\assign{o}{x \leftarrow \read{o}; \ R}\big) = \big(\assign{o}{\read{o}}\big) : I \, \setminus \, \{o\} \to \{o\}}\and
\inferrule*[right=fold-if-left]{o : \tau \in \Delta \\ \Delta; \ \cdot \vdash R : I \to \Bool \\ \Delta; \ \cdot \vdash S_1 : I \to \tau \\ \Delta; \ \cdot \vdash S_2 : I \to \tau}{\Delta \vdash \big(\new{l}{\tau}{\Par{\assign{o}{x \leftarrow R; \ \ifte{x}{{\color{red} \read{l}}}{S_2}}}{{\color{red} \assign{l}{S_1}}}}\big) = \\ \big(\assign{o}{x \leftarrow R; \ \ifte{x}{{\color{red} S_1}}{S_2}}\big) : I \, \setminus \, \{o\} \to \{o\}\hspace{-80pt}}\and
\inferrule*[right=fold-if-right]{o : \tau \in \Delta \\ \Delta; \ \cdot \vdash R : I \to \Bool \\ \Delta; \ \cdot \vdash S_1 : I \to \tau \\ \Delta; \ \cdot \vdash S_2 : I \to \tau}{\Delta \vdash \big(\new{r}{\tau}{\Par{\assign{o}{x \leftarrow R; \ \ifte{x}{S_1}{{\color{red} \read{r}}}}}{{\color{red} \assign{r}{S_2}}}}\big) = \\ \big(\assign{o}{x \leftarrow R; \ \ifte{x}{S_1}{{\color{red} S_2}}}\big) : I \, \setminus \, \{o\} \to \{o\}\hspace{-78pt}}\and
\inferrule*[right=fold-bind]{o : \tau \in \Delta \\ \Delta; \ \cdot \vdash R : I \to \sigma \\ \Delta; \ x : \sigma \vdash S : I \to \tau}{\Delta \vdash \big(\new{c}{\sigma}{\Par{\assign{o}{{\color{red} x \leftarrow \read{c};} \ S}}{{\color{red} \assign{c}{R}}}}\big) = \big(\assign{o}{{\color{red} x \leftarrow R; \ } S}\big) : I \, \setminus \, \{o\} \to \{o\}}\and
\inferrule*[right=subst]{o_1 \neq o_2 \\ o_1 : \tau_1, o_2 : \tau_2 \in \Delta \\ \Delta; \ \cdot \vdash R_1 : I \to \tau_1 \\ \Delta; \ x_1 : \tau_1 \vdash R_2 : I \to \tau_2 \\ \Delta; \ \cdot \vdash \big(x_1 \leftarrow R_1; \ {\color{red} x'_1 \leftarrow R_1; \ } \ret{(x_1,{\color{red} x'_1})}\big) = \big(x_1 \leftarrow R_1; \ \ret{(x_1,{\color{red} x_1})}\big) : I \to \tau_1 \times \tau_1}{\Delta \vdash \big(\Par{\assign{o_1}{R_1}}{\assign{o_2}{{\color{red} x_1 \leftarrow \read{o_1}; \ } R_2}}\big) = \big(\Par{\assign{o_1}{R_1}}{\assign{o_2}{{\color{red} x_1 \leftarrow R_1; \ } R_2}}\big) : I \, \setminus \, \{o_1,o_2\} \to \{o_1, o_2\}}\and
\inferrule*[right=drop]{\\ o_1 \neq o_2 \\ o_1 : \tau_1, o_2 : \tau_2 \in \Delta \\ \\ \Delta; \ \cdot \vdash R_1 : I \to \tau_1 \\ \Delta; \ \cdot \vdash R_2 : I \to \tau_2 \\ \Delta; \ \cdot \vdash \big({\color{red} x_1 \leftarrow R_1; \ } R_2\big) = R_2 : I \to \tau_2}{\Delta \vdash \big(\Par{\assign{o_1}{R_1}}{\assign{o_2}{{\color{red} x_1 \leftarrow \read{o_1}; \ } R_2}}\big) = \big(\Par{\assign{o_1}{R_1}}{\assign{o_2}{R_2}}\big) : I \, \setminus \, \{o_1,o_2\} \to \{o_1, o_2\}}
\end{mathpar}
\caption{Additional rules for exact equality of \ipdl protocols. Distinguishing changes of equalities are highlighed in {\color{red} red}.}
\label{fig:protocols_equality_strict_2}
\end{figure*}

\subsubsection{Approximate Equality}
The equational theory for the approximate fragment of \ipdl consists of two layers: one for the \emph{approximate equality} of protocols, and one for the \emph{asymptotic equality} of protocol families as functions of the security parameter $\lambda \in \nat$. Informally, if two protocol families are asymptotically equal, then any \emph{resource-bounded} adversary cannot distinguish them with greater than negligible error. Analogously to exact protocol equality, for approximate equality we assume an ambient finite set of \emph{approximate axioms} of the form $\Delta \vdash P \approx Q : I \to O$, where $\Delta \vdash P : I \to O$ and $\Delta \vdash Q : I \to O$. These axioms capture cryptographic assumptions on computational indistinguishability. The approximate equality of two such protocols has the form $\Delta \vdash \approxeq{P}{Q}{I}{O}{k}{l}$, and we think of this proof as corresponding to a specific security parameter $\lambda$. In the asymptotic equality judgement, both parameters $k, l$ become functions of the security parameter $\lambda$, and must be bounded by a polynomial in $\lambda$.

\begin{figure*}
\begin{mathpar}
\fbox{$\Delta \vdash \approxcong{P}{Q}{I}{O}{l}$}\\
\inferrule*[right=axiom]{\Delta \vdash P \approx Q : I \to O \ \axiom}{\Delta \vdash \approxcong{P}{Q}{I}{O}{0}}\and
\inferrule*[right=input-unused]{i \notin I \cup O \\ \Delta \vdash \approxcong{P}{Q}{I}{O}{l}}{\Delta \vdash \approxcong{P}{Q}{I \cup \{i\}}{O}{l}}\and
\inferrule*[right=embed]{\phi : \Delta_1 \to \Delta_2 \\ \Delta_2 \vdash \approxcong{P}{Q}{I}{O}{l}}{\Delta_1 \vdash \approxcong{\phi^\star(P)}{\phi^\star(Q)}{\phi^\star(I)}{\phi^\star(O)}{l}}\and
\inferrule*[right=cong-comp-left]{\Delta \vdash \approxcong{P}{P'}{I \cup O_2}{O_1}{l} \\ \Delta \vdash Q : I \cup O_1 \to O_2}{\Delta \vdash \approxcong{\Par{P}{Q}}{\Par{P'}{Q}}{I}{O_1 \cup O_2}{l + \tmnorm{Q} + 3}}\and
\inferrule*[right=cong-new]{\Delta, o : \tau \vdash \approxcong{P}{P'}{I}{O \cup \{o\}}{l}}{\Delta \vdash \approxcong{\big(\new{o}{\tau}{P}\big)}{\big(\new{o}{\tau}{P'}\big)}{I}{O}{l}}
\end{mathpar}
\caption{Approximate congruence of \ipdl protocols.}
\label{fig:protocols_congruence_approx}
\end{figure*}

\begin{figure*}
\begin{mathpar}
\fbox{$\Delta \vdash \approxeq{P}{Q}{I}{O}{k}{l}$}\\
\inferrule*[right=strict]{\Delta \vdash P = Q : I \to O}{\Delta \vdash \approxeq{P}{Q}{I}{O}{0}{0}}\and
\inferrule*[right=approx-cong]{\Delta \vdash \approxcong{P}{Q}{I}{O}{l}}{\Delta \vdash \approxeq{P}{Q}{I}{O}{1}{l}}\and
\inferrule*[right=sym]{\Delta \vdash \approxeq{P_1}{P_2}{I}{O}{k}{l}}{\Delta \vdash \approxeq{P_2}{P_1}{I}{O}{k}{l}}\and
\inferrule*[right=trans]{\Delta \vdash \approxeq{P_1}{P_2}{I}{O}{k_1}{l_1} \\ \Delta \vdash \approxeq{P_2}{P_3}{I}{O}{k_2}{l_2}}{\Delta \vdash \approxeq{P_1}{P_3}{I}{O}{k_1 + k_2}{\max(l_1, l_2)}}\and
\end{mathpar}
\caption{Approximate equality for \ipdl protocols.}
\label{fig:protocols_equality_approx}
\end{figure*}

Figure \ref{fig:protocols_equality_approx} shows the rules for the approximate equality of \ipdl protocols, where we chain together a sequence of strict equalities and \emph{approximate congruence} transformations, see Figure \ref{fig:protocols_congruence_approx}. The parameter $k \in \nat$ counts the number of axiom invocations. Applying a single approximate axiom incurs $k = 1$ (rule \textsc{approx-cong}, and the use of transitivity requires us to add up the respective values of $k$ (rule \textsc{trans}). Even though each individual axiom invocation introduces a negligible error, the sum of exponentially many negligible errors might not be negligible, which is why we later impose a polynomial bound on $k$.

The parameter $l$ tracks the increase in adversarial resources incurred by the proof. The bulk of the reasoning in \ipdl is done in the exact fragment, where a typical proof step transforms the protocol into a form where an approximate axiom applies. We subsequently carry out an approximate congruence step, where we use the approximate axiom to replace a small protocol fragment nested inside a larger program context by its computationally indistinguishable counterpart. The program context is formally a part of the adversary, and as such it must be resource-bounded for the indistinguishability assumption to apply. Some nesting patterns do not effect any change on the adversary's resources: for example, a simple renaming of channels (rule \textsc{embed}); the formal addition of an unused channel $i$ to the protocol's inputs $I$ (rule \textsc{input-unused}), in which case any value assigned by the adversary to channel $i$ will leave the protocol unchanged; or the introduction of an internal channel $o : \tau$ (rule \textsc{cong-new}), in which case the adversary will never query $o$ because internal channels are only visible in the scope of their declaration.

On the other hand, composing two approximately equal protocols $P \approx P'$ with another protocol $Q$ requires the adversary to simulate the interaction of the common protocol $Q$ with $P$ versus $P'$. In other words, the adversary \emph{absorbs} $Q$ and the protocol becomes part of the new adversary's code. In particular, the number of symbols needed to encode the adversary's code on a Turing Machine tape increases, and the parameter $l$ measures this increase. As we can see in rule \textsc{cong-comp-left}, we use $\tmnorm{Q} + 3$ additional symbols: $\tmnorm{Q}$ symbols for encoding the protocol $Q$; a parallel composition symbol to combine the original adversary code with the protocol $Q$; and two parenthesis symbols \textsf{``(''}, \textsf{``)''} for enclosing the composition. We emphasize that the exact numbers here are not crucial; what matters is that we eventually deliver a polynomial in $\lambda$. \smallskip

List the type constants declared in the signature $\Sigma$ as $\type_1,\ldots,\type_{|\Sigma_\type|}$. Unlike the parameter $k \in \nat$, the parameter $l$ is not a natural number but a function $l(t_1,\ldots,t_{|\Sigma_\type|}) : \nat^{|\Sigma_\type|} \to \nat$ that is \emph{monotonically increasing in each argument}. When encoding a protocol $Q$ as a sequence of symbols on a Turing Machine tape, we invariably encounter type annotations such as $\Var{x}{\tau}$. At this point, we do not know how many bits we will need to encode values of type $\tau$, because the type constants $\type \in \Sigma$ are as of yet uninterpreted. Instead, we leave the size of each type constant as a variable to the function $l$, which will later be instantiated by the appropriate natural number according to $\int{-}$. With this proviso, the Turing Machine bound of a type $\tau$ is straightforward:
\begin{align*}
\tmnorm{\type_i} & \coloneqq t_i \\
\tmnorm{\one} & \coloneqq 0 \\
\tmnorm{\Bool} & \coloneqq 1 \\
\tmnorm{\tau_1 \times \tau_2} & \coloneqq \tmnorm{\tau_1} + \tmnorm{\tau_2}
\end{align*}
For variables $\Var{x}{\tau}$, we use the symbols \textsf{``(''}, \textsf{``var''}, \textsf{``:''}, \textsf{``)''} in addition to the de Bruijn index of the variable $x$, encoded as a single symbol, and the encoding of the type annotation $\tau$. For expressions $\checkmark$, $\true$, $\false$, we use the corresponding symbols \textsf{``$\checkmark$''}, \textsf{``true''}, \textsf{``false''} and the two parenthesis symbols \textsf{``(''}, \textsf{``)''}. For an application $\App{\func}{\sigma}{\tau}{e}$, we use the symbols \textsf{``(''}, \textsf{``app''}, \textsf{``$\to$''}, \textsf{``)''} in addition to the function symbol $\func$, encoded as a single symbol, and the encodings of the two type annotations $\sigma, \tau$ and the expression $e$. To encode a pair $(e_1, e_2)$, we will only need the encodings of the two expressions $e_1$ and $e_2$. Finally, to encode first and second projections, we will use the symbols \textsf{``(''}, \textsf{``fst''} or \textsf{``snd''}, \textsf{``$\times$''}, \textsf{)''} in addition to the encodings of the two type annotations $\sigma, \tau$ and the expression $e$.
\begin{align*}
\tmnorm{\Var{x}{\tau}} & \coloneqq \tmnorm{\tau} + 5 \\
\tmnorm{\checkmark} & \coloneqq 3 \\
\tmnorm{\true} & \coloneqq 3 \\
\tmnorm{\false} & \coloneqq 3 \\
\tmnorm{\App{\func}{\sigma}{\tau}{e}} & \coloneqq \tmnorm{\sigma} + \tmnorm{\tau} + \tmnorm{e} + 5 \\
\tmnorm{(e_1, e_2)} & \coloneqq \tmnorm{e_1} + \tmnorm{e_2} \\
\tmnorm{\fst_{\sigma \times \tau} \ e} & \coloneqq \tmnorm{\sigma} + \tmnorm{\tau} + \tmnorm{e} + 5 \\
\tmnorm{\snd_{\sigma \times \tau} \ e} & \coloneqq \tmnorm{\sigma} + \tmnorm{\tau} + \tmnorm{e} + 5
\end{align*}
For a return $\ret{e}$, we use the symbols \textsf{``(''}, \textsf{``ret''}, \textsf{)''} in addition to the encoding of the expression $e$. For a sampling $\Samp{\dist}{\sigma}{\tau}{e}$, we use the symbols \textsf{``(''}, \textsf{``samp''}, \textsf{``$\twoheadrightarrow$''}, \textsf{``)''} in addition to the distribution symbol $\dist$, encoded as a single symbol, and the encodings of the two type annotations $\sigma, \tau$ and the expression $e$. For a read $\Read{c}{\tau}$, we use the symbols \textsf{``(''}, \textsf{``read''}, \textsf{``:''}, \textsf{``)''} in addition to the de Bruijn index of the channel $c$, encoded as a single symbol, and the encoding of the type annotation $\tau$. Furthermore, we will need one extra symbol: one of \textsf{``input-to-query''}, \textsf{``input-queried''}, \textsf{``input-not-to-query''}. When encoding a protocol $Q : I \cup O_1 \to O_2$ coming from the \textsc{comp-cong-left} rule, we use \textsf{``input-to-query''} or \textsf{``input-queried''} if we are reading from a channel $o_1 \in O_1$, according to whether we have already queried the channel $o_1$, and \textsf{``input-not-to-query''} otherwise. For a conditional $\ifte{e}{R_1}{R_2}$, we use the symbols \textsf{``(''}, \textsf{``if''}, \textsf{``then''}, \textsf{``else''}, \textsf{``)''} in addition to the encodings of the expression $e$ and the two reactions $R_1, R_2$. Finally, to encode a bind, we use the symbols \textsf{``$\{$''}, \textsf{``$\_$''}, \textsf{``:''}, \textsf{``$\leftarrow$''}, \textsf{``;''}, \textsf{``$\}$''} in addition to the encodings of the type annotation $\sigma$ and the two reactions $R$ and $S$. The symbol \textsf{``$\_$''} is used in lieu of the bound variable name $x$ and stands for de Bruijn index $0$.
\begin{align*}
\tmnorm{\ret{e}} & \coloneqq \tmnorm{e} + 3 \\
\tmnorm{\Samp{\dist}{\sigma}{\tau}{e}} & \coloneqq \tmnorm{\sigma} + \tmnorm{\tau} + \tmnorm{e} + 5 \\
\tmnorm{\Read{c}{\tau}} & \coloneqq \tmnorm{\tau} + 6 \\
\tmnorm{\ifte{e}{R_1}{R_2}} & \coloneqq \tmnorm{e} + \tmnorm{R_1} + \tmnorm{R_2} + 5 \\
\tmnorm{x : \sigma \leftarrow R; \ S} & \coloneqq \tmnorm{\sigma} + \tmnorm{R} + \tmnorm{S} + 6
\end{align*}
To encode the zero protocol $\zero$, we use the single symbol \textsf{``0''}. For an assignment $\assign{o}{R}$, we use the symbols \textsf{``[''}, \textsf{``$\coloneqq$''}, \textsf{``react''}, \textsf{``]''} in addition to the de Bruijn index of the channel $c$, encoded as a single symbol, and the encoding of the reaction $R$. For a parallel composition $\Par{P}{Q}$, we use the symbols \textsf{``(''}, \textsf{``$\|$''}, \textsf{``)''} in addition to the encodings of the two protocols $P$ and $Q$. Finally, for the declaration of a new channel $\new{o}{\tau}{P}$, we use the symbols \textsf{``new''}, \textsf{``$\_$''}, \textsf{``:''}, \textsf{``in''}, \textsf{``wen''} in addition to the encodings of the typing annotation $\tau$ and the protocol $P$. The symbol \textsf{``$\_$''} is used in lieu of the bound channel name $c$ and stands for de Bruijn index $0$. The symbol \textsf{``wen''} indicates the end of the binding scope.
\begin{align*}
\tmnorm{\zero} & \coloneqq 1 \\
\tmnorm{\assign{o}{R}} & \coloneqq \tmnorm{R} + 5 \\
\tmnorm{\Par{P}{Q}} & \coloneqq \tmnorm{P} + \tmnorm{Q} + 3 \\
\tmnorm{\new{c}{\tau}{P}} & \coloneqq \tmnorm{\tau} + \tmnorm{P} + 5
\end{align*}
We note that the Turing Machine bound of each construct is invariant under embeddings. \medskip

\noindent To make the ambient approximate theory with axioms $\Delta^1 \vdash P^1 \approx Q^1 : I^1 \to O^1, \ldots, \Delta^n \vdash P^n \approx Q^n : I^n \to O^n$ explicit, we write the approximate equality judgement as
\[\Delta^1 \vdash P^1 \approx Q^1 : I^1 \to O^1, \ldots, \Delta^n \vdash P^n \approx Q^n : I^n \to O^n \; \mathlarger{\mathlarger{\Rightarrow}} \; \Delta \vdash \approxeq{P}{Q}{I}{O}{k}{l}.\]
We also recall that the exact fragment of \ipdl is formulated with respect to ambient theories $\mathbb{T}_\mathsf{exp}$, $\mathbb{T}_\mathsf{dist}$, and $\mathbb{T}_\mathsf{prot}$ for expressions, distributions, and protocols. If we want to make these explicit, we combine them into a single exact \ipdl theory $\mathbb{T}_= \coloneqq (\mathbb{T}_\mathsf{exp}, \mathbb{T}_\mathsf{dist}, \mathbb{T}_\mathsf{prot})$, and write the approximate equality judgement as
\[\mathbb{T}_=; \, \Delta^1 \vdash P^1 \approx Q^1 : I^1 \to O^1, \ldots, \Delta^n \vdash P^n \approx Q^n : I^n \to O^n \; \mathlarger{\mathlarger{\Rightarrow}} \; \Delta \vdash \approxeq{P}{Q}{I}{O}{k}{l}.\]

\noindent For the asymptotic equality of \ipdl protocols, we assume a finite set $\mathbb{T}_\approx$ of \emph{approximate axiom families} of the form $\big\{\Delta_\lambda \vdash P_\lambda \approx Q_\lambda : I_\lambda \to O_\lambda\big\}_{\lambda \in \nat}$, where $\big\{\Delta_\lambda \vdash P_\lambda : I_\lambda \to O_\lambda\big\}_{\lambda \in \nat}$ and $\big\{\Delta_\lambda \vdash Q_\lambda : I_\lambda \to O_\lambda\big\}_{\lambda \in \nat}$ are two protocol families with pointwise-identical typing judgements. The asymptotic equality of two such protocol families has the form $\mathbb{T}_\approx \, \mathlarger{\mathlarger{\Rightarrow}} \, \big\{\Delta_\lambda \vdash P_\lambda \approx Q_\lambda : I_\lambda \to O_\lambda\big\}_{\lambda \in \nat}$, see Figure \ref{fig:protocols_equality_asympto}, where the left-hand side of $\Rightarrow$ lists the approximate axiom families comprising the asymptotic \ipdl theory $\mathbb{T}_\approx$.

Specifically, for any fixed $\lambda$ we obtain an approximate theory by selecting from each axiom family the particular axiom corresponding to $\lambda$. Similarly, from each of the two protocol families we select the protocol corresponding to $\lambda$, which gives us two concrete protocols to equate approximately. We recall that an approximate equality judgement is tagged by a pair of parameters $k \in \nat$ and $l(t_1,\ldots,t_{|\Sigma_\type|}) : \nat^{|\Sigma_\type|} \to \nat$, where $|\Sigma_\type|$ is the number of type constants declared in our ambient signature $\Sigma$. Letting $\lambda \in \nat$ vary thus gives us two functions $k_\lambda : \nat \to \nat$ and $l_\lambda : \nat^{|\Sigma_\type|+1} \to \nat$, and we require that these be bounded by polynomials in the appropriate number of variables.

\begin{figure*}
\begin{mathpar}
\fbox{$\big\{\Delta^1_\lambda \vdash P^1_\lambda \approx Q^1_\lambda : I^1_\lambda \to O^1_\lambda\big\}_{\lambda \in \nat}, \ldots, \big\{\Delta^n_\lambda \vdash P^n_\lambda \approx Q^n_\lambda : I^n_\lambda \to O^n_\lambda\big\}_{\lambda \in \nat} \; \mathlarger{\mathlarger{\mathlarger{\Rightarrow}}} \; \big\{\Delta_\lambda \vdash P_\lambda \approx Q_\lambda : I_\lambda \to O_\lambda\big\}_{\lambda \in \nat}$}\\
\inferrule{\forall \lambda, \Delta^1_\lambda \vdash P^1_\lambda \approx
    Q^1_\lambda : I^1_\lambda \to O^1_\lambda, \ldots, \Delta^n_\lambda \vdash
    P^n_\lambda \approx Q^n_\lambda : I^n_\lambda \to O^n_\lambda \; \mathlarger{\mathlarger{\Rightarrow}} \; \Delta_\lambda \vdash
    \approxeq{P_\lambda}{Q_\lambda}{I_\lambda}{O_\lambda}{k_\lambda}{l_\lambda}
    \\ k_\lambda = \mathsf{O}(\poly(\lambda)) \\ l_\lambda = \mathsf{O}\big(\poly(\lambda,t_1,\ldots,t_{|\Sigma_\type|})\big)}{\big\{\Delta^1_\lambda \vdash P^1_\lambda \approx Q^1_\lambda : I^1_\lambda \to O^1_\lambda\big\}_{\lambda \in \nat}, \ldots, \big\{\Delta^n_\lambda \vdash P^n_\lambda \approx Q^n_\lambda : I^n_\lambda \to O^n_\lambda\big\}_{\lambda \in \nat} \; \mathlarger{\mathlarger{\Rightarrow}} \; \big\{\Delta_\lambda \vdash P_\lambda \approx Q_\lambda : I_\lambda \to O_\lambda\big\}_{\lambda \in \nat}}
\end{mathpar}
\caption{Asymptotic equality for \ipdl protocol families.}
\label{fig:protocols_equality_asympto}
\end{figure*}

Whenever we want to make the underlying exact theory explicit, we write the asymptotic equality judgement as $\mathbb{T}_=; \, \mathbb{T}_\approx \; \mathlarger{\mathlarger{\Rightarrow}} \; \big\{\Delta_\lambda \vdash P_\lambda \approx Q_\lambda : I_\lambda \to O_\lambda\big\}_{\lambda \in \nat}$.

%\section{Operational Semantics of \ipdl}
%In this section we define an operational semantics for expressions, reactions, and protocols. This semantics will validate the \emph{exact} fragment of our equational logic and prove perfect observational equivalence. To give semantics to user-defined symbols, we define interpretations:

\begin{definition}[Interpretation]
An interpretation $\sem{-}$ for a signature $\Sigma$ associates:
\begin{itemize}
\item to each type symbol $\type$ a bitstring length $\sem{\type} \in \nat$;
\item to each function symbol $\func : \sigma \to \tau$ a function $\sem{\func}$ from bitstrings $\{0,1\}^{\sem{\sigma}}$ to bitstrings $\{0,1\}^{\sem{\tau}}$;
\item to each distribution symbol $\dist : \sigma \to \tau$ a function $\sem{\dist}$ from bitstrings $\{0,1\}^{\sem{\sigma}}$ to \emph{distributions} on bitstrings $\{0,1\}^{\sem{\tau}}$. 
\end{itemize}
\end{definition}

\noindent In the above, we naturally lift the interpretation $\sem{-}$ to data types by setting
\begin{align*}
\sem{\one} & \coloneqq 0 \\
\sem{\Bool} & \coloneqq 1 \\
\sem{\tau \times \sigma} & \coloneqq \sem{\tau} + \sem{\sigma}
\end{align*}

\noindent To handle partial computations, we augment the syntax of expressions, reactions, and protocols to contain intermediate bitstring values $v \in \{0,1\}^\star$:\smallskip

\begin{syntax}
\category[Valued Expressions]{e}
\alternative{{\color{red} v}} \alternative{\dots}
\category[Valued Reactions]{R, S}
\alternative{{\color{red} \val{v}}} \alternative{\dots}
\category[Valued Protocols]{P}
\alternative{{\color{red} \assign{o}{v}}} \alternative{\dots}
\end{syntax}\smallskip

\noindent Given an ambient interpretation $\sem{-}$ for the signature $\Sigma$, we can type the valued counterpart of \ipdl constructs as expected: in addition to the regular typing rules, we have

\begin{mathpar}
\inferrule*{v \in \{0,1\}^{\sem{\tau}}}{\Gamma \vdash v : \tau}\and
\inferrule*{v \in \{0,1\}^{\sem{\tau}}}{\Delta; \ \Gamma \vdash \val{v} : I \to \tau}\and
\inferrule*{o : \tau \in \Delta \\ o \notin I \\ v \in \{0,1\}^{\sem{\tau}}}{\Delta \vdash (\assign{o}{v}) : I \to \{o\}}
\end{mathpar}

\noindent The big-step semantics $e \Downarrow v$ for expressions is straightforward -- see Figure \ref{fig:expressions_semantics}, where we denote the empty bitstring by $()$ and use $v_1 v_2$ for bitstring concatenation. Pairing is given by the aforementioned bitstring concatenation (rule \textsc{pair}), and the projections $\fst_{\sigma \times \tau}$ and $\snd_{\sigma \times \tau}$ unambiguously split the pair according to $\sem{\sigma}$ and $\sem{\tau}$, respectively (rules \textsc{fst} and \textsc{snd}).

\begin{lemma}[Determinism of $\Downarrow$ for expressions]
For any well-typed expression $\Gamma \vdash e : \tau$ there exists a unique value $v$ such that $e \Downarrow v$, and $v \in \{0,1\}^{\sem{\tau}}$.
\end{lemma}

Reactions have a straightforward small-step semantics of the form $R \to \eta$, where $\eta$ is a probability distribution over reactions. Figure \ref{fig:reactions_semantics} shows the rules, where we write $1[R]$ for the distribution with unit mass at the reaction $R$, and freely use a distribution in place of a value (rule \textsc{samp}) or a reaction (rule \textsc{bind-react}) to indicate the obvious lifting of the corresponding construct to distributions on reactions. All distributions are implicitly finitely supported. Crucially, there is no semantic rule for stepping the reaction $\read{c}$ -- we model communication via semantics for protocols, which substitute all instances of $\mathsf{read}$ for values. 

We give semantics to protocols via two main small-step rules, see Figure \ref{fig:protocols_semantics}, where we analogously write $1[P]$ for the distribution with unit mass at the protocol $P$, and freely use a distribution in place of a reaction (in rule \textsc{step-react}) or a protocol (rules \textsc{step-comp-left}, \textsc{step-comp-right}, and \textsc{step-new}) to indicate the obvious lifting of the corresponding construct to distributions on protocols.

First we have the \emph{output} relation $P \outstep{o}{v} Q$, which is enabled when the reaction for channel $o$ in $P$ terminates, resulting in value $v$ (rule \textsc{out-val}). When this happens, the value of $o$ is broadcast through the protocol context enveloping $P$ (rules \textsc{out-comp-left}, \textsc{out-comp-left}, and \textsc{out-new}), resulting in each $\read{o}$ command in other reactions to be substituted with $\val{v}$. Note that the value of $o$ is not broadcast above the $\mathsf{new}$ quantifier when the local channel introduced is equal to $o$.

Next we have the \emph{internal stepping} relation $P \to \eta$, specified similarly to the small-step relation for reactions. The rule \textsc{step-react} lifts the stepping relation for $R$ to the stepping relation for $\assign{o}{R}$, while the three rules \textsc{step-comp-left}, \textsc{step-comp-right}, \textsc{step-new} simply propagate the stepping relation through parallel composition and the $\mathsf{new}$ quantifier. The last rule \textsc{out-new-hide} links the output relation with the stepping relation: whenever $P$ steps to $P'$, resulting in the output $\assign{o}{v}$, we have that $\new{o}{\tau}{P}$ steps with unit mass to $\new{o}{\tau}{P'}$.

\begin{figure}
\begin{mathpar}
\fbox{$e \Downarrow v$}\\
\inferrule*[right=val]{ }{v \Downarrow v}\and
\inferrule*[right=check]{ }{\checkmark \Downarrow ()}\and
\inferrule*[right=true]{ }{\true \Downarrow 1}\and
\inferrule*[right=false]{ }{\false \Downarrow 0}\and
\inferrule*[right=app]{e \Downarrow v}{(\func \ e) \Downarrow \sem{\func}(v)}\and
\inferrule*[right=pair]{e_1 \Downarrow v_1 \\ e_2 \Downarrow v_2}{(e_1,e_2) \Downarrow v_1 v_2}\and
\inferrule*[right=fst]{e \Downarrow v_1 v_2 \\ v_1 \in \{0,1\}^{\sem{\sigma}}}{(\fst_{\sigma \times \tau} \ e) \Downarrow v_1}\and
\inferrule*[right=snd]{e \Downarrow v_1 v_2 \\ v_2 \in \{0,1\}^{\sem{\tau}}}{(\snd_{\sigma \times \tau}) \ e \Downarrow v_2}
\end{mathpar}
\caption{Big-step operational semantics for $\ipdl$ expressions.}
\label{fig:expressions_semantics}
\end{figure}

\begin{figure}
\begin{mathpar}
\fbox{$R \to \eta$}\\
\inferrule*[right=ret]{e \Downarrow v}{\ret{e} \to 1[\val{v}]}\and
\inferrule*[right=samp]{e \Downarrow v}{\samp{(\dist \ e)} \to \val{\sem{\dist}(v)}}\and
\inferrule*[right=if-true]{e \Downarrow 1}{(\ifte{e}{R_1}{R_2}) \to 1[R_1]}\and
\inferrule*[right=if-false]{e \Downarrow 0}{(\ifte{e}{R_1}{R_2}) \to 1[R_2]}\and
\inferrule*[right=bind-val]{ }{(x : \sigma \leftarrow \val{v}; \ S) \to 1\big[S[\assign{x}{v}]\big]}\and
\inferrule*[right=bind-react]{R \to \eta}{(x : \sigma \leftarrow R; \ S) \to {(x : \sigma \leftarrow \eta; \ S)}}
\end{mathpar}
\caption{Small-step operational semantics for $\ipdl$ reactions.} 
\label{fig:reactions_semantics}
\end{figure}

\begin{figure}
\begin{mathpar}
\fbox{$P \outstep{o}{v} Q$}\\
\inferrule*[right=out-react]{ }{(\assign{o}{\val{v}}) \outstep{o}{v} (\assign{o}{v})}\and
\inferrule*[right=out-comp-left]{P \outstep{o}{v} P'}{(\Par{P}{Q}) \outstep{o}{v} \big(\Par{P'}{Q[\assign{\read{o}}{\val{v}}]}\big)}\and
\inferrule*[right=out-comp-right]{Q \outstep{o}{v} Q'}{(\Par{P}{Q}) \outstep{o}{v} \big(\Par{P[\assign{\read{o}}{\val{v}}]}{Q'}\big)}\and
\inferrule*[right=out-new]{P \outstep{o}{v} P' \\ o \neq c}{(\new{c}{\tau}{P}) \outstep{o}{v} (\new{c}{\tau}{P'})}\\\\
\fbox{$P \to \eta$}\\
\inferrule*[right=step-react]{R \to \eta}{(\assign{o}{R}) \to (\assign{o}{\eta})}\and
\inferrule*[right=step-comp-left]{P \to \eta}{(\Par{P}{Q}) \to (\Par{\eta}{Q})}\and
\inferrule*[right=step-comp-right]{Q \to \eta}{(\Par{P}{Q}) \to (\Par{P}{\eta})}\and
\inferrule*[right=step-new]{P \to \eta}{(\new{c}{\tau}{P}) \to (\new{c}{\tau}{\eta})}\and
\inferrule*[right=out-new-hide]{P \outstep{c}{v} P'}{(\new{c}{\tau}{P}) \to 1[\new{c}{\tau}{P'}]}
\end{mathpar}
\caption{Small-step operational semantics for \ipdl protocols.}
\label{fig:protocols_semantics}
\end{figure}

The big-step operational semantics for reactions $R \Downarrow \eta$, see Figure \ref{fig:reactions_big_step}, performs as many steps as possible in an attempt to compute $R$, resulting in a distribution $\eta$ on reactions. A reaction that cannot step any further is \emph{final}. We can syntactically describe final reactions as those that have either yielded a final value or have an unresolved read in the leading position (\emph{i.e.}, are \emph{stuck}).

Similarly, the big-step operational semantics for protocols $P \Downarrow \eta$, see Figure \ref{fig:protocols_big_step}, performs as many output and internal steps as possible in an attempt to compute $P$, resulting in a distribution $\eta$ on protocols. Analogously to reactions, a protocol that cannot step any further is \emph{final}. We can syntactically describe final protocols as those where every channel, including the internal ones, carries either a final value or a reaction that is stuck.

\begin{figure}
\begin{mathpar}
\fbox{$R \ \stuck$}\\
\inferrule*{ }{(\read{c}) \ \stuck}\and
\inferrule*{R \ \stuck}{(x : \tau \leftarrow R; \ S) \ \stuck}\\
\fbox{$R \ \final$}\\
\inferrule*{ }{(\val{v}) \ \final}\and
\inferrule*{R \ \stuck}{R \ \final}\\
\fbox{$R \Downarrow \eta$}\\
\inferrule*{}{ }\and
\inferrule*{R \to \sum_i c_i \, 1[R_i] \\ R_i \Downarrow \eta_i}{R \Downarrow \sum_i c_i \, \eta_i}\and \\
\inferrule*{R \ \final}{R \Downarrow 1[R]}
\end{mathpar}
\caption{Big-step operational semantics for \ipdl reactions.}
\label{fig:reactions_big_step}
\end{figure}

\begin{figure}
\begin{mathpar}
\fbox{$P \ \final$}\\
\inferrule*{ }{\zero \ \final}\and
\inferrule*{ }{(\assign{o}{v}) \ \final}\and
\inferrule*{R \ \stuck}{(\assign{o}{R}) \ \final}\and
\inferrule*{P \ \final \\ Q \ \final}{(\Par{P}{Q}) \ \final}\and
\inferrule*{P \ \final}{(\new{o}{\tau}{P}) \ \final}\\\\
\fbox{$P \Downarrow \eta$}\\
\inferrule*{}{ }\and
\inferrule*{P \to \sum_i c_i \ 1[P_i] \\ P_i \Downarrow \eta_i}{P \Downarrow \sum_i c_i \ \eta_i}\and \\
\inferrule*{P \outstep{o}{v} Q \\ Q \Downarrow \eta}{P \Downarrow \eta}\and
\inferrule*{P \ \final}{P \Downarrow 1[P]}
\end{mathpar}
\caption{Big-step operational semantics for \ipdl protocols.}
\label{fig:protocols_big_step}
\end{figure}

Note that while the semantics for reactions is sequential, both output and internal step relations for protocols are non-deterministic. Indeed, any two channels in a protocol may output in any order. Ordinarily, this presents a problem for reasoning about cryptography, since non-deterministic choice may present a security leak. However, our language introduces \emph{no} way to exploit this extra non-determinism, essentially due to the $\mathsf{read}$ commands in reactions being blocking. This is formalized by a \emph{confluence} result for \textsf{IPDL}:

\begin{lemma}[Confluence]
If $\Delta \vdash P : I \to O$, then:
\begin{itemize}
\item If $P \outstep{o}{v_1} Q_1$ and $P \outstep{o}{v_2} Q_2$, then $v_1 = v_2$ and $Q_1 = Q_2$.
\item If $P \outstep{o_1}{v_1} Q_1$ and $P \outstep{o_2}{v_2} Q_2$ with
$o_1 \neq o_2$, then there exists $Q$ such that $Q_1 \outstep{o_2}{v_2} Q$ and $Q_2 \outstep{o_1}{v_1} Q$.
\item If $P \outstep{o}{v} Q$ and $P \to \eta$, then there exists $\eta'$ such that $\eta \outstep{o}{v} \eta'$ and $Q \to \eta'$.
\item If $P \to \eta_1$ and $P \to \eta_2$, then either $\eta_1 = \eta_2$ or there exists $\eta$ such that $\eta_1 \to \eta$ and $\eta_2 \to \eta$.
\end{itemize}
\end{lemma}

\noindent In the above lemma, we lift the two protocol stepping relations $\outstep{o}{v}$ and $\to$ to distributions in the natural way. \smallskip

To guarantee termination of the semantics for reactions, we count the maximum number of steps the reaction would take \emph{provided all reads were resolved}:
\begin{align*}
\norm{\val{v}} & \coloneqq 0 \\
\norm{\ret{e}} & \coloneqq 1 \\
\norm{\samp{(\dist \ e)}} & \coloneqq 1 \\
\norm{\read{c}} & \coloneqq 0 \\
\norm{\ifte{e}{R_1}{R_2}} & \coloneqq \mathsf{max} \, (\norm{R_1}, \norm{R_2}) + 1 \\
\norm{x : \tau \leftarrow R; \ S} & \coloneqq (\norm{R} + \norm{S}) + 1
\end{align*}
We note that $\norm{-}$ for reactions is invariant under substitutions, embeddings, and input assignment. As expected, stepping reduces the number of steps left, guaranteeing termination:

\begin{lemma}
If $R \to \sum_i c_i \ 1[R_i]$, $c_i \neq 0$, then $\norm{R_i} < \norm{R}$.
\end{lemma}

\begin{corollary}[Determinism of $\Downarrow$ for reactions]
For any well-typed reaction $\Delta; \cdot \vdash R : I \to \tau$ there exists a unique distribution $\eta$ such that $R \Downarrow \eta$. We will denote $\eta$  by $\eval{R}$.
\end{corollary}

\noindent To guarantee termination of the semantics for protocols, we analogously count the maximum number of steps the protocol would take \emph{provided all reads in reactions were resolved}:
\begin{align*}
\norm{\zero} & \coloneqq 0 \\
\norm{\assign{o}{v}} & \coloneqq 0 \\
\norm{\assign{o}{R}} & \coloneqq \norm{R} + 1 \\
\norm{\Par{P}{Q}} & \coloneqq \norm{P} + \norm{Q} \\
\norm{\new{c}{\tau}{P}} & \coloneqq \norm{P}
\end{align*}
As for reactions, $\norm{-}$ for protocols is invariant under embeddings and input assignment, and stepping reduces the number of steps left:

\begin{lemma}
If $P \outstep{o}{v} Q$, then $\norm{Q} < \norm{P}$, and if $P \to \sum_i c_i \ 1[P_i]$, $c_i \neq 0$, then $\norm{P_i} < \norm{P}$.
\end{lemma}

\noindent Together with confluence, termination gives us the desired result:

\begin{corollary}[Determinism of $\Downarrow$ for protocols]
For any well-typed protocol $\Delta \vdash P : I \to O$ there exists a unique distribution $\eta$ such that $P \Downarrow \eta$. We will denote $\eta$ by $\eval{P}$.
\end{corollary}

%\section{Soundness of Exact Equality in \ipdl}
%Soundness of equality at the expression level means that if we substitute the same valued expression for each free variable, the resulting closed expressions will compute to the same value:

\begin{definition}
An axiom $\Gamma \vdash e_1 = e_2 : \tau$ is \emph{sound} if for any valued substitution $\theta : \cdot \to \Gamma$, we have $\eval{\theta^\star(e_1)} = \eval{\theta^\star(e_2)}$.
\end{definition}

\noindent The ambient \ipdl theory for expressions is said to be sound if each of its axioms is sound. It is straightforward to show that this implies overall soundness:
 
\begin{lemma}[Soundness of equality of expressions]
If the ambient \ipdl theory for expressions is sound, then for any equal expressions $\Gamma \vdash e_1 = e_2 : \tau$ and any valued substitution $\theta : \cdot \to \Gamma$, we have that $\eval{\theta^\star(e_1)} = \eval{\theta^\star(e_1)}$.
\end{lemma}

At the reaction level, two equal reactions should behave in a way that is indistinguishable by an external observer. We formally capture this notion of indistinguishability by a logical relation known as a \emph{bisimulation} -- a binary relation on distributions on reactions that satisfies certain closure properties, together with the crucial \emph{valuation property} that allows us to jointly partition two related distributions so that any two corresponding components are again related and have the same \emph{value}: a reaction $R$ is said to have value $v$ if $R$ is of the form $\val{v}$ (otherwise the value is undefined), and we lift this notion to distributions on reactions in the obvious way. At the reaction level, we only require the valuation property for those distributions that are \emph{final}, \emph{i.e.}, no reaction in the support steps.

\begin{definition}[Reaction bisimulation]
A \emph{reaction bisimulation} $\sim$ is a binary relation on distributions on reactions $\Delta; \ \cdot \vdash R : I \to \tau$ satisfying the following conditions:
\begin{itemize}
\item \emph{Closure under convex combinations}: For any distributions $\eta_1 \sim \varepsilon_1$ and $\eta_2 \sim \varepsilon_2$, and any coefficients $c_1, c_2 > 0$ with $c_1 + c_2 = 1$, we have $c_1 \eta_1 + c_2 \eta_2 \sim c_1 \varepsilon_1 + c_2 \varepsilon_2$.

\item \emph{Closure under input assignment}: For any distributions $\eta \sim \varepsilon$, input channel $i \in I$ of type $\tau$, and value $v \in \{0,1\}^{\sem{\tau}}$, we have $\eta[\read{i} \coloneqq \val{v}] \sim \varepsilon[\read{i} \coloneqq \val{v}]$.

\item \emph{Closure under computation}: For any distributions $\eta \sim \varepsilon$, we have $\eval{\eta} \sim \eval{\varepsilon}$.

\item \emph{Valuation property}: For any distributions $\eta \sim \varepsilon$ that are final, there exists a joint convex combination \[\eta = \sum_i c_i \, \eta_i \; \sim \, \sum_i c_i \, \varepsilon_i = \varepsilon\]
with $c_i > 0$ and $\sum_i c_i = 1$, such that
\begin{itemize}
\item the respective components $\eta_i \sim \varepsilon_i$ are again related, and
\item the distributions $\eta_i$ and $\varepsilon_j$ have the same value $v$ or lack thereof if and only if $i = j$.
\end{itemize}
\end{itemize}
\end{definition}

\noindent Crucially, we note that the joint convex combination in the valuation property is unique up to the order of the summands. We now describe one canonical way to construct reaction bisimulations:

\begin{definition}
Let $\sim$ be an arbitrary binary relation on distributions on reactions $\Delta; \ \cdot \vdash R : I \to \tau$. The \emph{lifting} $\lift(\sim)$ is the closure of $\sim$ under joint convex combinations. Explicitly, $\lift(\sim)$ is defined by
\[\sum_i c_i \, \eta_i \; \lift(\sim) \; \sum_i c_i \, \varepsilon_i\]
for coefficients $c_i > 0$ with $\sum_i c_i = 1$ and distributions $\eta_i \sim \varepsilon_i$.
\end{definition}

\begin{lemma}\label{lem:reaction_seed}
Let $\sim$ be a binary relation on distributions on reactions $\Delta; \ \cdot \vdash R : I \to \tau$ with the following properties:
\begin{itemize}
\item \emph{Closure under input assignment}: For any distributions $\eta \sim \varepsilon$, input channel $i \in I$ of type $\tau$, and value $v \in \{0,1\}^{\sem{\tau}}$, we have $\eta[\read{i} \coloneqq \val{v}] \sim \varepsilon[\read{i} \coloneqq \val{v}]$.

\item \emph{Lifting closure under computation}: For any distributions $\eta \sim \varepsilon$, we have $\eval{\eta} \lift(\sim) \, \eval{ \varepsilon}$.

\item \emph{Valuation property}: For any distributions $\eta \sim \varepsilon$ that are final, there exists a joint convex combination \[\eta = \sum_i c_i \, \eta_i \; \sim \, \sum_i c_i \, \varepsilon_i = \varepsilon\]
with $c_i > 0$ and $\sum_i c_i = 1$, such that
\begin{itemize}
\item the respective components $\eta_i \sim \varepsilon_i$ are again related, and
\item the distributions $\eta_i$ and $\varepsilon_j$ have the same value $v$ or lack thereof if and only if $i = j$.
\end{itemize}
\end{itemize}
Then the lifting $\lift(\sim)$ is a reaction bisimulation.
\end{lemma}

\begin{lemma}
We have the following: 
\begin{itemize}
\item The identity relation is a reaction bisimulation.
\item The inverse of a reaction bisimulation is a reaction bisimulation.
\item The composition of two reaction bisimulations is a reaction bisimulation.
\end{itemize}
\end{lemma}

\begin{example}
Fix two expressions $\cdot \vdash e_1 : \sigma$ and $\cdot \vdash e_2 : \sigma$ such that $\eval{e_1} = \eval{e_2}$. Then the relation $\sim$ defined by
\begin{itemize}
\item $1[R(x \coloneqq e_1)] \sim 1[R(x \coloneqq e_2)]$ for reaction $\Delta; \ x : \sigma \vdash R : I \to \tau$
\end{itemize}
is a reaction bisimulation.
\end{example}

\noindent Having defined reaction bisimulations, we can now formally state what it means for reaction equality to be sound:

\begin{definition}
An axiom $\Delta; \ \Gamma \vdash R_1 = R_2 : I \to \tau$ is \emph{sound} if there is a reaction bisimulation $\sim$ such that for any valued substitution $\theta : \cdot \to \Gamma$, we have $1[\theta^\star(R_1)] \sim 1[\theta^\star(R_2)]$.
\end{definition}

\noindent The ambient \ipdl theory for reactions is said to be sound if each of its axioms is sound. We now show that this implies overall soundness:

\begin{lemma}[Soundness of equality of reactions]
If the ambient \ipdl theory for reactions is sound, then for any equal reactions $\Delta; \ \Gamma \vdash R_1 = R_2 : I \to \tau$,     there exists a reaction bisimulation $\sim$ such that for any valued substitution $\theta : \cdot \to \Gamma$, we have $1[\theta^\star(R_1)] \sim 1[\theta^\star(R_2)]$.
\end{lemma}

\begin{proof}
We first replace the exchange rule \textsc{exch} by the three rules \textsc{exch-samp-samp}, \textsc{exch-samp-read}, and \textsc{exch-read-read} in Figure \ref{fig:exch_alt}; it is easy to see that this new set of rules is equivalent to the original one. We now proceed by induction on the alternative set of rules for reaction equality. We will freely use a distribution in place of a value (rule \textsc{exch-samp-read}) or a reaction (rules \textsc{embed}, \textsc{cong-bind}) to indicate the obvious lifting of the corresponding construct to distributions on reactions.

\begin{itemize}
\item \textsc{refl}: Our desired bisimulation is the identity relation.
\item \textsc{sym}: Our desired bisimulation is the inverse of the bisimulation obtained from the premise.
\item \textsc{trans}: Our desired bisimulation is the composition of the two bisimulations obtained from the two premises.
\item \textsc{axiom}: The desired bisimulation exists by assumption.
\item \textsc{input-unused}: Our desired bisimulation is precisely the bisimulation obtained from the premise, seen as a bisimulation on distributions on reactions with the additional input $i$.
\item \textsc{subst}: Our desired bisimulation is precisely the bisimulation obtained from the premise.
\item \textsc{embed}: Let $\sim$ be the bisimulation obtained from the premise. Our desired bisimulation $\sim_\phi$ is defined by
\begin{itemize}
\item $\phi^\star(\eta) \sim_\phi \phi^\star(\varepsilon)$ if $\eta \sim \varepsilon$
\end{itemize}
\item \textsc{cong-ret}: Our desired bisimulation is the lifting of the relation $\sim$ defined by
\begin{itemize}
\item $1[\ret{e}] \sim 1[\ret{e'}]$ for
\begin{itemize}
\item expressions $\cdot \vdash e : \tau$ and $\cdot \vdash e' : \tau$such that $\eval{e} = \eval{e'}$
\end{itemize}
\item $1[\val{v}] \sim 1[\val{v}]$ for value $v \in \{0,1\}^{\sem{\tau}}$
\end{itemize}
\item \textsc{cong-samp}: Our desired bisimulation is the lifting of the relation $\sim$ defined by
\begin{itemize}
\item $1[\samp{(\dist \ e)}] \sim 1[\samp{(\dist \ e')}]$ for
\begin{itemize}
\item expressions $\cdot \vdash e : \tau$ and $\cdot \vdash e' : \tau$such that $\eval{e} = \eval{e'}$
\end{itemize}
\item $1[\val{v}] \sim 1[\val{v}]$ for value $v \in \{0,1\}^{\sem{\tau}}$
\end{itemize}
\item \textsc{cong-if}: Let $\sim_1$ and $\sim_2$ be the two bisimulations obtained from the two premises. Our desired bisimulation is the lifting of the relation $\sim_\mathsf{if}$ defined by
\begin{itemize}
\item $1[\ifte{e}{R_1}{R_2}] \sim_\mathsf{if} \, 1[\ifte{e'}{R'_1}{R'_2}]$ for 
\begin{itemize}
\item expressions $\cdot \vdash e : \Bool$ and $\cdot \vdash e' : \Bool $such that $\eval{e} = \eval{e'}$
\item reactions $\Delta; \ \cdot \vdash R_1 : I \to \tau$ and $\Delta; \ \cdot \vdash R'_1 : I \to \tau$ such that $1[R_1] \sim_1 1[R'_1]$
\item reactions $\Delta; \ \cdot \vdash R_2 : I \to \tau$ and $\Delta; \ \cdot \vdash R'_2 : I \to \tau$ such that $1[R_2] \sim_2 1[R'_2]$
\end{itemize}
\item $\eta_1 \sim_\mathsf{if} \eta'_1$ if $\eta_1 \sim_1 \eta_1'$
\item $\eta_2 \sim_\mathsf{if} \eta'_2$ if $\eta_2 \sim_2 \eta_2'$
\end{itemize}
\item \textsc{cong-bind}: Let $\sim_1$ and $\sim_2$ be the two bisimulations obtained from the two premises. Our desired bisimulation is the lifting of the relation $\sim_\mathsf{bind}$ defined by
\begin{itemize}
\item $(x \leftarrow \eta; \ S) \sim_\mathsf{bind} (x \leftarrow \eta'; \ S')$ for
\begin{itemize}
\item distributions $\eta \sim_1 \eta'$
\item reactions $\Delta; \ x : \sigma \vdash S : I \to \tau$ and $\Delta; \ x : \sigma \vdash S' : I \to \tau$ such that for any value $v \in \{0,1\}^{\sem{\sigma}}$, we have $1[S(x \coloneqq v)] \sim_2 1[S'(x \coloneqq v)]$
\end{itemize}
\item $\varepsilon \sim_\mathsf{bind} \varepsilon'$ if $\varepsilon \sim_2 \varepsilon'$
\end{itemize}
\item \textsc{ret-bind}: Our desired bisimulation is the lifting of the relation $\sim$ defined by
\begin{itemize}
\item $1[x \leftarrow \ret{e}; \ R] \sim 1[R(x \coloneqq e)]$ for expression $\cdot \vdash e : \sigma$ and reaction $\Delta; \ x : \sigma \vdash R : I \to \tau$
\item $1[R(x \coloneqq v)] \sim 1[R(x \coloneqq e)]$ for
\begin{itemize}
\item reaction $\Delta; \ x : \sigma \vdash R : I \to \tau$
\item expression $\cdot \vdash e : \sigma$ and value $v \in \{0,1\}^{\sem{\sigma}}$ such that $e \Downarrow v$
\end{itemize}
\end{itemize}
\item \textsc{bind-ret}: Our desired bisimulation is the lifting of the relation $\sim$ defined by
\begin{itemize}
\item $1[x \leftarrow R; \ \ret{x}] \sim 1[R]$ for reaction $\Delta; \ \cdot \vdash R : I \to \tau$
\item $1[\val{v}] \sim 1[\val{v}]$ for value $v \in \{0,1\}^{\sem{\tau}}$
\end{itemize}
\item \textsc{bind-bind}: Our desired bisimulation is the lifting of the relation $\sim$ defined by
\begin{itemize}
\item $1[x_2 \leftarrow (x_1 \leftarrow R_1; \ R_2); \ S] \sim 1[x_1 \leftarrow R_1; \ x_2 \leftarrow R_2; \ S]$ for
\begin{itemize}
\item reaction $\Delta; \ \cdot \vdash R_1 : I \to \sigma_1$
\item reaction $\Delta; \ x_1 : \sigma_1 \vdash R_2 : I \to \sigma_2$
\item reaction $\Delta; \ x_2 : \sigma_2 \vdash S : I \to \tau$
\end{itemize}
\item $1[x_2 \leftarrow R_2; \ S] \sim 1[x_2 \leftarrow R_2; \ S]$ for
\begin{itemize}
\item reaction $\Delta; \ \cdot \vdash R_2 : I \to \sigma_2$
\item reaction $\Delta; \ x_2 : \sigma_2 \vdash S : I \to \tau$
\end{itemize}
\item $1[S] \sim 1[S]$ for reaction $\Delta; \ \cdot \vdash S : I \to \tau$
\end{itemize}
\item \textsc{samp-pure}: Our desired bisimulation is the lifting of the relation $\sim$ defined by
\begin{itemize}
\item $1[x \leftarrow \samp{(\dist \ e)}; \ R] \sim 1[R]$ for reaction $\Delta; \ \cdot \vdash R : I \to \tau$
\item $1[R] \sim 1[R]$ for reaction $\Delta; \ \cdot \vdash R : I \to \tau$
\end{itemize}
\item \textsc{read-det}: Our desired bisimulation is the lifting of the relation $\sim$ defined by
\begin{itemize}
\item $1[x \leftarrow \read{i}; \ y \leftarrow \read{i}; \ R] \sim 1[x \leftarrow \read{i}; \ R(y \coloneqq x)]$ for reaction $\Delta; \ x : \sigma, y : \sigma \vdash R : I \to \tau$
\item $1[x \leftarrow \val{v}; \ y \leftarrow \val{v}; \ R] \sim 1[x \leftarrow \val{v}; \ R(y \coloneqq x)]$ for
\begin{itemize}
\item reaction $\Delta; \ x : \sigma, y : \sigma \vdash R : I \to \tau$ 
\item value $v \in \{0,1\}^{\sem{\sigma}}$
\end{itemize}
\item $1[R] \sim 1[R]$ for reaction $\Delta; \ \cdot \vdash R : I \to \tau$
\end{itemize}
\item \textsc{if-left}: Our desired bisimulation is the lifting of the relation $\sim$ defined by
\begin{itemize}
\item $1[\ifte{\true}{R_1}{R_2}] \sim 1[R_1]$ for reactions $\Delta; \ \cdot \vdash R_1 : I \to \tau$ and $\Delta; \ \cdot \vdash R_2 : I \to \tau$
\item $1[R_1] \sim 1[R_1]$ for reaction $\Delta; \ \cdot \vdash R_1 : I \to \tau$
\end{itemize}
\item \textsc{if-right}: Our desired bisimulation is the lifting of the relation $\sim$ defined by
\begin{itemize}
\item $1[\ifte{\false}{R_1}{R_2}] \sim 1[R_2]$ for reactions $\Delta; \ \cdot \vdash R_1 : I \to \tau$ and $\Delta; \ \cdot \vdash R_2 : I \to \tau$
\item $1[R_2] \sim 1[R_2]$ for reaction $\Delta; \ \cdot \vdash R_2 : I \to \tau$
\end{itemize}
\item \textsc{if-ext}: Our desired bisimulation is the lifting of the relation $\sim$ defined by
\begin{itemize}
\item $1[R(x \coloneqq e)] \sim 1[\ifte{e}{R(x \coloneqq \true)}{R(x \coloneqq \false)]}$ for
\begin{itemize}
\item reaction $\Delta; \ x : \Bool \vdash R : I \to \tau$
\item expression $\cdot \vdash e : \Bool$
\end{itemize}
\item $1[R(x \coloneqq e)] \sim 1[R(x \coloneqq \true)]$ for
\begin{itemize}
\item reaction $\Delta; \ x : \Bool \vdash R : I \to \tau$
\item expression $\cdot \vdash e : \Bool$ such that $\eval{e} = 1$
\end{itemize}
\item $1[R(x \coloneqq e)] \sim 1[R(x \coloneqq \false)]$ for
\begin{itemize}
\item reaction $\Delta; \ x : \Bool \vdash R : I \to \tau$
\item expression $\cdot \vdash e : \Bool$ such that $\eval{e} = 0$
\end{itemize}
\end{itemize}
\item \textsc{exch-samp-samp}: Our desired bisimulation is the lifting of the relation $\sim$ defined by
\begin{itemize}
\item $1[x_1 \leftarrow \samp{(\dist_1 \ e_1)}; \ x_2 \leftarrow \samp{(\dist_2 \ e_2)}; \ \ret{(x_1,x_2)}] \sim \\ 1[x_2 \leftarrow \samp{(\dist_2 \ e_2)}; \ x_1 \leftarrow \samp{(\dist_1 \ e_1)}; \ \ret{(x_1,x_2)}]$ for
\begin{itemize}
\item expressions $\cdot \vdash e_1 : \sigma_1$ and $\cdot \vdash e_2 : \sigma_2$
\end{itemize}
\item $1[\val{v_1 v_2}] \sim 1[\val{v_1 v_2}]$ for values $v_1 \in \{0,1\}^{\sem{\tau_1}}$ and $v_2 \in \{0,1\}^{\sem{\tau_2}}$
\end{itemize}
\item \textsc{exch-samp-read}: Our desired bisimulation is the lifting of the relation $\sim$ defined by
\begin{itemize}
\item $1[x_1 \leftarrow \samp{(\dist \ e)}; \ x_2 \leftarrow \read{i}; \ \ret{(x_1,x_2)}] \sim 1[x_2 \leftarrow \read{i}; \ x_1 \leftarrow \samp{(\dist \ e)}; \ \ret{(x_1,x_2)}]$ for
\begin{itemize}
\item expression $\cdot \vdash e : \sigma$
\end{itemize}
\item $1[x_1 \leftarrow \samp{(\dist \ e)}; \ x_2 \leftarrow \val{v_2}; \ \ret{(x_1,x_2)}] \sim 1[x_2 \leftarrow \val{v_2}; \ x_1 \leftarrow \samp{(\dist \ e)}; \ \ret{(x_1,x_2)}]$ for
\begin{itemize}
\item expression $\cdot \vdash e : \sigma$
\item value $v_2 \in \{0,1\}^{\sem{\tau_2}}$
\end{itemize}
\item $\big(x_2 \leftarrow \read{i}; \ \ret{(\sem{\dist}(v),x_2)}\big) \sim 1[x_2 \leftarrow \read{i}; \ x_1 \leftarrow \samp{(\dist \ e)}; \ \ret{(x_1,x_2)}]$ for
\begin{itemize}
\item expression $\cdot \vdash e : \sigma$ and value $v \in \{0,1\}^{\sem{\sigma}}$ such that $e \Downarrow v$
\end{itemize}
\item $\big(x_2 \leftarrow \val{v_2}; \ \ret{(\sem{\dist}(\eval{e}),x_2)}\big) \sim 1[x_2 \leftarrow \val{v_2}; \ x_1 \leftarrow \samp{(\dist \ e)}; \ \ret{(x_1,x_2)}]$ for
\begin{itemize}
\item expression $\cdot \vdash e : \sigma$
\item value $v_2 \in \{0,1\}^{\sem{\tau_2}}$
\end{itemize}
\item $1[\val{v_1 v_2}] \sim 1[\val{v_1 v_2}]$ for values $v_1 \in \{0,1\}^{\sem{\tau_1}}$ and $v_2 \in \{0,1\}^{\sem{\tau_2}}$
\end{itemize}
\item \textsc{exch-read-read}: Our desired bisimulation is the lifting of the relation $\sim$ defined by
\begin{itemize}
\item $1[x_1 \leftarrow \read{i_1}; \ x_2 \leftarrow \read{i_2}; \ \ret{(x_1,x_2)}] \sim 1[x_2 \leftarrow \read{i_2}; \ x_1 \leftarrow \read{i_1}; \ \ret{(x_1,x_2)}]$
\item $1[x_1 \leftarrow \val{v_1}; \ x_2 \leftarrow \read{i_2}; \ \ret{(x_1,x_2)}] \sim 1[x_2 \leftarrow \read{i_2}; \ x_1 \leftarrow \val{v_1}; \ \ret{(x_1,x_2)}]$ for
\begin{itemize}
\item value $v_1 \in \{0,1\}^{\sem{\tau_1}}$
\end{itemize}
\item $1[x_1 \leftarrow \read{i_1}; \ x_2 \leftarrow \val{v_2}; \ \ret{(x_1,x_2)}] \sim 1[x_2 \leftarrow \val{v_2}; \ x_1 \leftarrow \read{i_1}; \ \ret{(x_1,x_2)}]$ for
\begin{itemize}
\item value $v_2 \in \{0,1\}^{\sem{\tau_2}}$
\end{itemize}
\item $1[x_1 \leftarrow \val{v_1}; \ x_2 \leftarrow \val{v_2}; \ \ret{(x_1,x_2)}] \sim 1[x_2 \leftarrow \val{v_2}; \ x_1 \leftarrow \val{v_1}; \ \ret{(x_1,x_2)}]$ for
\begin{itemize}
\item values $v_1 \in \{0,1\}^{\sem{\tau_1}}$ and $v_2 \in \{0,1\}^{\sem{\tau_2}}$
\end{itemize}
\item $1[x_2 \leftarrow \read{i_2}; \ \ret{(v_1,x_2)}] \sim 1[x_2 \leftarrow \read{i_2}; \ x_1 \leftarrow \val{v_1}; \ \ret{(x_1,x_2)}]$ for value $v_1 \in \{0,1\}^{\sem{\tau_1}}$
\item $1[x_1 \leftarrow \read{i_1}; \ x_2 \leftarrow \val{v_2}; \ \ret{(x_1,x_2)}] \sim 1[x_1 \leftarrow \read{i_1}; \ \ret{(x_1,v_2)}]$ for value $v_2 \in \{0,1\}^{\sem{\tau_2}}$
\item $1[x_2 \leftarrow \val{v_2}; \ \ret{(v_1,x_2)}] \sim 1[x_2 \leftarrow \val{v_2}; \ x_1 \leftarrow \val{v_1}; \ \ret{(x_1,x_2)}]$ for
\begin{itemize}
\item values $v_1 \in \{0,1\}^{\sem{\tau_1}}$ and $v_2 \in \{0,1\}^{\sem{\tau_2}}$
\end{itemize}
\item $1[x_1 \leftarrow \val{v_1}; \ x_2 \leftarrow \val{v_2}; \ \ret{(x_1,x_2)}] \sim 1[x_1 \leftarrow \val{v_1}; \ \ret{(x_1,v_2)}]$ for
\begin{itemize}
\item values $v_1 \in \{0,1\}^{\sem{\tau_1}}$ and $v_2 \in \{0,1\}^{\sem{\tau_2}}$
\end{itemize}
\item $1[\val{v_1 v_2}] \sim 1[\val{v_1 v_2}]$ for values $v_1 \in \{0,1\}^{\sem{\tau_1}}$ and $v_2 \in \{0,1\}^{\sem{\tau_2}}$
\end{itemize}
\end{itemize}
\end{proof}

\begin{figure}[ht!]
\begin{mathpar}
\inferrule*[right=exch-samp-samp]{\dist_1 : \sigma_1 \twoheadleftarrow \tau_1, \dist_2 : \sigma_2 \twoheadleftarrow \tau_2 \in \Sigma \\ \Gamma \vdash e_1 : \sigma_1 \\ \Gamma \vdash e_2 : \sigma_2}{\Delta; \ \Gamma \vdash \big(x_1 : \tau_1 \leftarrow \samp{(\dist_1 \ e_1)}; \ x_2 : \tau_2 \leftarrow \samp{(\dist_2 \ e_2)}; \ \ret{(x_1,x_2)}\big) = \\ \big(x_2 : \tau_2 \leftarrow \samp{(\dist_2 \ e_2)}; \ x_1 : \tau_1 \leftarrow \samp{(\dist_1 \ e_1)}; \ \ret{(x_1,x_2)}\big) : I \to \tau_1 \times \tau_2}\and
\inferrule*[right=exch-samp-read]{\dist : \sigma \to \tau_1 \in \Sigma \\ \Gamma \vdash e : \sigma \\ i : \tau_2 \in \Delta \\ i \in I}{\Delta; \ \Gamma \vdash \big(x_1 : \tau_1 \twoheadleftarrow \samp{(\dist \ e)}; \ x_2 : \tau_2 \leftarrow \read{i}; \ \ret{(x_1,x_2)}\big) =  \\ \big(x_2 : \tau_2 \leftarrow \read{i}; \ x_1 : \tau_1 \leftarrow \samp{(\dist \ e)}; \ \ret{(x_1,x_2)}\big) : I \to \tau_1 \times \tau_2}\and
\inferrule*[right=exch-read-read]{i_1 : \tau_1, i_2 : \tau_2 \in \Delta \\ i_1, i_2 \in I}{\Delta; \ \Gamma \vdash \big(x_1 : \tau_1 \leftarrow \read{i_1}; \ x_2 : \tau_2 \leftarrow \read{i_2}; \ \ret{(x_1,x_2)}\big) = \hspace{52.3pt} \\ \big(x_2 : \tau_2 \leftarrow \read{i_2}; \ x_1 : \tau_1 \leftarrow \read{i_1}; \ \ret{(x_1,x_2)}\big) : I \to \tau_1 \times \tau_2 \hspace{-39.8pt}}
\end{mathpar}
\caption{Alternative formulation of the \textsc{exch} rule for reaction equality.}
\label{fig:exch_alt}
\end{figure}

At last we get to the protocol level. A protocol bisimulation is entirely analogous to a reaction bisimulation, except we require the valuation property to hold: \emph{i)} per output channel $o$, and \emph{ii)} for all distributions (not necessarily final).

\begin{definition}[Protocol bisimulation]
A \emph{protocol bisimulation} $\sim$ is a binary relation on distributions on protocols $\Delta \vdash P : I \to O$ satisfying the following conditions:
\begin{itemize}
\item \emph{Closure under convex combinations}: For any distributions $\eta_1 \sim \varepsilon_1$ and $\eta_2 \sim \varepsilon_2$, and any coefficients $c_1, c_2 > 0$ with $c_1 + c_2 = 1$, we have $c_1 \eta_1 + c_2 \eta_2 \sim c_1 \varepsilon_1 + c_2 \varepsilon_2$.

\item \emph{Closure under input assignment}: For any distributions $\eta \sim \varepsilon$, input channel $i \in I$ of type $\tau$, and value $v \in \{0,1\}^{\sem{\tau}}$, we have $\eta[\read{i} \coloneqq \val{v}] \sim \varepsilon[\read{i} \coloneqq \val{v}]$.

\item \emph{Closure under computation}: For any distributions $\eta \sim \varepsilon$, we have $\eval{\eta} \sim \eval{\varepsilon}$.

\item \emph{Valuation property}: For any output channel $o \in O$, and any distributions $\eta \sim \varepsilon$, there exists a joint convex combination \[\eta = \sum_i c_i \, \eta_i \; \sim \, \sum_i c_i \, \varepsilon_i = \varepsilon\]
with $c_i > 0$ and $\sum_i c_i = 1$, such that
\begin{itemize}
\item the respective components $\eta_i \sim \varepsilon_i$ are again related, and
\item the distributions $\eta_i$ and $\varepsilon_j$ have the same value $v$ or lack thereof on $o$ if and only if $i = j$.
\end{itemize}
\end{itemize}
\end{definition}

\noindent Just like for reaction bisimulations, the joint sum in the valuation property is unique up to the order of the summands. We have an analogous canonical way of constructing protocol bisimulations:

\begin{definition}
Let $\sim$ be an arbitrary binary relation on distributions on protocols $\Delta \vdash P : I \to O$. The \emph{lifting} $\lift(\sim)$ is the closure of $\sim$ under joint convex combinations. Explicitly, $\lift(\sim)$ is defined by
\[\sum_i c_i \, \eta_i \; \lift(\sim) \; \sum_i c_i \, \varepsilon_i\]
for coefficients $c_i > 0$ with $\sum_i c_i = 1$ and distributions $\eta_i \sim \varepsilon_i$.
\end{definition}

\begin{lemma}\label{lem:protocol_seed}
Let $\sim$ be a binary relation on distributions on protocols $\Delta \vdash P : I \to O$ with the following properties:
\begin{itemize}
\item \emph{Closure under input assignment}: For any distributions $\eta \sim \varepsilon$, input channel $i \in I$ of type $\tau$, and value $v \in \{0,1\}^{\sem{\tau}}$, we have $\eta[\read{i} \coloneqq \val{v}] \sim \varepsilon[\read{i} \coloneqq \val{v}]$.

\item \emph{Lifting closure under computation}: For any distributions $\eta \sim \varepsilon$, we have $\eval{\eta} \lift(\sim) \, \eval{\varepsilon}$.

\item \emph{Valuation property}: For any output channel $o \in O$, and any distributions $\eta \sim \varepsilon$, there exists a joint convex combination \[\eta = \sum_i c_i \, \eta_i \; \sim \, \sum_i c_i \, \varepsilon_i = \varepsilon\]
with $c_i > 0$ and $\sum_i c_i = 1$, such that
\begin{itemize}
\item the respective components $\eta_i \sim \varepsilon_i$ are again related, and
\item the distributions $\eta_i$ and $\varepsilon_j$ have the same value $v$ or lack thereof on $o$ if and only if $i = j$.
\end{itemize}
\end{itemize}
Then the lifting $\lift(\sim)$ is a protocol bisimulation.
\end{lemma}

\begin{lemma}
We have the following: 
\begin{itemize}
\item The identity relation is a protocol bisimulation.
\item The inverse of a protocol bisimulation is a protocol bisimulation.
\item The composition of two protocol bisimulations is a protocol bisimulation.
\end{itemize}
\end{lemma}

\noindent We can now formally state what it means for exact protocol equality to be sound:

\begin{definition}
An axiom $\Delta \vdash P_1 = P_2 : I \to O$ is \emph{sound} if there is a protocol bisimulation $\sim$ such that $1[P_1] \sim 1[P_2]$.
\end{definition}

\noindent The ambient \ipdl theory for protocols is said to be sound if each of its axioms is sound. We now show that this implies overall soundness for exact equality:

\begin{lemma}[Soundness of exact equality of protocols]
If the ambient \ipdl theory for protocols is sound, then for any equal protocols $\Delta \vdash P_1 = P_2 : I \to O$, there exists a protocol bisimulation $\sim$ such that $1[P_1] \sim 1[P_2]$.
\end{lemma}

\begin{proof}
We first replace the rules \textsc{fold-if-left} and \textsc{fold-if-right} by the equivalent formulation in Figure \ref{fig:fold_if_alt}. We now proceed by induction on this alternative set of rules for exact protocol equality. We will freely use a measure in place of a reaction (rule \textsc{cong-react}) or a protocol (rules \textsc{embed}, \textsc{absorb-left}) to indicate the obvious lifting of the corresponding construct to measures on protocols.

\begin{itemize}
\item \textsc{refl}: Our desired bisimulation is the identity relation.
\item \textsc{sym}: Our desired bisimulation is the inverse of the bisimulation obtained from the premise.
\item \textsc{trans}: Our desired bisimulation is the composition of the two bisimulations obtained from the two premises.
\item \textsc{axiom}: The desired bisimulation exists by assumption.
\item \textsc{input-unused}: Our desired bisimulation is precisely the bisimulation obtained from the premise, seen as a bisimulation on distributions on protocols with the additional input $i$.
\item \textsc{embed}: Let $\sim$ be the bisimulation obtained from the premise. Our desired bisimulation $\sim_\phi$ is defined by
\begin{itemize}
\item $\phi^\star(\eta) \sim_\phi \phi^\star(\varepsilon)$ if $\eta \sim \varepsilon$
\end{itemize}
\item \textsc{cong-react}: Let $\sim$ be the reaction bisimulation obtained from the premise. Our desired bisimulation is the lifting of the relation $\sim_{\mathsf{react}}$ defined by
\begin{itemize}
\item $(o \coloneqq \eta) \sim_{\mathsf{react}} (o \coloneqq \eta')$ for distributions $\eta \sim \eta'$
\item $1[o \coloneqq v] \sim_{\mathsf{react}} 1[o \coloneqq v]$ for  value $v \in \{0,1\}^{\sem{\tau}}$
\end{itemize}
\item \textsc{cong-comp-left}: Let $\sim$ be the bisimulation obtained from the premise. Our desired bisimulation is the lifting of the relation $\sim_{\mathsf{par}}$ defined by
\begin{itemize}
\item $(\Par{\eta}{Q}) \sim_{\mathsf{par}} (\Par{\eta'}{Q})$ for $\eta \sim \eta'$ and protocol $\Delta \vdash Q : I \cup O_1 \to O_2$
\end{itemize}
The fact that this is indeed a bisimulation requires a fair amount of work; see Lemma \ref{lem:compositionality_exact}.
\item \textsc{cong-new}: Let $\sim$ be the bisimulation obtained from the premise. Our desired bisimulation $\sim_{\mathsf{new}}$ is defined by
\begin{itemize}
\item $(\new{o}{\tau}{\eta}) \sim_{\mathsf{new}} (\new{o}{\tau}{\eta'})$ if $\eta \sim \eta'$
\end{itemize}
\item \textsc{comp-comm}: Our desired bisimulation is the lifting of the relation $\sim$ defined by
\begin{itemize}
\item $1[\Par{P_1}{P_2}] \sim 1[\Par{P_2}{P_1}]$ for protocols $\Delta \vdash P_1 : I \cup O_2 \to O_1$ and $\Delta \vdash P_2 : I \cup O_1 \to O_2$
\end{itemize}
\item \textsc{comp-assoc}: Our desired bisimulation is the lifting of the relation $\sim$ defined by
\begin{itemize}
\item $1\big[\Par{(\Par{P_1}{P_2})}{P_3}\big] \sim 1\big[\Par{P_1}{(\Par{P_2}{P_3})}\big]$ for
\begin{itemize}
\item protocol $\Delta \vdash P_1 : I \cup O_2 \cup O_3 \to O_1$
\item protocol $\Delta \vdash P_2 : I \cup O_1 \cup O_3 \to O_2$
\item protocol $\Delta \vdash P_3 : I \cup O_1 \cup O_2 \to O_3$
\end{itemize}
\end{itemize}
\item \textsc{new-exch}: The desired bisimulation is the lifting of the relation $\sim$ defined by
\begin{itemize}
\item $1[\new{o_1}{\tau_1}{\new{o_2}{\tau_2}{P}}] \sim 1[\new{o_2}{\tau_2}{\new{o_1}{\tau_1}{P}}]$ for
\begin{itemize}
\item protocol $\Delta, o_1 : \tau_1, o_2 : \tau_2 \vdash P : I \to O \cup \{o_1,o_2\}$
\end{itemize}
\end{itemize}
\item \textsc{comp-new}: Our desired bisimulation is the lifting of the relation $\sim$ defined by
\begin{itemize}
\item $1[\Par{P}{(\new{o}{\tau}{Q})}] \sim 1[\new{o}{\tau}{(\Par{P}{Q})}]$ for
\begin{itemize}
\item protocol $\Delta \vdash P : I \cup O_2 \to O_1$
\item protocol $\Delta, o : \tau \vdash Q : I \cup O_1 \to O_2 \cup \{o\}$
\end{itemize}
\end{itemize}
\item \textsc{absorb-left}: Our desired bisimulation is the lifting of the relation $\sim$ defined by
\begin{itemize}
\item $1[\Par{P}{Q}] \sim 1[P]$ for protocols $\Delta \vdash P : I \to O$ and $\Delta \vdash Q : I \cup O \to \emptyset$
\end{itemize}
\item \textsc{diverge}: Our desired bisimulation is the lifting of the relation $\sim$ defined by
\begin{itemize}
\item $1[\assign{o}{x \leftarrow \read{o}; \ R}] \sim 1[\assign{o}{\read{o}}]$ for reaction $\Delta; \ \cdot \vdash R : I \cup \{o\} \to \tau$
\end{itemize}
\item \textsc{fold-if-left}: Our desired bisimulation is the lifting of the relation $\sim$ defined by
\begin{itemize}
\item $1[\new{l}{\tau}{\Par{\assign{o}{x \leftarrow \read{b}; \ \ifte{x}{\read{l}}}{S_2}}{\assign{l}{x \leftarrow \read{b}; \ S_1}}}] \sim \\ 1[\assign{o}{x \leftarrow \read{b}; \ \ifte{x}{S_1}{S_2}}]$ for
\begin{itemize}
\item reaction $\Delta; \ \cdot \vdash S_1 : I \cup \{o\} \to \tau$
\item reaction $\Delta; \ \cdot \vdash S_2 : I \cup \{o\} \to \tau$
\end{itemize}
\item $1[\new{l}{\tau}{\Par{\assign{o}{x \leftarrow \val{v}; \ \ifte{x}{\read{l}}}{S_2}}{\assign{l}{x \leftarrow \val{v}; \ S_1}}}] \sim \\ 1[\assign{o}{x \leftarrow \val{v}; \ \ifte{x}{S_1}{S_2}}]$ for
\begin{itemize}
\item value $v \in \{0,1\}$
\item reaction $\Delta; \ \cdot \vdash S_1 : I \cup \{o\} \to \tau$
\item reaction $\Delta; \ \cdot \vdash S_2 : I \cup \{o\} \to \tau$
\end{itemize}
\item $1[\new{l}{\tau}{\Par{\assign{o}{\read{l}}}{\assign{l}{S_1}}}] \sim 1[\assign{o}{S_1}]$ for reaction $\Delta; \ \cdot \vdash S_1 : I \cup \{o\} \to \tau$
\item $1[\new{l}{\tau}{\Par{\assign{o}{S_2}}{\assign{l}{S_1}}}] \sim 1[\assign{o}{S_2}]$ for
\begin{itemize}
\item reaction $\Delta; \ \cdot \vdash S_1 : I \cup \{o\} \to \tau$
\item reaction $\Delta; \ \cdot \vdash S_2 : I \cup \{o\} \to \tau$
\end{itemize}
\item $1[\new{l}{\tau}{\Par{\assign{o}{v_2}}{\assign{l}{S_1}}}] \sim 1[\assign{o}{v_2}]$ for
\begin{itemize}
\item reaction $\Delta; \ \cdot \vdash S_1 : I \cup \{o\} \to \tau$
\item value $v_2 \in \{0,1\}^{\sem{\tau}}$
\end{itemize}
\item $1[\new{l}{\tau}{\Par{\assign{o}{S_2}}{\assign{l}{v_1}}}] \sim 1[\assign{o}{S_2}]$ for
\begin{itemize}
\item value $v_1 \in \{0,1\}^{\sem{\tau}}$
\item reaction $\Delta; \ \cdot \vdash S_2 : I \cup \{o\} \to \tau$
\end{itemize}
\item $1[\new{l}{\tau}{\Par{\assign{o}{v_2}}{\assign{l}{v_1}}}] \sim 1[\assign{o}{v_2}]$ for values $v_1,v_2 \in \{0,1\}^{\sem{\tau}}$
\end{itemize}
\item \textsc{fold-if-right}: Our desired bisimulation is the lifting of the relation $\sim$ defined by
\begin{itemize}
\item $1[\new{r}{\tau}{\Par{\assign{o}{x \leftarrow \read{b}; \ \ifte{x}{S_1}{\read{r}}}}{\assign{r}{x \leftarrow \read{b}; \ S_2}}}] \sim \\ 1[\assign{o}{x \leftarrow \read{b}; \ \ifte{x}{S_1}{S_2}}]$ for
\begin{itemize}
\item reaction $\Delta; \ \cdot \vdash S_1 : I \cup \{o\} \to \tau$
\item reaction $\Delta; \ \cdot \vdash S_2 : I \cup \{o\} \to \tau$
\end{itemize}
\item $1[\new{r}{\tau}{\Par{\assign{o}{x \leftarrow \val{v}; \ \ifte{x}{S_1}{\read{r}}}}{\assign{r}{x \leftarrow \val{v}; \ S_2}}}] \sim \\ 1[\assign{o}{x \leftarrow \val{v}; \ \ifte{x}{S_1}{S_2}}]$ for
\begin{itemize}
\item value $v \in \{0,1\}$
\item reaction $\Delta; \ \cdot \vdash S_1 : I \cup \{o\} \to \tau$
\item reaction $\Delta; \ \cdot \vdash S_2 : I \cup \{o\} \to \tau$
\end{itemize}
\item $1[\new{r}{\tau}{\Par{\assign{o}{\read{r}}}{\assign{r}{S_2}}}] \sim 1[\assign{o}{S_2}]$ for reaction $\Delta; \ \cdot \vdash S_2 : I \cup \{o\} \to \tau$
\item $1[\new{r}{\tau}{\Par{\assign{o}{S_1}}{\assign{r}{S_2}}}] \sim 1[\assign{o}{S_1}]$ for
\begin{itemize}
\item reaction $\Delta; \ \cdot \vdash S_1 : I \cup \{o\} \to \tau$
\item reaction $\Delta; \ \cdot \vdash S_2 : I \cup \{o\} \to \tau$
\end{itemize}
\item $1[\new{r}{\tau}{\Par{\assign{o}{v_1}}{\assign{r}{S_2}}}] \sim 1[\assign{o}{v_1}]$ for
\begin{itemize}
\item value $v_1 \in \{0,1\}^{\sem{\tau}}$
\item reaction $\Delta; \ \cdot \vdash S_2 : I \cup \{o\} \to \tau$
\end{itemize}
\item $1[\new{r}{\tau}{\Par{\assign{o}{S_1}}{\assign{r}{v_2}}}] \sim 1[\assign{o}{S_1}]$ for
\begin{itemize}
\item reaction $\Delta; \ \cdot \vdash S_1 : I \cup \{o\} \to \tau$
\item value $v_2 \in \{0,1\}^{\sem{\tau}}$
\end{itemize}
\item $1[\new{r}{\tau}{\Par{\assign{o}{v_1}}{\assign{r}{v_2}}}] \sim 1[\assign{o}{v_1}]$ for values $v_1, v_2 \in \{0,1\}^{\sem{\tau}}$
\end{itemize}
\item \textsc{fold-bind}: Our desired bisimulation is the lifting of the relation $\sim$ defined by
\begin{itemize}
\item $1[\new{c}{\tau_1}{\Par{\assign{o}{x \leftarrow \read{c};} \ R_2}{\assign{c}{R_1}}}] \sim 1[\assign{o}{x \leftarrow R_1; \ R_2}]$ for
\begin{itemize}
\item reaction $\Delta; \ \cdot \vdash R_1 : I \cup \{o\} \to \tau_1$
\item reaction $\Delta; \ x : \tau_1 \vdash R_2 : I \cup \{o\} \to \tau_2$
\end{itemize}
\item $1[\new{c}{\tau_1}{\Par{\assign{o}{R_2}}{\assign{c}{v_1}}}] \sim 1[\assign{o}{R_2}]$ for
\begin{itemize}
\item value $v_1 \in \{0,1\}^{\sem{\tau_1}}$
\item reaction $\Delta; \ \cdot \vdash R_2 : I \cup \{o\} \to \tau_2$
\end{itemize}
\item $1[\new{c}{\tau_1}{\Par{\assign{o}{v_2}}{\assign{c}{v_1}}}] \sim 1[\assign{o}{v_2}]$ for
\begin{itemize}
\item values $v_1 \in \{0,1\}^{\sem{\tau_1}}$ and $v_2 \in \{0,1\}^{\sem{\tau_2}}$
\end{itemize}
\end{itemize}
\item \textsc{subsume}: Our desired bisimulation is the lifting of the relation $\sim$ defined by
\begin{itemize}
\item $1[\Par{\assign{o_1}{x_0 \leftarrow \read{o_0}; \ R_1}}{\assign{o_2}{x_0 \leftarrow \read{o_0}; \ x_1 \leftarrow \read{o_1}; \ R_2}}] \sim \\ 1[\Par{\assign{o_1}{x_0 \leftarrow \read{o_0}; \ R_1}}{\assign{o_2}{x_1 \leftarrow \read{o_1}; \ R_2}}]$ for
\begin{itemize}
\item reaction $\Delta; \ x_0 : \tau_0 \vdash R_1 : I \cup \{o_1,o_2\} \to \tau_1$
\item reaction $\Delta; \ x_1 : \tau_1 \vdash R_2 : I \cup \{o_1,o_2\} \to \tau_2$
\end{itemize}
\item $1[\Par{\assign{o_1}{x_0 \leftarrow \val{v_0}; \ R_1}}{\assign{o_2}{x_0 \leftarrow \val{v_0}; \ x_1 \leftarrow \read{o_1}; \ R_2}}] \sim \\ 1[\Par{\assign{o_1}{x_0 \leftarrow \val{v_0}; \ R_1}}{\assign{o_2}{x_1 \leftarrow \read{o_1}; \ R_2}}]$ for
\begin{itemize}
\item value $v_0 \in \{0,1\}^{\sem{\tau_0}}$
\item reaction $\Delta; \ x_0 : \tau_0 \vdash R_1 : I \cup \{o_1,o_2\} \to \tau_1$
\item reaction $\Delta; \ x_1 : \tau_1 \vdash R_2 : I \cup \{o_1,o_2\} \to \tau_2$
\end{itemize}
\item $1[\Par{\assign{o_1}{R_1}}{\assign{o_2}{x_1 \leftarrow \read{o_1}; \ R_2}}] \sim 1[\Par{\assign{o_1}{R_1}}{\assign{o_2}{x_1 \leftarrow \read{o_1}; \ R_2}}]$ for
\begin{itemize}
\item reaction $\Delta; \ \cdot \vdash R_1 : I \cup \{o_1,o_2\} \to \tau_1$
\item reaction $\Delta; \ x_1 : \tau_1 \vdash R_2 : I \cup \{o_1,o_2\} \to \tau_2$
\end{itemize}
\item $1[\Par{\assign{o_1}{v_1}}{\assign{o_2}{R_2}}] \sim 1[\Par{\assign{o_1}{v_1}}{\assign{o_2}{R_2}}]$ for
\begin{itemize}
\item value $v_1 \in \{0,1\}^{\sem{\tau_1}}$
\item reaction $\Delta; \ \cdot \vdash R_2 : I \cup \{o_1,o_2\} \to \tau_2$
\end{itemize}
\item $1[\Par{\assign{o_1}{v_1}}{\assign{o_2}{v_2}}] \sim 1[\Par{\assign{o_1}{v_1}}{\assign{o_2}{v_2}}]$ for values $v_1 \in \{0,1\}^{\sem{\tau_1}}$ and $v_2 \in \{0,1\}^{\sem{\tau_2}}$
\end{itemize}


\item \textsc{subst}: Let $\sim$ be the reaction bisimulation obtained from the premise. Our desired bisimulation is the lifting of the relation $\sim_\mathsf{subst}$ defined by
\begin{itemize}
\item $\big(\Par{\assign{o_1}{\eta}}{\assign{o_2}{x_1 \leftarrow \read{o_1}; \ R_2}}\big) \sim_\mathsf{subst} \big(\Par{\assign{o_1}{\eta}}{\assign{o_2}{x_1 \leftarrow \eta; \ R_2}}\big)$ for
\begin{itemize}
\item distribution $\eta$ on reactions $\Delta; \ \cdot \vdash R_1 : I \cup \{o_1,o_2\} \to \tau_1$
\item reaction $\Delta; \ \cdot \vdash R_1 : I \cup \{o_1,o_2\} \to \tau_1$ evaluating to the same distribution as $\eta$
\item reaction $\Delta; \ x_1 : \tau_1 \vdash R_2 : I \cup \{o_1,o_2\} \to \tau_2$
\end{itemize}
such that $1[x_1 \leftarrow R_1; \ x_1' \leftarrow R_1; \ \ret{(x_1,x_1')}] \sim 1[x_1 \leftarrow R_1; \ \ret{(x_1,x_1)}]$
\item $1[\Par{\assign{o_1}{v_1}}{\assign{o_2}{R_2}}] \sim_\mathsf{subst} 1[\Par{\assign{o_1}{v_1}}{\assign{o_2}{R_2}}]$ for
\begin{itemize}
\item value $v_1 \in \{0,1\}^{\sem{\tau_1}}$
\item reaction $\Delta; \ \cdot \vdash R_2 : I \cup \{o_1,o_2\} \to \tau_2$
\end{itemize}
\item $1[\Par{\assign{o_1}{v_1}}{\assign{o_2}{v_2}}] \sim_\mathsf{subst} 1[\Par{\assign{o_1}{v_1}}{\assign{o_2}{v_2}}]$ for values $v_1 \in \{0,1\}^{\sem{\tau_1}}$ and $v_2 \in \{0,1\}^{\sem{\tau_2}}$
\end{itemize}
\item \textsc{drop}: Let $\sim$ be the reaction bisimulation obtained from the premise. Our desired bisimulation is the lifting of the relation $\sim_{\mathsf{drop}}$ defined by
\begin{itemize}
\item $\big(\Par{\assign{o_1}{\eta_1}}{\assign{o_2}{x_1 \leftarrow \read{o_1}; \ R_2}}\big) \sim_{\mathsf{drop}} \big(\Par{\assign{o_1}{\eta_1}}{\assign{o_2}{\eta_2}}\big)$ for
\begin{itemize}
\item measure $\eta_1$ on reactions $\Delta; \ \cdot \vdash R_1 : I \cup \{o_1,o_2\} \to \tau_1$
\item reaction $\Delta; \ \cdot \vdash R_1 : I \cup \{o_1,o_2\} \to \tau_1$ such that
\begin{itemize}
\item[\emph{i)}] $R_1$ either evaluates to the same distribution as $\eta_1$, or
\item[\emph{ii)}] there exists a measure $\overline{\eta_1}$ on reactions $\Delta; \ \cdot \vdash R_1 : I \cup \{o_1,o_2\} \to \tau_1$ such that $R_1$ evaluates to the same distribution as $\eta_1 + \overline{\eta_1}$
\end{itemize}
\item distribution $\eta_2$ on reactions $\Delta; \ \cdot \vdash R_2 : I \cup \{o_1,o_2\} \to \tau_2$
\item reaction $\Delta; \ \cdot \vdash R_2 : I \cup \{o_1,o_2\} \to \tau_2$ evaluating to the same distribution as $\eta_2$
\end{itemize}
such that $1[x_1 \leftarrow R_1; \ R_2] \sim 1[R_2]$
\item $(\Par{\assign{o_1}{v_1}}{\assign{o_2}{R_2}}) \sim_{\mathsf{drop}} (\Par{\assign{o_1}{v_1}}{\assign{o_2}{R_2}})$ for
\begin{itemize}
\item value $v_1 \in \{0,1\}^{\sem{\tau_1}}$
\item reaction $\Delta; \ \cdot \vdash R_2 : I \cup \{o_1,o_2\} \to \tau_2$
\end{itemize}
\item $(\Par{\assign{o_1}{v_1}}{\assign{o_2}{v_2}}) \sim_{\mathsf{drop}} (\Par{\assign{o_1}{v_1}}{\assign{o_2}{v_2}})$ for values $v_1 \in \{0,1\}^{\sem{\tau_1}}$ and $v_2 \in \{0,1\}^{\sem{\tau_2}}$
\end{itemize}
\end{itemize}
\end{proof}

\begin{figure*}[h]
\begin{mathpar}
\inferrule*[right=fold-if-left]{o \notin I \\ b \in I \\ b : \Bool, o : \tau \in \Delta \\ \Delta; \ \cdot \vdash S_1 : I \cup \{o\} \to \tau \\ \Delta; \ \cdot \vdash S_2 : I \cup \{o\} \to \tau}{\Delta \vdash \big(\new{l}{\tau}{\Par{\assign{o}{x : \Bool \leftarrow \read{b}; \ \ifte{x}{{\color{red} \read{l}}}{S_2}}}{{\color{red} \assign{l}{x : \Bool \leftarrow \read{b}; \ S_1}}}}\big) = \\ \big(\assign{o}{x : \Bool \leftarrow \read{b}; \ \ifte{x}{{\color{red} S_1}}{S_2}}\big) : I \to \{o\}\hspace{15pt}}\and
\inferrule*[right=fold-if-right]{o \notin I \\ b \in I \\ b : \Bool, o : \tau \in \Delta \\ \Delta; \ \cdot \vdash S_1 : I \cup \{o\} \to \tau \\ \Delta; \ \cdot \vdash S_2 : I \cup \{o\} \to \tau}{\Delta \vdash \big(\new{r}{\tau}{\Par{\assign{o}{x : \Bool \leftarrow \read{b}; \ \ifte{x}{S_1}{{\color{red} \read{r}}}}}{{\color{red} \assign{r}{x : \Bool \leftarrow \read{b}; \ S_2}}}}\big) = \\ \big(\assign{o}{x : \Bool \leftarrow \read{b}; \ \ifte{x}{S_1}{{\color{red} S_2}}}\big) : I \to \{o\}\hspace{25pt}}
\end{mathpar}
\caption{Alternative formulation of the \textsc{fold-if-left} and \textsc{fold-if-right} rules.}
\label{fig:fold_if_alt}
\end{figure*}

\noindent The remainder of this section is devoted to proving the following lemma:

\begin{lemma}[Compositionality for the exact equality of protocols]\ref{lem:compositionality_exact}
Let $\sim$ be a bisimulation on protocols $\Delta \vdash P : I \cup O_2 \to O_1$. Then the lifting of the relation $\sim_{\mathsf{par}}$ defined by
\begin{itemize}
\item $(\Par{\eta}{Q}) \sim_{\mathsf{par}} (\Par{\eta'}{Q})$ for $\eta \sim \eta'$ and protocol $\Delta \vdash Q : I \cup O_1 \to O_2$
\end{itemize}
is a protocol bisimulation.
\end{lemma}

\begin{proof}
The one property difficult to verify is lifting closure under computation: for any protocol $\Delta \vdash Q : I \cup O_1 \to O_2$, and any distributions $\eta \sim \eta'$, we have $\eval{(\Par{\eta}{Q})} \lift(\sim_\mathsf{par}) \eval{(\Par{\eta'}{Q})}$. The difficulty arises from the \emph{global} nature of the protocol semantics: in the composition $\Par{P}{Q}$, a step of the form $P \outstep{o}{v} P'$ \emph{changes} the protocol $Q$ (specifically to $Q[\assign{\read{o}}{\val{v}}]$). This makes it hard to express the computation of $\Par{P}{Q}$ in terms of the computation of $P$, because in the course of the latter we are simultaneously probabilistically updating $Q$.
%
%We solve this problem by defining an alternate \emph{local} form of interaction for \ipdl protocols, and showing that the resulting operational semantics agrees with the original one. Informally speaking, the global semantics of \ipdl protocols has a \emph{push} character -- the moment a value $v$ on a channel $o$ is computed, every $\read{o}$ command in all other reactions is replaced by $\val{v}$. In contrast, the local form of the semantics that we are about to define has a \emph{pull} character -- a reaction containing a $\read{o}$ command extracts the value $v$ from channel $o$, if possible, and replaces this \emph{particular} occurrence of $\read{o}$ by $\val{v}$. We formally define this mechanism in Figure {fig:protocols_semantics_local}. The relation 
%
%
%For protocols, the relation $P \in{o}{v} Q$ indicates that we have replaced one occurrence of the command $\read{o}$ in a \emph{single} reaction in $P$ by $\val{v}$, yielding the protocol $Q$. The protocol $Q$ may be seen as a single-step approximation towards the protocol obtained by performing the output assignment $o \coloneqq v$ in $P$.
%
%\begin{figure}
%\begin{mathpar}
%\fbox{$R \in{o}{v} S$}\\
%\inferrule*{ }{\read{o} \in{o}{v} \val{v}}\and
%\inferrule*{R_1 \in{o}{v} R_1'}{(\ifte{e}{R_1}{R_2}) \in{o}{v} (\ifte{e}{R_1'}{R_2})}\and
%\inferrule*{R_2 \in{o}{v} R_2'}{(\ifte{e}{R_1}{R_2}) \in{o}{v} (\ifte{e}{R_2'}{R_2})}\and
%\inferrule*{R \in{o}{v} R'}{(x : \sigma \leftarrow R; \ S) \in{o}{v} {(x : \sigma \leftarrow R'; \ S)}}\and
%\inferrule*{S \in{o}{v} S'}{(x : \sigma \leftarrow R; \ S) \in{o}{v} {(x : \sigma \leftarrow R; \ S')}}\\\\
%\fbox{$P \in{o}{v} Q$}\\
%\inferrule*{R \in{o}{v} S}{(\assign{o}{R}) \in{o}{v} (\assign{o}{S})}\and
%\inferrule*{P \in{o}{v} P'}{(\Par{P}{Q}) \in{o}{v} (\Par{P'}{Q})}\and
%\inferrule*{Q \in{o}{v} Q'}{(\Par{P}{Q}) \in{o}{v} (\Par{P}{Q'})}\and
%\inferrule*{P \in{o}{v} P' \\ o \neq c}{(\new{c}{\tau}{P}) \in{o}{v} (\new{c}{\tau}{P'})}\\\\
%\fbox{$P \hookrightarrow Q$}\\
%\inferrule*[right=pull-comp-left]{P \outstep{o}{v} P' \\ Q \in{o}{v} Q'}{(\Par{P}{Q}) \pull{o}{v} (\Par{P}{Q'})}\and
%\inferrule*[right=pull-comp-right]{Q \outstep{o}{v} Q' \\ P \in{o}{v} P'}{(\Par{P}{Q}) \pull{o}{v} (\Par{P'}{Q})}\and
%\inferrule*[right=pull-new]{P \pull{o}{v} P' \\ o \neq c}{(\new{c}{\tau}{P}) \pull{o}{v} (\new{c}{\tau}{P'})}
%\end{mathpar}
%\caption{The local form of small-step operational semantics for \ipdl protocols.}
%\label{fig:protocols_semantics_local}
%\end{figure}
%
%
We now have all the preliminaries necessary to prove that $\sim_\mathsf{par}$ enjoys lifting closure under computation.


 Since the set $O_1$ of outputs is finite, we can apply the valuation property of $\sim$ in succession for each output channel $o \in O_1$, until we end up with the special case when $\eta$ and $\eta'$ have the same value $v$ or lack thereof on each output channel. In other words, it suffices to prove the following:
%
%\emph{\begin{center}
%Claim 1: For any protocol $\Delta \vdash Q : I \cup O_1 \to O_2$, and any measures $\eta \sim \eta'$ that have the same value $v$ or lack thereof on any output channel $o \in O_1$, if $(\Par{\eta}{Q}) \Downarrow \varepsilon$ and $(\Par{\eta'}{Q}) \Downarrow \varepsilon'$, then $\varepsilon \lift(\sim_\mathsf{par}) \, \varepsilon'$.
%\end{center}}
%
%The remainder of this section is devoted to proving this claim.
%
%
%
%
%\emph{\begin{center}
%Claim 2: For any protocol $\Delta \vdash Q : I \cup O_1 \to O_2$, and any measures $\eta$ and $\eta'$ that have the same value $v$ or lack thereof on any output channel $o \in O_1$, if $(\Par{\eta}{Q}) \Downarrow \varepsilon$ and $(\Par{\eta'}{Q}) \Downarrow \varepsilon'$, then $\varepsilon \lift(\sim_\mathsf{par}) \, \varepsilon'$.
%\end{center}}
%
%
%The local form of the big-step operational semantics for protocols $P \Downarrow \eta$, see Figure \ref{fig:protocols_big_step_local}, performs as many output and internal steps as possible in an attempt to compute
%
%\begin{figure}
%\begin{mathpar}
%\fbox{$P \outstep{O} Q$}\\
%
%\inferrule*[right=out-val]{ }{(\assign{o}{\val{v}}) \outstep{o}{v} (\assign{o}{v})}\and
%\inferrule*[right=out-comp-left]{P \outstep{o}{v} P'}{(\Par{P}{Q}) \outstep{o}{v} \big(\Par{P'}{Q[\assign{\read{o}}{\val{v}}]}\big)}\and
%\inferrule*[right=out-comp-right]{Q \outstep{o}{v} Q'}{(\Par{P}{Q}) \outstep{o}{v} \big(\Par{P[\assign{\read{o}}{\val{v}}]}{Q'}\big)}\and
%\inferrule*[right=out-new]{P \outstep{o}{v} P' \\ o \neq c}{(\new{c}{\tau}{P}) \outstep{o}{v} (\new{c}{\tau}{P'})}\\\\
%
%
%\inferrule*{P \Downarrow \eta \\ P \outset{\out{P}} \eta}{P \ \lfinal}\\\\
%
%
%\fbox{$P \outset{O} Q$}\\
%\inferrule*{P \Downarrow \eta \\ P \outset{\out{P}} \eta}{P \ \lfinal}\\\\
%
%
%\fbox{$P \ \lfinal$}\\
%\inferrule*{P \Downarrow 1[Q] \\ P \outset{\out{P}} Q}{P \ \lfinal}\\\\
%\fbox{$P \Rightarrow \eta$}\\
%\inferrule*{}{ }\and
%\inferrule*{P \to \sum_i c_i \ 1[P_i] \\ P_i \Rightarrow \eta_i}{P \Rightarrow \sum_i c_i \ \eta_i}\and
%\inferrule*{P \hookrightarrow Q \\ Q \Rightarrow \eta}{P \Rightarrow \eta}\and
%\inferrule*{P \ \lfinal}{P \Rightarrow 1[P]}
%\end{mathpar}
%\caption{The local form of big-step operational semantics for \ipdl protocols.}
%\label{fig:protocols_big_step_local}
%\end{figure}
%
%
%
%
\end{proof}
%
%\begin{definition}[Sound exact theory]
%Fix a signature $\Sigma$ and an interpretation $\Int$. An \emph{exact} \ipdl theory $\mathbb{T}$ is a triple $(\mathbb{T}_e,\mathbb{T}_r,\mathbb{T}_p)$ of expression-level, reaction-level, and protocol-level \ipdl theories, respectively. The exact theory $\mathbb{T}$ is \emph{sound} with respect to $\Int$, written $\Int \vDash \mathbb{T}$, if each of $\mathbb{T}_e$, $\mathbb{T}_r$, and $\mathbb{T}_p$ is sound with respect to $\Int$.
%\end{definition}

\end{document}

%\section{Computational Semantics of \ipdl}

%\section{Case Studies in \ipdl}

\section{Maude Formalization}
\newcommand{\flip}{\mathsf{flip}}
\newcommand{\id}{\mathsf{id}}
\newcommand{\adv}{\mathsf{adv}}
\newcommand{\Alice}{\mathsf{Alice}}
\newcommand{\In}{\mathsf{In}}
\newcommand{\Ctxt}{\mathsf{Ctxt}}
\newcommand{\Key}{\mathsf{Key}}
\newcommand{\LeakCtxt}{\mathsf{LeakCtxt}}
\subsection{Maude}

Maude \cite{DBLP:conf/maude/2007} is a high-level declarative
language and 
a high-performance logical framework supporting both equational and rewriting logic computation for a wide range of applications. 
Maude features several kinds of modules:
\begin{itemize}
\item \emph{functional modules}, which are theories (with an initial
model semantics) in 
membership equational logic that allow definitions of 
data types and operations on them, via multiple sorts, subsort
relations between them, equations between terms, and
assertions of membership of a term to a sort,
\item \emph{system modules}, which are theories in rewriting logic
that extend functional modules with
definitions of rewrite rules, representing transitions between states, and
\item \emph{strategy modules}, which control 
the way the rewriting rules are applied, by means of strategy combinators,
such as concatenation, iterations and others.
\end{itemize}

We now present the features of the Maude language that we make use of
in formalizing \textsf{IPDL}.
Maude functional modules are introduced with the syntax
\code{fmod NAME is ... endfm}. In a functional module we can declare
sorts, using the keyword \code{sort}, state that two sorts are
in the subsort relation, written \code{subsort s1 < s2}, declare
operations on the sorts, using \code{op f : s1  ...  sk -> s} for 
an operation \code{f} with argument sorts \code{s1 ... sk} and 
result sort \code{s}. Moreover, operations
may have attributes, written in square brackets after their declarations,
like \code{comm} for commutativity or \code{assoc} for associativity.
In Maude, terms are rewritten to a normal form modulo the declared 
attributes and the equations of defined operations.
More precisely, 
equations are used as equational rules: instances of the left-hand side
pattern that match subterms of a term are replaces with the 
corresponding instances of the right-hand side.
The process is called term rewriting and the result of simplifying a term by complete application of equational rules is called its normal form. 
We can control which operations will appear in these ground
forms by adding the attribute \code{ctor} to them. An operation that
is not a constructor of a sort is regarded as defined. 
Equations are introduced by the syntax 
\code{eq t1 = t2}, where \code{t1} and \code{t2} are terms
of sorts related via subsorting. 
We can assert sort membership using the syntax 
\code{t : s}
where \code{t} is a term and \code{s} is a sort.
Conditional equations are written
\code{ceq t = t' if C1 $\land$ ... $\land$ Cn} 
where \code{Ci} is either an equation or a membership.
We may declare variables using the keyword \code{var}.
Functional modules are assumed to satisfy the executability requirements of confluence,
termination, and sort-decreasingness, see details
in \cite{DBLP:conf/maude/2007}.
The semantics of functional models is given in terms of the initial
model whose elements are ground equivalence classes of terms modulo
equations.

Rewriting logic extends equational logic by introducing the notion of rewrites corresponding to transitions between states.
Unlike equations, rewrites are not symmetric. Maude system modules
are introduced with the syntax \code{mod NAME is ... endm}. Rules
are declared with the syntax 
\code{rl [label] : t1 => t2}. 
Conditional rules are written with the keyword
\code{crl [label] : t => t' if C1 $\land$ ... $\land$ Cn} 
and their conditions
\code{Ci} may be equations, memberships or rewrites. 
Rewrites are not expected to be terminating, confluent or deterministic.
Rewrites denote transitions between the elements of the initial model
of the functional part of a system module.

Maude strategy modules are introduced with the syntax
\code{smod NAME is ... endsm}. In addition to declarations allowed in
system modules, we can have strategy declarations and definitions.
The main strategy combinators are \code{;} for concatenation of strategy 
expressions, \code{|} for alternative, \code{*} for iteration of an
expression zero or more times, \code{idle} for the strategy giving as result its argument, \code{fail} for the strategy that gives no result,
\code{s1 ? s2 : s3} for the strategy that attempts to run the
strategy \code{s1} then, if the run is successful, it runs \code{s2},
otherwise it runs \code{s3}.
Several other derived constructions are also
supported, \emph{e.g.}, \code{try s} for \code{s ? idle : idle} and \code{s1 or-else s2} for \code{s1 ? idle : s2}. The match and rewrite operator 
\code{matchrew} restricts the application of a strategy to a specific
subterm of the subject term, see details in \cite{DBLP:conf/maude/2007}.
Strategies are declared as \code{strat NAME : s1 ... sk @ s .},
where \code{s1 ... sk} are the sorts of the arguments of the strategy
and \code{s} is the subject sort to which the strategy is applied.
The syntax for definitions is \code{sd NAME(v1,..., vk) := Exp .} where
\code{vi} are variables of sort \code{si} and \code{Exp} is a strategy expression.


Maude supports module imports, using the keyword \code{protecting},
which means that no new elements of an imported sorts may be 
added and 
no identification between elements of an imported sorts via
equations are allowed. 
Two more importation modes are supported, but we do not make use 
of them.

Maude provides several predefined data types. We will use Booleans, natural
numbers, lists, sets and maps.
  
\subsection{Syntax}

We start with a sort \code{Type} for data types, together with 
constants \code{unit} and \code{bool} of sort \code{Type} and a
binary product on the sort \code{Type}. Expressions are built over
signatures, which are implemented as commutative lists of symbols,
where a function or distribution symbol pairs the symbol name with 
its arity. Signatures are valid if they don't contain multiple
occurences of same symbol name. Expressions are then implemented
as a sort \code{Expression} that includes as a subsort the identifiers,
which are provided by the default Maude sort \code{Qid},
such that we can use them for variable names. There are constructors
for \code{True}, \code{False} and \code{()}. Application is 
represented as \code{ap f e} where \code{ap} is a constructor, \code{f}
is an identifier standing for the name of the function symbol and
\code{e} is an expression. Moreover we have constructors for
pairs and projections on first and second component of a pair. 
Type contexts are implemented again as commutative lists of typed
variables, written \code{x : T}, where \code{x} is an identifier and
\code{T} is a type. Expression typing is 
implemented as a predicate 
\code{typeOf : Signature TypeContext Expression -> Bool}, while
we let Maude handle expression equality by only adding the expression
equality rules \textsc{fast-pair}, \textsc{snd-pair} and 
\textsc{pair-ext} as axioms, \emph{e.g.},
\code{eq fst pair(E1, E2) = E1 .} where \code{E1 E2 : Expression}.

Channel sets are simply sets of identifiers, standing for channel
names. Channel contexts are commutative lists of typed channel
names, written \code{c :: T}.

Reactions are introduced by the following constructors of the sort
\code{Reaction}, following the grammar for reactions. If \code{e} is an expression, \code{return e} is a reaction. If \code{d} is an identifier, standing for the name of a
distribution symbol, and \code{e} is an expression, \code{samp d < e >} 
is a reaction. If \code{c} is a channel name, \code{read c} is a reaction.
Moreover, we write \code{if e then R1 else R2} for branching, if \code{e} is an expression and \code{R1, R2} are reactions, and \code{x : T <- R1 ; R2} for binding, when \code{x} is an identifier, \code{T} is a type and 
\code{R1, R2} are reactions.
Typing of reactions $\Delta; \ \Gamma \vdash R : I \to \tau$ is given by a function
\code{typeOf : Signature ChannelContext TypeContext          
Set\{ChannelName\} Reaction -> Type }, 
with the meaning that
we compute the type of a reaction in the context given by a signature, a channel context, a type context and a set of inputs, 
i.e. $\code{Delta}; \ \code{Gamma} \vdash \code{R} : \code{I} \to \code{T}$ 
if and only if 
\code{typeOf(Sigma, Delta, Gamma, I, R) = T}, where \code{Sigma} is 
current signature. 
Maude allows us to write the typing
judgements in a very similar way to their original formulation, \emph{e.g.}, the typing rule for binding is written as
\begin{lstlisting}

 ceq typeOf(Sigma, Delta, Gamma, I, x : T1 <- R1 ; R2) = 
     typeOf(Sigma, Delta, Gamma (x : T1), I, R2)
     if typeOf(Sigma, Delta, Gamma, I, R1) == T1 .

\end{lstlisting}

Protocols also follow the grammar for protocols, using the following
constructors for the sort \code{Protocol}. For the empty protocol we write
\code{emptyProtocol}. If \code{c} is a channel name and \code{R} is a 
reaction, \code{c ::= R} is a protocol. If \code{P1, P2} are protocols,
so is \code{P1 || P2}. Finally. \code{new c : T in P} is a protocol,
if \code{c} is a channel name, \code{T} is a type and \code{P} is a protocol. Typing of protocols $\Delta \vdash P : I \to O$ is implemented as a predicate
\code{typeOf : Signature ChannelContext Set{ChannelName} Protocol -> Bool}, with the meaning that the protocol typechecks 
in the context given by a signature, a channel context and a set of inputs.
Note that since the set of outputs can be computed from a protocol, we
do not add it as a parameter of the type checking predicate, so we will have that
$\code{Delta} \vdash \code{P} : \code{I} \to \code{getOutputs(P)}$ 
if and only if
\code{typeOf(Sigma, Delta, I, P)},
where \code{Sigma} is the current signature and
\code{getOutputs} computes the outputs of \code{P}.
The implementation splits the typechecking into a check that the inputs are valid w.r.t. \code{Delta} and a recursive function that does the rest of typechecking:

\begin{lstlisting}
eq typeOf(Sigma, Delta, I, A, P) = 
    validChanSet I Delta A 
    and 
    typeOfAux(Sigma, Delta, I, A, P)
\end{lstlisting}

For example, the typing rule for \code{new} checks that \code{c} is new and that \code{P} typechecks
when extending \code{Delta} with the typed channel \code{ c :: T}:
\begin{lstlisting}
 eq typeOfAux(Sigma, Delta, I, new c : T in P) =
         not occurs c Delta
     and typeOfAux(Sigma, Delta (c :: T), I, P) .   
\end{lstlisting}


\subsection{Exact equality}



At the reaction level, exact equality is given with axioms of the form
$\Delta; \ \Gamma \vdash R_1 = R_2 : I \to \tau$. 
Let us consider the following example:

$\code{(A :: bool) (B :: bool)}; \code{empty} \vdash$

$\code{x : bool <- return True  ; if x then read A else read B} = \code{read A} : \code{\{A, B\}} \to \code{bool}$. 

The proof of this is obtained by applying the \textsc{trans} axiom to

$\code{(A :: bool) (B :: bool)}; \code{empty} \vdash$

$\code{x : bool <- return True  ; if x then read A else read B} = 
$

$\code{if True then read A else read B} : \code{\{A, B\}} \to \code{bool}$

\noindent that we prove by \textsc{ret-bind} and

$\code{(A :: bool) (B :: bool)}; \code{empty} \vdash \code{if True then read A else read B} = \code{read A} : \code{\{A, B\}} \to \code{bool}$

\noindent that we prove by \textsc{if-left}. 

From a practical point of view, it is inconvenient to write this proof in this way, because we have to make explicit all intermediate steps, 
and this is tedious and error-prone. Instead, we will work with a
transition system. Its states are \emph{configurations} containing the context, i.e. the current signature \code{Sigma, Delta, Gamma, I, T},
and the current reaction: \code{rConfig(Sigma, Delta, Gamma, R, I, T)}.
The transitions in the system are determined by rewrite rules, 
which are obtained by orienting the axioms of the exact equality calculus
from left to right. Since we can apply the \textsc{sym} axiom,
the choice of direction is not important.
For example,
the \textsc{if-left} axiom becomes
\begin{lstlisting}
crl [if-left] : 
     rConfig(Sigma, Delta, Gamma, if True then R1 else R2, I, A, T) 
     =>  
     rConfig(Sigma, Delta, Gamma, R1, I, A, T)
 if
     typeOf(Sigma, Delta, Gamma, I, A, R1) == T
     /\
     typeOf(Sigma, Delta, Gamma, I, A, R2) == T 
 .
\end{lstlisting}

We also employ the Maude strategy language to conveniently write 
application of the \textsc{trans} axiom as rule composition, denoted \code{;}. The proof in the example above becomes
\begin{lstlisting}
srew 
 rConfig(emptySig, (A :: bool) (B :: bool), emptyTypeCtx, 
         x : bool <- return True  ; if x then read A else read B, 
         (A, B), bool)
using ret-bind ; if-left .
\end{lstlisting}
\noindent and Maude returns the following result
\begin{lstlisting}
Solution 1
rewrites: 28 in 0ms cpu (0ms real) (~ rewrites/second)
result ReactionConfig: 
 rConfig(emptySig, (A :: bool) (B :: bool), emptyTypeCtx, 
         read A, (A , B), bool)
\end{lstlisting}

If the condition of a rule is a rewrite, we will need to explicitly
provide a sub-proof for that step as well.
For example, the rule \textsc{cong-bind} is
\begin{lstlisting}
 crl [cong-bind] :
     rConfig(Sigma, Delta, Gamma, x : T1 <- R1 ; R2, I , A, T2) 
     => 
     rConfig(Sigma, Delta, Gamma, x : T1 <- R3 ; R4, I, A, T2) 
     if
     rConfig(Sigma, Delta, Gamma, R1, I, A, T1)  
     => 
     rConfig(Sigma, Delta, Gamma, R3, I, A, T1) 
     /\
     rConfig(Sigma, Delta, Gamma (x : T1), R2, I, A, T2)
     => 
     rConfig(Sigma, Delta, Gamma (x : T1), R4, I, A, T2) .
\end{lstlisting}
\noindent and we can apply it to rewrite the reaction
\code{x : bool <- if True then read A else read B ; return x}
to
\code{x : bool <- read A ; return x}
by writing
\code{cong-bind\{if-left, idle\}}.

The \textsc{if-ext} axiom has the particularity that 
it establishes an equality between reactions 
where a variable has been substituted with a term.
Maude cannot apply this rule, because it cannot do the matching.
For this reason, we have omitted this rule and replaced it with
several rules that we can prove using \textsc{if-ext}. We have also
introduced an alpha-renaming rule for convenience, as we can also derive it
from the exact equality axioms.

The same principle is applied for exact equality of protocols.
This time we rewrite protocol configurations, of the form
\code{pConfig(Sigma, Delta, P, I, O)}.
The rules of exact equality for protocols may make use of
exact equality of reactions. For example, the \textsc{cong-react} rule
is
\begin{lstlisting}
 crl [CONG-REACT] : 
     pConfig(Sigma, Delta (c :: T), c ::= R, I, c) 
     =>
     pConfig(Sigma, Delta (c :: T), c ::= R', I, c)
     if
     rConfig(Sigma, Delta (c :: T), emptyTypeCtx, R, 
             insert(c, I), T)
     =>
     rConfig(Sigma, Delta (c :: T), emptyTypeCtx, R', I', T) 
     /\ I' == insert(c, I)
     /\ not c in I .
\end{lstlisting}

\subsection{Normal Forms}

We work with protocols that start with a list of declarations of 
internal channels, using $\code{new}$, followed by a parallel compositions of channel assignments. The reactions in these
assignments can be transformed into a list of binds of the form
\code{x : T <- read c},
called bind-read reactions, followed by a reaction without binds.
The list of binds can be regarded as commutative, 
as two reactions with the same list of binds in different order
are equivalent due to the reaction equivalence rule \textsc{exch}.
Similarly, different order of declarations of internal channels gives
equivalent protocols, by using the protocol equivalence rule
\textsc{new-exch}. When writing equivalence proofs, we do not want to 
make the use of these rules explicit. Instead, we want to be able to apply
the rules as though we could freely consider a certain declaration of an 
internal channel or a certain bind read reaction as the first.

Therefore, we introduce normal forms of reactions and protocols. 
For reactions, normal forms 
\code{nf(L, R, O)}
consist of a commutative list \code{L} of bind-read reactions,
a bind-free reaction \code{R}
and a chosen order \code{O} of the names of the variables occuring in
the binds in \code{L}, given as a list of names.
The latter will be used to determine how to turn the normal form 
of a reaction into a regular reaction.
For example, the normal form of
\begin{lstlisting}
 'd : bool <- read 'ce ;
 'm0 : bool <- read 'in0 ; 
 'm1 : bool <- read 'in1 ;
 'k0 : bool <- read 'key0 ;
 'k1 : bool <- read 'key1 ;
  if 'd then return 'k0 else return 'k1
\end{lstlisting}
is
\begin{lstlisting}
 nf(
  ('d : bool <- read 'ce)
  ('m0 : bool <- read 'in0) 
  ('m1 : bool <- read 'in1)
  ('k0 : bool <- read 'key0)
  ('k1 : bool <- read 'key1),
  if 'd then return 'k0 else return 'k1,
  'd :: 'm0 :: 'm1 :: 'k0 :: 'k1
 )
\end{lstlisting}
 
During equivalence proofs, we may obtain 
in a normal form \code{nf(L, R, O)}
either arbitrary binds in \code{L} 
(\emph{e.g.}, by substituting a read from a channel with the reaction
assigned to that channel)
or reactions \code{R} that are not bind-free.
This will be represented as a pre-normal-form, 
written 
\code{preNF(L, R, O)}, 
which is a normal form without restrictions on the occuring reactions.
If \code{L} contains a bind that is not a read bind, we will write it
as \code{x1 : T1 <$\sim$ R1}. 
The general strategy will be to transform pre-normal-forms 
\code{preNF(L, R, O)}
to normal forms using the following steps:
\begin{itemize}
\item if \code{x1 : T1 <$\sim$ R1} is in \code{L} and \code{R1} is of the form
\code{nf(L2, R2, O2)}, move the inner binds from \code{L1} at the level of \code{L}, and simplify the reaction of \code{x1} to \code{R2}.
\item if  \code{x1 : T1 <$\sim$ R1} is in \code{L} and \code{R1} is bind-free,
rewrite the entire reaction as 
\code{preNF(L', x1 : T1 <- R1 ; R, O')}, where \code{L'} and \code{O'}
are obtained by removing \code{x1 : T1 <$\sim$ R1} and \code{x1} from \code{L} and \code{O}, respectively.
\item apply reaction-level axioms to \code{R} to bring it in the form
\code{L' ; R'}, where \code{L'} is a list of bind reads and \code{R'} is
bind-free, then move \code{L'} at the outer level of \code{L}.
\end{itemize}    

At the level of protocols, normal forms  
\code{newNf(L, P, O)}
consist of a commutative list \code{L} of declarations of internal
channels, a protocol \code{P} that does not start with internal channel declarations
and again a designated order \code{O} 
for the names of internal channels occuring in the declarations in 
\code{L}.
For example, the normal form of
\begin{lstlisting}
new 'ce : bool in 
new 'key0 : bool in 
new 'key1 : bool in 
new 'flip : bool in 
new 'choice : bool in
 (
 ('ce ::= 'f : bool <- read 'flip ;
          'c : bool <- read 'choice ;
          if 'f then 
           (if 'c then return False else return True)
          else 
           (if 'c then return True else return False)
 )
 || ('msgenc0 ::= 'd : bool <- read 'ce ;
                  'm0 : bool <- read 'in0 ; 
                  'm1 : bool <- read 'in1 ;
                  'k0 : bool <- read 'key0 ;
                  'k1 : bool <- read 'key1 ;
                  if 'd then return 'k0 else return 'k1)
  || ('key0 ::= return True) 
  || ('key1 ::= return False)
  || ('flip ::= return True) 
  || ('choice ::= return False) 
)
\end{lstlisting}
\noindent is
\begin{lstlisting}
newNF(
 ('ce : bool) ('key0 : bool) ('key1 : bool)
 ('flip : bool) ('choice : bool),
 
 ('ce ::= 'f : bool <- read 'flip ;
          'c : bool <- read 'choice ;
          if 'f then 
           (if 'c then return False else return True)
          else 
           (if 'c then return True else return False)
 )
 || ('msgenc0 ::= 'd : bool <- read 'ce ;
                  'm0 : bool <- read 'in0 ; 
                  'm1 : bool <- read 'in1 ;
                  'k0 : bool <- read 'key0 ;
                  'k1 : bool <- read 'key1 ;
                  if 'd then return 'k0 else return 'k1)
  || ('key0 ::= return True) 
  || ('key1 ::= return False)
  || ('flip ::= return True) 
  || ('choice ::= return False),
  
  'ce :: 'key0 :: 'key1 :: 'flip :: 'choice
)
\end{lstlisting}

\subsection{Families of protocols}

Families of protocols provide a convenient abbreviation for 
semantically related protocols \code{P[0]...P[n]}, where 
the value of \code{n} is typically not known. The semantical relation
translates in the protocols being assigned similar reactions.
We illustrate the syntax with the help of an example:
\begin{lstlisting}
(family 'SumCommit 'i (bound (n + 2)) ::= 
        (when ('i =T= 0) --> nf(emptyBRList, return False, emptyCNameList)) ;; 
        (when ('i =T= (n + 2)) --> 
               nf( ('x : bool <- read ('SumCommit [ n + 1 ])) 
                   ('f : bool <- read 'LastCommit) ,
                   return (ap 'xor pair('x, 'f)) ,
                   'x :: 'f :: emptyCNameList )
        )          ;;
        (otherwise --> nf(('x : bool <- read ('SumCommit ['i -- 1])) 
                          ('c : bool <- read ('Commit ['i -- 1])),
                          return (ap 'xor pair('x, 'c)),
                          'x :: 'c :: emptyCNameList )
        )                  
    )
\end{lstlisting}    
Here \code{'i} is an index 
variable ranging between \code{0} and \code{n + 2}.
The reaction assigned to the protocol \code{'SumCommit['i]} is given
with alternatives. 
We allow the bound to be a natural number, an identifier denoting a 
natural number or an expression involving natural numbers and
identifiers. 
We represent this as a sort \code{NatTerm} that is a super-sort of
\code{Qid} and \code{Nat} together with 
addition, deletion (written \code{--}) and multiplication on that sort,
extending in the expected way the corresponding operations on natural
numbers. The conditions occuring in the alternatives are of sort
\code{BoolTerm}, and can be comparisons between \code{NatTerm}s
(\code{=T=, <T, <=T}),
user-defined predicates (\code{apply 'p t}) where \code{'p} is 
the name of the predicate and \code{t} is a \code{NatTerm} or negation
of a \code{BoolTerm}.

In this new setting, we may introduce a channel directly or via a family
of protocols. 
Channel names, which were identifiers so far, must be extended to
indexed identifiers. We implement them as a sort
\code{ChannelName} which includes as a sub-sort the sort of identifiers
and  has a constructor \code{\_[\_] : Qid List\{NatTerm\} -> ChannelName},
and thus \code{'c['i 'j]} is an example of a channel name. 
Taking this into account, we
have also extended channel sets and channel contexts to keep track of the 
bound of a family of protocols. Channel sets are implemented as
sets of bounded channel names, which are written \code{c @ l }, where \code{c} is an identifier and 
\code{l} is a list of bounds for the family named
\code{c}. We use an empty list for a regular protocol, with no indices. Channel contexts are commutative lists of 
typed bounded channel names, written 
\code{ c @ l :: T}, where \code{T} is a type.

We allow families of protocols with two indices as well. 
We write \code{family 'F ('i 'j) ((bound m) (uniformBound n))}
for a family indexed by \code{'i} ranging from \code{0} to
\code{m} and by \code{'j} ranging from \code{0} to
\code{n}. We also allow the second bound to vary for each \code{'i},
but we did not use this in the case studies so far.

The equality calculus must be adapted to the new notation. We
have introduced rules that apply the core equality rules over the new
notation, with the meaning that the rules are applied in parallel for
each index.
We need to record the assumptions made about indices, so we extend
the protocol configuration with a new component of sort
\code{Set\{BoolTerm\}}.
 Moreover, we have a rule for induction proofs. 
We present here the variant for one index, 
as the one for two indices is similar.
The goal is to rewrite
\code{ P || family C q (bound nt1) ::= cases} 
to 
\code{ P || family C q (bound nt1) ::= cases'} 
by induction on the index
\begin{lstlisting}
crl [INDUCTION-when-one] :
     pConfig(Sigma, Delta, 
       P || (family C q (bound nt1) ::= cases), I, O, A)
     =>
     pConfig(Sigma, Delta , 
       P || (family C q (bound nt1) ::= cases'), I, O, A)
\end{lstlisting} 
We start with the base case and we must provide a proof that
if we assign \code{C[0]} its corresponding case from \code{cases}
with \code{q = 0}, we can rewrite the resulting protocol to the protocol
that we get by assigning \code{C[0]} its corresponding case from \code{cases'} with \code{q = 0}. We record the assumption \code{q = 0} in
the set of index assumptions 
\code{A}. We also need to update the current outputs, by removing the outputs of the family \code{C} and adding \code{C[0]}:
      
\begin{lstlisting}
     if 
     pConfig(Sigma, Delta , 
       P || (projectIndex (family C q (bound nt1) ::= cases) 0 A empty ), I, 
       insert( C[0] @ nil, O \ (C @ nt1)), 
       insert(q =T= 0, A)
       ) 
     => 
     pConfig(Sigma, Delta , P2 , I, O', A')
     /\
     O' == insert( C[0] @ nil, O \ (C @ nt1))
     /\
     A' ==  insert(q =T= 0, A) 
     /\
     P2 == P || (projectIndex (family C q (bound nt1) ::= cases') 0 A empty)
\end{lstlisting}
The induction step assumes that we have successfully proven the 
property up to index \code{'k}, so now we can
make use of \code{family C q (bound 'k) ::= cases'} when proving the
property for index \code{'k + 1}, where \code{'k} is arbitrary.
We record in A the assumption that \code{'k + 1} must be in bounds. 
We also need to update the current outputs, by removing the outputs of the family \code{C} and adding \code{C[k + 1]} and the outputs of the family \code{C} with the new bound \code{k}:
\begin{lstlisting}     
     /\
     pConfig(Sigma, Delta , 
       P || (family C q (bound 'k) ::= cases') || 
       (projectIndex (family C q (bound nt1) ::= cases) ('k ++ 1) A empty), I, 
       insert(C @ 'k, insert(C['k ++ 1] @ nil, O \ (C @ nt1))), 
       insert('k ++ 1 <=T nt1, A)
       )
       =>
     pConfig(Sigma, Delta, P3, I, O'', A'')
     /\
     O'' = insert(C @ 'k, insert(C['k ++ 1] @ nil, O \ (C @ nt1)))
     /\
     A'' == insert('k ++ 1 <=T nt1, A) 
     /\
     P3 == (
             P || (family C q (bound 'k) ::= cases') ||
             (projectIndex (family C q (bound nt1) ::= cases') ('k ++ 1) A empty)
           )
       [nonexec] . 
\end{lstlisting}

A strategy will call the induction rule using
\begin{lstlisting}
INDUCTION-when-one[
         C:Qid <- Q, 
         cases':Cases <- 'the cases that we want to get after the induction proof'
        ]
        { 'proof of induction base',
          'proof of induction step'        
        }
\end{lstlisting}

\subsection{Strategies}

We now come to the strategies that will appear in proofs. It is possible that some of them will make use of other substrategies, but as these will not be in use, we refrain from including them here. To ease presentation,
we group strategies by the main core rule that is applied. They may 
have several forms to be applied in different contexts  
\emph{e.g.}, channels, families, groups of families, cases.

The rules \textsc{refl}, \textsc{trans}, \textsc{axiom} and
\textsc{embed} are not explicitly applied, as they are implied by
the properties of rewrite relation in Maude. 
The rule \textsc{sym} does not require the use of a strategy, and
in order to apply it we must specify explicitly the protocol that we 
rewrite from. More precisely, if the current protocol is \texttt{P}, 
we write \texttt{SYM[P1:Protocol <- P']\{proof\}} where 
\texttt{proof} is an
exact equality proof rewriting \texttt{P'} to \texttt{P}. 
The rule
\textsc{input-unused} is embedded in the application of other rules,
in the way the conditions on inputs are given.
The rules \textsc{cong-react}, \textsc{cong-new} and \textsc{cong-comp}
are applied inside the strategy definition, and their usage is not visible
to the user. The rules \textsc{cong-comm} and
\textsc{cong-assoc} are not applied explicitly, as
it suffices to specify the parallel composition in Maude as a commutative
and associative operation.

\subsubsection{SUBST}

Here we have the largest number of variations, because we need rules for
substituting a channel in a family, a family in a family taking into 
account whether they have one or two indices and so on. Since the number
of parameters varies, we cannot have a meta-strategy that tries all
possible variants. In the future we plan to generate the extra
arguments from the context where the rule applies, and thus reduce the arguments of all strategies to the name of the channel/family that gets
substituted and the the name of the channel/family where the 
substitution takes place. Thus, we will be able to introduce a
meta-strategy that greatly simplifies substitutions.

We have the following substitution strategies:
\begin{itemize}
 \item  \texttt{substNF(C1, C2)} substitutes the channel \texttt{C1} in \texttt{C2}. Both channels must be in normal form. The strategy also gets the pre-normal form resulting from the substitution to a normal form, as described above.
   \item  \texttt{substNFRead(C1,C2)} is a simpler particular case of substitution when the channel \texttt{C} that we substitute reads from another channel. Both channels must be in normal form.
   \item  \texttt{smart-subst-nf(C1, C2)} tries to apply \texttt{substNFRead} and if that fails, applies \texttt{substNF}. 
In the future we plan to plug all substitution strategies under this
meta-strategy.
   \item  \texttt{substNFFamiliesOne(C1, C2, R)} substitutes in the family \texttt{C2} with one index a read from \texttt{C1[i]} for some index \texttt{i} with the corresponding reaction \texttt{R} assigned in the family \texttt{C1} to the index \texttt{i}. The family \texttt{C2} must be in normal form.
   \item  \texttt{applyCaseDistSubst(q1, q2, q3, q4, pr)} works under the assumption that we have two groups, \texttt{q1, q2}, and \texttt{q1} is defined with cases. The rule does a \texttt{substNFRead} equivalent for the families \texttt{q3, q4} with two indices that come from the first branch of \texttt{q1} and from \texttt{q2}. The protocol \texttt{pr} is then used in a \textsc{sym} proof to redo the grouping.
   \item  \texttt{substChannelFamilyOne(C1, C2)} substitute a channel in a family with one index.
   \item  \texttt{applySubstChannelBranch(C1, q2)} substitutes the channel 
   \texttt{C1} in the family \texttt{C2}, in the first branch of a group defined with cases.
   \item  \texttt{applyCaseDistBranch2(q1, q2)} substitutes a channel in a family on the left branch of the right branch of a group.
   \item  \texttt{applyBranch2SubstRev(q1, q2, nt, x, T, R)} applies a reverse substitution on the left branch of a family. The parameters \texttt{nt, x, T, R} are the index, the name of the bind variable, the type and the reaction of the channel that is reversely substituted. 
   \item  \texttt{applySubstRevFamily(Q, C2, T)} does a reverse substitution on a branch of a family with cases. The parameter \texttt{T} is the type of the reaction that is reversely substituted.
   \item  \texttt{substNFReadFamilyOneChannel(C1, C2)} is a \texttt{substNFRead} equivalent for a family with one index and a channel.
   \item  \texttt{substNFReadFamilyTwoChannel(C1, C2)} is a \texttt{substNFRead} equivalent for a  family with two indices and a channel.
   \item  \texttt{substRevFamilyChannel(Q, C, nt, T)} does the reverse substitution of a family \texttt{Q} in a channel \texttt{C}. The parameters \texttt{nt, T} are the index and the type of the channel that is reversely substituted.
   \item  \texttt{substNFFamilyOneChannel(C1, C2, R)} is the \texttt{substNF} equivalent for a family with one index and a channel.
   \item  \texttt{applySubstNFLeft(q1, q2, R)} applies \texttt{substNFFamiliesOne} on the left branch of a family defined with cases.
\end{itemize}  

\subsubsection{DROP}

\begin{itemize}
\item \texttt{applyDropNF(C1, C2)} applies the normal form version of \textsc{drop}.
\item \texttt{applyDropPreNF(C1, C2)} applies the pre-normal form
version of \textsc{drop}.
\end{itemize}

\subsubsection{ABSORB}

\begin{itemize}
\item \texttt{absorbChannel(C)} applies the new-normal-form version of
\textsc{absorb} for the channel \texttt{C}.
\item \texttt{absorbFamily(Q)} applies the new-normal-form version
of \textsc{absorb} for the family \texttt{Q}.
\item \texttt{applyAbsorbReverse(P)} applies the reverse of the 
\textsc{absorb} rule for the protocol \texttt{P}.
\item \texttt{addNewFamilyToGroup(P, Q1, Q2)} adds the family \texttt{Q2},
introduced by the protocol \texttt{P}, to the group \texttt{Q1}.
\item \texttt{applyCaseDistAbsorb(q1, q2, q3, pr)} operates
under the assumption that the current protocol is of the form
\texttt{family q2 ::= P || family q1 ::= when cond1 --> P1 ;; otherwise --> P2} and applies the \textsc{absorb} rule on the protocol
\texttt{P || P1} then it reconstructs the original shape of the
current protocol.
\end{itemize}

\subsubsection{FOLD}

\begin{itemize}
\item \texttt{foldNF(C1, C2)} applies the normal form version of the 
\textsc{fold} rule, when the channel \texttt{C1} gets folded in the 
channel \texttt{C2}.
\item \texttt{foldNFPre(C1, C2)} applies the pre-normal form version of the 
\textsc{fold} rule, when the channel \texttt{C1} gets folded in the 
channel \texttt{C2}.
\item \texttt{foldNFFamily(Q1, Q2)} applies the normal form version 
of the 
\textsc{fold} rule, when the family \texttt{Q1} gets folded in the 
family \texttt{Q2}.
\end{itemize}

\subsubsection{READ-INSIDE-IF}

This is a rule derived from \textsc{if-ext} and allows us to rewrite
\texttt{x : T1 <- read i ; if M then R1 else R2} to 
\texttt{if M then x : T1 <- read i ; R1 else x : T1 <- read i ; R2}.

\begin{itemize}
\item \texttt{applyReadInsideIfPre(C)} applies 
\textsc{read-inside-if} to a protocol in new-normal-form.
\end{itemize}

\subsubsection{Purely syntactic transformations}

Under this heading we group a number of strategy that change
only the shape of a protocol. The rules that apply are
derivable from the core rules.

\begin{itemize}
\item \texttt{applyAddToGroup(Q1, Q2)} moves the family 
\texttt{Q2} inside the group \texttt{Q1}.
\item \texttt{changeOrder(C, ql)} changes the specified order
of reads in a normal or pre-normal form. \texttt{C} is the name of 
the channel that is assigned the (pre-)normal form and
\texttt{ql} is the new order, given as a list of variable names.
\item \texttt{applyReorderNF(Q, ql)}: on the first branch of the family
\texttt{Q}, changes the order in the normal form as specified in 
\texttt{ql}.
\item \texttt{nf2PreNF(C)} turn the normal form assigned to the 
channel \texttt{C} to a pre-normal form
\item \texttt{applyGroupFamilies(Q1, Q2)} composes the families 
\texttt{Q1, Q2} to a new family \texttt{'Comp[Q1 Q2]} that assigns to
each index \texttt{i} the protocol
\texttt{(Q1[i] ::= R1) || (Q2[i] ::= R2} where \texttt{R1, R2} are
the reactions assigned to the index \texttt{i} by \texttt{Q1, Q2}.
This transformation is needed for some induction proofs.
\item \texttt{applyUngroupFamilies(Q1, Q2)} is the reverse transformation
of the previous strategy.
\item \texttt{moveProtocolUnderNewNF} if the current protocol is the parallel composition of a protocol \texttt{P} with a new-normal-form, move
\texttt{P} inside the new-normal-form.
\item \texttt{applyDeleteEmptyNF(Q)} if the new-normal-form assigned to the
group \texttt{Q} has no new declarations, keeps only its protocol.
\item \texttt{applyDropName(Q)} removes the group name \texttt{Q}.
\item \texttt{applyCombine(Q)} if the group \texttt{Q} is defined using cases, removes the group name and moves the cases inside the families of the group.
\item \texttt{applyAlphaNFPr(C, QL)} does an alpha-renaming of the bind
variables of a normal form assigned to the channel \texttt{C}.
 \texttt{QL} specifies the
renaming.
\item \texttt{applyBranch2Alpha(q1, QL)} does an alpha-renaming on the otherwise branch of a family \texttt{q1}. \texttt{QL} specifies the
renaming.
\item \texttt{moveBindInPre(C, Q)} if the channel \texttt{C} is assigned
a normal form, move the bind assigned to the variable \texttt{Q} in the
reaction of the normal form, and turn the normal form to a pre-normal form.
\item \texttt{applyBranch2MoveReads(q1, ql)} moves the reads 
specified by the list \texttt{ql} from bind list of a normal form to the reaction of the normal form on the left branch of a family 
\texttt{q1}.
\item \texttt{moveReadsToRFamily(C,cnl)} move the reads specified in \texttt{cnl} from the bind list of a normal form to the reaction of the normal form assigned to the family \texttt{C}.

\end{itemize}  

\subsection{A simple example}

In our simplest example, the \ipdl \emph{Hello World} analogue, Alice receives a Boolean message, encodes it by xor-ing it with a randomly generated Boolean, and leaks the encoding to the adversary. We show that this is equal to leaking a randomly generated ciphertext.

Formally, our signature consists of two symbols: $\oplus : \Bool \times \Bool \to \Bool$ for the Boolean sum, and $\flip : \one \twoheadrightarrow \Bool$ for the uniform distribution on Booleans. We write $x \oplus y$ in place of $\oplus \ (x,y)$.

\subsubsection{The Assumptions}
Our single axiom states that $\flip$ is invariant under xor-ing with a fixed Boolean:
\begin{itemize}
\item $\cdot; \ x : \Bool \vdash (y \leftarrow \flip; \ \ret{x \oplus y}) = \samp{\flip} : \emptyset \to \Bool$
\end{itemize}
This is indeed the case if (and only if) $\flip$ is uniform.

\subsubsection{The Ideal Functionality}
Upon receiving the input message, the ideal functionality generates a random ciphertext on an internal channel $\Ctxt$ and leaks its value to the adversary:
\begin{itemize}
\item $\Ctxt \coloneqq m \leftarrow \In; \samp{\flip}$ 
\item $\LeakCtxt^\id_\adv \coloneqq \read{\Ctxt}$
\end{itemize}

\subsubsection{The Real Protocol}
In the real protocol, Alice generates a random Boolean key on an internal channel $\Key$, constructs the ciphertext by xor-ing the input message with the key, and leaks the resulting ciphertext to the adversary:
\begin{itemize}
\item $\Key \coloneqq \samp{\flip}$
\item $\Ctxt \coloneqq m \leftarrow \In; \ k \leftarrow \Key; \ \ret{m \oplus k}$
\item $\LeakCtxt^\Alice_\adv \coloneqq \read{\Ctxt}$
\end{itemize}

\subsubsection{The Simulator}
The simulator mediates between the two leakage channels $\LeakCtxt^\id_\adv$ and $\LeakCtxt^\Alice_\adv$ by forwarding the former to the latter:
\begin{itemize}
\item $\LeakCtxt^\Alice_\adv \coloneqq \read{\LeakCtxt^\id_\adv}$
\end{itemize}

\subsubsection{Real = Ideal + Sim}
On the left-hand side of the above equality we have the real protocol. On the right-hand side, we have the composition of the ideal functionality with the simulator, followed by the hiding of the channel $\LeakCtxt^\id_\adv$. The two protocols now have identical inputs (the channel $\In$) as well as outputs (the channel $\LeakCtxt^\Alice_\adv$).

We now simplify both protocols so that they have the same internal structure. On the left-hand side, we fold the internal channel
\begin{itemize}
\item $\Key \coloneqq \samp{\flip}$
\end{itemize}
into the channel
\begin{itemize}
\item $\Ctxt \coloneqq m \leftarrow \In; \ k \leftarrow \Key; \ \ret{m \oplus k}$,
\end{itemize}
yielding
\begin{itemize}
\item $\Ctxt \coloneqq m \leftarrow \In; \ k \leftarrow \flip; \ \ret{m \oplus k}$
\end{itemize}
and on the right-hand side we fold the internal channel
\begin{itemize}
\item $\LeakCtxt^\id_\adv \coloneqq \read{\Ctxt}$
\end{itemize}
into the channel
\begin{itemize}
\item $\LeakCtxt^\Alice_\adv \coloneqq \read{\LeakCtxt^\id_\adv}$,
\end{itemize}
yielding
\begin{itemize}
\item $\LeakCtxt^\Alice_\adv \coloneqq \read{\Ctxt}$.
\end{itemize}
The two protocols now both have an internal channel $\Ctxt$ and an output channel $\LeakCtxt^\Alice_\adv$.

To finish the proof, we fold the internal channel into the output channel in both protocols, yielding the two single-reaction protocols
\begin{itemize}
\item $\LeakCtxt^\Alice_\adv \coloneqq m \leftarrow \In; \ k \leftarrow \flip; \ \ret{m \oplus k}$, and
\item $\LeakCtxt^\Alice_\adv \coloneqq m \leftarrow \In; \ \samp{\flip}$
\end{itemize}
The equality between these two now follows immediately from our axiom, and we are done.

\subsubsection{Maude implementation}

Assume we work in the file \code{helloWorld.maude} placed in the 
\code{lib} folder of the IPDL-Maude repository. We start by importing
the strategies and starting a new module that extends \code{APPROX-EQUALITY}, which provides both exact and approximate equality
\begin{lstlisting}
load ../src/strategies

mod HELLO-WORLD is
 protecting APPROX-EQUALITY .

\end{lstlisting}

We have to define the signature.
Our example works with booleans, 
so we will not introduce new datatypes. 
When needed, we define them as new constants of sort \code{Type}.

We must introduce a function symbol for $\oplus$ and 
a distribution symbol for $\flip$:
\begin{lstlisting}
op xorF : -> SigElem .
eq xorF = 'xor : (bool * bool) ~> bool .

op flipF : -> SigElem .
eq flipF = 'flip : unit ~>> bool .

op sig : -> Signature .
eq sig = xorF flipF .
\end{lstlisting}
We then write the protocol resulting from composing the ideal functionality
with the simulator followed by hiding the channel 
\code{'LeakCtxt\_id\_adv}
and the protocol \code{real}:
\begin{lstlisting}
 op idealPlusSim : -> Protocol .
 eq idealPlusSim = 
    new 'Ctxt : bool in
    new 'LeakCtxt_id_adv : bool in
    (
    ('LeakCtxt_id_adv ::= 
       nf(('c : bool <- read 'Ctxt),
          return 'c,
          'c :: emptyCNameList )
    )  
    ||
    ('LeakCtxt_Alice_adv ::= 
       nf('c : bool <- read 'LeakCtxt_id_adv,
          return 'c, 
          'c :: emptyCNameList)
    )
    ||
    ('Ctxt ::= 
       nf( 'm : bool <- read 'In,
           samp ('flip < () >),
           'm :: emptyCNameList)
    )
    )
 .
 
  op real : -> Protocol .
  eq real = 
    new 'Key : bool in
    new 'Ctxt : bool in
    (
     ('Key ::= samp ('flip < () >))
     || 
     ('Ctxt ::= 
       nf( ('m : bool <- read 'In)
            'k : bool <- read 'Key,
            return (ap 'xor pair('m, 'k)),
           'm :: 'k :: emptyCNameList
               )
     )
     ||
     ('LeakCtxt_Alice_adv ::= 
       nf( 'c : bool <- read 'Ctxt ,
           return 'c,
          'c :: emptyCNameList )  
     )
    ) 
         
  .
\end{lstlisting}

We then close the \code{HELLO-WORLD} module and
open a new module, \code{EXECUTE}, importing \code{HELLO-WORLD} and
\code{STRATEGIES}, where we add the assumptions and typically strategies
for using them in proofs
\begin{lstlisting}
smod EXECUTE is
  protecting STRATEGIES .
  protecting HELLO-WORLD .
\end{lstlisting}

The assumption is introduced at the level of reactions
\begin{lstlisting}
rl [assumption] :
    rConfig(Sigma, Delta, Gamma (x : bool),
            y : bool <- samp flip ; 
            return (ap 'xor pair(x, y)), I, A, bool
            )
     => 
     rConfig(Sigma, Delta , Gamma (x : bool),
             samp flip, I, A, bool)
  .  
\end{lstlisting}
\noindent and we will want to apply it in the 
main reaction of a pre-normal form, so we need the following strategy:
\begin{lstlisting}
strat applyAssumption : ChannelName @ ProtocolConfig .
  sd applyAssumption(cn) := 
   match pConf s.t. startsWithNew pConf
    ? CONG-NEW-NF{applyAssumption(cn)}
    : matchrew pConf s.t. pConfig(Sigma, Delta, P, I, O, A) := pConf by pConf 
       using CONG-REACT[o:ChannelName <- cn]
                { cong-pre-nf{assumption} ; 
                  try (pre2Nf)}   
  .
\end{lstlisting}
\noindent which states, read in reverse order, that we apply 
\textsc{cong-pre-nf} with the assumption as a sub-proof 
to the reaction \code{R} that is assigned to the channel \code{cn}
(thus requiring to apply \textsc{cong-react}) 
that is inside a new normal form, and then we must apply
\code{cong-new-nf}. Moreover, \code{pre2Nf} converts the
pre-normal form to a normal form, as now we no longer have binds in the
reaction of the pre-normal form.

We now get to writing the proof. The initial configuration is
\begin{lstlisting}
pConfig(sig, 
        ('In @ nil :: bool) 
        ('LeakCtxt_Alice_adv @ nil :: bool), 
        idealPlusSim, 
        'In @ nil, 
        getOutputs(idealPlusSim), 
        empty)
\end{lstlisting}
\noindent where 
\begin{itemize}
 \item \code{sig} is the signature defined above,
 \item the channel context contains the input channel \code{'In}
       and the output channel \code{'LeakCtxt\_Alice\_adv}. Both are of type
       \code{bool} and they have no indices,
 \item the protocol that we want to rewrite,
 \item the set of its inputs, here we have just one input channel,      
 \item the set of its outputs, which can be computed, so it is more
       convenient to call the function \code{getOutputs} than to
       enumerate all inputs,
 \item since we have no indices, the set of assumptions about them is empty.      
\end{itemize}

We first turn the protocol in new normal form, then we do the two folds:
\begin{lstlisting}
srew [1] 
pConfig(sig, 
         ('In @ nil :: bool) 
         ('LeakCtxt_Alice_adv @ nil :: bool), 
         idealPlusSim, 
         'In @ nil, 
         getOutputs(idealPlusSim), 
         empty)
using 
       sugar-newNF
     ; foldNF('LeakCtxt_id_adv, 'LeakCtxt_Alice_adv) 
     ; foldNF('Ctxt, 'LeakCtxt_Alice_adv)    
.

\end{lstlisting}

Similarly, for \code{real}, we turn the protocol in new normal form,
we do the folds and then we apply the assumption:
\begin{lstlisting}
srew [1] 
 pConfig(sig, 
         ('In @ nil :: bool) 
         ('LeakCtxt_Alice_adv @ nil :: bool), 
         real, 
         'In @ nil, 
         getOutputs(real), 
         empty)
using 
       sugar-newNF
     ; foldNF('Key, 'Ctxt)
     ; foldNF('Ctxt, 'LeakCtxt_Alice_adv)
     ; applyAssumption('LeakCtxt_Alice_adv)
. 
\end{lstlisting}

In both cases we get the same result, so we now can combine the two proofs in one, using \textsc{sym}:
\begin{lstlisting}
srew [1] 
 pConfig(sig, 
         ('In @ nil :: bool) 
         ('LeakCtxt_Alice_adv @ nil :: bool), 
         real, 
         'In @ nil, 
         getOutputs(real), 
         empty)
using 
       sugar-newNF
     ; foldNF('Key, 'Ctxt)
     ; foldNF('Ctxt, 'LeakCtxt_Alice_adv)
     ; applyAssumption('LeakCtxt_Alice_adv)
     ; SYM[P1:Protocol <- idealPlusSim]{
       sugar-newNF
     ; foldNF('LeakCtxt_id_adv, 'LeakCtxt_Alice_adv) 
     ; foldNF('Ctxt, 'LeakCtxt_Alice_adv)  
     }
.  
\end{lstlisting}

When running the proof in Maude, we get \code{idealPlusSim} in the 
protocol in the resulting \code{pConfig}, as expected. Maude also reports
on the number of rewrites and the time spent doing the proof:´
\begin{lstlisting} 
$  maude lib/helloWorld.maude 
		     \||||||||||||||||||/
		   --- Welcome to Maude ---
		     /||||||||||||||||||\
	    Maude 3.2.1 built: Feb 21 2022 18:21:17
	     Copyright 1997-2022 SRI International
		   Tue Mar 14 04:52:45 2023
==========================================
srewrite [1] in EXECUTE : pConfig(sig, ('In @ nil :: bool)
    'LeakCtxt_Alice_adv @ nil :: bool, real, 'In @ nil, getOutputs(real),
    empty) using sugar-newNF ; foldNF('Key, 'Ctxt) ; foldNF('Ctxt,
    'LeakCtxt_Alice_adv) ; applyAssumption('LeakCtxt_Alice_adv) ; SYM[
    P1:Protocol <- idealPlusSim]{sugar-newNF ; foldNF('LeakCtxt_id_adv,
    'LeakCtxt_Alice_adv) ; foldNF('Ctxt, 'LeakCtxt_Alice_adv)} .

Solution 1
rewrites: 516 in 0ms cpu (1ms real) (~ rewrites/second)
result ProtocolConfig: pConfig(('xor : bool * bool ~> bool) 'flip : unit ~>>
    bool, ('In @ nil :: bool) 'LeakCtxt_Alice_adv @ nil :: bool, new 'Ctxt :
    bool in new 'LeakCtxt_id_adv : bool in ('Ctxt ::= nf('m : bool <- read
    'In, samp ('flip < () >), 'm :: emptyCNameList)) || ('LeakCtxt_Alice_adv
    ::= nf('c : bool <- read 'LeakCtxt_id_adv, return 'c, 'c ::
    emptyCNameList)) || 'LeakCtxt_id_adv ::= nf('c : bool <- read 'Ctxt,
    return 'c, 'c :: emptyCNameList), 'In @ nil, 'LeakCtxt_Alice_adv @ nil,
    empty)

\end{lstlisting}

\subsection{Approximate equality}

We start by defining wrappers for width and length
\begin{lstlisting}  
  sort Width .
  sort Length .
  
  op width_ : Nat -> Width [ctor] .
  op length_ : Nat -> Length [ctor] .
\end{lstlisting}  
\noindent and the measure functions  
\begin{lstlisting}
  op |_| : Protocol -> Nat .
  op |_| : Reaction -> Nat .
  op |_| : Expression -> Nat .
\end{lstlisting}

The configurations \code{aConfig(Sigma, Delta, P, I, O, A, w, l)}
for approximate equality
extend protocol configurations with fields for width and length.
The main idea is that we set these at 0 at the start of a proof,
and we keep track of their modifications with the help of the 
approximate equality rules.
The \textsc{strict} rule allows us to apply exact equality calculus
without modifying the width and the length
\begin{lstlisting}
 crl [STRICT] :
   aConfig(Sigma, Delta, P, I, O, A, w, l) 
   =>
   aConfig(Sigma, Delta, Q, I, O, A, w, l)
 if
  pConfig(Sigma, Delta, P, I, O, A)
  => 
  pConfig(Sigma, Delta, Q, I, O, A)
 . 
\end{lstlisting}
The \textsc{trans} rule
\begin{lstlisting}
crl [TRANS] :
  aConfig(Sigma, Delta, P, I, O, A, width nw, length nl)
  => 
  aConfig(Sigma, Delta, P2, I, O, A, width (nw + nw1 + nw2), length (nl + defMax(nl1, nl2)))
 if
  aConfig(Sigma, Delta, P, I, O, A, width 0, length 0)
  => 
  aConfig(Sigma, Delta, P1, I, O, A, width nw1, length nl1)
  /\ 
  aConfig(Sigma, Delta, P1, I, O, A, width 0, length 0)
  => 
  aConfig(Sigma, Delta, P2, I, O, A, width nw2, length nl2)
 . 
\end{lstlisting}
\noindent adds to the current value of width the values
computed in the sub-proofs and to the current value of length
the maximum of the two lengths computed in the subproofs.

Approximate equality axioms combine the rules \textsc{axiom} and
\textsc{input-unused}: if we rewrite a \code{aConfig} whose width is 
\code{w} and whose length is \code{l},
we get a \code{aConfig} whose width is \code{w + 1} and
whose length is \code{l + | I \ I'| } where \code{I} is the set of 
inputs of the configuration and \code{I'} is the set of inputs used 
by the protocol we want to rewrite with the axiom.
 
Strategies for applying an approximate equality axiom are of the form
\begin{lstlisting}
 strat S : Param @ ApproxEqConfig .  
 sd S(X) := 
    match aConf s.t. aConfStartsWithNew aConf
    ? CONG-NEW-NF-APPROX{S(X)} 
    : matchrew aConf s.t. 
       aConfig(Sigma, Delta, P, I, O, A, width w, length l) := aConf 
      by aConf 
      using
      CONG-COMP-APPROX{
         axiom(X)
      } 
    .
\end{lstlisting}

Proofs will typically consist of a composition, using \textsc{trans} rule,
of several \textsc{strict} steps with a number of applications of the axioms, using their corresponding strategies.

%%%\subsection{Case studies}

\bibliography{bib} 
\bibliographystyle{ieeetr}

%\appendix
%\section{Derived rules, reaction equivalence}

\begin{lstlisting}
 *** rules for normal forms: 
 
 
 crl [cong-pre-nf] :
     rConfig(Sigma, Delta, Gamma, preNF(BL , R1 , QL), I, A, T)
     => 
     rConfig(Sigma, Delta, Gamma, preNF(BL , R2 , QL), I, A, T)
     if 
     rConfig(Sigma, Delta, addDeclarations BL Gamma, R1, I, A, T) 
     => 
     rConfig(Sigma, Delta, Gamma', R2, I, A, T) 
     /\
     Gamma' == addDeclarations BL Gamma [nonexec] .
     
  crl [cong-nf] :
     rConfig(Sigma, Delta, Gamma, nf(BL , R1 , QL), I, A, T)
     => 
     rConfig(Sigma, Delta, Gamma, nf(BL , R2 , QL), I, A, T)
     if 
     rConfig(Sigma, Delta, addDeclarations BL Gamma, R1, I, A, T) 
     => 
     rConfig(Sigma, Delta, Gamma', R2, I, A, T) 
     /\  Gamma' == addDeclarations BL Gamma 
     [nonexec] . 
     
 crl [read-det-pre] :
     rConfig(Sigma, Delta, Gamma, 
               preNF( (x : T1 <- read i) (y : T1 <- read i) BL , R , QL), I, A, T2) 
     =>
     rConfig(Sigma, Delta, Gamma, 
               preNF( (x : T1 <- read i) BL , R [y / x] , del y QL), I, A, T2) 
 if isElemB(i, I, A)  /\ elem (toBound i) T1 Delta A
 /\ typeOf(Sigma, Delta, addDeclarations BL (Gamma (x : T1) (y : T1)), I, A, R) == T2        
 .
                
 crl [bind-ret-2-pre] : 
     rConfig(Sigma, Delta, Gamma, 
               preNF( (x : T1 <~ R1) BL , return x , QL), I, A, T1) 
     =>
     rConfig(Sigma, Delta, Gamma, 
               preNF( BL , R1 , del x QL), I, A, T1) 
 if typeOf(Sigma, Delta, addDeclarations BL Gamma, I, A, R1) == T1 
  .         
     
 crl [read2Binds] : 
    rConfig(Sigma, Delta, Gamma, preNF(BL (x : T1 <~ read i), R , QL), I, A, T) 
    =>
    rConfig(Sigma, Delta, Gamma, preNF(BL (x : T1 <- read i), R , QL), I, A, T) 
    if isElemB(i, I, A)  and elem (toBound i) T1 Delta A .
    
 crl [pre2Nf] : preNF(BRL, R, QL) => nf(BRL, R, QL) if R : BindFreeReaction .
 
 rl [nf2Pre] : nf(BL, R, QL) => preNF(BL, R, QL) .
      
 crl [merge-pre] :
     rConfig(Sigma, Delta, Gamma, preNF(BL (x : T1 <~ R1) , R2 , QL), I, A, T2)
     =>
     rConfig(Sigma, Delta, Gamma, preNF(BL , x : T1 <- R1 ; R2 , del x QL), I, A, T2) 
   if typeOf(Sigma, Delta, addDeclarations BL Gamma, I, A, R1) == T1
     /\ typeOf(Sigma, Delta, addDeclarations BL (Gamma (x : T1)), I, A, R2) == T2
 .
    
 crl [bind2R-nf] :   
 rConfig(Sigma, Delta, Gamma, nf(BRL (x : T1 <- R1) , R2 , QL), I, A, T2)
     =>
     rConfig(Sigma, Delta, Gamma, preNF(BRL , x : T1 <- R1 ; R2 , del x QL), I, A, T2) 
 if typeOf(Sigma, Delta, addDeclarations BRL Gamma, I, A, R1) == T1
 /\ typeOf(Sigma, Delta, addDeclarations BRL (Gamma (x : T1)), I, A, R2) == T2 
     .
    
 crl [bind2R-pre] :
     rConfig(Sigma, Delta, Gamma, preNF(BL (x : T1 <- R1) , R2 , QL), I, A, T2)
     =>
     rConfig(Sigma, Delta, Gamma, preNF(BL , x : T1 <- R1 ; R2 , del x QL), I, A, T2) 
 if typeOf(Sigma, Delta, addDeclarations BL Gamma, I, A, R1) == T1
 /\ typeOf(Sigma, Delta, addDeclarations BL (Gamma (x : T1)), I, A, R2) == T2 
     .
  
 crl [bind2R-pre-reverse] :
     rConfig(Sigma, Delta, Gamma, 
             preNF(BL , x : T1 <- R1 ; R2 , QL), 
             I, A, T2)
     =>
     rConfig(Sigma, Delta, Gamma, 
             preNF(BL (x : T1 <- R1) , R2 , x :: QL),
             I, A, T2) 
 if typeOf(Sigma, Delta, addDeclarations BL Gamma, I, A, R1) == T1
 /\ typeOf(Sigma, Delta, addDeclarations BL (Gamma (x : T1)), I, A, R2) == T2
 .  
 
 crl [ret-bind-pre] :
     rConfig(Sigma, Delta, Gamma, preNF((x : T1 <~ return M) BL, R , QL), I, A, T2) 
     =>
     rConfig(Sigma, Delta, Gamma, preNF(BL, R [x / M] , del x QL), I, A, T2) 
 if
     typeOf(Sigma, Gamma, M) == T1 
     /\  typeOf(Sigma, Delta, addDeclarations BL (Gamma (x : T1)), I, A, R) == T2 
 .    

 
 crl [bind-bind-pre] :
     rConfig(Sigma, Delta, Gamma, 
              preNF((x2 : T2 <~ nf(BRL, R2, QL')) BL, R1, QL), I, A, T1)
     =>  
     rConfig(Sigma, Delta, Gamma, 
              preNF(BRL (x2 : T2 <~ R2) BL, 
                    R1, addListBefore QL' x2 QL), I, A, T1) 
 if  
     typeOf(Sigma, Delta, 
            addDeclarations BRL (addDeclarations BL Gamma), I, A, R2) == T2
     /\          
     typeOf(Sigma, Delta, 
            addDeclarations BL (Gamma (x2 : T2)), I, A, R1) == T1
 .  
 
  crl [bind-bind-pre-pre] :
     rConfig(Sigma, Delta, Gamma, 
              preNF((x2 : T2 <~ preNF(BRL, R2, QL')) BL, R1, QL), I, A, T1)
     =>  
     rConfig(Sigma, Delta, Gamma, 
              preNF(BRL (x2 : T2 <~ R2) BL, 
                    R1, addListBefore QL' x2 QL), I, A, T1) 
 if  
     typeOf(Sigma, Delta, 
            addDeclarations BRL (addDeclarations BL Gamma), I, A, R2) == T2
     /\          
     typeOf(Sigma, Delta, 
            addDeclarations BL (Gamma (x2 : T2)), I, A, R1) == T1
 . 
            
 *** derived rules:
  
  rl [change-order] :
     rConfig(Sigma, Delta, Gamma, nf(BRL, R, QL), I, A, T)
     =>
     rConfig(Sigma, Delta, Gamma, nf(BRL, R, QL'), I, A, T) 
     [nonexec] . 
     
    rl [change-order-pre] :
     rConfig(Sigma, Delta, Gamma, preNF(BRL, R, QL), I, A, T)
     =>
     rConfig(Sigma, Delta, Gamma, preNF(BRL, R, QL'), I, A, T) 
     [nonexec] . 
    
 
  crl [same-reaction-if] : 
     rConfig(Sigma, Delta, Gamma, if M then R else R, I, A, T)
     => 
     rConfig(Sigma, Delta, Gamma, R, I, A, T) 
  if typeOf(Sigma, Delta, Gamma, I, A, R) == T  
  /\ typeOf(Sigma, Gamma, M) == bool 
  .
 
  crl [cong-branch-refl] : 
     rConfig(Sigma, Delta, Gamma, if M then R1 else R2, I, A, T)
     => 
     rConfig(Sigma, Delta, Gamma, if M then R3 else R4, I, A, T)
     if
     typeOf(Sigma, Gamma, M) == bool 
     /\
     rConfig(Sigma, Delta, Gamma, R1, I, A, T) => 
     rConfig(Sigma, Delta, Gamma, R3, I, A, T)
     /\
     rConfig(Sigma, Delta, Gamma, R2, I, A, T) => 
     rConfig(Sigma, Delta, Gamma, R4, I, A, T) . 
      
 crl [read-inside-if] : 
    rConfig(Sigma, Delta, Gamma, 
             x : T1 <- read i ; if M then R1 else R2, I, A, T) 
    => 
    rConfig(Sigma, Delta, Gamma, 
             if M then x : T1 <- read i ; R1 
                  else x : T1 <- read i ; R2, I, A, T) 
    if isElemB(i, I, A)  /\ elem (toBound i) T1 Delta A .
   
 crl [read-outside-if] : 
    rConfig(Sigma, Delta, Gamma, if M then x : T1 <- read i ; R1 
                                      else x : T1 <- read i ; R2
            , I, A, T) 
    => 
    rConfig(Sigma, Delta, Gamma, 
             x : T1 <- read i ; if M then R1 else R2, I, A, T) 
    if isElemB(i, I, A)  /\ elem (toBound i) T1 Delta A
    .
                    
 rl [if-over-bind-same-2] :
     rConfig(Sigma, Delta, Gamma, x : T1 <- if M1 
                                             then if M2 then R1 else R2
                                             else if M2 then R3 else R4 ;
                                  if M1 
                                     then if M2 then S1 else S2
                                     else if M2 then S3 else S4,
                                 I, A, T)
    => 
    rConfig(Sigma, Delta, Gamma, if M1 
                                    then if M2 then (x : T1 <- R1 ; S1) 
                                               else (x : T1 <- R2 ; S2)
                                    else if M2 then (x : T1 <- R3 ; S3) 
                                               else (x : T1 <- R4 ; S4),          
                                 I, A, T) .                             
           
 crl [if-over-bind-same] :
    rConfig(Sigma, Delta, Gamma, 
              x : T1 <- if M then R1 else R2 ; 
              if M then R3 else R4, I, A, T)
    =>
    rConfig(Sigma, Delta, Gamma, 
             if M then x : T1 <- R1 ; R3 
                  else x : T1 <- R2 ; R4 , 
             I, A, T)                   
 if typeOf(Sigma, Delta, Gamma, I, A, R1) == T1  /\
       typeOf(Sigma, Delta, Gamma, I, A, R2) == T1  /\
       typeOf(Sigma, Delta, Gamma (x : T1), I, A, R3) == T /\
       typeOf(Sigma, Delta, Gamma (x : T1), I, A, R4) == T /\
       typeOf(Sigma, Gamma, M) == bool 
 .
 
 crl [if-over-bind] : 
    rConfig(Sigma, Delta, Gamma, x : T1 <- if M then R1 else R2 ; R , I, A, T) 
    =>
    rConfig(Sigma, Delta, Gamma, if M then x : T1 <- R1 ; R else x : T1 <- R2 ; R , 
            I, A, T) 
 if typeOf(Sigma, Delta, Gamma, I, A, R1) == T1  /\
       typeOf(Sigma, Delta, Gamma, I, A, R2) == T1  /\
       typeOf(Sigma, Delta, Gamma (x : T1), I, A, R) == T /\
       typeOf(Sigma, Gamma, M) == bool
   .
      
 crl [if-over-read] : 
     rConfig(Sigma, Delta, Gamma, if M then x : T1 <- R1 ; R else x : T1 <- R1 ; S, I, A, T)
     =>        
     rConfig(Sigma, Delta, Gamma, x : T1 <- R1 ; if M then R else S, I, A, T)
 if
     typeOf(Sigma, Delta, Gamma, I, A, R1) == T1 /\
     typeOf(Sigma, Delta, Gamma (x : T1), I, A, R) == T /\
     typeOf(Sigma, Delta, Gamma (x : T1), I, A, S) == T /\
     typeOf(Sigma, Gamma, M) == bool 
 .
 
 rl [samp-over-if] :
    rConfig(Sigma, Delta, Gamma, 
            preNF((x : T1 <~ samp Dist) BL, 
                  if M then R1 else R2, QL), I, A, T) 
    => 
    rConfig(Sigma, Delta, Gamma, 
            preNF(BL, if M then (x : T1 <- samp Dist ; R1) 
                           else (x : T1 <- samp Dist ; R2), 
                           del x QL), 
            I, A, T) 
    .
 
 var vx vy : Qid . 
        
 crl [alpha] :
     rConfig(Sigma, Delta, Gamma, vx : T1 <- R1 ; R2 , I, A, T2 )
     =>
     rConfig(Sigma, Delta, Gamma, vy : T1 <- R1 ; (R2 [vx / vy]), I, A, T2)
     if typeOf(Sigma, Delta, Gamma, I, A, R1) == T1 /\
        typeOf(Sigma, Delta, Gamma (vx : T1), I, A, R2) == T2 [nonexec] .
        
 rl [alpha-nf] : 
    rConfig(Sigma, Delta, Gamma, 
            nf((vx : T1 <- R1) BRL,
               R2,
               QL
              ),
            I, A, T2    
           )
     => 
     rConfig(Sigma, Delta, Gamma, 
            nf((vy : T1 <- R1) BRL,
               R2 [vx / vy] ,
               replace vx vy QL
              ),
            I, A, T2    
           )   
      [nonexec]        

\end{lstlisting}

\section{Derived rules, protocol equivalence}
\begin{lstlisting}
********************************************
 *** derived rules and rules for normal forms
 ********************************************
 
 rl [desugar-newNF] :
     pConfig(Sigma, Delta, newNF(ltq, P1, ql), I, O, A)
     =>
     pConfig(Sigma, Delta, newNF2New(newNF(ltq, P1, ql)), I, O, A) 
 .
     
 rl [sugar-newNF] :
    pConfig(Sigma, Delta, P, I, O, A)
    => 
    pConfig(Sigma, Delta, new2NF(P), I, O, A) 
 .
    
 rl [delete-empty-newNF] :
     pConfig(Sigma, Delta, newNF(emptyTypedCNameList, P, emptyCNameList), I, O, A)
     => 
     pConfig(Sigma, Delta, P, I, O, A) 
 . 
     
 rl [reorder-newNF] :
     pConfig(Sigma, Delta, newNF(ltq, P, ql1), I, O, A)
     =>
     pConfig(Sigma, Delta, newNF(ltq, P, ql2), I, O, A)
    [nonexec]
 . 
        
  crl [CONG-NEW-NF] : 
    pConfig(Sigma, Delta, newNF(ltq, P1, ql), I, O, A) 
    => 
    pConfig(Sigma, Delta, newNF(ltq, P2, ql), I, O, A)
    if
    pConfig(Sigma, addChannels ltq Delta, P1, I, union(chansInList ltq, O), A)
    =>
    pConfig(Sigma, Delta', P2, I, O', A) 
    /\
    Delta' == addChannels ltq Delta
    /\
    O' == union(chansInList ltq, O)
  .    
 
 crl [absorb-new-nf] :
     pConfig(Sigma, Delta, newNF(< c : T > ltq, P || (c ::= R), ql), I, O, A) 
     => 
     pConfig(Sigma, Delta, newNF(ltq, P, del c ql), I, O, A) 
 if
     typeOf(Sigma, addChannels ltq (Delta (toBound c :: T)), emptyTypeContext, 
            insert(toBound c, union(I, getOutputs(P))), A, R) == T 
     /\
     typeOf(Sigma, addChannels ltq Delta, I, A, P)       
     /\ 
     getOutputs(newNF2New(newNF(ltq, P, ql))) == O 
   .
   
 crl [comp-new-nf-left] : 
     pConfig(Sigma, Delta, newNF(< c : T > ltq, P || (c ::= R), ql), I, O, A) 
    => 
     pConfig(Sigma, Delta, newNF(ltq, P || new c : T in (c ::= R), del c ql), I, O, A) 
    if
     typeOf(Sigma, addChannels ltq Delta, I, A, P) 
     /\
     typeOf(Sigma, addChannels ltq (Delta ((toBound c) :: T)), emptyTypeContext, 
            insert(toBound c, union(I, getOutputs(P))), A, R) == T 
 . 
      
 crl [comp-new-nf-right] :
    pConfig(Sigma, Delta, newNF(ltq, P || new c : T in Q, ql), I, O, A)
    =>
    pConfig(Sigma, Delta, newNF(< c : T > ltq, P || Q, c :: ql), I, O, A) 
    if
     typeOf(Sigma, addChannels ltq Delta, union(I, getOutputs(Q)), A, P) 
     /\
     typeOf(Sigma, addChannels ltq (Delta ((toBound c) :: T)), 
            union(I, getOutputs(P)), A, Q) 
 .
 
  crl [fold-bind-new] :
    pConfig(Sigma, Delta, 
             newNF(< c : T > ltq, P || (c ::= R) || (o ::= x : T <- read c ; S), ql),
            I, O, A)
    => 
    pConfig(Sigma, Delta, 
             newNF(ltq, P || (o ::= x : T <- R ; S), del c ql),
            I, O, A) 
    if 
    typeOf(Sigma, addChannels ltq (Delta ((toBound c) :: T)), emptyTypeContext, 
            insert(toBound o, insert(toBound c, union(I, getOutputs(P)))), A, R) == T . 
           
  crl [fold-bind-new-nf] :
    pConfig(Sigma, Delta, 
             newNF(< c : T > ltq, P || (c ::= R) || 
                                 (o ::= nf((x : T <- read c) BRL, S, QL)), 
                   ql),
            I, O, A)
    => 
    pConfig(Sigma, Delta, 
             newNF(ltq, P || (o ::= preNF((x : T <~ R) BRL, S, QL)), del c ql),
            I, O, A) 
    if typeOf(Sigma, addChannels ltq (Delta ((toBound c):: T)),
               addDeclarations BRL emptyTypeContext, 
               insert(toBound o, 
                      insert(toBound c, union(I,  getOutputs(P)))), 
              A, R) == T 
  .  
            
  crl [fold-bind-new-prenf] :
    pConfig(Sigma, Delta, 
             newNF(< c : T > ltq, P || (c ::= R) || 
                                 (o ::= preNF((x : T <- read c) BRL, S, QL)), ql),
            I, O, A)
    => 
    pConfig(Sigma, Delta, 
             newNF(ltq, P || (o ::= preNF((x : T <~ R) BRL, S, QL)), del c ql),
            I, O, A) 
    if        
    typeOf(Sigma, addChannels ltq (Delta ((toBound c) :: T)), 
            emptyTypeContext, 
            insert(toBound o, 
                   insert(toBound c, union(I, getOutputs(P)))), 
            A, R) == T .
            
 crl [COMP-NEW-newNF] :
     pConfig(Sigma, Delta, P || newNF(ltq, Q, QL), I, O, A)
     =>   
     pConfig(Sigma, Delta, newNF(ltq, P || Q, QL), I, O, A)
 if
     typeOf(Sigma, addChannels ltq Delta, union(I, getOutputs(P)), A, Q)
     /\
     typeOf(Sigma, Delta, I, 
            union(I, getOutputs(newNF(ltq, Q, QL))), A, P)     
 .
   
   rl [lift-inner-new-nf] :
     pConfig(Sigma, Delta, newNF(ltq1, newNF(ltq2, P, ql2) , ql1), I, O, A)
     => 
     pConfig(Sigma, Delta, newNF(ltq1 ltq2, P, ql1 ++ ql2), I, O, A)
   .
      
  crl [UNUSED-nf] :
    pConfig(Sigma, Delta,
            (cn1 ::= samp Dist) || (cn2 ::= nf( (x : T1 <- read cn1) BRL , R2, QL) ),
            I, O, A) 
    =>           
     pConfig(Sigma, Delta,
            (cn1 ::= samp Dist) || (cn2 ::= nf( BRL , R2, del x QL) ),
            I, O, A) 
    if
     typeOf(Sigma, Delta, addDeclarations BRL (x : T1), 
             insert(toBound cn1, insert(toBound cn2, I)), A, R2) ==
     typeInCtx(toBound cn2, A, Delta)
     /\
     elem (toBound cn1) T1 Delta A        
    . 
    
   crl [UNUSED-nf-copy] :
    pConfig(Sigma, Delta,
            (cn1 ::= nf(emptyBRList, samp Dist, emptyCNameList)) || 
            (cn2 ::= nf( (x : T1 <- read cn1) BRL , R2, QL) ),
            I, O, A) 
    =>           
     pConfig(Sigma, Delta,
            (cn1 ::= nf(emptyBRList, samp Dist, emptyCNameList)) || 
            (cn2 ::= nf( BRL , R2, del x QL) ),
            I, O, A) 
    if 
     typeOf(Sigma, Delta, addDeclarations BRL (x : T1), 
             insert(toBound cn1, insert(toBound cn2, I)), A, R2) ==
     typeInCtx(toBound cn2, A, Delta)
     /\
     elem (toBound cn1) T1 Delta A
    .    
                   
  crl [UNUSED-pre-nf] :
    pConfig(Sigma, Delta,
            (q1 ::= samp Dist) || (q2 ::= preNF( (x : T1 <- read q1) BRL , R2, QL) ),
            I, O, A) 
    =>           
     pConfig(Sigma, Delta,
            (q1 ::= samp Dist) || (q2 ::= preNF( BRL , R2, del x QL) ),
            I, O, A) 
     if
     typeOf(Sigma, Delta, addDeclarations BRL (x : T1), 
             insert(toBound q1, insert(toBound q2, I)), A, R2) ==
     typeInCtx(toBound q2, A, Delta)
             /\
     elem (toBound q1) T1 Delta A 
     .    
   
  crl [SUBST-nf] : 
     pConfig(Sigma, Delta,
             (cn1 ::= R1) || (cn2 ::= nf( (x1 : T1 <- read cn1) BRL , R2, QL) ),
             I, O, A)  
     => 
     pConfig(Sigma, Delta,
             (cn1 ::= R1) || (cn2 ::= preNF((x1 : T1 <~ R1) BRL , R2, QL)),
             I, O, A) 
     if isSampFree(R1) /\
        O == insert(toBound cn1, toBound cn2) /\ 
     typeOf(Sigma, Delta, emptyTypeContext, 
            insert(toBound cn1, insert(toBound cn2, I)), A, R1) == T1 /\
     typeOf(Sigma, Delta, addDeclarations BRL (x1 : T1), 
            insert(toBound cn1, insert(toBound cn2, I)), A, R2) ==
     typeInCtx(toBound cn2, A, Delta)
     /\
     elem (toBound cn1) T1 Delta A 
     . 
     
     
  crl [moveReadInnerNf] :
    pConfig(Sigma, Delta,
             cn1 ::= nf((x : T <- read cn2) BRL , R1, QL) ,
             I, O, A)  
     => 
       pConfig(Sigma, Delta,
             cn1 ::= preNF(BRL , x : T <- read cn2 ; R1, del x QL) ,
             I, O, A)    
  if elem (toBound cn2) T Delta A 
  /\ typeOf(Sigma, Delta, addDeclarations BRL (x : T), 
            insert(toBound cn1, I), A, R1) == 
     typeInCtx(toBound cn1, A, Delta)
  .  
  
    crl [moveReadInnerPreNf] :
    pConfig(Sigma, Delta,
             cn1 ::= preNF((x : T <- read cn2) BRL , R1, QL) ,
             I, O, A)  
     => 
       pConfig(Sigma, Delta,
             cn1 ::= preNF(BRL , x : T <- read cn2 ; R1, del x QL) ,
             I, O, A)  
    if elem (toBound cn2) T Delta A 
     /\ typeOf(Sigma, Delta, addDeclarations BRL (x : T), 
            insert(toBound cn1, I), A, R1) == 
        typeInCtx(toBound cn1, A, Delta)          
  .   
    
  rl [SUBST-nf-rev] : 
     pConfig(Sigma, Delta,
             (cn1 ::= R1) || (cn2 ::= preNF( BRL , R1, QL) ),
             I, O, A)  
     => 
     pConfig(Sigma, Delta,
             (cn1 ::= R1) || (cn2 ::= preNF((x1 : T1 <- read cn1) BRL , return x1, x1 :: QL)),
             I, O, A)   
  [nonexec] .  
  
  crl [SUBST-nf-read] : 
     pConfig(Sigma, Delta,
             (cn1 ::= nf((x2 : T1 <- read C), return x2, x2 :: emptyCNameList )) || 
             (cn2 ::= nf( (x1 : T1 <- read cn1) BRL , R2, QL) ),
             I, O, A)  
     => 
     pConfig(Sigma, Delta,
             (cn1 ::= nf((x2 : T1 <- read C), return x2, x2 :: emptyCNameList )) || 
             (cn2 ::= nf((x2 : T1 <- read C) BRL , R2 [x1 / x2], replace x1 x2 QL)),
             I, O, A) 
     if
     isElemB(C, I, A) /\
     O == insert(toBound cn1, toBound cn2) /\ 
     typeOf(Sigma, Delta, addDeclarations BRL (x1 : T1), 
             insert(toBound cn1, insert(toBound cn2, I)), A, R2) ==
     typeInCtx(toBound cn2, A, Delta)
             /\
     elem (toBound cn1) T1 Delta A 
             /\
     elem (toBound C) T1 Delta A           
     .   
 
 *********************
 *** rules for families
 *********************

  rl [alpha-family] :
   pConfig(Sigma, Delta, 
            newNF(ltq {C q (bound n) : T}, 
                  P || family C q (bound n) ::= cases, QL), 
            I, O, A)
   => 
   pConfig(Sigma, Delta, 
           newNF(ltq {C q' (bound n) : T}, 
                 P || (family C q' (bound n) ::= 
                        (alphaCases cases q q')), QL), 
                 I, O, A)
 [nonexec]
 .
  
 rl [addToGroup] :
   pConfig(Sigma, Delta, 
           (family C q (bound nt) ::= P) || 
            family C' q' (bound nt1) ::= cases, I, O, A)
   => 
   pConfig(Sigma, Delta, 
           family C q (bound nt) ::= 
               (P || family C' q' (bound nt1) ::= cases), 
           I, O, A)
 .  
    
  crl [absorb-new-nf-family-one] : 
     pConfig(Sigma, Delta, 
             newNF({C q (bound n) : T }  ltq, 
                   P || (family C q (bound n) ::= cases), ql), 
             I, O, A) 
     => 
     pConfig(Sigma, Delta, newNF(ltq, P, del C ql), I, O, A) 
  if
     typeOf(Sigma, addChannels ({C q (bound n) : T }  ltq) Delta, 
            insert(C @ n, union(I, getOutputs(P))), A, 
            family C q (bound n) ::= cases)
     /\
     typeOf(Sigma, addChannels ltq Delta, I, A, P)       
     /\ 
     getOutputs(newNF2New(newNF(ltq, P, ql))) == O 
 .
 
 crl [absorb-new-nf-family-two] : 
     pConfig(Sigma, Delta, 
             newNF({C (q q')((bound n) (uniformBound nt)) : T }  ltq, 
                   P || 
                   (family C (q q')
                             ((bound n) (uniformBound nt)) ::= cases), 
                   ql), 
             I, O, A) 
     => 
     pConfig(Sigma, Delta, newNF(ltq, P, del C ql), I, O, A) 
 if typeOf(Sigma, 
           addChannels ({C (q q')
                           ((bound n) (uniformBound nt)) : T } 
                        ltq) Delta, 
           insert(C @ (n nt), union(I, getOutputs(P))), 
           A, 
           family C (q q')((bound n) (uniformBound nt)) ::= cases
          )
     /\
     typeOf(Sigma, addChannels ltq Delta, I, A, P)       
     /\ 
     getOutputs(newNF2New(newNF(ltq, P, ql))) == O     
 .  
 
      
 crl [absorb-reverse-new-nf-family] : 
     pConfig(Sigma, Delta, newNF(ltq, P, ql), I, O, A)
     =>
     pConfig(Sigma, Delta, 
             newNF({q nlist blist : T } ltq, 
                   P || (family q nlist blist ::= cases), q :: ql), I, O, A) 
 if   
     typeOf(Sigma, addChannels ltq Delta, I, A, P) 
     /\
     typeOf(Sigma, addChannels ({q nlist blist : T } ltq) Delta, 
             insert(toBounds q blist, union(I, O)), A,
             (family q nlist blist ::= cases)) 
     /\
     getOutputs(newNF2New(newNF(ltq, P, ql))) == O 
 [nonexec] 
 . 
     
   crl [comp-new-nf-right-family] :
    pConfig(Sigma, Delta, 
            newNF(ltq, P || newfamily c nlist blist : T in Q, ql), 
            I, O, A)
    =>
    pConfig(Sigma, Delta, 
            newNF({ c nlist blist : T } ltq, P || Q, c :: ql), 
            I, O, A) 
    if
     typeOf(Sigma, addChannels ltq Delta, union(I, getOutputs(Q)), A, P) 
     /\
     typeOf(Sigma, addChannels ltq (Delta ((toBounds c blist) :: T)),
            union(I, getOutputs(P)), A, Q) 
    .   
    
    rl [COMP-NEW-newNF-newfamily] :
     pConfig(Sigma, Delta, 
             P || (newfamily c nlist blist : T in Q), I, O, A)
     =>  
     pConfig(Sigma, Delta, 
             newNF( {c nlist blist : T}, P || Q, c :: emptyCNameList), 
             I, O, A)
     if 
     typeOf(Sigma, Delta ((toBounds c blist) :: T), 
            union(I, getOutputs(P)), A, Q)
     /\
     typeOf(Sigma, Delta, 
            union(I ,(getOutputs(Q) \ (toBounds c blist))), A, P) 
     .              
 
 crl [moveReadInnerNfFamily] :
    pConfig(Sigma, Delta,
             family C lq blist ::= nf((x : T <- read cn2) BRL , R1, QL) ,
             I, O, A)  
     => 
       pConfig(Sigma, Delta,
             family C lq blist ::= 
              preNF(BRL , x : T <- read cn2 ; R1, del x QL) ,
             I, O, A)
   if          
   elem (toBound cn2) T Delta A 
  /\ typeOf(Sigma, Delta, addDeclarations BRL (x : T), 
            insert(toBounds C blist , I), A, R1) == 
     typeInCtx(toBounds C blist, A, Delta)              
  .  
  
    crl [moveReadInnerPreNfFamily] :
    pConfig(Sigma, Delta,
             family C lq blist ::= 
               preNF((x : T <- read cn2) BRL , R1, QL) ,
             I, O, A)  
     => 
       pConfig(Sigma, Delta,
             family C lq blist ::= 
               preNF(BRL , x : T <- read cn2 ; R1, del x QL) ,
             I, O, A) 
             if          
   elem (toBound cn2) T Delta A 
  /\ typeOf(Sigma, Delta, addDeclarations BRL (x : T), 
            insert(toBounds C blist , I), A, R1) == 
     typeInCtx(toBounds C blist, A, Delta)          
  .   
     crl [SUBST-nf-rev-families-one] : 
     pConfig(Sigma, Delta,
             (family C1 q (bound nt1) ::= cases) 
             || 
             (family C q' (bound nt2) ::= preNF( BRL , R1, QL) ),
             I, O, A)  
     => 
     pConfig(Sigma, Delta,
             (family C1 q (bound nt1) ::= cases) 
             || 
              (family C q' (bound nt2) ::= 
                preNF((x1 : T1 <- read (C1[nj])) BRL , 
                       return x1, x1 :: QL)),
             I, O, A) 
     if 
     typeOf(Sigma, Delta, emptyTypeContext, 
            insert(toBounds C1 (bound nt1), 
             insert(toBounds C (bound nt2) , I)), 
             insert(q =T= nj, A), R2) 
     == 
     typeInCtx(toBounds C1 (bound nt1), insert(q =T= nj, A), Delta) 
     /\
     typeOf(Sigma, Delta, addDeclarations BRL emptyTypeContext, 
            insert(toBounds C1 (bound nt1), 
               insert(toBounds C (bound nt2) , I)), A, R1) 
     == 
     typeInCtx(toBounds C (bound nt2), A, Delta) 
     /\       
     (projectIndex (family C1 q (bound nt1)  ::= cases) nj 
          insert(q =T= nj, A) empty) == (C1[nj] ::= R2)
     /\ 
     R1 == convertNF(R2)               
  [nonexec] . 
  
    crl [subst-families-one] :
     pConfig(Sigma, Delta, 
             (family C q (bound nt) ::= cases)
             ||
             (cn2 ::= nf((x : T <- read (C[nj])) BRL, R, ql)),
             I, O, A)
     => 
     pConfig(Sigma, Delta, 
             (family C q (bound nt) ::= cases)
             ||
             (cn2 ::= preNF((x : T <~ R2) BRL, R, ql)
             )
             ,
             I, O, A) 
     if 
      typeOf(Sigma, Delta, addDeclarations BRL (x : T), 
            insert(toBounds C (bound nt), 
             insert(toBound cn2 , I)), A, R) 
     == 
     typeInCtx(toBound cn2, A, Delta) 
     /\ 
     typeInCtx(C @ nt, A, Delta) == T
     /\ 
     (projectIndex (family C q (bound nt)  ::= cases) nj 
          insert(q =T= nj, A) empty) == (C[nj] ::= R2)
         
     [nonexec] 
     .
     
     var C1 : Qid .
     
     crl [subst-channel-one-family] :
     pConfig(Sigma, Delta, 
             (cn ::= R2)
             ||
             (family C1 q' (bound nt1) ::= 
                nf((x : T <- read cn) BRL, R, ql)
             ),
             I, O, A)
     => 
     pConfig(Sigma, Delta, 
             (cn ::= R2)
             ||
             (family C1 q' (bound nt1) ::= 
               preNF((x : T <~ R2) BRL, R, ql)
             ),
             I, O, A) 
     if isSampFree(R2) /\
        O == insert(toBound cn, C1 @ nt1) /\ 
     typeOf(Sigma, Delta, emptyTypeContext, 
            insert(toBound cn, insert(C1 @ nt1, I)), A, R2) == T /\
     typeOf(Sigma, Delta, addDeclarations BRL (x : T), 
            insert(toBound cn, insert(C1 @ nt1, I)), A, R) == 
     typeInCtx(C1 @ nt1, A, Delta)
     /\ 
     elem (toBound cn) T Delta A 
     . 
     
              
     crl [subst-families-one-family] :
     pConfig(Sigma, Delta, 
             (family C q (bound nt) ::= cases)
             ||
             (family C1 q' (bound nt1) ::= 
               nf((x : T <- read (C[nj])) BRL, R, ql)),
             I, O, A)
     => 
     pConfig(Sigma, Delta, 
             (family C q (bound nt) ::= cases)
             ||
             (family C1 q' (bound nt1) ::= 
               preNF((x : T <~ R2) BRL, R, ql)
             )
             ,
             I, O, A) 
     if (projectIndex (family C q (bound nt)  ::= cases) nj 
          insert(q =T= nj, A) empty) == (C[nj] ::= R2)
     [nonexec] 
     .
              
     crl [subst-2-families-one] :
     pConfig(Sigma, Delta, 
             (family C q (bound nt) ::= cases')
             ||
             (cn2 ::= cases),
             I, O, A)
     => 
     pConfig(Sigma, Delta, 
             (family C q (bound nt) ::= cases')
             ||
             (cn2 ::= nf('x : T <- read (C[nj]), 
                         return 'x, 
                         'x :: emptyCNameList)
             ),
             I, O, A) 
     if (projectIndex (family C q (bound nt)  ::= cases') nj 
          insert(q =T= nj, A) empty) == (C[nj] ::= cases)
     [nonexec] 
     .
     
     crl [subst-families-two] :
     pConfig(Sigma, Delta, 
             (family C (q q') ((bound n) (uniformBound nt)) ::= cases')
             ||
             (cn2 ::= cases),
             I, O, A)
     => 
    pConfig(Sigma, Delta, 
             (family C (q q') ((bound n) (uniformBound nt))  ::= cases')
             ||
             (cn2 ::= nf('x : T <- read (C[q nj]), return 'x, 'x :: emptyCNameList)),
             I, O, A) 
     if (projectIndex 
          (family C (q q') 
                    ((bound n) (uniformBound nt)) ::= cases') nj A empty)
         == 
        (C[q nj] ::= cases)
     [nonexec] .
     
     rl [subst-rev-families] :
     pConfig(Sigma, Delta, 
             (family C q (bound n) ::= 
                whenList1 ;; (when (q =T= nj) --> R) ;; whenList2)
             ||
             (cn2 ::= R),
             I, O, A)
     => 
    pConfig(Sigma, Delta, 
             (family C q (bound n) ::= 
               whenList1 ;; (when (q =T= nj) --> R) ;; whenList2)
             ||
             (cn2 ::= 
               nf('x : T <- read (C[nj]), 
                  return 'x, 'x :: emptyCNameList)),
             I, O, A) 
    [nonexec] .             
     
     crl [SUBST-nf-read-family-one] : 
     pConfig(Sigma, Delta ,
             (family q1 q (bound n) ::= 
                nf((x2 : T1 <- read cn1), 
                   return x2, x2 :: emptyCNameList )) 
              || 
             (cn2 ::= nf( (x1 : T1 <- read (q1[nj])) BRL , R2, QL) ),
             I, O, A)  
     => 
     pConfig(Sigma, Delta,
             (family q1 q (bound n) ::= 
               nf((x2 : T1 <- read cn1), return x2, x2 :: emptyCNameList ))
             || 
             (cn2 ::= 
               nf((x2 : T1 <- read (evalCName cn1 (q |-> nj))) BRL , 
                  R2 [x1 / x2], replace x1 x2 QL)),
             I, O, A) 
     if 
     isElemB(cn1, I, A) /\
     elem (toBounds q1 (bound n)) T1 Delta A /\
     elem (toBound cn1) T1 Delta A /\
     O == insert(toBounds q1 (bound n), toBound cn2)  /\
     typeOf(Sigma, Delta, addDeclarations BRL (x1 : T1), 
            insert(toBounds q1 (bound n), insert(toBound cn1, I)), A, R2)
     == typeInCtx(toBound cn2, A, Delta) 
           . 
     
     rl [SUBST-nf-read-family-two] : 
     pConfig(Sigma, Delta ,
             (family q1 (q q') ((bound n1) (uniformBound nt)) ::= 
                 nf((x2 : T1 <- read cn1), 
                    return x2, x2 :: emptyCNameList )) 
             || 
             (cn2 ::= nf( (x1 : T1 <- read (q1[q nj])) BRL , R2, QL) ),
             I, O, A)  
     => 
     pConfig(Sigma, Delta,
             (family q1 (q q') ((bound n1) (uniformBound nt)) ::= 
                nf((x2 : T1 <- read cn1), 
                   return x2, x2 :: emptyCNameList ))  
              || 
             (cn2 ::= 
              nf((x2 : T1 <- read (evalCName cn1 (q' |-> nj))) BRL , 
                 R2 [x1 / x2], replace x1 x2 QL)),
             I, O, A) 
    if
     isElemB(cn1, I, A)  /\
     elem (toBounds q1 (bound n)) T1 Delta A /\
     elem (toBound cn1) T1 Delta A /\
     O == insert(toBounds q1 (bound n), toBound cn2)  /\
     typeOf(Sigma, Delta, addDeclarations BRL (x1 : T1), 
            insert(toBounds q1 (bound n), insert(toBound cn1, I)), A, R2)
     == typeInCtx(toBound cn2, A, Delta) 
        . 
     
     var q'' : Qid .
     
     rl [SUBST-nf-read-family-two-family] : 
     pConfig(Sigma, Delta ,
             (family q1 (q q') ((bound n1) (uniformBound nt)) ::= 
                 nf((x2 : T1 <- read cn1), 
                    return x2, x2 :: emptyCNameList )) 
             || 
             (family q2 q'' (bound nt2) ::= 
                 nf( (x1 : T1 <- read (q1[q nj])) BRL , R2, QL) ),
             I, O, A)  
     => 
     pConfig(Sigma, Delta,
             (family q1 (q q') ((bound n1) (uniformBound nt)) ::= 
                nf((x2 : T1 <- read cn1), 
                   return x2, x2 :: emptyCNameList ))  
             || 
             (family q2 q'' (bound nt2) ::= 
                nf((x2 : T1 <- read (evalCName cn1 (q' |-> nj))) BRL ,
                   R2 [x1 / x2], 
                   replace x1 x2 QL)),
             I, O, A) . 
      
     crl [SUBST-nf-read-family] : 
     pConfig(Sigma, Delta ,
             (family q1 ntl bounds ::= 
                nf((x2 : T1 <- read cn1), 
                   return x2, x2 :: emptyCNameList )) 
             || 
             (cn2 ::= 
                nf( (x1 : T1 <- read (q1[nj])) BRL , R2, QL) ),
             I, O, A)  
     => 
     pConfig(Sigma, Delta,
             (family q1 ntl bounds ::= 
               nf((x2 : T1 <- read cn1), return x2, x2 :: emptyCNameList))
             || 
             (cn2 ::= 
               nf((x2 : T1 <- read cn1) BRL , 
                  R2 [x1 / x2], 
                  replace x1 x2 QL)),
             I, O, A) 
     if 
     isElemB(cn1, I, A) /\
     elem (toBounds q1 bounds) T1 Delta A /\
     elem (toBound cn1) T1 Delta A /\
     O == insert(toBounds q1 bounds, toBound cn2)  /\
     typeOf(Sigma, Delta, addDeclarations BRL (x1 : T1), 
            insert(toBounds q1 bounds, insert(toBound cn1, I)), A, R2) ==
     typeInCtx(toBound cn2, A, Delta) .

  
     rl [comp-new-families] : 
     pConfig(Sigma, Delta, 
              newNF(ltq1, P1, ql1) || newNF(ltq2, P2, ql2), I, O, A)
     => 
     pConfig(Sigma, Delta, 
              newNF(ltq1 ltq2, P1 || P2, ql1 ++ ql2), I, O, A) 
     .
     
      crl [use-family-p] : 
     pConfig(Sigma, Delta, P1 || family C lq blist ::= P, I, O, A)
     => 
     pConfig(Sigma, Delta, P2 || family C lq blist ::= P, I, O, A)
     if 
     pConfig(Sigma, Delta, P1 || P, I, O, A)
     => 
     pConfig(Sigma, Delta, P2 || P, I, O, A)
     [nonexec] .
      
      
    crl [fold-bind-new-nf-families] :
    pConfig(Sigma, Delta, 
             newNF({ C q (bound n) : T } ltq, 
                   P || 
                   (family C q (bound n) ::= R) || 
                   (family C' q' (bound n1) ::= 
                      nf((x : T <- read (C[q'])) BRL, S, QL)), 
                   ql),
            I, O, A)
    => 
    pConfig(Sigma, Delta, 
             newNF(ltq,
                   P || 
                   (family C' q' (bound n1) ::= 
                     preNF((x : T <~ R) BRL, S, QL)), del C ql),
            I, O, A)  
    if typeOf(Sigma, addChannels ltq (Delta ((toBound c):: T)),
              addDeclarations BRL emptyTypeContext, 
              insert(toBound o, 
               insert(toBound c, union(I, getOutputs(P)))), A, R) == T 
  . 
      
     *** congruence rules for families
       
     crl [CONG-NEWFAMILY] : 
     pConfig(Sigma, Delta, newfamily C lq blist : T in P1, I, O, A)
     =>  
     pConfig(Sigma, Delta, newfamily C lq blist : T in P2, I, O, A)
     if 
     pConfig(Sigma, Delta ((toBounds C blist) :: T), P1, 
             I, insert(toBounds C blist, O), A)
     =>
     pConfig(Sigma, Delta', P2, I, O', A)
     /\
     O' == insert(toBounds C blist, O)
     /\
     Delta' == Delta ((toBounds C blist) :: T)
     [nonexec]
     .

        
      
     crl [CONG-FAMILY-R] : 
     pConfig(Sigma, Delta, family C lq blist ::= R, I, O, A)
     => 
     pConfig(Sigma, Delta, family C lq blist ::= R', I, O, A)
     if 
     rConfig(Sigma, Delta, emptyTypeContext, R, 
             insert(toBounds C blist, I), A, 
             typeInCtx(toBounds C blist, A, Delta))
     =>
     rConfig(Sigma, Delta, emptyTypeContext, R', I', A, T) 
     /\ I' == insert(toBounds C blist, I)  
     /\ T == typeInCtx(toBounds C blist, A, Delta)
     [nonexec] .
     
           
     crl [CONG-FAMILY-P] : 
     pConfig(Sigma, Delta, family C lq blist ::= P1, I, O, A)
     => 
     pConfig(Sigma, Delta, family C lq blist ::= P2, I, O, A)
     if
     pConfig(Sigma, Delta, P1, I, O, A)
     =>
     pConfig(Sigma, Delta, P2, I, O, A) [nonexec] .
    
     var whenCond : WhenCond .
     var whenList1 whenList2 whenList whenList' : WhenList .
     
     crl [CONG-FAMILY-WHENLIST-P] : 
     pConfig(Sigma, Delta, 
             family C lq blist ::= 
               whenList1 ;; (when bt --> P1) ;; whenList2, I, O, A)
     => 
     pConfig(Sigma, Delta, 
              family C lq blist ::= 
                whenList1 ;; (when bt --> P2) ;; whenList2, I, O, A) 
     if 
     pConfig(Sigma, Delta, P1, I, O, insert(bt, A) )
     =>  
     pConfig(Sigma, Delta, P2, I, O, A')
     /\ 
     A' == insert(bt, A) [nonexec] . 

     var A' : Set{BoolTerm} . 

     crl [CONG-FAMILY-WHENLIST-R] : 
     pConfig(Sigma, Delta, 
       family C lq blist ::= 
         (whenList1 ;; (when bt --> R1) ;; whenList2), I, O, A)
     => 
     pConfig(Sigma, Delta, 
       family C lq blist ::= 
        (whenList1 ;; (when bt --> R2) ;; whenList2), I, O, A) 
     if 
     rConfig(Sigma, Delta, emptyTypeContext, R1, 
             insert(toBounds C blist, I), 
             insert(bt, A), T)
     =>  
     rConfig(Sigma, Delta, emptyTypeContext, R2, I', A', T)
     /\
     A' == insert(bt, A) 
     /\
     I' == insert(toBounds C blist, I) [nonexec] . 

     *** case distinction
     
     var aP1 aP2 : Protocol .
     var A'' : Set{BoolTerm} .
     var aQid : Qid .
     
     crl [CASE-DISTINCTION-one] :
     pConfig (Sigma, Delta, 
               P || family aQid q (bound n) ::= 
                      ((when bt --> aP1) ;; whenList),
               I, O, A)
     => 
     pConfig(Sigma, Delta, 
              P || family aQid q (bound n) ::= 
                     ((when bt --> aP2) ;; whenList'),
               I, O, A) 
     if 
     pConfig(Sigma, Delta, P || aP1, I, O, insert(bt, A))
     =>
     pConfig(Sigma, Delta, P || aP2, I, O, A')
     /\
     A' == insert(bt, A)
     /\
     pConfig (Sigma, Delta, 
               P || family aQid q (bound n) ::= whenList,
               I, O, insert(neg bt, A)) 
     =>
     pConfig(Sigma, Delta, 
              P || family aQid q (bound n) ::= whenList',
               I, O, A'')
     /\
     A'' == insert(neg bt, A)              [nonexec] .   
               
     crl [CASE-DISTINCTION-one-end] :
     pConfig (Sigma, Delta, 
               P || family C q (bound n) ::= (otherwise --> aP1),
               I, O, A)
     => 
     pConfig(Sigma, Delta, 
              P || family C q (bound n) ::= (otherwise --> aP2),
               I, O, A) 
     if 
     pConfig(Sigma, Delta, P || aP1, I, O, A)
     =>
     pConfig(Sigma, Delta, P || aP2, I, O, A)            
     [nonexec] .

     crl [CASE-DISTINCTION-one-R] :
     pConfig (Sigma, Delta, 
               P || family aQid q (bound n) ::= 
                     ((when bt --> R1) ;; whenList),
               I, O, A)
     => 
     pConfig(Sigma, Delta, 
              P || family aQid q (bound n) ::= 
                    ((when bt --> R2) ;; whenList'),
               I, O, A) 
     if 
     pConfig(Sigma, Delta, 
              P || family aQid q (bound n) ::= R1, 
              I, O, insert(bt, A))
     =>
     pConfig(Sigma, Delta, 
             P || family aQid q (bound n) ::= R2, 
             I, O, A')
     /\
     A' == insert(bt, A)
     /\
     pConfig (Sigma, Delta, 
               P || family aQid q (bound n) ::= whenList,
               I, O, insert(neg bt, A)) 
     =>
     pConfig(Sigma, Delta, 
              P || family aQid q (bound n) ::= whenList',
               I, O, A'')
     /\
     A'' == insert(neg bt, A)              [nonexec] .   
               
     crl [CASE-DISTINCTION-one-R-end] :
     pConfig (Sigma, Delta, 
               P || family C q (bound n) ::= (otherwise --> R1),
               I, O, A)
     => 
     pConfig(Sigma, Delta, 
              P || family C q (bound n) ::= (otherwise --> R2),
               I, O, A) 
     if 
     pConfig(Sigma, Delta, P || family C q (bound n) ::= R1, I, O, A) 
     =>
     pConfig(Sigma, Delta, P || family C q (bound n) ::= R2, I, O, A)            
     [nonexec] .
     
     
     *** induction
     
     
     var k n1 n2 : Nat . 
     var R'' : Reaction .
     
     crl [INDUCTION-R-one] :
     pConfig(Sigma, Delta ((C @ n) :: T), 
       P || (family C q (bound n) ::= R), I, O, A)
     =>
     pConfig(Sigma, Delta ((C @ n) :: T), 
       P || (family C q (bound n) ::= R'), I, O, A)
     if 
     pConfig(Sigma, Delta ((C @ n) :: T), 
       P || (C [0] ::= replaceVars(R, q |-> 0)), I, O, 
       insert(q =T= 0, A)) 
     => 
     pConfig(Sigma, Delta ((C @ n) :: T), 
       P || (C [0] ::= R''), I, O, A')
     /\
     R'' == replaceVars(R', q |-> 0)
     /\
     A' == insert(q =T= 0, A)
     /\
     pConfig(Sigma, Delta ((C @ n) :: T), 
       P || (family C q (bound 'k) ::= R') || 
       (C['k ++ 1] ::= replaceVars(R, q |-> 'k ++ 1)), I, O, A) 
     =>
     pConfig(Sigma, Delta ((C @ n) :: T), 
       P || (family C q (bound 'k) ::= R') || 
       (C['k ++ 1] ::= R3), I, O, A) 
     /\
     R3 == replaceVars(R, q |-> 'k ++ 1)
       [nonexec] .
   
       
     crl [INDUCTION-R-two] :
     pConfig(Sigma, Delta ((C @ (n1 n2)) :: T), 
       P || (family C (q q') ((bound n1) (uniformBound n2)) ::= R), I, O, A)
     =>
     pConfig(Sigma, Delta ((C @ (n1 n2)) :: T), 
       P || (family C (q q') ((bound n1) (uniformBound n2)) ::= R'), I, O, A)
     if 
     pConfig(
       Sigma, Delta ((C @ (n1 n2)) :: T), 
       P || (C[q 0] ::= replaceVars(R, q' |-> 0)), I, 
       insert(C[q 0] @ nil, getOutputs(P)), A) 
     => 
     pConfig(Sigma, Delta ((C @ (n1 n2)):: T), 
       P || (C[q 0] ::= R2), I, O', A)
     /\ O' == insert(C[q 0] @ nil, getOutputs(P)) 
     /\
     R2 == replaceVars(R', q' |-> 0)
     /\
     pConfig(Sigma, Delta ((C @ (n1 n2)) :: T), 
       P || (family C (q q') ((bound n1) (uniformBound 'k)) ::= R') 
       || (C[q ('k ++ 1)] ::= replaceVars(R, q' |-> 'k ++ 1 )), I, 
       insert(C @ (n1 'k),  
                insert(C[q ('k ++ 1)] @ nil, getOutputs(P))), A) 
     =>
     pConfig(Sigma, Delta ((C @ (n1 n2)) :: T), 
       P || (family C (q q') ((bound n1) (uniformBound 'k)) ::= R')
         || (C[q ('k ++ 1)] ::= R3), I, O'', A) 
     /\
     O'' == insert(C @ (n1 'k),  
                insert(C[q ('k ++ 1)] @ nil, getOutputs(P)))
     /\
     R3 == replaceVars(R', q' |-> 'k ++ 1 )
       [nonexec] .  
  
       
     crl [INDUCTION-when-one] :
     pConfig(Sigma, Delta, 
       P || (family C q (bound nt1) ::= cases), I, O, A)
     =>
     pConfig(Sigma, Delta , 
       P || (family C q (bound nt1) ::= cases'), I, O, A)
     if 
     pConfig(Sigma, Delta , 
       P || (projectIndex (family C q (bound nt1) ::= cases) 0 A empty ), I, 
       insert( C[0] @ nil, O \ (C @ nt1)), 
       insert(q =T= 0, A)
       ) 
     => 
     pConfig(Sigma, Delta , P2 , I, O', A')
     /\
     O' == insert( C[0] @ nil, O \ (C @ nt1))
     /\
     A' ==  insert(q =T= 0, A) 
     /\
     P2 == P || (projectIndex (family C q (bound nt1) ::= cases') 0 A empty)
     /\
     pConfig(Sigma, Delta , 
       P || (family C q (bound 'k) ::= cases') || 
       (projectIndex 
        (family C q (bound nt1) ::= cases) ('k ++ 1) A empty), I, 
       insert(C @ 'k, insert(C['k ++ 1] @ nil, O \ (C @ nt1))), 
       insert('k ++ 1 <=T nt1, A)
       )
       =>
     pConfig(Sigma, Delta, P3, I, O'', A'')
     /\
     O'' = insert(C @ 'k, insert(C['k ++ 1] @ nil, O \ (C @ nt1)))
     /\
     A'' == insert('k ++ 1 <=T nt1, A) 
     /\
     P3 == (
             P || (family C q (bound 'k) ::= cases') ||
             (projectIndex 
               (family C q (bound nt1) ::= cases') ('k ++ 1) A empty
             )
           )
       [nonexec] . 
       
      crl [INDUCTION-when-one-comb] :
     pConfig(Sigma, Delta, 
       P || (family 'Comp[C C'] q (bound nt1) ::= 
                    ((C[q] ::= R1) || (C'[q] ::= R2)) ), I, O, A)
     =>
     pConfig(Sigma, Delta , 
       P || (family 'Comp[C C'] q (bound nt1) ::= 
                    ((C[q] ::= R3) || (C'[q] ::= R4)) ), I, O, A)
     if 
     pConfig(Sigma, Delta , 
       P || ((C [0] ::= replaceVars(R1, q |-> 0)) || 
             (C'[0] ::= replaceVars(R2, q |-> 0))
            ), I, 
       insert( C[0] @ nil, 
         insert( C'[0] @ nil, O \ (insert(C @ nt1, C' @ nt1)))), 
       insert(q =T= 0, A)
       ) 
     => 
     pConfig(Sigma, Delta , P2 , I, O', A')
     /\
     O' == insert( C[0] @ nil, 
            insert( C'[0] @ nil, O \ (insert(C @ nt1, C' @ nt1))))
     /\
     A' ==  insert(q =T= 0, A) 
     /\
     P2 == P || ((C [0] ::= replaceVars(R3, q |-> 0)) || 
                 (C'[0] ::= replaceVars(R4, q |-> 0)))
     /\
     pConfig(Sigma, Delta , 
       P || (family 'Comp[C C'] q (bound 'k) ::= 
                    ((C[q] ::= R3) || (C'[q] ::= R4)) ) || 
                      (C['k ++ 1] ::= replaceVars(R1, q |-> 'k ++ 1) ) 
                      ||
                      (C'['k ++ 1] ::= replaceVars(R2, q |-> 'k ++ 1)), I, 
       insert(C @ 'k, 
        insert(C' @ 'k,
         insert(C['k ++ 1] @ nil,
           insert(C'['k ++ 1] @ nil,
                  O \ (insert(C @ nt1, C' @ nt1)
                      )
                  )))),
       insert('k ++ 1 <=T nt1, A)
       )
       =>
     pConfig(Sigma, Delta, P3, I, O'', A'')
     /\
     O'' = insert(C @ 'k, 
        insert(C' @ 'k,
         insert(C['k ++ 1] @ nil,
           insert(C'['k ++ 1] @ nil,
                  O \ (insert(C @ nt1, C' @ nt1)
                      )
                  ))))
     /\
     A'' == insert('k ++ 1 <=T nt1, A) 
     /\
     P3 == (P || (family 'Comp[C C'] q (bound 'k) ::= 
                    ((C[q] ::= R3) || (C'[q] ::= R4)) ) || 
                      (C['k ++ 1] ::= replaceVars(R3, q |-> 'k ++ 1) ) 
                      ||
                      (C'['k ++ 1] ::= replaceVars(R4, q |-> 'k ++ 1))
           )
       [nonexec] .   
       
     crl [INDUCTION-when-two] :
     pConfig(Sigma, Delta, 
       P || (family C (q q') ((bound n1) (uniformBound n2)) ::= cases), 
       I, O, A)
     =>
     pConfig(Sigma, Delta , 
       P || (family C (q q') ((bound n1) (uniformBound n2)) ::= cases'), 
       I, O, A)
     if 
     pConfig(Sigma, Delta , 
       P || 
       (projectIndex 
         (family C (q q') ((bound n1) (uniformBound n2)) ::= cases) 
         0 A empty 
       ), I, 
       union (O \ (C @ (n1 n2)), C[q 0] @ nil), 
       insert(q' =T= 0, insert(q <=T n1, A))
       ) 
     => 
     pConfig(Sigma, Delta , 
       P || P2, I, O', A') 
     /\
     O' == union (O \ (C @ (n1 n2)), C[q 0] @ nil)
     /\
     A' == insert(q' =T= 0, insert(q <=T n1, A))
     /\
     P2 == projectIndex (
             family C (q q') ((bound n1) (uniformBound n2)) ::= cases') 
             0 A empty
     /\
     pConfig(Sigma, Delta , 
       P || (family C (q q') ((bound n1) (uniformBound 'k)) ::= cases') 
       || (projectIndex 
            (family C (q q') ((bound n1) (uniformBound n2)) ::= cases) 
            ('k ++ 1) A empty), 
             I, 
             insert( C @ (n1 'k) , 
                     insert( 
                       C[q ('k ++ 1)] @ nil, 
                       (O \ (C @ (n1 n2)) )
                     ) 
            ), 
             insert('k <T n2, 
               insert('k ++ 1 <=T n2, insert(q <=T n1, A)))
             )
       =>
     pConfig(Sigma, Delta , P4, I, O'', A'')  
     /\
     O'' == insert( C @ (n1 'k) , 
                     insert( 
                       C[q ('k ++ 1)] @ nil , 
                       (O \ (C @ (n1 n2)) )
                     ) 
            )
     /\
     A'' == insert('k <T n2, 
              insert('k ++ 1 <=T n2, insert(q <=T n1, A)))
     /\
     P4 == 
           P || 
           (family C (q q') ((bound n1) (uniformBound 'k)) ::= cases') || 
           (projectIndex 
             (family C (q q') ((bound n1) (uniformBound n2)) ::= cases')
             ('k ++ 1) A empty)
       [nonexec] .  
                              
    rl [convert-combined] :
       family C q (bound n) ::=
        (
         (when bt --> P1 )
         ;;
         (otherwise --> P2)
        )
       => 
       combine q n bt P1 P2 .
       
   rl [drop-group-name] :
     family C q (bound n) ::= P 
     => 
     P .     

   crl [wrap-channel-family] :  
     pConfig(Sigma, Delta, 
             newNF( ltq { C q (bound nt) : T } < C[nt + 1] : T >
                   , 
                   P || ( C[nt + 1] ::= R ) || 
                   family C q (bound nt) ::= 
                    (whenList1 ;; (when bt --> R) ;; whenList2), 
                   QL),
             I, O, A)  
     => 
     pConfig(Sigma, Delta, 
             newNF(ltq { C q (bound (nt + 1)) : T }
                   , 
                   P || 
                   family C q (bound (nt + 1)) ::=  
                    (whenList1 ;; (when bt --> R) ;; whenList2), 
                   del (C[nt + 1]) QL),
             I, O, A)  
     if A |= bt with (q |-> (nt + 1))
     .
     
     rl [unwrap-channel-family] :
     pConfig(Sigma, Delta, 
             newNF( ltq { C q (bound (nt + 1)) : T },
                    P  || family C q (bound (nt + 1)) ::= cases, 
                   QL),
             I, O, A)  
     => 
     pConfig(Sigma, Delta, 
             newNF( ltq { C q (bound nt) : T } < C[nt + 1] : T >,
                    P || 
                   (projectIndex 
                      (family C q (bound (nt + 1)) ::= cases) 
                      (nt + 1) A empty
                   ) ||
                   (family C q (bound nt) ::= cases), 
                   C[nt + 1] :: QL),
             I, O, A)  
    .  
    
    rl [wrap-channel-family-new-R] :
     pConfig(Sigma, Delta, 
             newNF( ltq { C q (bound nt) : T } < C[nt + 1] : T >
                   , 
                   P || ( C[nt + 1] ::= R1 ) || 
                  family C q (bound nt) ::= R2,
                   QL),
             I, O, A) 
     =>
     pConfig(Sigma, Delta, 
             newNF( ltq { C q (bound (nt + 1)) : T }
                   , 
                   P || family C q (bound (nt + 1)) ::= 
                          ((when (q =T= (nt + 1)) --> R1) ;; 
                            otherwise --> R2),
                   del (C[nt + 1]) QL),
             I, O, A)
    .  
    
     rl [wrap-channel-family-new-W] :
     pConfig(Sigma, Delta, 
             newNF( ltq { C q (bound nt) : T } < C[nt + 1] : T >
                   , 
                   P || ( C[nt + 1] ::= R1 ) || 
                   family C q (bound nt) ::= whenList,
                   QL),
             I, O, A) 
     =>
     pConfig(Sigma, Delta, 
             newNF( ltq { C q (bound (nt + 1)) : T }
                   , 
                   P || family C q (bound (nt + 1)) ::= 
                          ((when (q =T= (nt + 1)) --> R1) ;; 
                            whenList),
                   del (C[nt + 1]) QL),
             I, O, A)
    . 
      
    crl [merge-cases] :                 
    pConfig(Sigma, Delta, 
             newNF( ltq { C q (bound nt) : T }
                   , 
                   P || family C q (bound nt) ::= 
                         (whenList1 ;;
                          (when (q =T= nt1) --> R1) ;;
                          whenList2 ;;
                          (otherwise --> R2)),
                   QL),
             I, O, A) 
     => 
        pConfig(Sigma, Delta, 
             newNF( ltq { C q (bound nt) : T }
                   , 
                   P || family C q (bound nt) ::= 
                         (whenList1 ;;
                          whenList2 ;;
                          (otherwise --> R2)),
                   QL),
             I, O, A) 
     if replaceVars(R2, q |-> nt1) == R1 
   .   

\end{lstlisting}

\end{document}