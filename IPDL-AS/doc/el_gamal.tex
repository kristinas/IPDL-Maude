\renewcommand{\key}{\mathsf{key}}
\renewcommand{\msg}{\mathsf{msg}}
\renewcommand{\unif}{\mathsf{unif}}
\renewcommand{\gen}{\mathsf{g}}
\renewcommand{\mul}{\mathsf{mul}}
\renewcommand{\inv}{\mathsf{inv}}
\renewcommand{\exp}{\mathsf{exp}}
\renewcommand{\id}{\mathsf{id}}
\renewcommand{\adv}{\mathsf{adv}}
\renewcommand{\net}{\mathsf{net}}
\newcommand{\rec}{\mathsf{rec}}
\renewcommand{\In}{\mathsf{In}}
\renewcommand{\Out}{\mathsf{Out}}
\renewcommand{\Key}{\mathsf{Key}}
\renewcommand{\Send}{\mathsf{Send}}
\renewcommand{\Recv}{\mathsf{Recv}}
\renewcommand{\LeakMsgRcvd}{\mathsf{LeakMsgRcvd}}
\renewcommand{\OkMsg}{\mathsf{OkMsg}}
\renewcommand{\LeakCtxt}{\mathsf{LeakCtxt}}
\renewcommand{\OkCtxt}{\mathsf{OkCtxt}}
\renewcommand{\SecretKey}{\mathsf{SecretKey}}
\renewcommand{\PublicKey}{\mathsf{PublicKey}}
\newcommand{\SecretEphemeralKey}{\mathsf{SecretEphemeralKey}}
\newcommand{\PublicEphemeralKey}{\mathsf{PublicEphemeralKey}}
\newcommand{\SessionKey}{\mathsf{SessionKey}}
\renewcommand{\LeakPublicKey}{\mathsf{LeakPublicKey}}
\renewcommand{\DDH}{\mathsf{DDH}}

The El Gamal protocol is a CPA-secure public-key encryption schema based on the Diffie-Hellman Key Exchange. As in our previous case studies, Alice wants to communicate $q$ messages to Bob using an authenticated channel. In the symmetric-key setting, the two parties had to agree on a secret shared key that was used both for encryption and decryption. In the public-key setting of El Gamal, Bob generates two keys: the public key is used for encryption and is known to both Alice and the adversary, whereas the secret key is used for decryption and known only to Bob.

Formally, we assume a type $\key$ of secret keys (elements of $\{0,\ldots,p-1\}$, where $p$ is a prime); a type $\msg$ of messages, also serving as public keys (elements of a cyclic group $G = \{g^0,\ldots,g^{p-1}\}$); a uniform distribution $\unif_\key : \one \twoheadrightarrow \key$ on keys; a uniform distribution $\unif_\msg : \one \twoheadrightarrow \msg$ on messages; the generator $\gen : \one \rightarrow \msg$ of $G$; the group multiplication function $\mul : \msg \times \msg \rightarrow \msg$, where we write $m * k$ in place of $\mul \ (m,k)$; the group inverse function $\inv : \msg \rightarrow \msg$, where we write $k^{-1}$ in place of $\inv \ k$; and the exponentiation function $\exp : \msg \times \key \rightarrow \msg$ that raises an element of $G$ to the power of $k \in \{0,\ldots,p-1\}$. We will write $m^k$ in place of $\exp \ (m,k)$.

\subsection{The Assumptions}
At the level of expressions, we only need to know that exponents commute and inverses cancel on the right:
\begin{itemize}
\item $k : \key, l : \key \vdash (\gen^l)^k = (\gen^k)^l : \msg$, and
\item $m : \msg, k : \key \vdash (m * k) * k^{-1} = m : \msg$
\end{itemize}
At the level of distributions, we need to know that sampling a random secret key $k$ from $\unif_\key$ and returning $g^k$ is the same as sampling from $\unif_\msg$ directly (a consequence of $\unif_\key$ being uniform),
\begin{itemize}
\item $\cdot \vdash \big(k \leftarrow \unif_\key; \ \ret{\gen^k}\big) = \unif_\msg : \msg$
\end{itemize}
and that the distribution $\unif_\msg$ on messages is invariant under the operation of multiplication with a fixed message (a consequence of $\unif_\msg$ being uniform),
\begin{itemize}
\item $m : \msg \vdash \big(k \leftarrow \unif_\msg; \ \ret{m * k}\big) = \unif_\msg : \msg$.
\end{itemize}
Finally, at the level of protocols we need the \emph{decisional Diffie-Hellman (DDH)} cryptographic assumption: as long as the secret keys $k,l$ are generated uniformly, even if the adversary knows the values $g^k$ and $g^l$, they will be unable to distinguish $(g^k)^l$ from a uniformly generated element of $G$. The corresponding protocol-level axiom states that in the channel context $\DDH_3 : \msg \times (\msg \times \msg)$, the single-reaction no-input protocol
\begin{itemize}
\item $\DDH_3 \coloneqq k_B \leftarrow \unif_\key; \ k_A \leftarrow \unif_\key; \ \ret{\big(\gen^{k_B}, \big(\gen^{k_A}, (\gen^{k_B})^{k_A}\big)\big)}$
\end{itemize}
rewrites approximately to the protocol
\begin{itemize}
\item $\DDH_3 \coloneqq k_B \leftarrow \unif_\key; \ k_A \leftarrow \unif_\key; \ k_R \leftarrow \unif_\key; \ \ret{\big(\gen^{k_B}, \big(\gen^{k_A}, \gen^{k_R}\big)\big)}$
\end{itemize}

\noindent We now show that if the DDH assumption holds, the protocol
\begin{itemize}
\item $\PublicKey \coloneqq k_B \leftarrow \unif_\key; \ \ret{\gen^{k_B}}$
\item $\DDH_2 \coloneqq p_B \leftarrow \PublicKey; \ k_A \leftarrow \unif_\key; \ \ret{\big(\gen^{k_A}, p_B^{k_A}\big)}$
\end{itemize}
rewrites approximately to the protocol
\begin{itemize}
\item $\PublicKey \coloneqq k_B \leftarrow \unif_\key; \ \ret{\gen^{k_B}}$
\item $\DDH_2 \coloneqq k_A \leftarrow \unif_\key; \ k_R \leftarrow \unif_\key; \ \ret{\big(\gen^{k_A}, \gen^{k_R}}\big)$
\end{itemize}

\noindent To show this, we first unfold the key samplings from the channels $\PublicKey$ and $\DDH_2$ into new internal channels $\SecretKey$ and $\SecretEphemeralKey$, respectively:

\begin{itemize}
\item {\color{red} $\SecretKey \coloneqq \samp{\unif_\key}$}
\item {\color{red} $\SecretEphemeralKey \coloneqq \unif_\key$}
\item $\PublicKey \coloneqq k_B \leftarrow \SecretKey; \ \ret{\gen^{k_B}}$
\item {\color{red} $\DDH_2 \coloneqq p_B \leftarrow \PublicKey; \ k_A \leftarrow \SecretEphemeralKey; \ \ret{\big(\gen^{k_A}, p_B^{k_A}\big)}$}
\end{itemize}

\noindent Next we substitute the channel $\PublicKey$ into the channel $\DDH_2$, yielding

\begin{itemize}
\item $\SecretKey \coloneqq \samp{\unif_\key}$
\item $\SecretEphemeralKey \coloneqq \samp{\unif_\key}$
\item $\PublicKey \coloneqq k_B \leftarrow \SecretKey; \ \ret{\gen^{k_B}}$
\item {\color{red} $\DDH_2 \coloneqq k_B \leftarrow \SecretKey; \ k_A \leftarrow \SecretEphemeralKey; \ \ret{\big(\gen^{k_A}, (\gen^{k_B})^{k_A}\big)}$}
\end{itemize}

\noindent We subsequently add a gratuitous dependency on the channel $\SecretEphemeralKey$ into the channel $\PublicKey$:

\begin{itemize}
\item $\SecretKey \coloneqq \samp{\unif_\key}$
\item $\SecretEphemeralKey \coloneqq \samp{\unif_\key}$
\item {\color{red} $\PublicKey \coloneqq k_B \leftarrow \SecretKey; \ k_A \leftarrow \SecretEphemeralKey; \ \ret{\gen^{k_B}}$}
\item $\DDH_2 \coloneqq k_B \leftarrow \SecretKey; \ k_A \leftarrow \SecretEphemeralKey; \ \ret{\big(\gen^{k_A}, (\gen^{k_B})^{k_A}\big)}$
\end{itemize}

\noindent We can now extract the DDH triple into a new internal channel $\DDH_3$:

\begin{itemize}
\item $\SecretKey \coloneqq \samp{\unif_\key}$
\item $\SecretEphemeralKey \coloneqq \samp{\unif_\key}$
\item {\color{red} $\DDH_3 \coloneqq k_B \leftarrow \SecretKey; \ k_A \leftarrow \SecretEphemeralKey; \ \ret{\big(\gen^{k_B}, \big(\gen^{k_A}, (\gen^{k_B})^{k_A}\big)\big)}$}
\item {\color{red} $\PublicKey \coloneqq \big(p_B, \mathit{ddh}_2\big) \leftarrow \DDH_3; \ \ret{p_B}$}
\item {\color{red} $\DDH_2 \coloneqq \big(p_B, \mathit{ddh}_2\big) \leftarrow \DDH_3; \ \ret{\mathit{ddh}_2}$}
\end{itemize}

\noindent The internal channels $\SecretKey$ and $\SecretEphemeralKey$ are now only used in the channel $\DDH_3$, so we can fold them in:

\begin{itemize}
\item {\color{red} $\DDH_3 \coloneqq k_B \leftarrow \unif_\key; \ k_A \leftarrow \unif_\key; \ \ret{\big(\gen^{k_B}, \big(\gen^{k_A}, (\gen^{k_B})^{k_A}\big)\big)}$}
\item $\PublicKey \coloneqq \big(p_B, \mathit{ddh}_2\big) \leftarrow \DDH_3; \ \ret{p_B}$
\item $\DDH_2 \coloneqq \big(p_B, \mathit{ddh}_2\big) \leftarrow \DDH_3; \ \ret{\mathit{ddh}_2}$
\end{itemize}

\noindent Using the DDH assumption, we can approximately rewrite the above protocol to

\begin{itemize}
\item {\color{red} $\DDH_3 \coloneqq k_B \leftarrow \unif_\key; \ k_A \leftarrow \unif_\key; \ k_R \leftarrow \unif_\key; \ \ret{\big(\gen^{k_B}, \big(\gen^{k_A}, \gen^{k_R}\big)\big)}$}
\item $\PublicKey \coloneqq \big(p_B, \mathit{ddh}_2\big) \leftarrow \DDH_3; \ \ret{p_B}$
\item $\DDH_2 \coloneqq \big(p_B, \mathit{ddh}_2\big) \leftarrow \DDH_3; \ \ret{\mathit{ddh}_2}$
\end{itemize}

\noindent Unfolding the three samplings from the channel $\DDH_3$ into new internal channels $\SecretKey$, $\SecretEphemeralKey$, $\Key$ yields

\begin{itemize}
\item {\color{red} $\SecretKey \coloneqq \samp{\unif_\key}$}
\item {\color{red} $\SecretEphemeralKey \coloneqq \samp{\unif_\key}$}
\item {\color{red} $\Key \coloneqq \samp{\unif_\key}$}
\item {\color{red} $\DDH_3 \coloneqq k_B \leftarrow \SecretKey; \ k_A \leftarrow \SecretEphemeralKey; \ k_R \leftarrow \Key; \ \ret{\big(\gen^{k_B}, \big(\gen^{k_A}, \gen^{k_R}\big)\big)}$}
\item $\PublicKey \coloneqq \big(p_B, \mathit{ddh}_2\big) \leftarrow \DDH_3; \ \ret{p_B}$
\item $\DDH_2 \coloneqq \big(p_B, \mathit{ddh}_2\big) \leftarrow \DDH_3; \ \ret{\mathit{ddh}_2}$
\end{itemize}

\noindent The internal channel $\DDH_3$ can now be substituted away:

\begin{itemize}
\item $\SecretKey \coloneqq \samp{\unif_\key}$
\item $\SecretEphemeralKey \coloneqq \samp{\unif_\key}$
\item $\Key \coloneqq \samp{\unif_\key}$
\item {\color{red} $\PublicKey \coloneqq k_B \leftarrow \SecretKey; \ k_A \leftarrow \SecretEphemeralKey; \ k_R \leftarrow \Key; \ \ret{\gen^{k_B}}$}
\item {\color{red} $\DDH_2 \coloneqq k_B \leftarrow \SecretKey; \ k_A \leftarrow \SecretEphemeralKey; \ k_R \leftarrow \Key; \ \ret{\big(\gen^{k_A}, \gen^{k_R}\big)}$}
\end{itemize}

\noindent After dropping the gratuitous dependencies -- on channels $\SecretEphemeralKey$, $\Key$ in the channel $\PublicKey$, and channel $\SecretKey$ in the channel $\DDH_2$ -- we end up with the following:

\begin{itemize}
\item $\SecretKey \coloneqq \samp{\unif_\key}$
\item $\SecretEphemeralKey \coloneqq \samp{\unif_\key}$
\item $\Key \coloneqq \samp{\unif_\key}$
\item {\color{red} $\PublicKey \coloneqq k_B \leftarrow \SecretKey; \ \ret{\gen^{k_B}}$}
\item {\color{red} $\DDH_2 \coloneqq k_A \leftarrow \SecretEphemeralKey; \ k_R \leftarrow \Key; \ \ret{\big(\gen^{k_A}, \gen^{k_R}\big)}$}
\end{itemize}

\noindent The channel $\SecretKey$ is now only used in the channel $\PublicKey$, and the channels $\SecretEphemeralKey$ and $\Key$ are only used in the channel $\DDH_2$. Performing the appropriate foldings results in the protocol

\begin{itemize}
\item $\PublicKey \coloneqq k_B \leftarrow \unif_\key; \ \ret{\gen^{k_B}}$
\item $\DDH_2 \coloneqq k_A \leftarrow \unif_\key; \ k_R \leftarrow \unif_\key; \ \ret{\big(\gen^{k_A}, \gen^{k_R}\big)}$
\end{itemize}

\noindent and we are done. Generalizing this argument to $q$ sessions, we have just shown that the protocol

\begin{itemize}
\item $\PublicKey \coloneqq k_B \leftarrow \unif_\key; \ \ret{\gen^{k_B}}$
\item $\DDH_2(i) \coloneqq p_B \leftarrow \PublicKey; \ k_A \leftarrow \unif_\key; \ \ret{\big(\gen^{k_A}, p_B^{k_A}\big)}$ for $0 \leq i < q$
\end{itemize}
rewrites approximately to the protocol
\begin{itemize}
\item $\PublicKey \coloneqq k_B \leftarrow \unif_\key; \ \ret{\gen^{k_B}}$
\item $\DDH_2(i) \coloneqq k_A \leftarrow \unif_\key; \ k_R \leftarrow \unif_\key; \ \ret{\big(\gen^{k_A}, \gen^{k_R}\big)}$ for $0 \leq i < q$
\end{itemize}

\subsection{The Ideal Functionality}
The ideal functionality reads the input message, leaks a confirmation to the adversary to signal that the message has been received, and, upon the approval from the adversary, outputs the message:
\begin{itemize}
\item $\LeakMsgRcvd(i)^\id_\adv \coloneqq m \leftarrow \In(i); \ \ret{\checkmark}$ for $0 \leq i < q$
\item $\Out(i) \coloneqq \_ \leftarrow \OkMsg(i)^\adv_\id; \ \read{\In(i)}$ for $0 \leq i < q$
\end{itemize}

\subsection{The Real Protocol}
The real-world protocol consists of Alice, Bob, and an authenticated channel. Bob randomly generates a secret key $k_B$ visible only to him:
\begin{itemize}
\item $\SecretKey \coloneqq \samp{\unif_\key}$
\end{itemize}
He computes the value $g^{k_B}$ as the public key visible to everybody, including the adversary; we model the adversary's access to the public key by an explicit leakage function:
\begin{itemize}
\item $\PublicKey \coloneqq k_B \leftarrow \SecretKey; \ \ret{\gen^{k_B}}$
\item $\LeakPublicKey^\rec_\adv \coloneqq \read{\PublicKey}$
\end{itemize}
Using the public key $g^{k_B}$, Alice encrypts each message by randomly generating a secret ephemeral key $k_A$ and constructing the ciphertext in two parts. The first is the public ephemeral key $g^{k_A}$, and the second is the encoding of the input message by multiplying it with the session key $(g^{k_B})^{k_A}$:
\begin{itemize}
\item $\SecretEphemeralKey(i) \coloneqq \samp{\unif_\key}$ for $0 \leq i < q$
\item $\PublicEphemeralKey(i) \coloneqq k_A \leftarrow \SecretEphemeralKey(i); \ \ret{\gen^{k_A}}$
\item $\SessionKey(i) \coloneqq p_B \leftarrow \PublicKey; \ k_A \leftarrow \SecretEphemeralKey(i); \ \ret{p_B^{k_A}}$ for $0 \leq i < q$
\item $\Send(i) \coloneqq m \leftarrow \In(i); \ k \leftarrow \SessionKey(i); \ p_A \leftarrow \PublicEphemeralKey(i); \ \ret{\big(p_A, m * k\big)}$ for $0 \leq i < q$
\end{itemize}
The authenticated channel leaks each ciphertext to the adversary and waits for their approval before forwarding it to Bob:
\begin{itemize}
\item $\LeakCtxt(i)^\adv_\net \coloneqq \read{\Send(i)}$ for $0 \leq i < q$
\item $\Recv(i) \coloneqq \_ \leftarrow \OkCtxt(i)^\adv_\net; \ \read{\Send(i)}$ for $0 \leq i < q$
\end{itemize}
At last, Bob decrypts the ciphertext by multiplying the message encoding (the second half) by the inverse of the session key $(g^{k_A})^{k_B}$. He constructs the session key from the the public ephemeral key (the first half) and his own secret key $k_B$:
\begin{itemize}
\item $\Out(i) \coloneqq \big(p_A, c\big) \leftarrow \Recv(i); \ k_B \leftarrow \SecretKey; \ \ret{c * \big(p_A^{k_B}\big)^{-1}}$ for $0 \leq i < q$
\end{itemize}
Thus, we have the following code for Alice:
\begin{itemize}
\item $\SecretEphemeralKey(i) \coloneqq \samp{\unif_\key}$ for $0 \leq i < q$
\item $\PublicEphemeralKey(i) \coloneqq k_A \leftarrow \SecretEphemeralKey(i); \ \ret{\gen^{k_A}}$ for $0 \leq i < q$
\item $\SessionKey(i) \coloneqq p_B \leftarrow \PublicKey; \ k_A \leftarrow \SecretEphemeralKey(i); \ \ret{p_B^{k_A}}$ for $0 \leq i < q$
\item $\Send(i) \coloneqq m \leftarrow \In(i); \ k \leftarrow \SessionKey(i); \ p_A \leftarrow \PublicEphemeralKey(i); \ \ret{\big(p_A, m * k\big)}$ for $0 \leq i < q$
\end{itemize}
The code for Bob is below:
\begin{itemize}
\item $\SecretKey \coloneqq \samp{\unif_\key}$
\item $\PublicKey \coloneqq k_B \leftarrow \SecretKey; \ \ret{\gen^{k_B}}$
\item $\LeakPublicKey^\rec_\adv \coloneqq \read{\PublicKey}$
\item $\Out(i) \coloneqq \big(p_A, c\big) \leftarrow \Recv(i); \ k_B \leftarrow \SecretKey; \ \ret{c * \big(p_A^{k_B}\big)^{-1}}$ for $0 \leq i < q$
\end{itemize}
Finally, we recall the code for the authenticated channel:
\begin{itemize}
\item $\LeakCtxt(i)^\adv_\net \coloneqq \read{\Send(i)}$ for $0 \leq i < q$
\item $\Recv(i) \coloneqq \_ \leftarrow \OkCtxt(i)^\adv_\net; \ \read{\Send(i)}$ for $0 \leq i < q$
\end{itemize}
Composing all of this together and hiding the internal communication yields the real-world protocol.

\subsection{The Simulator}
The channels $\OkCtxt(-)^\adv_\net$, $\LeakMsgRcvd(-)^\id_\adv$ are inputs to the simulator, whereas the channels $\OkMsg(i)^\adv_\id$, $\LeakPublicKey^\rec_\adv$, $\LeakCtxt(-)^\adv_\net$ are the outputs. The simulator constructs the public key and the public ephemeral key exactly as in the real world: by randomly generating the secret key $k_A$ and the secret ephemeral key $k_B$, and computing $g^{k_A}$ and $g^{k_B}$. Upon receiving the empty message from the ideal functionality to indicate that a message has been received, the simulator leaks the ciphertext by pairing up the public ephemeral key with a random element of $G$:

\begin{itemize}
\item $\SecretKey \coloneqq \samp{\unif_\key}$
\item $\PublicKey \coloneqq k_B \leftarrow \SecretKey; \ \ret{\gen^{k_B}}$
\item $\LeakPublicKey^\rec_\adv \coloneqq \read{\PublicKey}$
\item $\SecretEphemeralKey(i) \coloneqq \samp{\unif_\key}$ for $0 \leq i < q$
\item $\PublicEphemeralKey(i) \coloneqq k_A \leftarrow \SecretEphemeralKey(i); \ \ret{\gen^{k_A}}$ for $0 \leq i < q$
\item $\LeakCtxt(i)^\adv_\net \coloneqq \_ \leftarrow \LeakMsgRcvd(i)^\id_\adv; \ c \leftarrow \unif_\msg; \ p_A \leftarrow \PublicEphemeralKey(i); \ \ret{\big(p_A, c\big)}$ for $0 \leq i < q$
\item $\OkMsg(i)^\adv_\id \coloneqq \read{\OkCtxt(i)^\adv_\net}$ for $0 \leq i < q$
\end{itemize}

\subsection{Real \texorpdfstring{$\approx$}{Approximates} Ideal + Simulator}
Plugging the simulator into the ideal functionality and substituting away the internal channels $\LeakMsgRcvd(-)^\id_\adv$ and $\OkMsg(-)^\adv_\id$ that originally served as a line of communication for the adversary yields the following:

\begin{itemize}
\item $\SecretKey \coloneqq \samp{\unif_\key}$
\item $\PublicKey \coloneqq k_B \leftarrow \SecretKey; \ \ret{\gen^{k_B}}$
\item $\LeakPublicKey^\rec_\adv \coloneqq \read{\PublicKey}$
\item $\SecretEphemeralKey(i) \coloneqq \samp{\unif_\key}$ for $0 \leq i < q$
\item $\PublicEphemeralKey(i) \coloneqq k_A \leftarrow \SecretEphemeralKey(i); \ \ret{\gen^{k_A}}$ for $0 \leq i < q$
\item $\LeakCtxt(i)^\adv_\net \coloneqq \_ \leftarrow \In(i); \ c \leftarrow \unif_\msg; \ p_A \leftarrow \PublicEphemeralKey(i); \ \ret{\big(p_A, c\big)}$ for $0 \leq i < q$
\item $\Out(i) \coloneqq \_ \leftarrow \OkCtxt(i)^\adv_\net; \ \read{\In(i)}$ for $0 \leq i < q$
\end{itemize}

\noindent Next we move on to simplifying the real protocol.  Explicitly, we have the code below:

\begin{itemize}
\item $\SecretKey \coloneqq \samp{\unif_\key}$
\item $\PublicKey \coloneqq k_B \leftarrow \SecretKey; \ \ret{\gen^{k_B}}$
\item $\LeakPublicKey^\rec_\adv \coloneqq \read{\PublicKey}$
\item $\SecretEphemeralKey(i) \coloneqq \samp{\unif_\key}$ for $0 \leq i < q$
\item $\PublicEphemeralKey(i) \coloneqq k_A \leftarrow \SecretEphemeralKey(i); \ \ret{\gen^{k_A}}$ for $0 \leq i < q$
\item $\SessionKey(i) \coloneqq p_B \leftarrow \PublicKey; \ k_A \leftarrow \SecretEphemeralKey(i); \ \ret{p_B^{k_A}}$ for $0 \leq i < q$
\item $\Send(i) \coloneqq m \leftarrow \In(i); \ k \leftarrow \SessionKey(i); \ p_A \leftarrow \PublicEphemeralKey(i); \ \ret{\big(p_A, m * k\big)}$ for $0 \leq i < q$
\item $\LeakCtxt(i)^\adv_\net \coloneqq \read{\Send(i)}$ for $0 \leq i < q$
\item $\Recv(i) \coloneqq \_ \leftarrow \OkCtxt(i)^\adv_\net; \ \read{\Send(i)}$ for $0 \leq i < q$
\item $\Out(i) \coloneqq \big(p_A, c\big) \leftarrow \Recv(i); \ k_B \leftarrow \SecretKey; \ \ret{c * \big(p_A^{k_B}\big)^{-1}}$ for $0 \leq i < q$
\end{itemize}

\noindent We first substitute away the internal channels $\Send(-)$ and $\Recv(-)$:

\begin{itemize}
\item $\SecretKey \coloneqq \samp{\unif_\key}$
\item $\PublicKey \coloneqq k_B \leftarrow \SecretKey; \ \ret{\gen^{k_B}}$
\item $\LeakPublicKey^\rec_\adv \coloneqq \read{\PublicKey}$
\item $\SecretEphemeralKey(i) \coloneqq \samp{\unif_\key}$ for $0 \leq i < q$
\item $\PublicEphemeralKey(i) \coloneqq k_A \leftarrow \SecretEphemeralKey(i); \ \ret{\gen^{k_A}}$ for $0 \leq i < q$
\item $\SessionKey(i) \coloneqq p_B \leftarrow \PublicKey; \ k_A \leftarrow \SecretEphemeralKey(i); \ \ret{p_B^{k_A}}$ for $0 \leq i < q$
\item {\color{red} $\LeakCtxt(i)^\adv_\net \coloneqq m \leftarrow \In(i); \ k \leftarrow \SessionKey(i); \ p_A \leftarrow \PublicEphemeralKey(i); \ \ret{\big(p_A, m * k\big)}$ for $0 \leq i < q$}
\item {\color{red} $\Out(i) \coloneqq \_ \leftarrow \OkCtxt(i)^\adv_\net; \ m \leftarrow \In(i); \ k \leftarrow \SessionKey(i); \ p_A \leftarrow \PublicEphemeralKey(i); \ k_B \leftarrow \SecretKey; \\ \ret{(m * k) * \big(p_A^{k_B}\big)^{-1}}$ for $0 \leq i < q$}
\end{itemize}

\noindent Substituting the channels $\PublicKey$, $\SessionKey(i)$, $\PublicEphemeralKey(i)$ into the channel $\Out(i)$ yields
\begin{itemize}
\item $\Out(i) \coloneqq \_ \leftarrow \OkCtxt(i)^\adv_\net; \ m \leftarrow \In(i); \ k_B \leftarrow \SecretKey; \ k_A \leftarrow \SecretEphemeralKey(i); \\ \ret{\big(m * (\gen^{k_B})^{k_A}\big) * \big((\gen^{k_A})^{k_B}\big)^{-1}}$ for $0 \leq i < q$
\end{itemize}
Since exponents commute, we can rewrite the channel $\Out(i)$ equivalently as
\begin{itemize}
\item $\Out(i) \coloneqq \_ \leftarrow \OkCtxt(i)^\adv_\net; \ m \leftarrow \In(i); \ k_B \leftarrow \SecretKey; \ k_A \leftarrow \SecretEphemeralKey(i); \\ \ret{\big(m * (\gen^{k_A})^{k_B}\big) * \big((\gen^{k_A})^{k_B}\big)^{-1}}$ for $0 \leq i < q$
\end{itemize}
We can now cancel out the two applications of $*$ to get:
\begin{itemize}
\item $\Out(i) \coloneqq \_ \leftarrow \OkCtxt(i)^\adv_\net; \ m \leftarrow \In(i); \ k_B \leftarrow \SecretKey; \ k_A \leftarrow \SecretEphemeralKey(i); \ \ret{m}$ for $0 \leq i < q$
\end{itemize}
Dropping the gratuitous dependencies on the channels $\SecretKey$ and $\SecretEphemeralKey(i)$ yields
\begin{itemize}
\item $\Out(i) \coloneqq \_ \leftarrow \OkCtxt(i)^\adv_\net; \ m \leftarrow \In(i); \ \ret{m}$ for $0 \leq i < q$
\end{itemize}
which simplifies to
\begin{itemize}
\item $\Out(i) \coloneqq \_ \leftarrow \OkCtxt(i)^\adv_\net; \ \read{\In(i)}$ for $0 \leq i < q$
\end{itemize}

\noindent To summarize, our cleaned-up version of the real protocol looks as follows:

\begin{itemize}
\item $\SecretKey \coloneqq \samp{\unif_\key}$
\item $\PublicKey \coloneqq k_B \leftarrow \SecretKey; \ \ret{\gen^{k_B}}$
\item $\LeakPublicKey^\rec_\adv \coloneqq \read{\PublicKey}$
\item $\SecretEphemeralKey(i) \coloneqq \samp{\unif_\key}$ for $0 \leq i < q$
\item $\PublicEphemeralKey(i) \coloneqq k_A \leftarrow \SecretEphemeralKey(i); \ \ret{\gen^{k_A}}$ for $0 \leq i < q$
\item $\SessionKey(i) \coloneqq p_B \leftarrow \PublicKey; \ k_A \leftarrow \SecretEphemeralKey(i); \ \ret{p_B^{k_A}}$ for $0 \leq i < q$
\item $\LeakCtxt(i)^\adv_\net \coloneqq m \leftarrow \In(i); \ k \leftarrow \SessionKey(i); \ p_A \leftarrow \PublicEphemeralKey(i); \ \ret{\big(p_A, m * k\big)}$ for $0 \leq i < q$
\item $\Out(i) \coloneqq \_ \leftarrow \OkCtxt(i)^\adv_\net; \ \read{\In(i)}$ for $0 \leq i < q$
\end{itemize}

\noindent The channel $\SecretKey$ is now only used in $\PublicKey$, so we can fold it in:

\begin{itemize}
\item {\color{red} $\PublicKey \coloneqq k_B \leftarrow \unif_\key; \ \ret{\gen^{k_B}}$}
\item $\LeakPublicKey^\rec_\adv \coloneqq \read{\PublicKey}$
\item $\SecretEphemeralKey(i) \coloneqq \samp{\unif_\key}$ for $0 \leq i < q$
\item $\PublicEphemeralKey(i) \coloneqq k_A \leftarrow \SecretEphemeralKey(i); \ \ret{\gen^{k_A}}$ for $0 \leq i < q$
\item $\SessionKey(i) \coloneqq p_B \leftarrow \PublicKey; \ k_A \leftarrow \SecretEphemeralKey(i); \ \ret{p_B^{k_A}}$ for $0 \leq i < q$
\item $\LeakCtxt(i)^\adv_\net \coloneqq m \leftarrow \In(i); \ k \leftarrow \SessionKey(i); \ p_A \leftarrow \PublicEphemeralKey(i); \ \ret{\big(p_A, m * k\big)}$ for $0 \leq i < q$
\item $\Out(i) \coloneqq \_ \leftarrow \OkCtxt(i)^\adv_\net; \ \read{\In(i)}$ for $0 \leq i < q$
\end{itemize}

\noindent Adding a gratuitous dependency on the channel $\PublicKey$ into the channel $\PublicEphemeralKey(i)$ yields

\begin{itemize}
\item $\PublicKey \coloneqq k_B \leftarrow \unif_\key; \ \ret{\gen^{k_B}}$
\item $\LeakPublicKey^\rec_\adv \coloneqq \read{\PublicKey}$
\item $\SecretEphemeralKey(i) \coloneqq \samp{\unif_\key}$ for $0 \leq i < q$
\item {\color{red} $\PublicEphemeralKey(i) \coloneqq p_B \leftarrow \PublicKey; \ k_A \leftarrow \SecretEphemeralKey(i); \ \ret{\gen^{k_A}}$ for $0 \leq i < q$}
\item $\SessionKey(i) \coloneqq p_B \leftarrow \PublicKey; \ k_A \leftarrow \SecretEphemeralKey(i); \ \ret{p_B^{k_A}}$ for $0 \leq i < q$
\item $\LeakCtxt(i)^\adv_\net \coloneqq m \leftarrow \In(i); \ k \leftarrow \SessionKey(i); \ p_A \leftarrow \PublicEphemeralKey(i); \ \ret{\big(p_A, m * k\big)}$ for $0 \leq i < q$
\item $\Out(i) \coloneqq \_ \leftarrow \OkCtxt(i)^\adv_\net; \ \read{\In(i)}$ for $0 \leq i < q$
\end{itemize}

\noindent We can now combine the construction of the public ephemeral key and the session key into a new internal channel $\DDH_2(i)$:

\begin{itemize}
\item $\PublicKey \coloneqq k_B \leftarrow \unif_\key; \ \ret{\gen^{k_B}}$
\item $\LeakPublicKey^\rec_\adv \coloneqq \read{\PublicKey}$
\item $\SecretEphemeralKey(i) \coloneqq \samp{\unif_\key}$ for $0 \leq i < q$
\item {\color{red} $\DDH_2(i) \coloneqq p_B \leftarrow \PublicKey; \ k_A \leftarrow \SecretEphemeralKey(i); \ \ret{\big(\gen^{k_A}, p_B^{k_A}\big)}$ for $0 \leq i < q$}
\item {\color{red} $\PublicEphemeralKey(i) \coloneqq \big(p_A, k\big) \leftarrow \DDH_2(i); \ \ret{p_A}$ for $0 \leq i < q$}
\item {\color{red} $\SessionKey(i) \coloneqq \big(p_A, k\big) \leftarrow \DDH_2(i); \ \ret{k}$ for $0 \leq i < q$}
\item $\LeakCtxt(i)^\adv_\net \coloneqq m \leftarrow \In(i); \ k \leftarrow \SessionKey(i); \ p_A \leftarrow \PublicEphemeralKey(i); \ \ret{\big(p_A, m * k\big)}$ for $0 \leq i < q$
\item $\Out(i) \coloneqq \_ \leftarrow \OkCtxt(i)^\adv_\net; \ \read{\In(i)}$ for $0 \leq i < q$
\end{itemize}

\noindent The channel $\SecretEphemeralKey(i)$ is now only used in $\DDH_2(i)$, so we can fold it in:

\begin{itemize}
\item $\PublicKey \coloneqq k_B \leftarrow \unif_\key; \ \ret{\gen^{k_B}}$
\item $\LeakPublicKey^\rec_\adv \coloneqq \read{\PublicKey}$
\item {\color{red} $\DDH_2(i) \coloneqq p_B \leftarrow \PublicKey; \ k_A \leftarrow \unif_\key; \ \ret{\big(\gen^{k_A}, p_B^{k_A}\big)}$ for $0 \leq i < q$}
\item $\PublicEphemeralKey(i) \coloneqq \big(p_A, k\big) \leftarrow \DDH_2(i); \ \ret{p_A}$ for $0 \leq i < q$
\item $\SessionKey(i) \coloneqq \big(p_A, k\big) \leftarrow \DDH_2(i); \ \ret{k}$ for $0 \leq i < q$
\item $\LeakCtxt(i)^\adv_\net \coloneqq m \leftarrow \In(i); \ k \leftarrow \SessionKey(i); \ p_A \leftarrow \PublicEphemeralKey(i); \ \ret{\big(p_A, m * k\big)}$ for $0 \leq i < q$
\item $\Out(i) \coloneqq \_ \leftarrow \OkCtxt(i)^\adv_\net; \ \read{\In(i)}$ for $0 \leq i < q$
\end{itemize}

\noindent As we observed earlier, the protocol snippet

\begin{itemize}
\item $\PublicKey \coloneqq k_B \leftarrow \unif_\key; \ \ret{\gen^{k_B}}$
\item $\DDH_2(i) \coloneqq p_B \leftarrow \PublicKey; \ k_A \leftarrow \unif_\key; \ \ret{\big(\gen^{k_A}, p_B^{k_A}\big)}$ for $0 \leq i < q$
\end{itemize}

\noindent rewrites approximately to

\begin{itemize}
\item $\PublicKey \coloneqq k_B \leftarrow \unif_\key; \ \ret{\gen^{k_B}}$
\item $\DDH_2(i) \coloneqq k_A \leftarrow \unif_\key; \ k_R \leftarrow \unif_\key; \ \ret{\big(\gen^{k_A}, \gen^{k_R}\big)}$ for $0 \leq i < q$
\end{itemize}

\noindent Our real-world protocol therefore approximately rewrites to the protocol

\begin{itemize}
\item $\PublicKey \coloneqq k_B \leftarrow \unif_\key; \ \ret{\gen^{k_B}}$
\item $\LeakPublicKey^\rec_\adv \coloneqq \read{\PublicKey}$
\item {\color{red} $\DDH_2(i) \coloneqq k_A \leftarrow \unif_\key; \ k_R \leftarrow \unif_\key; \ \ret{\big(\gen^{k_A}, \gen^{k_R}\big)}$ for $0 \leq i < q$}
\item $\PublicEphemeralKey(i) \coloneqq \big(p_A, k\big) \leftarrow \DDH_2(i); \ \ret{p_A}$ for $0 \leq i < q$
\item $\SessionKey(i) \coloneqq \big(p_A, k\big) \leftarrow \DDH_2(i); \ \ret{k}$ for $0 \leq i < q$
\item $\LeakCtxt(i)^\adv_\net \coloneqq m \leftarrow \In(i); \ k \leftarrow \SessionKey(i); \ p_A \leftarrow \PublicEphemeralKey(i); \ \ret{\big(p_A, m * k\big)}$ for $0 \leq i < q$
\item $\Out(i) \coloneqq \_ \leftarrow \OkCtxt(i)^\adv_\net; \ \read{\In(i)}$ for $0 \leq i < q$
\end{itemize}

\noindent Unfolding the two samplings from $\DDH_2(i)$ into new internal channels $\SecretEphemeralKey(i)$ and $\Key(i)$ yields

\begin{itemize}
\item $\PublicKey \coloneqq k_B \leftarrow \unif_\key; \ \ret{\gen^{k_B}}$
\item $\LeakPublicKey^\rec_\adv \coloneqq \read{\PublicKey}$
\item {\color{red} $\SecretEphemeralKey(i) \coloneqq \samp{\unif_\key}$ for $0 \leq i < q$}
\item {\color{red} $\Key(i) \coloneqq \samp{\unif_\key}$ for $0 \leq i < q$}
\item {\color{red}  $\DDH_2(i) \coloneqq k_A \leftarrow \SecretEphemeralKey(i); \ k_R \leftarrow \Key(i); \ \ret{\big(\gen^{k_A}, \gen^{k_R}\big)}$ for $0 \leq i < q$}
\item $\PublicEphemeralKey(i) \coloneqq \big(p_A, k\big) \leftarrow \DDH_2(i); \ \ret{p_A}$ for $0 \leq i < q$
\item $\SessionKey(i) \coloneqq \big(p_A, k\big) \leftarrow \DDH_2(i); \ \ret{k}$ for $0 \leq i < q$
\item $\LeakCtxt(i)^\adv_\net \coloneqq m \leftarrow \In(i); \ k \leftarrow \SessionKey(i); \ p_A \leftarrow \PublicEphemeralKey(i); \ \ret{\big(p_A, m * k\big)}$ for $0 \leq i < q$
\item $\Out(i) \coloneqq \_ \leftarrow \OkCtxt(i)^\adv_\net; \ \read{\In(i)}$ for $0 \leq i < q$
\end{itemize}

\noindent The channel $\DDH_2(i)$ can now be substituted away:

\begin{itemize}
\item $\PublicKey \coloneqq k_B \leftarrow \unif_\key; \ \ret{\gen^{k_B}}$
\item $\LeakPublicKey^\rec_\adv \coloneqq \read{\PublicKey}$
\item $\SecretEphemeralKey(i) \coloneqq \samp{\unif_\key}$ for $0 \leq i < q$
\item $\Key(i) \coloneqq \samp{\unif_\key}$ for $0 \leq i < q$
\item {\color{red} $\PublicEphemeralKey(i) \coloneqq k_A \leftarrow \SecretEphemeralKey(i); \ k_R \leftarrow \Key(i); \ \ret{\gen^{k_A}}$ for $0 \leq i < q$}
\item {\color{red} $\SessionKey(i) \coloneqq k_A \leftarrow \SecretEphemeralKey(i); \ k_R \leftarrow \Key(i); \ \ret{\gen^{k_R}}$ for $0 \leq i < q$}
\item $\LeakCtxt(i)^\adv_\net \coloneqq m \leftarrow \In(i); \ k \leftarrow \SessionKey(i); \ p_A \leftarrow \PublicEphemeralKey(i); \ \ret{\big(p_A, m * k\big)}$ for $0 \leq i < q$
\item $\Out(i) \coloneqq \_ \leftarrow \OkCtxt(i)^\adv_\net; \ \read{\In(i)}$ for $0 \leq i < q$
\end{itemize}

\noindent After dropping the gratuitous dependencies -- on channel $\Key(i)$ from the channel $\PublicEphemeralKey(i)$, and channel $\SecretEphemeralKey(i)$ from the channel $\SessionKey(i)$ -- we end up with the following:

\begin{itemize}
\item $\PublicKey \coloneqq k_B \leftarrow \unif_\key; \ \ret{\gen^{k_B}}$
\item $\LeakPublicKey^\rec_\adv \coloneqq \read{\PublicKey}$
\item $\SecretEphemeralKey(i) \coloneqq \samp{\unif_\key}$ for $0 \leq i < q$
\item $\Key(i) \coloneqq \samp{\unif_\key}$ for $0 \leq i < q$
\item {\color{red} $\PublicEphemeralKey(i) \coloneqq k_A \leftarrow \SecretEphemeralKey(i); \ \ret{\gen^{k_A}}$ for $0 \leq i < q$}
\item {\color{red} $\SessionKey(i) \coloneqq k_R \leftarrow \Key(i); \ \ret{\gen^{k_R}}$ for $0 \leq i < q$}
\item $\LeakCtxt(i)^\adv_\net \coloneqq m \leftarrow \In(i); \ k \leftarrow \SessionKey(i); \ p_A \leftarrow \PublicEphemeralKey(i); \ \ret{\big(p_A, m * k\big)}$ for $0 \leq i < q$
\item $\Out(i) \coloneqq \_ \leftarrow \OkCtxt(i)^\adv_\net; \ \read{\In(i)}$ for $0 \leq i < q$
\end{itemize}

\noindent We now unfold the sampling from $\PublicKey$ into a new internal channel $\SecretKey$:

\begin{itemize}
\item {\color{red}$\SecretKey \coloneqq \samp{\unif_\key}$}
\item {\color{red} $\PublicKey \coloneqq k_B \leftarrow \SecretKey; \ \ret{\gen^{k_B}}$}
\item $\LeakPublicKey^\rec_\adv \coloneqq \read{\PublicKey}$
\item $\SecretEphemeralKey(i) \coloneqq \samp{\unif_\key}$ for $0 \leq i < q$
\item $\Key(i) \coloneqq \samp{\unif_\key}$ for $0 \leq i < q$
\item $\PublicEphemeralKey(i) \coloneqq k_A \leftarrow \SecretEphemeralKey(i); \ \ret{\gen^{k_A}}$ for $0 \leq i < q$
\item $\SessionKey(i) \coloneqq k_R \leftarrow \Key(i); \ \ret{\gen^{k_R}}$ for $0 \leq i < q$
\item $\LeakCtxt(i)^\adv_\net \coloneqq m \leftarrow \In(i); \ k \leftarrow \SessionKey(i); \ p_A \leftarrow \PublicEphemeralKey(i); \ \ret{\big(p_A, m * k\big)}$ for $0 \leq i < q$
\item $\Out(i) \coloneqq \_ \leftarrow \OkCtxt(i)^\adv_\net; \ \read{\In(i)}$ for $0 \leq i < q$
\end{itemize}

\noindent The channel $\Key(i)$ is now only used in $\SessionKey(i)$, so we can fold it in:

\begin{itemize}
\item $\SecretKey \coloneqq \samp{\unif_\key}$
\item $\PublicKey \coloneqq k_B \leftarrow \SecretKey; \ \ret{\gen^{k_B}}$
\item $\LeakPublicKey^\rec_\adv \coloneqq \read{\PublicKey}$
\item $\SecretEphemeralKey(i) \coloneqq \samp{\unif_\key}$ for $0 \leq i < q$
\item $\PublicEphemeralKey(i) \coloneqq k_A \leftarrow \SecretEphemeralKey(i); \ \ret{\gen^{k_A}}$ for $0 \leq i < q$
\item {\color{red} $\SessionKey(i) \coloneqq k_R \leftarrow \unif_\key; \ \ret{\gen^{k_R}}$ for $0 \leq i < q$}
\item $\LeakCtxt(i)^\adv_\net \coloneqq m \leftarrow \In(i); \ k \leftarrow \SessionKey(i); \ p_A \leftarrow \PublicEphemeralKey(i); \ \ret{\big(p_A, m * k\big)}$ for $0 \leq i < q$
\item $\Out(i) \coloneqq \_ \leftarrow \OkCtxt(i)^\adv_\net; \ \read{\In(i)}$ for $0 \leq i < q$
\end{itemize}

\noindent By assumption, sampling a random secret key $k_R$ from $\unif_\key$ and returning $g^{k_R}$ is the same as sampling from $\unif_\msg$ directly, so we can simplify the channel $\SessionKey(i)$ as follows:

\begin{itemize}
\item $\SecretKey \coloneqq \samp{\unif_\key}$
\item $\PublicKey \coloneqq k_B \leftarrow \SecretKey; \ \ret{\gen^{k_B}}$
\item $\LeakPublicKey^\rec_\adv \coloneqq \read{\PublicKey}$
\item $\SecretEphemeralKey(i) \coloneqq \samp{\unif_\key}$ for $0 \leq i < q$
\item $\PublicEphemeralKey(i) \coloneqq k_A \leftarrow \SecretEphemeralKey(i); \ \ret{\gen^{k_A}}$ for $0 \leq i < q$
\item {\color{red} $\SessionKey(i) \coloneqq \samp{\unif_\msg}$ for $0 \leq i < q$}
\item $\LeakCtxt(i)^\adv_\net \coloneqq m \leftarrow \In(i); \ k \leftarrow \SessionKey(i); \ p_A \leftarrow \PublicEphemeralKey(i); \ \ret{\big(p_A, m * k\big)}$ for $0 \leq i < q$
\item $\Out(i) \coloneqq \_ \leftarrow \OkCtxt(i)^\adv_\net; \ \read{\In(i)}$ for $0 \leq i < q$
\end{itemize}

\noindent We can now extract the encoding of the input message into a new internal channel $\Enc(i)$:

\begin{itemize}
\item $\SecretKey \coloneqq \samp{\unif_\key}$
\item $\PublicKey \coloneqq k_B \leftarrow \SecretKey; \ \ret{\gen^{k_B}}$
\item $\LeakPublicKey^\rec_\adv \coloneqq \read{\PublicKey}$
\item $\SecretEphemeralKey(i) \coloneqq \samp{\unif_\key}$ for $0 \leq i < q$
\item $\PublicEphemeralKey(i) \coloneqq k_A \leftarrow \SecretEphemeralKey(i); \ \ret{\gen^{k_A}}$ for $0 \leq i < q$
\item $\SessionKey(i) \coloneqq \samp{\unif_\msg}$ for $0 \leq i < q$
\item {\color{red} $\Enc(i) \coloneqq m \leftarrow \In(i); \ k \leftarrow \SessionKey(i); \ \ret {m * k}$ for $0 \leq i < q$}
\item {\color{red} $\LeakCtxt(i)^\adv_\net \coloneqq c \leftarrow \Enc(i); \ p_A \leftarrow \PublicEphemeralKey(i); \ \ret{\big(p_A, c\big)}$ for $0 \leq i < q$}
\item $\Out(i) \coloneqq \_ \leftarrow \OkCtxt(i)^\adv_\net; \ \read{\In(i)}$ for $0 \leq i < q$
\end{itemize}

\noindent At this point, the channel $\SessionKey(i)$ is only used in $\Enc(i)$, so we can fold it in:

\begin{itemize}
\item $\SecretKey \coloneqq \samp{\unif_\key}$
\item $\PublicKey \coloneqq k_B \leftarrow \SecretKey; \ \ret{\gen^{k_B}}$
\item $\LeakPublicKey^\rec_\adv \coloneqq \read{\PublicKey}$
\item $\SecretEphemeralKey(i) \coloneqq \samp{\unif_\key}$ for $0 \leq i < q$
\item $\PublicEphemeralKey(i) \coloneqq k_A \leftarrow \SecretEphemeralKey(i); \ \ret{\gen^{k_A}}$ for $0 \leq i < q$
\item {\color{red} $\Enc(i) \coloneqq m \leftarrow \In(i); \ k \leftarrow \unif_\key; \ \ret {m * k}$ for $0 \leq i < q$}
\item $\LeakCtxt(i)^\adv_\net \coloneqq c \leftarrow \Enc(i); \ p_A \leftarrow \PublicEphemeralKey(i); \ \ret{\big(p_A, c\big)}$ for $0 \leq i < q$
\item $\Out(i) \coloneqq \_ \leftarrow \OkCtxt(i)^\adv_\net; \ \read{\In(i)}$ for $0 \leq i < q$
\end{itemize}

\noindent By assumption, the distribution $\unif_\msg$ on messages is invariant under the operation of multiplication with a fixed message, so we can simplify the channel $\Enc(i)$ as follows:

\begin{itemize}
\item $\SecretKey \coloneqq \samp{\unif_\key}$
\item $\PublicKey \coloneqq k_B \leftarrow \SecretKey; \ \ret{\gen^{k_B}}$
\item $\LeakPublicKey^\rec_\adv \coloneqq \read{\PublicKey}$
\item $\SecretEphemeralKey(i) \coloneqq \samp{\unif_\key}$ for $0 \leq i < q$
\item $\PublicEphemeralKey(i) \coloneqq k_A \leftarrow \SecretEphemeralKey(i); \ \ret{\gen^{k_A}}$ for $0 \leq i < q$
\item {\color{red} $\Enc(i) \coloneqq \_ \leftarrow \In(i); \ \samp{\unif_\key}$ for $0 \leq i < q$}
\item $\LeakCtxt(i)^\adv_\net \coloneqq c \leftarrow \Enc(i); \ p_A \leftarrow \PublicEphemeralKey(i); \ \ret{\big(p_A, c\big)}$ for $0 \leq i < q$
\item $\Out(i) \coloneqq \_ \leftarrow \OkCtxt(i)^\adv_\net; \ \read{\In(i)}$ for $0 \leq i < q$
\end{itemize}

\noindent The channel $\Enc(i)$ is now only used in $\LeakCtxt(i)^\adv_\net$, so we can fold it in:

\begin{itemize}
\item $\SecretKey \coloneqq \samp{\unif_\key}$
\item $\PublicKey \coloneqq k_B \leftarrow \SecretKey; \ \ret{\gen^{k_B}}$
\item $\LeakPublicKey^\rec_\adv \coloneqq \read{\PublicKey}$
\item $\SecretEphemeralKey(i) \coloneqq \samp{\unif_\key}$ for $0 \leq i < q$
\item $\PublicEphemeralKey(i) \coloneqq k_A \leftarrow \SecretEphemeralKey(i); \ \ret{\gen^{k_A}}$ for $0 \leq i < q$
\item {\color{red} $\LeakCtxt(i)^\adv_\net \coloneqq \_ \leftarrow \In(i); \ c \leftarrow \unif_\key; \ p_A \leftarrow \PublicEphemeralKey(i); \ \ret{\big(p_A, c\big)}$ for $0 \leq i < q$}
\item $\Out(i) \coloneqq \_ \leftarrow \OkCtxt(i)^\adv_\net; \ \read{\In(i)}$ for $0 \leq i < q$
\end{itemize}

\noindent But this is precisely the simplified composition of the ideal functionality and the simulator from the beginning of this section.