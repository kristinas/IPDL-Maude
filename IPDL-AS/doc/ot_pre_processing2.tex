\renewcommand{\msg}{\mathsf{msg}}
\newcommand{\flip}{\mathsf{flip}}
\renewcommand{\unif}{\mathsf{unif}}
\renewcommand{\id}{\mathsf{id}}
\renewcommand{\adv}{\mathsf{adv}}
\newcommand{\sen}{\mathsf{sen}}
\renewcommand{\rec}{\mathsf{rec}}
\newcommand{\ot}{\mathsf{ot}}
\renewcommand{\Msg}{\mathsf{Msg}}
\renewcommand{\Out}{\mathsf{Out}}
\newcommand{\Choice}{\mathsf{Choice}}
\newcommand{\OTMsg}{\mathsf{OTMsg}}
\newcommand{\OTOut}{\mathsf{OTOut}}
\newcommand{\OTChoice}{\mathsf{OTChoice}}
\newcommand{\Flip}{\mathsf{Flip}}
\renewcommand{\Key}{\mathsf{Key}}
\newcommand{\ChoiceEnc}{\mathsf{ChoiceEnc}}
\newcommand{\MsgEnc}{\mathsf{MsgEnc}}
\newcommand{\KeyPair}{\mathsf{KeyPair}}
\newcommand{\PrivateMsg}{\mathsf{PrivateMsg}}
\renewcommand{\LeakMsgRcvd}{\mathsf{MsgRcvd}}
\newcommand{\LeakOTMsgRcvd}{\mathsf{OTMsgRcvd}}
\newcommand{\LeakChoice}{\mathsf{Choice}}
\newcommand{\LeakOTChoice}{\mathsf{OTChoice}}
\newcommand{\LeakFlip}{\mathsf{Flip}}
\newcommand{\LeakOut}{\mathsf{Out}}
\newcommand{\LeakOTOut}{\mathsf{OTOut}}
\newcommand{\LeakChoiceEnc}{\mathsf{ChoiceEnc}}
\newcommand{\LeakMsgEnc}{\mathsf{MsgEnc}}

In this case study, Alice and Bob carry out a 1-Out-Of-2 Oblivious Transfer (OT) separated into an \emph{offline} phase, where Alice and Bob exchange a key using a single idealized 1-Out-Of-2 OT instance, and an \emph{online} phase that relies on the shared key and requires no cryptographic assumptions at all, thereby being very fast. We prove the protocol semi-honest secure in the case when the receiver is corrupt. Formally, we assume a type $\msg$ of messages; a coin-flip distribution $\flip : \one \twoheadrightarrow \Bool$; a uniform distribution $\unif_\msg : \one \twoheadrightarrow \msg$ on messages; and a bitwise xor function $\oplus : \msg \times \msg \rightarrow \msg$, where we write $x \oplus y$ in place of $\oplus \ (x,y)$.

\subsection{The Assumptions}
At the expression level, we assume that the operation of bitwise xor with a fixed message is self-inverse: \emph{i.e.}, we have the two axioms
\begin{itemize}
\item $x : \msg, y : \msg \vdash x \oplus (x \oplus y) = y : \msg$, and
\item $x : \msg, y : \msg \vdash (x \oplus y) \oplus y = x : \msg$.
\end{itemize}
At the distribution level, we assume that the coin flip is fair via the following axiom:
\begin{itemize}
\item $\cdot \vdash \big(f \leftarrow \flip; \ \ifte{f}{\ret{\false}}{\ret{\true}}\big) = \samp{\flip} : \emptyset \to \Bool$.
\end{itemize}
Finally, we assume that the distribution $\unif_\msg$ on messages is invariant under the operation of xor-ing with a fixed message (as is indeed the case when $\unif_\msg$ is uniform):
\begin{itemize}
\item $\cdot \ ; \ x : \msg \vdash (y \leftarrow \unif_\msg; \ \ret{x \oplus y}) = \samp{\unif_\msg} : \emptyset \to \msg$, and
\item $\cdot \ ; \ y : \msg \vdash (x \leftarrow \unif_\msg; \ \ret{x \oplus y}) = \samp{\unif_\msg} : \emptyset \to \msg$.
\end{itemize}

\subsection{The Ideal Functionality}
In its basic form, the ideal functionality reads two messages $m_0,m_1$ from the sender, and one Boolean $c$ from the receiver, and outputs the following message:
\[\begin{cases*}
m_0 & \text{if $c = \false$} \\
m_1 & \text{if $c = \true$}
\end{cases*}\]
In code:
\begin{itemize}
\item $\Out \coloneqq m_0 \leftarrow \Msg(0); \ m_1 \leftarrow \Msg(1); \ c \leftarrow \Choice; \ \ifte{c}{\ret{m_1}}{\ret{m_0}}$
\end{itemize}
Since the sender is honest, only the timing information for their messages is leaked:
\begin{itemize}
\item {\color{blue} $\LeakMsgRcvd(0)^\id_\adv \coloneqq m_0 \leftarrow \Msg(0); \ \ret{\checkmark}$}
\item {\color{blue} $\LeakMsgRcvd(1)^\id_\adv \coloneqq m_1 \leftarrow \Msg(1); \ \ret{\checkmark}$}
\end{itemize}
Since the receiver is semi-honest, the choice as well as selected message are leaked to the adversary:
\begin{itemize}
\item {\color{blue} $\LeakChoice^\id_\adv \coloneqq \read{\Choice}$}
\item {\color{blue} $\LeakOut^\id_\adv \coloneqq \read{\Out}$}
\end{itemize}

\subsection{The Real Protocol}
For the offline phase, we assume an ideal OT functionality. Alice randomly generates a new pair of messages, to be treated as keys, and sends them to the OT functionality:
\begin{itemize}
\item $\Key(0) \coloneqq \samp{\unif_\msg}$
\item $\Key(1) \coloneqq \samp{\unif_\msg}$
\item $\OTMsg(0) \coloneqq \read{\Key(0)}$
\item $\OTMsg(1) \coloneqq \read{\Key(1)}$
\end{itemize}
Bob flips a coin to decide which key he will ask for and informs the adversary:
\begin{itemize}
\item $\Flip \coloneqq \samp{\flip}$
\item {\color{blue} $\LeakFlip^\rec_\adv \coloneqq \read{\Flip}$}
\end{itemize}
He subsequently sends his choice to the OT functionality:
\begin{itemize}
\item $\OTChoice \coloneqq \read{\Flip}$
\end{itemize}
The OT functionality selects the corresponding key and sends it to Bob, accompanied by the requisite leakages:
\begin{itemize}
\item $\OTOut \coloneqq m_0 \leftarrow \OTMsg(0); \ m_1 \leftarrow \OTMsg(1); \ c \leftarrow \OTChoice; \ \ifte{c}{\ret{m_1}}{\ret{m_0}}$
\item {\color{blue} $\LeakOTMsgRcvd(0)^\ot_\adv \coloneqq m_0 \leftarrow \OTMsg(0); \ \ret{\checkmark}$}
\item {\color{blue} $\LeakOTMsgRcvd(1)^\ot_\adv \coloneqq m_1 \leftarrow \OTMsg(1); \ \ret{\checkmark}$}
\item {\color{blue} $\LeakOTChoice^\ot_\adv \coloneqq \read{\OTChoice}$}
\item {\color{blue} $\LeakOTOut^\ot_\adv \coloneqq \read{\OTOut}$}
\end{itemize}
Bob stores the result of the OT exchange as the key shared between him and Alice:
\begin{itemize}
\item $\SharedKey \coloneqq \read{\OTOut}$
\end{itemize}
This ends the offline phase. The online phase starts by Bob's informing the adversary about his choice of message:
\begin{itemize}
\item {\color{blue} $\LeakChoice^\rec_\adv \coloneqq \read{\Choice}$}
\end{itemize}
Bob subsequently encrypts this choice by xor-ing it with the shared key established in the pre-processing phase, and sends the encryption to Alice while leaking its value:
\begin{itemize}
\item $\ChoiceEnc \coloneqq f \leftarrow \Flip; \ c \leftarrow \Choice; \\ \ifte{c}{\big(\ifte{f}{\ret{\false}}{\ret{\true}}\big)}{\big(\ifte{f}{\ret{\true}}{\ret{\false}}\big)}$
\item {\color{blue} $\LeakChoiceEnc^\rec_\adv \coloneqq \read{\ChoiceEnc}$}
\end{itemize}
Upon receiving Bob's encrypted choice, Alice encrypts her messages by bitwise xor-ing them with the keys - either their own respective keys in case Bob's encrypted choice is $\false$, or the mutually-swapped keys if Bob's encrypted choice is $\true$:
\begin{itemize}
\item $\MsgEnc(0) \coloneqq m_0 \leftarrow \Msg(0); \ m_1 \leftarrow \Msg(1); \ k_0 \leftarrow \Key(0); \ k_1 \leftarrow \Key(1); \ e \leftarrow \ChoiceEnc; \\ \ifte{e}{\ret{m_0 \oplus k_1}}{\ret{m_0 \oplus k_0}}$
\item $\MsgEnc(1) \coloneqq m_0 \leftarrow \Msg(0); \ m_1 \leftarrow \Msg(1); \ k_0 \leftarrow \Key(0); \ k_1 \leftarrow \Key(1); \ e \leftarrow \ChoiceEnc; \\ \ifte{e}{\ret{m_1 \oplus k_0}}{\ret{m_1 \oplus k_1}}$
\end{itemize}
After receiving Alice's encrypted messages, Bob leaks them to the adversary:
\begin{itemize}
\item {\color{blue} $\LeakMsgEnc(0)^\rec_\adv \coloneqq \read{\MsgEnc(0)}$}
\item {\color{blue} $\LeakMsgEnc(1)^\rec_\adv \coloneqq \read{\MsgEnc(1)}$}
\end{itemize}
He then selects the encryption of the message he wants, decrypts it by xor-ing it with the shared key, and outputs the result while leaking its value:
\begin{itemize}
\item $\Out \coloneqq e_0 \leftarrow \MsgEnc(0); \ e_1 \leftarrow \MsgEnc(1); \ s \leftarrow \SharedKey; \ c \leftarrow \Choice; \ \ifte{c}{\ret{e_1 \oplus s}}{\ret{e_0 \oplus s}}$
\item {\color{blue} $\LeakOut^\rec_\adv \coloneqq \read{\Out}$}
\end{itemize}
Thus, we have the following code for Alice:
\begin{itemize}
\item $\Key(0) \coloneqq \samp{\unif_\msg}$
\item $\Key(1) \coloneqq \samp{\unif_\msg}$
\item $\OTMsg(0) \coloneqq \read{\Key(0)}$
\item $\OTMsg(1) \coloneqq \read{\Key(1)}$
\item $\MsgEnc(0) \coloneqq m_0 \leftarrow \Msg(0); \ m_1 \leftarrow \Msg(1); \ k_0 \leftarrow \Key(0); \ k_1 \leftarrow \Key(1); \ e \leftarrow \ChoiceEnc; \\ \ifte{e}{\ret{m_0 \oplus k_1}}{\ret{m_0 \oplus k_0}}$
\item $\MsgEnc(1) \coloneqq m_0 \leftarrow \Msg(0); \ m_1 \leftarrow \Msg(1); \ k_0 \leftarrow \Key(0); \ k_1 \leftarrow \Key(1); \ e \leftarrow \ChoiceEnc; \\ \ifte{e}{\ret{m_1 \oplus k_0}}{\ret{m_1 \oplus k_1}}$
\end{itemize}
The code for Bob has the following form:
\begin{itemize}
\item $\Flip \coloneqq \samp{\flip}$
\item {\color{blue} $\LeakFlip^\rec_\adv \coloneqq \read{\Flip}$}
\item $\OTChoice \coloneqq \read{\Flip}$
\item $\SharedKey \coloneqq \read{\OTOut}$
\item {\color{blue} $\LeakChoice^\rec_\adv \coloneqq \read{\Choice}$}
\item $\ChoiceEnc \coloneqq f \leftarrow \Flip; \ c \leftarrow \Choice; \\ \ifte{c}{\big(\ifte{f}{\ret{\false}}{\ret{\true}}\big)}{\big(\ifte{f}{\ret{\true}}{\ret{\false}}\big)}$
\item {\color{blue} $\LeakChoiceEnc^\rec_\adv \coloneqq \read{\ChoiceEnc}$}
\item {\color{blue} $\LeakMsgEnc(0)^\rec_\adv \coloneqq \read{\MsgEnc(0)}$}
\item {\color{blue} $\LeakMsgEnc(1)^\rec_\adv \coloneqq \read{\MsgEnc(1)}$}
\item $\Out \coloneqq e_0 \leftarrow \MsgEnc(0); \ e_1 \leftarrow \MsgEnc(1); \ s \leftarrow \SharedKey; \ c \leftarrow \Choice; \ \ifte{c}{\ret{e_1 \oplus s}}{\ret{e_0 \oplus s}}$
\item {\color{blue} $\LeakOut^\rec_\adv \coloneqq \read{\Out}$}
\end{itemize}
Finally, we recall the code for the OT functionality:
\begin{itemize}
\item $\OTOut \coloneqq m_0 \leftarrow \OTMsg(0); \ m_1 \leftarrow \OTMsg(1); \ c \leftarrow \OTChoice; \ \ifte{c}{\ret{m_1}}{\ret{m_0}}$
\item {\color{blue} $\LeakOTMsgRcvd(0)^\ot_\adv \coloneqq m_0 \leftarrow \OTMsg(0); \ \ret{\checkmark}$}
\item {\color{blue} $\LeakOTMsgRcvd(1)^\ot_\adv \coloneqq m_1 \leftarrow \OTMsg(1); \ \ret{\checkmark}$}
\item {\color{blue} $\LeakOTChoice^\ot_\adv \coloneqq \read{\OTChoice}$}
\item {\color{blue} $\LeakOTOut^\ot_\adv \coloneqq \read{\OTOut}$}
\end{itemize}
Composing all of this together and hiding the internal communication yields the real-world protocol.

\subsection{Real = Ideal + Simulator}
Our goal is to simplify the real protocol until it becomes clear how to separate it out into the ideal functionality part and the simulator part. We recall the code:

\begin{itemize}
\item $\Key(0) \coloneqq \samp{\unif_\msg}$
\item $\Key(1) \coloneqq \samp{\unif_\msg}$
\item $\Flip \coloneqq \samp{\flip}$
\item {\color{blue} $\LeakFlip^\rec_\adv \coloneqq \read{\Flip}$}
\item $\OTMsg(0) \coloneqq \read{\Key(0)}$
\item $\OTMsg(1) \coloneqq \read{\Key(1)}$
\item $\OTChoice \coloneqq \read{\Flip}$
\item $\OTOut \coloneqq m_0 \leftarrow \OTMsg(0); \ m_1 \leftarrow \OTMsg(1); \ c \leftarrow \OTChoice; \ \ifte{c}{\ret{m_1}}{\ret{m_0}}$
\item {\color{blue} $\LeakOTMsgRcvd(0)^\ot_\adv \coloneqq m_0 \leftarrow \OTMsg(0); \ \ret{\checkmark}$}
\item {\color{blue} $\LeakOTMsgRcvd(1)^\ot_\adv \coloneqq m_1 \leftarrow \OTMsg(1); \ \ret{\checkmark}$}
\item {\color{blue} $\LeakOTChoice^\ot_\adv \coloneqq \read{\OTChoice}$}
\item {\color{blue} $\LeakOTOut^\ot_\adv \coloneqq \read{\OTOut}$}
\item $\SharedKey \coloneqq \read{\OTOut}$
\item {\color{blue} $\LeakChoice^\rec_\adv \coloneqq \read{\Choice}$}
\item $\ChoiceEnc \coloneqq f \leftarrow \Flip; \ c \leftarrow \Choice; \\ \ifte{c}{\big(\ifte{f}{\ret{\false}}{\ret{\true}}\big)}{\big(\ifte{f}{\ret{\true}}{\ret{\false}}\big)}$
\item {\color{blue} $\LeakChoiceEnc^\rec_\adv \coloneqq \read{\ChoiceEnc}$}
\item $\MsgEnc(0) \coloneqq m_0 \leftarrow \Msg(0); \ m_1 \leftarrow \Msg(1); \ k_0 \leftarrow \Key(0); \ k_1 \leftarrow \Key(1); \ e \leftarrow \ChoiceEnc; \\ \ifte{e}{\ret{m_0 \oplus k_1}}{\ret{m_0 \oplus k_0}}$
\item $\MsgEnc(1) \coloneqq m_0 \leftarrow \Msg(0); \ m_1 \leftarrow \Msg(1); \ k_0 \leftarrow \Key(0); \ k_1 \leftarrow \Key(1); \ e \leftarrow \ChoiceEnc; \\ \ifte{e}{\ret{m_1 \oplus k_0}}{\ret{m_1 \oplus k_1}}$
\item {\color{blue} $\LeakMsgEnc(0)^\rec_\adv \coloneqq \read{\MsgEnc(0)}$}
\item {\color{blue} $\LeakMsgEnc(1)^\rec_\adv \coloneqq \read{\MsgEnc(1)}$}
\item $\Out \coloneqq e_0 \leftarrow \MsgEnc(0); \ e_1 \leftarrow \MsgEnc(1); \ s \leftarrow \SharedKey; \ c \leftarrow \Choice; \  \ifte{c}{\ret{e_1 \oplus s}}{\ret{e_0 \oplus s}}$
\item {\color{blue} $\LeakOut^\rec_\adv \coloneqq \read{\Out}$}
\end{itemize}

\noindent We start by eliminating all of the internal OT channels:

\begin{itemize}
\item $\Key(0) \coloneqq \samp{\unif_\msg}$
\item $\Key(1) \coloneqq \samp{\unif_\msg}$
\item $\Flip \coloneqq \samp{\flip}$
\item {\color{blue} $\LeakFlip^\rec_\adv \coloneqq \read{\Flip}$}
\item {\color{red} $\LeakOTMsgRcvd(0)^\ot_\adv \coloneqq k_0 \leftarrow \Key(0); \ \ret{\checkmark}$}
\item {\color{red} $\LeakOTMsgRcvd(1)^\ot_\adv \coloneqq k_1 \leftarrow \Key(1); \ \ret{\checkmark}$}
\item {\color{red} $\LeakOTChoice^\ot_\adv \coloneqq \read{\Flip}$}
\item {\color{red} $\LeakOTOut^\ot_\adv \coloneqq \read{\SharedKey}$}
\item {\color{red} $\SharedKey \coloneqq k_0 \leftarrow \Key(0); \ k_1 \leftarrow \Key(1); \ f \leftarrow \Flip; \ \ifte{f}{\ret{k_1}}{\ret{k_0}}$}
\item {\color{blue} $\LeakChoice^\rec_\adv \coloneqq \read{\Choice}$}
\item $\ChoiceEnc \coloneqq f \leftarrow \Flip; \ c \leftarrow \Choice; \\ \ifte{c}{\big(\ifte{f}{\ret{\false}}{\ret{\true}}\big)}{\big(\ifte{f}{\ret{\true}}{\ret{\false}}\big)}$
\item {\color{blue} $\LeakChoiceEnc^\rec_\adv \coloneqq \read{\ChoiceEnc}$}
\item $\MsgEnc(0) \coloneqq m_0 \leftarrow \Msg(0); \ m_1 \leftarrow \Msg(1); \ k_0 \leftarrow \Key(0); \ k_1 \leftarrow \Key(1); \ e \leftarrow \ChoiceEnc; \\ \ifte{e}{\ret{m_0 \oplus k_1}}{\ret{m_0 \oplus k_0}}$
\item $\MsgEnc(1) \coloneqq m_0 \leftarrow \Msg(0); \ m_1 \leftarrow \Msg(1); \ k_0 \leftarrow \Key(0); \ k_1 \leftarrow \Key(1); \ e \leftarrow \ChoiceEnc; \\ \ifte{e}{\ret{m_1 \oplus k_0}}{\ret{m_1 \oplus k_1}}$
\item {\color{blue} $\LeakMsgEnc(0)^\rec_\adv \coloneqq \read{\MsgEnc(0)}$}
\item {\color{blue} $\LeakMsgEnc(1)^\rec_\adv \coloneqq \read{\MsgEnc(1)}$}
\item $\Out \coloneqq e_0 \leftarrow \MsgEnc(0); \ e_1 \leftarrow \MsgEnc(1); \ s \leftarrow \SharedKey; \ c \leftarrow \Choice; \  \ifte{c}{\ret{e_1 \oplus s}}{\ret{e_0 \oplus s}}$
\item {\color{blue} $\LeakOut^\rec_\adv \coloneqq \read{\Out}$}
\end{itemize}

\noindent Substituting the channel $\ChoiceEnc$ into $\MsgEnc(0)$ and $\MsgEnc(1)$ yields
\begin{itemize}
\item $\MsgEnc(0) \coloneqq m_0 \leftarrow \Msg(0); \ m_1 \leftarrow \Msg(1); \ k_0 \leftarrow \Key(0); \ k_1 \leftarrow \Key(1); \ {\color{red} f \leftarrow \Flip; \ c \leftarrow \Choice;}$ \\ ${\color{red} \ifte{c}{\big(\ifte{f}{\ret{m_0 \oplus k_0}}{\ret{m_0 \oplus k_1}}\big)}{\big(\ifte{f}{\ret{m_0 \oplus k_1}}{\ret{m_0 \oplus k_0}}\big)}}$
\item $\MsgEnc(1) \coloneqq m_0 \leftarrow \Msg(0); \ m_1 \leftarrow \Msg(1); \ k_0 \leftarrow \Key(0); \ k_1 \leftarrow \Key(1); \ {\color{red} f \leftarrow \Flip; \ c \leftarrow \Choice;}$ \\ ${\color{red} \ifte{c}{\big(\ifte{f}{\ret{m_1 \oplus k_1}}{\ret{m_1 \oplus k_0}}\big)}{\big(\ifte{f}{\ret{m_1 \oplus k_0}}{\ret{m_1 \oplus k_1}}\big)}}$
\end{itemize}
Substituting the channel $\SharedKey$ into $\Out$ yields
\begin{itemize}
\item $\Out \coloneqq e_0 \leftarrow \MsgEnc(0); \ e_1 \leftarrow \MsgEnc(1); \ {\color{red} k_0 \leftarrow \Key(0); \ k_1 \leftarrow \Key(1); \ f \leftarrow \Flip; \ } c \leftarrow \Choice; $ \\ ${\color{red} \ifte{c}{\big(\ifte{f}{\ret{e_1 \oplus k_1}}{\ret{e_1 \oplus k_0}}\big)}{\big(\ifte{f}{\ret{e_0 \oplus k_1}}{\ret{e_0 \oplus k_0}}\big)}}$
\end{itemize}
Further substituting the channels $\MsgEnc(0)$ and $\MsgEnc(1)$ into $\Out$ yields
\begin{itemize}
\item $\Out \coloneqq {\color{red} m_0 \leftarrow \Msg(0); \ m_1 \leftarrow \Msg(1); \ } k_0 \leftarrow \Key(0); \ k_1 \leftarrow \Key(1); \ f \leftarrow \Flip; \ c \leftarrow \Choice;$ \\ $\mathsf{if} \ c \ \mathsf{then} \ \ifte{f}{\ret{{\color{red} (m_1 \oplus k_1)} \oplus k_1}}{\ret{{\color{red} (m_1 \oplus k_0)} \oplus k_0}}$ \\ ${\color{white} \mathsf{if} \ c \ } \mathsf{else} \ \ifte{f}{\ret{{\color{red} (m_0 \oplus k_1)} \oplus k_1}}{\ret{{\color{red} (m_0 \oplus k_0)} \oplus k_0}}$
\end{itemize}
We can now cancel out the two applications of $\oplus$ since they are mutually inverse by assumption:
\begin{itemize}
\item $\Out \coloneqq m_0 \leftarrow \Msg(0); \ m_1 \leftarrow \Msg(1); \ k_0 \leftarrow \Key(0); \ k_1 \leftarrow \Key(1); \ f \leftarrow \Flip; \ c \leftarrow \Choice; \\ \ifte{c}{\big(\ifte{f}{\ret{{\color{red} m_1}}}{\ret{{\color{red} m_1}}}\big)}{\big(\ifte{f}{\ret{{\color{red} m_0}}}{\ret{{\color{red} m_0}}}\big)}$
\end{itemize}
After simplifying we get
\begin{itemize}
\item $\Out \coloneqq m_0 \leftarrow \Msg(0); \ m_1 \leftarrow \Msg(1); \ c \leftarrow \Choice; \ {\color{red} \ifte{c}{\ret{m_1}}{\ret{m_0}}}$
\end{itemize}
Summarizing, the cleaned-up version of the real protocol is below:

\begin{itemize}
\item $\Key(0) \coloneqq \samp{\unif_\msg}$
\item $\Key(1) \coloneqq \samp{\unif_\msg}$
\item $\Flip \coloneqq \samp{\flip}$
\item {\color{blue} $\LeakFlip^\rec_\adv \coloneqq \read{\Flip}$}
\item {\color{blue} $\LeakOTMsgRcvd(0)^\ot_\adv \coloneqq k_0 \leftarrow \Key(0); \ \ret{\checkmark}$}
\item {\color{blue} $\LeakOTMsgRcvd(1)^\ot_\adv \coloneqq k_1 \leftarrow \Key(1); \ \ret{\checkmark}$}
\item {\color{blue} $\LeakOTChoice^\ot_\adv \coloneqq \read{\Flip}$}
\item {\color{blue} $\LeakOTOut^\ot_\adv \coloneqq \read{\SharedKey}$}
\item $\SharedKey \coloneqq k_0 \leftarrow \Key(0); \ k_1 \leftarrow \Key(1); \ f \leftarrow \Flip; \ \ifte{f}{\ret{k_1}}{\ret{k_0}}$
\item {\color{blue} $\LeakChoice^\rec_\adv \coloneqq \read{\Choice}$}
\item $\ChoiceEnc \coloneqq f \leftarrow \Flip; \ c \leftarrow \Choice; \\ \ifte{c}{\big(\ifte{f}{\ret{\false}}{\ret{\true}}\big)}{\big(\ifte{f}{\ret{\true}}{\ret{\false}}\big)}$
\item {\color{blue} $\LeakChoiceEnc^\rec_\adv \coloneqq \read{\ChoiceEnc}$}
\item $\MsgEnc(0) \coloneqq m_0 \leftarrow \Msg(0); \ m_1 \leftarrow \Msg(1); \ k_0 \leftarrow \Key(0); \ k_1 \leftarrow \Key(1); \ f \leftarrow \Flip; \ c \leftarrow \Choice;$ \\ $\ifte{c}{\big(\ifte{f}{\ret{m_0 \oplus k_0}}{\ret{m_0 \oplus k_1}}\big)}{\big(\ifte{f}{\ret{m_0 \oplus k_1}}{\ret{m_0 \oplus k_0}}\big)}$
\item $\MsgEnc(1) \coloneqq m_0 \leftarrow \Msg(0); \ m_1 \leftarrow \Msg(1); \ k_0 \leftarrow \Key(0); \ k_1 \leftarrow \Key(1); \ f \leftarrow \Flip; \ c \leftarrow \Choice;$ \\ $\ifte{c}{\big(\ifte{f}{\ret{m_1 \oplus k_1}}{\ret{m_1 \oplus k_0}}\big)}{\big(\ifte{f}{\ret{m_1 \oplus k_0}}{\ret{m_1 \oplus k_1}}\big)}$
\item {\color{blue} $\LeakMsgEnc(0)^\rec_\adv \coloneqq \read{\MsgEnc(0)}$}
\item {\color{blue} $\LeakMsgEnc(1)^\rec_\adv \coloneqq \read{\MsgEnc(1)}$}
\item $\Out \coloneqq m_0 \leftarrow \Msg(0); \ m_1 \leftarrow \Msg(1); \ c \leftarrow \Choice; \ \ifte{c}{\ret{m_1}}{\ret{m_0}}$
\item {\color{blue} $\LeakOut^\rec_\adv \coloneqq \read{\Out}$}
\end{itemize}

\noindent Since both keys are generated from the same distribution, the coin flip that distinguishes between them can be eliminated (\emph{``decoupling''}). To show this, we introduce an internal channel $\KeyPair$ that constructs the pair of two keys, where the first one is shared and the second one is private:

\begin{itemize}
\item $\Key(0) \coloneqq \samp{\unif_\msg}$
\item $\Key(1) \coloneqq \samp{\unif_\msg}$
\item $\Flip \coloneqq \samp{\flip}$
\item {\color{blue} $\LeakFlip^\rec_\adv \coloneqq \read{\Flip}$}
\item {\color{blue} $\LeakOTMsgRcvd(0)^\ot_\adv \coloneqq k_0 \leftarrow \Key(0); \ \ret{\checkmark}$}
\item {\color{blue} $\LeakOTMsgRcvd(1)^\ot_\adv \coloneqq k_1 \leftarrow \Key(1); \ \ret{\checkmark}$}
\item {\color{blue} $\LeakOTChoice^\ot_\adv \coloneqq \read{\Flip}$}
\item {\color{blue} $\LeakOTOut^\ot_\adv \coloneqq \read{\SharedKey}$}
\item {\color{red} $\KeyPair \coloneqq k_0 \leftarrow \Key(0); \ k_1 \leftarrow \Key(1); \ f \leftarrow \Flip; \ \ifte{f}{\ret{(k_1,k_0)}}{\ret{(k_0,k_1)}}$}
\item {\color{red} $\SharedKey \coloneqq k \leftarrow \KeyPair; \ \ret{(\fst \ k)}$}
\item {\color{blue} $\LeakChoice^\rec_\adv \coloneqq \read{\Choice}$}
\item $\ChoiceEnc \coloneqq f \leftarrow \Flip; \ c \leftarrow \Choice; \\ \ifte{c}{\big(\ifte{f}{\ret{\false}}{\ret{\true}}\big)}{\big(\ifte{f}{\ret{\true}}{\ret{\false}}\big)}$
\item {\color{blue} $\LeakChoiceEnc^\rec_\adv \coloneqq \read{\ChoiceEnc}$}
\item $\MsgEnc(0) \coloneqq m_0 \leftarrow \Msg(0); \ m_1 \leftarrow \Msg(1); \ {\color{red} k \leftarrow \KeyPair; \ } c \leftarrow \Choice;$ \\ {\color{red} $\ifte{c}{\ret{m_0 \oplus (\snd \ k)}}{\ret{m_0 \oplus (\fst \ k)}}$}
\item $\MsgEnc(1) \coloneqq m_0 \leftarrow \Msg(0); \ m_1 \leftarrow \Msg(1); \ {\color{red} k \leftarrow \KeyPair; \ } c \leftarrow \Choice;$ \\ {\color{red} $\ifte{c}{\ret{m_1 \oplus (\fst \ k)}}{\ret{m_1 \oplus (\snd \ k)}}$}
\item {\color{blue} $\LeakMsgEnc(0)^\rec_\adv \coloneqq \read{\MsgEnc(0)}$}
\item {\color{blue} $\LeakMsgEnc(1)^\rec_\adv \coloneqq \read{\MsgEnc(1)}$}
\item $\Out \coloneqq m_0 \leftarrow \Msg(0); \ m_1 \leftarrow \Msg(1); \ c \leftarrow \Choice; \ \ifte{c}{\ret{m_1}}{\ret{m_0}}$
\item {\color{blue} $\LeakOut^\rec_\adv \coloneqq \read{\Out}$}
\end{itemize}

\noindent The internal channels $\Key(0)$ and $\Key(1)$ are now only used in the single channel $\KeyPair$. We can therefore fold the two key samplings into the channel $\KeyPair$:

\begin{itemize}
\item $\Flip \coloneqq \samp{\flip}$
\item {\color{blue} $\LeakFlip^\rec_\adv \coloneqq \read{\Flip}$}
\item {\color{blue} $\LeakOTMsgRcvd(0)^\ot_\adv \coloneqq k_0 \leftarrow \Key(0); \ \ret{\checkmark}$}
\item {\color{blue} $\LeakOTMsgRcvd(1)^\ot_\adv \coloneqq k_1 \leftarrow \Key(1); \ \ret{\checkmark}$}
\item {\color{blue} $\LeakOTChoice^\ot_\adv \coloneqq \read{\Flip}$}
\item {\color{blue} $\LeakOTOut^\ot_\adv \coloneqq \read{\SharedKey}$}
\item $\KeyPair \coloneqq {\color{red} k_0 \leftarrow \unif_\msg; \ k_1 \leftarrow \unif_\msg; \ } f \leftarrow \Flip; \ \ifte{f}{\ret{(k_1,k_0)}}{\ret{(k_0,k_1)}}$
\item $\SharedKey \coloneqq k \leftarrow \KeyPair; \ \ret{(\fst \ k)}$
\item {\color{blue} $\LeakChoice^\rec_\adv \coloneqq \read{\Choice}$}
\item $\ChoiceEnc \coloneqq f \leftarrow \Flip; \ c \leftarrow \Choice; \\ \ifte{c}{\big(\ifte{f}{\ret{\false}}{\ret{\true}}\big)}{\big(\ifte{f}{\ret{\true}}{\ret{\false}}\big)}$
\item {\color{blue} $\LeakChoiceEnc^\rec_\adv \coloneqq \read{\ChoiceEnc}$}
\item $\MsgEnc(0) \coloneqq m_0 \leftarrow \Msg(0); \ m_1 \leftarrow \Msg(1); \ k \leftarrow \KeyPair; \ c \leftarrow \Choice;$ \\ $\ifte{c}{\ret{m_0 \oplus (\snd \ k)}}{\ret{m_0 \oplus (\fst \ k)}}$
\item $\MsgEnc(1) \coloneqq m_0 \leftarrow \Msg(0); \ m_1 \leftarrow \Msg(1); \ k \leftarrow \KeyPair; \ c \leftarrow \Choice;$ \\ $\ifte{c}{\ret{m_1 \oplus (\fst \ k)}}{\ret{m_1 \oplus (\snd \ k)}}$
\item {\color{blue} $\LeakMsgEnc(0)^\rec_\adv \coloneqq \read{\MsgEnc(0)}$}
\item {\color{blue} $\LeakMsgEnc(1)^\rec_\adv \coloneqq \read{\MsgEnc(1)}$}
\item $\Out \coloneqq m_0 \leftarrow \Msg(0); \ m_1 \leftarrow \Msg(1); \ c \leftarrow \Choice; \ \ifte{c}{\ret{m_1}}{\ret{m_0}}$
\item {\color{blue} $\LeakOut^\rec_\adv \coloneqq \read{\Out}$}
\end{itemize}

\noindent Rearranging the order of bindings inside $\KeyPair$ yields
\begin{itemize}
\item $\KeyPair \coloneqq f \leftarrow \Flip; \ k_0 \leftarrow \unif_\msg; \ k_1 \leftarrow \unif_\msg; \ \ifte{f}{\ret{(k_1,k_0)}}{\ret{(k_0,k_1)}}$
\end{itemize}
Since sampling and branching are interchangeable, the three reaction snippets
\begin{align*}
& \bullet \; k_0 \leftarrow \unif_\msg; \ k_1 \leftarrow \unif_\msg; \ \ifte{f}{\ret{(k_1,k_0)}}{\ret{(k_0,k_1)}} \\
& \bullet \; \ifte{f}{\big({\color{red} k_0 \leftarrow \unif_\msg; \ k_1 \leftarrow \unif_\msg; \ } \ret{(k_1,k_0)}\big)}{\big(k_0 \leftarrow \unif_\msg; \ k_1 \leftarrow \unif_\msg; \ \ret{(k_0,k_1)}\big)} \\
& \bullet \; \ifte{f}{\big({\color{red} k_1 \leftarrow \unif_\msg; \ k_0 \leftarrow \unif_\msg; \ } \ret{(k_1,k_0)}\big)}{\big(k_0 \leftarrow \unif_\msg; \ k_1 \leftarrow \unif_\msg; \ \ret{(k_0,k_1)}\big)}
\end{align*}
are equivalent. But the last snippet amounts to doing the same thing either way, so we might just as well not flip:
\begin{itemize}
\item $\KeyPair \coloneqq k_0 \leftarrow \unif_\msg; \ k_1 \leftarrow \unif_\msg; \ \ret{(k_0,k_1)}$
\end{itemize}
Unfolding the samplings back thus gives us:

\begin{itemize}
\item {\color{red} $\Key(0) \coloneqq \samp{\unif_\msg}$}
\item {\color{red} $\Key(1) \coloneqq \samp{\unif_\msg}$}
\item $\Flip \coloneqq \samp{\flip}$
\item {\color{blue} $\LeakFlip^\rec_\adv \coloneqq \read{\Flip}$}
\item {\color{blue} $\LeakOTMsgRcvd(0)^\ot_\adv \coloneqq k_0 \leftarrow \Key(0); \ \ret{\checkmark}$}
\item {\color{blue} $\LeakOTMsgRcvd(1)^\ot_\adv \coloneqq k_1 \leftarrow \Key(1); \ \ret{\checkmark}$}
\item {\color{blue} $\LeakOTChoice^\ot_\adv \coloneqq \read{\Flip}$}
\item {\color{blue} $\LeakOTOut^\ot_\adv \coloneqq \read{\SharedKey}$}
\item $\KeyPair \coloneqq {\color{red} k_0 \leftarrow \Key(0); \ k_1 \leftarrow \Key(1); \ } \ret{(k_0,k_1)}$
\item $\SharedKey \coloneqq k \leftarrow \KeyPair; \ \ret{(\fst \ k)}$
\item {\color{blue} $\LeakChoice^\rec_\adv \coloneqq \read{\Choice}$}
\item $\ChoiceEnc \coloneqq f \leftarrow \Flip; \ c \leftarrow \Choice; \\ \ifte{c}{\big(\ifte{f}{\ret{\false}}{\ret{\true}}\big)}{\big(\ifte{f}{\ret{\true}}{\ret{\false}}\big)}$
\item {\color{blue} $\LeakChoiceEnc^\rec_\adv \coloneqq \read{\ChoiceEnc}$}
\item $\MsgEnc(0) \coloneqq m_0 \leftarrow \Msg(0); \ m_1 \leftarrow \Msg(1); \ k \leftarrow \KeyPair; \ c \leftarrow \Choice;$ \\ $\ifte{c}{\ret{m_0 \oplus (\snd \ k)}}{\ret{m_0 \oplus (\fst \ k)}}$
\item $\MsgEnc(1) \coloneqq m_0 \leftarrow \Msg(0); \ m_1 \leftarrow \Msg(1); \ k \leftarrow \KeyPair; \ c \leftarrow \Choice;$ \\ $\ifte{c}{\ret{m_1 \oplus (\fst \ k)}}{\ret{m_1 \oplus (\snd \ k)}}$
\item {\color{blue} $\LeakMsgEnc(0)^\rec_\adv \coloneqq \read{\MsgEnc(0)}$}
\item {\color{blue} $\LeakMsgEnc(1)^\rec_\adv \coloneqq \read{\MsgEnc(1)}$}
\item $\Out \coloneqq m_0 \leftarrow \Msg(0); \ m_1 \leftarrow \Msg(1); \ c \leftarrow \Choice; \ \ifte{c}{\ret{m_1}}{\ret{m_0}}$
\item {\color{blue} $\LeakOut^\rec_\adv \coloneqq \read{\Out}$}
\end{itemize}

\noindent The internal channel $\KeyPair$ can now be substituted away:

\begin{itemize}
\item $\Key(0) \coloneqq \samp{\unif_\msg}$
\item $\Key(1) \coloneqq \samp{\unif_\msg}$
\item $\Flip \coloneqq \samp{\flip}$
\item {\color{blue} $\LeakFlip^\rec_\adv \coloneqq \read{\Flip}$}
\item {\color{blue} $\LeakOTMsgRcvd(0)^\ot_\adv \coloneqq k_0 \leftarrow \Key(0); \ \ret{\checkmark}$}
\item {\color{blue} $\LeakOTMsgRcvd(1)^\ot_\adv \coloneqq k_1 \leftarrow \Key(1); \ \ret{\checkmark}$}
\item {\color{blue} $\LeakOTChoice^\ot_\adv \coloneqq \read{\Flip}$}
\item {\color{blue} $\LeakOTOut^\ot_\adv \coloneqq \read{\SharedKey}$}
\item {\color{red} $\SharedKey \coloneqq \read{\Key(0)}$}
\item {\color{blue} $\LeakChoice^\rec_\adv \coloneqq \read{\Choice}$}
\item $\ChoiceEnc \coloneqq f \leftarrow \Flip; \ c \leftarrow \Choice; \\ \ifte{c}{\big(\ifte{f}{\ret{\false}}{\ret{\true}}\big)}{\big(\ifte{f}{\ret{\true}}{\ret{\false}}\big)}$
\item {\color{blue} $\LeakChoiceEnc^\rec_\adv \coloneqq \read{\ChoiceEnc}$}
\item $\MsgEnc(0) \coloneqq m_0 \leftarrow \Msg(0); \ m_1 \leftarrow \Msg(1); \ c \leftarrow \Choice; \ {\color{red} k_0 \leftarrow \Key(0); \ k_1 \leftarrow \Key(1);}$ \\ $\ifte{c}{{\color{red} \ret{m_0 \oplus k_1}}}{{\color{red} \ret{m_0 \oplus k_0}}}$
\item $\MsgEnc(1) \coloneqq m_0 \leftarrow \Msg(0); \ m_1 \leftarrow \Msg(1); \ c \leftarrow \Choice; \ {\color{red} k_0 \leftarrow \Key(0); \ k_1 \leftarrow \Key(1);}$ \\ $\ifte{c}{{\color{red} \ret{m_1 \oplus k_0}}}{{\color{red} \ret{m_1 \oplus k_1}}}$
\item {\color{blue} $\LeakMsgEnc(0)^\rec_\adv \coloneqq \read{\MsgEnc(0)}$}
\item {\color{blue} $\LeakMsgEnc(1)^\rec_\adv \coloneqq \read{\MsgEnc(1)}$}
\item $\Out \coloneqq m_0 \leftarrow \Msg(0); \ m_1 \leftarrow \Msg(1); \ c \leftarrow \Choice; \ \ifte{c}{\ret{m_1}}{\ret{m_0}}$
\item {\color{blue} $\LeakOut^\rec_\adv \coloneqq \read{\Out}$}
\end{itemize}

\noindent The second key is now only referenced in the channels $\MsgEnc(0)$ and $\MsgEnc(1)$, where we use it to encrypt either the first or the second message, respectively. This encryption process can be extracted out into a new internal channel $\PrivateMsg$:

\begin{itemize}
\item $\Key(0) \coloneqq \samp{\unif_\msg}$
\item $\Key(1) \coloneqq \samp{\unif_\msg}$
\item $\Flip \coloneqq \samp{\flip}$
\item {\color{blue} $\LeakFlip^\rec_\adv \coloneqq \read{\Flip}$}
\item {\color{blue} $\LeakOTMsgRcvd(0)^\ot_\adv \coloneqq k_0 \leftarrow \Key(0); \ \ret{\checkmark}$}
\item {\color{blue} $\LeakOTMsgRcvd(1)^\ot_\adv \coloneqq k_1 \leftarrow \Key(1); \ \ret{\checkmark}$}
\item {\color{blue} $\LeakOTChoice^\ot_\adv \coloneqq \read{\Flip}$}
\item {\color{blue} $\LeakOTOut^\ot_\adv \coloneqq \read{\SharedKey}$}
\item $\SharedKey \coloneqq \read{\Key(0)}$
\item {\color{blue} $\LeakChoice^\rec_\adv \coloneqq \read{\Choice}$}
\item $\ChoiceEnc \coloneqq f \leftarrow \Flip; \ c \leftarrow \Choice; \\ \ifte{c}{\big(\ifte{f}{\ret{\false}}{\ret{\true}}\big)}{\big(\ifte{f}{\ret{\true}}{\ret{\false}}\big)}$
\item {\color{blue} $\LeakChoiceEnc^\rec_\adv \coloneqq \read{\ChoiceEnc}$}
\item {\color{red} $\PrivateMsg \coloneqq m_0 \leftarrow \Msg(0); \ m_1 \leftarrow \Msg(1); \ k_1 \leftarrow \Key(1); \ c \leftarrow \Choice;$} \\ {\color{red} $\ifte{c}{\ret{m_0 \oplus k_1}}{\ret{m_1 \oplus k_1}}$}
\item $\MsgEnc(0) \coloneqq m_0 \leftarrow \Msg(0); \ m_1 \leftarrow \Msg(1); \ c \leftarrow \Choice; \ k_0 \leftarrow \Key(0); \ {\color{red} p \leftarrow \PrivateMsg;}$ \\ $\ifte{c}{{\color{red} \ret{p}}}{\ret{m_0 \oplus k_0}}$
\item $\MsgEnc(1) \coloneqq m_0 \leftarrow \Msg(0); \ m_1 \leftarrow \Msg(1); \ c \leftarrow \Choice; \ k_0 \leftarrow \Key(0); \ {\color{red} p \leftarrow \PrivateMsg;}$ \\ $\ifte{c}{\ret{m_1 \oplus k_0}}{{\color{red} \ret{p}}}$
\item {\color{blue} $\LeakMsgEnc(0)^\rec_\adv \coloneqq \read{\MsgEnc(0)}$}
\item {\color{blue} $\LeakMsgEnc(1)^\rec_\adv \coloneqq \read{\MsgEnc(1)}$}
\item $\Out \coloneqq m_0 \leftarrow \Msg(0); \ m_1 \leftarrow \Msg(1); \ c \leftarrow \Choice; \ \ifte{c}{\ret{m_1}}{\ret{m_0}}$
\item {\color{blue} $\LeakOut^\rec_\adv \coloneqq \read{\Out}$}
\end{itemize}

\noindent We can now fold the internal channel $\Key(1)$ into the channel $\PrivateMsg$:

\begin{itemize}
\item $\Key(0) \coloneqq \samp{\unif_\msg}$
\item $\Flip \coloneqq \samp{\flip}$
\item {\color{blue} $\LeakFlip^\rec_\adv \coloneqq \read{\Flip}$}
\item {\color{blue} $\LeakOTMsgRcvd(0)^\ot_\adv \coloneqq k_0 \leftarrow \Key(0); \ \ret{\checkmark}$}
\item {\color{blue} $\LeakOTMsgRcvd(1)^\ot_\adv \coloneqq k_1 \leftarrow \Key(1); \ \ret{\checkmark}$}
\item {\color{blue} $\LeakOTChoice^\ot_\adv \coloneqq \read{\Flip}$}
\item {\color{blue} $\LeakOTOut^\ot_\adv \coloneqq \read{\SharedKey}$}
\item $\SharedKey \coloneqq \read{\Key(0)}$
\item {\color{blue} $\LeakChoice^\rec_\adv \coloneqq \read{\Choice}$}
\item $\ChoiceEnc \coloneqq f \leftarrow \Flip; \ c \leftarrow \Choice; \\ \ifte{c}{\big(\ifte{f}{\ret{\false}}{\ret{\true}}\big)}{\big(\ifte{f}{\ret{\true}}{\ret{\false}}\big)}$
\item {\color{blue} $\LeakChoiceEnc^\rec_\adv \coloneqq \read{\ChoiceEnc}$}
\item $\PrivateMsg \coloneqq m_0 \leftarrow \Msg(0); \ m_1 \leftarrow \Msg(1); \ {\color{red} k_1 \leftarrow \unif_\msg; \ } c \leftarrow \Choice;$ \\ $\ifte{c}{\ret{m_0 \oplus k_1}}{\ret{m_1 \oplus k_1}}$
\item $\MsgEnc(0) \coloneqq m_0 \leftarrow \Msg(0); \ m_1 \leftarrow \Msg(1); \ c \leftarrow \Choice; \ k_0 \leftarrow \Key(0); \ p \leftarrow \PrivateMsg;$ \\ $\ifte{c}{\ret{p}}{\ret{m_0 \oplus k_0}}$
\item $\MsgEnc(1) \coloneqq m_0 \leftarrow \Msg(0); \ m_1 \leftarrow \Msg(1); \ c \leftarrow \Choice; \ k_0 \leftarrow \Key(0); \ p \leftarrow \PrivateMsg;$ \\ $\ifte{c}{\ret{m_1 \oplus k_0}}{\ret{p}}$
\item {\color{blue} $\LeakMsgEnc(0)^\rec_\adv \coloneqq \read{\MsgEnc(0)}$}
\item {\color{blue} $\LeakMsgEnc(1)^\rec_\adv \coloneqq \read{\MsgEnc(1)}$}
\item $\Out \coloneqq m_0 \leftarrow \Msg(0); \ m_1 \leftarrow \Msg(1); \ c \leftarrow \Choice; \ \ifte{c}{\ret{m_1}}{\ret{m_0}}$
\item {\color{blue} $\LeakOut^\rec_\adv \coloneqq \read{\Out}$}
\end{itemize}

\noindent Rearranging the order of bindings inside $\PrivateMsg$ yields
\begin{itemize}
\item $\PrivateMsg \coloneqq m_0 \leftarrow \Msg(0); \ m_1 \leftarrow \Msg(1); \ c \leftarrow \Choice; \ k_1 \leftarrow \unif_\msg; \\ \ifte{c}{\ret{m_0 \oplus k_1}}{\ret{m_1 \oplus k_1}}$
\end{itemize}
Since sampling and branching are interchangeable, and by assumption $\unif_\msg$ is invariant under xor-ing with a fixed message, the three reaction snippets
\begin{align*}
& \bullet \; k_1 \leftarrow \unif_\msg; \ \ifte{c}{\ret{m_0 \oplus k_1}}{\ret{m_1 \oplus k_1}} \\
& \bullet \; \ifte{c}{\big(k_1 \leftarrow \unif_\msg; \ \ret{m_0 \oplus k_1}\big)}{\big(k_1 \leftarrow \unif_\msg; \ \ret{m_1 \oplus k_1}\big)} \\
& \bullet \; \ifte{c}{\samp{\unif_\msg}}{\samp{\unif_\msg}}
\end{align*}
are equivalent. So we may just as well not branch:
\begin{itemize}
\item $\PrivateMsg \coloneqq m_0 \leftarrow \Msg(0); \ m_1 \leftarrow \Msg(1); \ c \leftarrow \Choice; \ \samp{\unif_\msg}$
\end{itemize}
Unfolding the sampling back gives us:

\begin{itemize}
\item $\Key(0) \coloneqq \samp{\unif_\msg}$
\item {\color{red} $\Key(1) \coloneqq \samp{\unif_\msg}$}
\item $\Flip \coloneqq \samp{\flip}$
\item {\color{blue} $\LeakFlip^\rec_\adv \coloneqq \read{\Flip}$}
\item {\color{blue} $\LeakOTMsgRcvd(0)^\ot_\adv \coloneqq k_0 \leftarrow \Key(0); \ \ret{\checkmark}$}
\item {\color{blue} $\LeakOTMsgRcvd(1)^\ot_\adv \coloneqq k_1 \leftarrow \Key(1); \ \ret{\checkmark}$}
\item {\color{blue} $\LeakOTChoice^\ot_\adv \coloneqq \read{\Flip}$}
\item {\color{blue} $\LeakOTOut^\ot_\adv \coloneqq \read{\SharedKey}$}
\item $\SharedKey \coloneqq \read{\Key(0)}$
\item {\color{blue} $\LeakChoice^\rec_\adv \coloneqq \read{\Choice}$}
\item $\ChoiceEnc \coloneqq f \leftarrow \Flip; \ c \leftarrow \Choice; \\ \ifte{c}{\big(\ifte{f}{\ret{\false}}{\ret{\true}}\big)}{\big(\ifte{f}{\ret{\true}}{\ret{\false}}\big)}$
\item {\color{blue} $\LeakChoiceEnc^\rec_\adv \coloneqq \read{\ChoiceEnc}$}
\item $\PrivateMsg \coloneqq m_0 \leftarrow \Msg(0); \ m_1 \leftarrow \Msg(1); \ c \leftarrow \Choice; \ {\color{red} \read{\Key(1)}}$
\item $\MsgEnc(0) \coloneqq m_0 \leftarrow \Msg(0); \ m_1 \leftarrow \Msg(1); \ c \leftarrow \Choice; \ k_0 \leftarrow \Key(0); \ p \leftarrow \PrivateMsg;$ \\ $\ifte{c}{\ret{p}}{\ret{m_0 \oplus k_0}}$
\item $\MsgEnc(1) \coloneqq m_0 \leftarrow \Msg(0); \ m_1 \leftarrow \Msg(1); \ c \leftarrow \Choice; \ k_0 \leftarrow \Key(0); \ p \leftarrow \PrivateMsg;$ \\ $\ifte{c}{\ret{m_1 \oplus k_0}}{\ret{p}}$
\item {\color{blue} $\LeakMsgEnc(0)^\rec_\adv \coloneqq \read{\MsgEnc(0)}$}
\item {\color{blue} $\LeakMsgEnc(1)^\rec_\adv \coloneqq \read{\MsgEnc(1)}$}\medskip
\item $\Out \coloneqq m_0 \leftarrow \Msg(0); \ m_1 \leftarrow \Msg(1); \ c \leftarrow \Choice; \ \ifte{c}{\ret{m_1}}{\ret{m_0}}$
\item {\color{blue} $\LeakOut^\rec_\adv \coloneqq \read{\Out}$}
\end{itemize}

\noindent The internal channel $\PrivateMsg$ can now be substituted away, yielding the final version of the real protocol:

\begin{itemize}
\item $\Key(0) \coloneqq \samp{\unif_\msg}$
\item $\Key(1) \coloneqq \samp{\unif_\msg}$
\item $\Flip \coloneqq \samp{\flip}$
\item {\color{blue} $\LeakFlip^\rec_\adv \coloneqq \read{\Flip}$}
\item {\color{blue} $\LeakOTMsgRcvd(0)^\ot_\adv \coloneqq k_0 \leftarrow \Key(0); \ \ret{\checkmark}$}
\item {\color{blue} $\LeakOTMsgRcvd(1)^\ot_\adv \coloneqq k_1 \leftarrow \Key(1); \ \ret{\checkmark}$}
\item {\color{blue} $\LeakOTChoice^\ot_\adv \coloneqq \read{\Flip}$}
\item {\color{blue} $\LeakOTOut^\ot_\adv \coloneqq \read{\SharedKey}$}
\item $\SharedKey \coloneqq \read{\Key(0)}$
\item {\color{blue} $\LeakChoice^\rec_\adv \coloneqq \read{\Choice}$}
\item $\ChoiceEnc \coloneqq f \leftarrow \Flip; \ c \leftarrow \Choice; \\ \ifte{c}{\big(\ifte{f}{\ret{\false}}{\ret{\true}}\big)}{\big(\ifte{f}{\ret{\true}}{\ret{\false}}\big)}$
\item {\color{blue} $\LeakChoiceEnc^\rec_\adv \coloneqq \read{\ChoiceEnc}$}
\item $\MsgEnc(0) \coloneqq m_0 \leftarrow \Msg(0); \ m_1 \leftarrow \Msg(1); \ c \leftarrow \Choice; \ k_0 \leftarrow \Key(0); \ {\color{red} k_1 \leftarrow \Key(1);}$ \\ $\ifte{c}{{\color{red} \ret{k_1}}}{\ret{m_0 \oplus k_0}}$
\item $\MsgEnc(1) \coloneqq m_0 \leftarrow \Msg(0); \ m_1 \leftarrow \Msg(1); \ c \leftarrow \Choice; \ k_0 \leftarrow \Key(0); \ {\color{red} k_1 \leftarrow \Key(1);}$ \\ $\ifte{c}{\ret{m_1 \oplus k_0}}{{\color{red} \ret{k_1}}}$
\item {\color{blue} $\LeakMsgEnc(0)^\rec_\adv \coloneqq \read{\MsgEnc(0)}$}
\item {\color{blue} $\LeakMsgEnc(1)^\rec_\adv \coloneqq \read{\MsgEnc(1)}$}
\item $\Out \coloneqq m_0 \leftarrow \Msg(0); \ m_1 \leftarrow \Msg(1); \ c \leftarrow \Choice; \ \ifte{c}{\ret{m_1}}{\ret{m_0}}$
\item {\color{blue} $\LeakOut^\rec_\adv \coloneqq \read{\Out}$}
\end{itemize}

\subsection{The Simulator}
The channel
\begin{itemize}
\item $\Out \coloneqq m_0 \leftarrow \Msg(0); \ m_1 \leftarrow \Msg(1); \ c \leftarrow \Choice; \ \ifte{c}{\ret{m_1}}{\ret{m_0}}$
\end{itemize}
can now be separated out as coming from the functionality, and the remainder of the protocol is turned into the simulator below. Plugging in the simulator into the ideal functionality and substituting away the internal channels $\LeakChoice^\id_\adv$ and $\LeakOut^\id_\adv$ that originally served as a line of communication for the adversary yields the final version of the real protocol, as desired.

\begin{itemize}
\item $\Key(0) \coloneqq \samp{\unif_\msg}$
\item $\Key(1) \coloneqq \samp{\unif_\msg}$
\item $\Flip \coloneqq \samp{\flip}$
\item {\color{blue} $\LeakFlip^\rec_\adv \coloneqq \read{\Flip}$}
\item {\color{blue} $\LeakOTMsgRcvd(0)^\ot_\adv \coloneqq k_0 \leftarrow \Key(0); \ \ret{\checkmark}$}
\item {\color{blue} $\LeakOTMsgRcvd(1)^\ot_\adv \coloneqq k_1 \leftarrow \Key(1); \ \ret{\checkmark}$}
\item {\color{blue} $\LeakOTChoice^\ot_\adv \coloneqq \read{\Flip}$}
\item {\color{blue} $\LeakOTOut^\ot_\adv \coloneqq \read{\SharedKey}$}
\item $\SharedKey \coloneqq \read{\Key(0)}$
\item {\color{red} $\LeakChoice^\rec_\adv \coloneqq \read{\LeakChoice^\id_\adv}$}
\item $\ChoiceEnc \coloneqq f \leftarrow \Flip; \ {\color{red} c \leftarrow \LeakChoice^\id_\adv;} \\ \ifte{c}{\big(\ifte{f}{\ret{\false}}{\ret{\true}}\big)}{\big(\ifte{f}{\ret{\true}}{\ret{\false}}\big)}$kip
\item {\color{blue} $\LeakChoiceEnc^\rec_\adv \coloneqq \read{\ChoiceEnc}$}
\item $\MsgEnc(0) \coloneqq {\color{red} m \leftarrow \LeakOut^\id_\adv; \ c \leftarrow \LeakChoice^\id_\adv; \ } k_0 \leftarrow \Key(0); \ k_1 \leftarrow \Key(1);$ \\ $\ifte{c}{\ret{k_1}}{{\color{red} \ret{m \oplus k_0}}}$
\item $\MsgEnc(1) \coloneqq {\color{red} m \leftarrow \LeakOut^\id_\adv; \ c \leftarrow \LeakChoice^\id_\adv; \ } k_0 \leftarrow \Key(0); \ k_1 \leftarrow \Key(1);$ \\ $\ifte{c}{{\color{red} \ret{m \oplus k_0}}}{\ret{k_1}}$
\item {\color{blue} $\LeakMsgEnc(0)^\rec_\adv \coloneqq \read{\MsgEnc(0)}$}
\item {\color{blue} $\LeakMsgEnc(1)^\rec_\adv \coloneqq \read{\MsgEnc(1)}$}
\item {\color{red} $\LeakOut^\rec_\adv \coloneqq \read{\LeakOut^\id_\adv}$}
\end{itemize}